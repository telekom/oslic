% Telekom osCompendium extract template
%
% (c) Karsten Reincke, Deutsche Telekom AG, Darmstadt 2011
%
% This LaTeX-File is licensed under the Creative Commons Attribution-ShareAlike
% 3.0 Germany License (http://creativecommons.org/licenses/by-sa/3.0/de/): Feel
% free 'to share (to copy, distribute and transmit)' or 'to remix (to adapt)'
% it, if you '... distribute the resulting work under the same or similar
% license to this one' and if you respect how 'you must attribute the work in
% the manner specified by the author ...':
%
% In an internet based reuse please link the reused parts to www.telekom.com and
% mention the original authors and Deutsche Telekom AG in a suitable manner. In
% a paper-like reuse please insert a short hint to www.telekom.com and to the
% original authors and Deutsche Telekom AG into your preface. For normal
% quotations please use the scientific standard to cite.
%
% [ File structure derived from 'mind your Scholar Research Framework' 
%   mycsrf (c) K. Reincke CC BY 3.0  http://mycsrf.fodina.de/ ]

%
% select the document class
% S.26: [ 10pt|11pt|12pt onecolumn|twocolumn oneside|twoside notitlepage|titlepage final|draft
%         leqno fleqn openbib a4paper|a5paper|b5paper|letterpaper|legalpaper|executivepaper openrigth ]
% S.25: { article|report|book|letter ... }
%
% oder koma-skript S.10 + 16
\documentclass[DIV=calc,BCOR=5mm,11pt,headings=small,oneside,abstract=true, toc=bib]{scrartcl}

%%% (1) general configurations %%%
\usepackage[utf8]{inputenc}

%%% (2) language specific configurations %%%
\usepackage[]{a4,ngerman}
\usepackage[english, ngerman]{babel}
\selectlanguage{ngerman}

% jurabib configuration
\usepackage[see]{jurabib}
\bibliographystyle{jurabib}
% Telekom osCompendium German Jurabib Configuration Include Module 
%
% (c) Karsten Reincke, Deutsche Telekom AG, Darmstadt 2011
%
% This LaTeX-File is licensed under the Creative Commons Attribution-ShareAlike
% 3.0 Germany License (http://creativecommons.org/licenses/by-sa/3.0/de/): Feel
% free 'to share (to copy, distribute and transmit)' or 'to remix (to adapt)'
% it, if you '... distribute the resulting work under the same or similar
% license to this one' and if you respect how 'you must attribute the work in
% the manner specified by the author ...':
%
% In an internet based reuse please link the reused parts to www.telekom.com and
% mention the original authors and Deutsche Telekom AG in a suitable manner. In
% a paper-like reuse please insert a short hint to www.telekom.com and to the
% original authors and Deutsche Telekom AG into your preface. For normal
% quotations please use the scientific standard to cite.
%
% [ File structure derived from 'mind your Scholar Research Framework' 
%   mycsrf (c) K. Reincke CC BY 3.0  http://mycsrf.fodina.de/ ]

% the first time cite with all data, later with shorttitle
\jurabibsetup{citefull=first}

%%% (1) author / editor list configuration
%\jurabibsetup{authorformat=and} % uses 'und' instead of 'u.'
% therefore define your own abbreviated conjunction: 
% an 'and before last author explicetly written conjunction

% for authors in citations
\renewcommand*{\jbbtasep}{ u. } % bta = between two authors sep
\renewcommand*{\jbbfsasep}{, } % bfsa = between first and second author sep
\renewcommand*{\jbbstasep}{ u. }% bsta = between second and third author sep
% for editors in citations
\renewcommand*{\jbbtesep}{ u. } % bta = between two authors sep
\renewcommand*{\jbbfsesep}{, } % bfsa = between first and second author sep
\renewcommand*{\jbbstesep}{ u. }% bsta = between second and third author sep

% for authors in literature list
\renewcommand*{\bibbtasep}{ u. } % bta = between two authors sep
\renewcommand*{\bibbfsasep}{, } % bfsa = between first and second author sep
\renewcommand*{\bibbstasep}{ u. }% bsta = between second and third author sep
% for editors  in literature list
\renewcommand*{\bibbtesep}{ u. } % bte = between two editors sep
\renewcommand*{\bibbfsesep}{, } % bfse = between first and second editor sep
\renewcommand*{\bibbstesep}{ u. }% bste = between second and third editor sep

% use: name, forname, forname lastname u. forname lastname
\jurabibsetup{authorformat=firstnotreversed}
\jurabibsetup{authorformat=italic}

%%% (2) title configuration
% in every case print the title, let it be seperated from the 
% author by a colon and use the slanted font
\jurabibsetup{titleformat={all,colonsep}}
%\renewcommand*{\jbtitlefont}{\textit}

%%% (3) seperators in bib data
% separate bibliographical hints and page hints by a comma
\jurabibsetup{commabeforerest}

%%% (4) specific configuration of bibdata in quotes / footnote
% use a.a.O if possible
\jurabibsetup{ibidem=strict}

% replace ugly a.a.O. by ders., a.a.O. resp. ders., ebda.
% but if there are more than one author or girl writers?
\AddTo\bibsgerman{
  \renewcommand*{\ibidemname}{Ds., a.a.O.}
  \renewcommand*{\ibidemmidname}{ds., a.a.O.}
}
\renewcommand*{\samepageibidemname}{Ds., ebda.}
\renewcommand*{\samepageibidemmidname}{ds., ebda.}

%%% (5) specific configuration of bibdata in bibliography
% ever an in: before journal and collection/book-tiltes 
\renewcommand*{\bibbtsep}{in: }
%\renewcommand*{\bibjtsep}{in: }

% ever a colon after author names 
\renewcommand*{\bibansep}{: }
% ever a semi colon after the title 
\renewcommand*{\bibatsep}{; }
% ever a comma before date/year
\renewcommand*{\bibbdsep}{, }

% let jurabib insert the S. and p. information
% no S. necessary in bib-files and in cites/footcites
\jurabibsetup{pages=format}

% use a compressed literature-list using a small line indent
\jurabibsetup{bibformat=compress}
\setlength{\jbbibhang}{1em}

% which follows the design of the cites and offers comments
\jurabibsetup{biblikecite}

% print annotations into bibliography
\jurabibsetup{annote}
\renewcommand*{\jbannoteformat}[1]{{ \itshape #1 }}

%refine the prefix of url download
\AddTo\bibsgerman{\renewcommand*{\urldatecomment}{Referenzdownload: }}

% we want to have the year of articles in brackets
\renewcommand*{\bibaldelim}{(}
\renewcommand*{\bibardelim}{)}

%Umformatierung des Reihentitels und der Reihennummer
\DeclareRobustCommand{\numberandseries}[2]{%
\unskip\unskip%,
\space\bibsnfont{(=~#2}%
\ifthenelse{\equal{#1}{}}{)}{, [Bd./Nr.]~#1)}%
}%

% Local Variables:
% mode: latex
% fill-column: 80
% End:


% language specific hyphenation
\input{../btexmat/oscHyphenationDeInc}

%%% (3) layout page configuration %%%

% select the visible parts of a page
% S.31: { plain|empty|headings|myheadings }
%\pagestyle{myheadings}
\pagestyle{headings}

% select the wished style of page-numbering
% S.32: { arabic,roman,Roman,alph,Alph }
\pagenumbering{arabic}
\setcounter{page}{1}

% select the wished distances using the general setlength order:
% S.34 { baselineskip| parskip | parindent }
% - general no indent for paragraphs
\setlength{\parindent}{0pt}
\setlength{\parskip}{1.2ex plus 0.2ex minus 0.2ex}


%%% (4) general package activation %%%
%\usepackage{utopia}
%\usepackage{courier}
%\usepackage{avant}
\usepackage[dvips]{epsfig}

% graphic
\usepackage{graphicx,color}
\usepackage{array}
\usepackage{shadow}
\usepackage{fancybox}

%- start(footnote-configuration)
%  flush the cite numbers out of the vertical line and let
%  the footnote text directly start in the left vertical line
\usepackage[marginal]{footmisc}
%- end(footnote-configuration)

\begin{document}

%% use all entries of the bliography

%%-- start(titlepage)
\titlehead{Literaturexzerpt}
\subject{Autor(en): Renner, Thomas \ldots}
\title{Titel: Open Source Software. Einsatzpotentiale ..}
\subtitle{Jahr: 2005 }
\author{K. Reincke\input{../btexmat/oscLicenseFootnoteInc}}

%\thanks{den Autoren von KOMA-Script und denen von Jurabib}
\maketitle
%%-- end(titlepage)
%\nocite{*}

\begin{abstract}
\noindent
Das Werk / The work\footcite[][]{RenVetRexKet2005a} \\
\noindent \itshape
\ldots Diskutiert Einsatzpotentiale von OSS und offeriert eine Methodik, die
Wirtschaftlichkeit ihres Einsatzes zu berechnen. Liefert eine gute
Begriffssystematik und einen Softwareüberblick. Die Berechnung der
Wirtschaftlichkeit beschränkt sich für Client-Maschinen auf eine Migration von
MS-Office nach Open-Office. Pikant sind zudem die auf Interviews beruhenden
Einschätzungen zu Vor- und Nachteilen von OSS. Auffallend ist auch, dass die
Lizenzerfüllung nicht thematisiert wird, obwohl hervorgehoben wird, dass OSS
gerade nicht frei von Lizenzbedingungen sei. \\
\noindent
\ldots The study discusses potentials of OSS and offers a method to compute its'
cost effectiveness. It presents a good survey of the underlying concepts and a
list of OS applications. For client mashines the computation of the cost
effectiveness refers only to the migration from MS-Office to Open-Office.
Unfortunately the list of advantages and disadvantages based on interviews is a
little inconsistent. Remarkably is also that the study doesn't focus on the act
of fulfilling an OS license although it highlights, that OSS is not free of
license conditions.

\end{abstract}
\footnotesize
%\tableofcontents
\normalsize

\section{Line of Thought}

Ziel der Studien ist eine Darstellung der \glqq{}Einsatzpotentiale von
Open Source Software\grqq{} und einer \glqq{}[\ldots] Methodik, um die
Wirtschaftlichkeit des Einsatzes zu
berechnen\grqq{}\footcite[vgl.][10]{RenVetRexKet2005a}.

Dazu werden zunächst - unter dem Stichwort \glqq{}Charakteristika von
Open Source Software\grqq{}\footcite[vgl.][12]{RenVetRexKet2005a} - die gängigen
Begriffe definiert\footcite[vgl.][12ff]{RenVetRexKet2005a}, Vor- und Nachteile
der Open Source Software aufgelistet\footcite[vgl.][16ff]{RenVetRexKet2005a},
gängigen Lizenzmodelle spezifziert\footcite[vgl.][19]{RenVetRexKet2005a},
Rechtsfragen erläutert\footcite[vgl.][23]{RenVetRexKet2005a} und
organisatorische Kontexte refiert\footcite[vgl.][29]{RenVetRexKet2005a}.

Daran schließt sich eine
\glqq{}Ist-Analyse\grqq{}\footcite[vgl.][31]{RenVetRexKet2005a} der Open Source
Software an, die im Wesentlichen aus einer auf den Verwendungskontext bezogenen
Gegenüberstellung von proprietärer und freier Software
besteht\footcite[vgl.][36]{RenVetRexKet2005a}. Diese Ist-Analyse mündet in einer
Ableitung der Vor- und Nachteile von Open Source Software aus Interviews, zum
einen aus Fraunhofer internen Interviews\footcite[vgl.][56f]{RenVetRexKet2005a},
zum anderen aus \glqq{}Befragung bei deutschen
Unternehmen\grqq{}\footcite[vgl.][68f]{RenVetRexKet2005a}. Die
Gegenüberstellung von \glqq{}Anwendungsfeldern\grqq{} und der darin relevanten
Programme wird in dann vertieft, wobei die jeweiligen \glqq{}Bewertungen
und Empfehlungen\grqq{} aus den \glqq{}Erhebungen\grqq{} abgeleitet
werden\footcite[vgl. dazu][70ff insbesondere S.82ff]{RenVetRexKet2005a}

Kern der Studie ist dann auch eine
\glqq{}Wirtschaftlichkeitsbetrachtung\grqq{}\footcite[vgl.][154]{RenVetRexKet2005a},
die sich - hinsichtlich der Nutzung von Clientrechner - auf eine Migration von
Microsoft-Office nach Open-Office und und Open-Groupware
'reduziert'\footcite[vgl.][156]{RenVetRexKet2005a}.

\section{Specific Aspects}

\subsection{Begriffsdefinition}

Die Studie definiert Open Source Software in direkter Anlehnung an die Open
Source Initiative\footcite[vgl.][12]{RenVetRexKet2005a}. 

Dem gegenüber stellt zunächst sie die Public Domain Software und weist explizit
daraufhin, dass dies ein für Deutschland nicht adäquates Konzept darstelle, weil
hier \glqq{}[\ldots] der Abtritt aller Urheberrechte an die Allgemeinheit gar
nicht möglich (sei)\grqq{}, weshalb Public Domain Software daher in Deutschland
\glqq{}[\ldots] (rechtlich) wie Freeware behandelt
(werde)\grqq{}\footcite[vgl.][14]{RenVetRexKet2005a}.

Unter dem Terminus \glqq{}Freeware\grqq{} wird sodann die Software
zusammengefasst, bei der Nutzungs und Verbreitungsrechte an Endnutzer abgetreten
werden, gelegentlich allerdings unter zusätzlichen Einschränkungen oder
Auflagen. Das Konzept der Offenlegung von Quellcode gehöre jedoch nicht zu den
konstitutiven Merkmalen von
\glqq{}Freeware\grqq{}\footcite[vgl.][14]{RenVetRexKet2005a}

Neben Open Source Software und Freeware unterscheidet die Studie auch noch
zwischen \glqq{}Shareware\grqq{} und \glqq{}kommerzieller Software\grqq{}, wobei
sie unterstreicht, dass die Idee von Shareware letztlich nur \glqq{}[\ldots] ein
besonderes Vermarktungsmodell für kommerzielle Software
(sei)\grqq{}\footcite[vgl.][14f]{RenVetRexKet2005a}

\subsection{Lizenzmodelle}

Statistisch bezieht die Studie sich bzgl. der Lizenzverteilung auf Angaben von
BerliOS\grqq{}\footcite[vgl.][22]{RenVetRexKet2005a}. Der angegebene Link 
\texttt{http://openfacts.berlios.de/index.phtml?title=Open-Source-Lizenzen}
läuft heute, 6 Jahre nach der Untersuchung, leider auf einen unerreichbaren
Server 195.37.77.139. Es wäre interessant gewesen, die Veränderungen sehen zu
können.

Die Studie unterstreicht, dass \glqq{}Open Source Software\grqq{} gerade nicht
\glqq{}[\ldots] frei von Lizenzbedingungen (sei)\grqq{}; sie bezeichnet die
gegenteilige Annahme als \glqq{}ein besonders oft anzutreffendes
Missverständnis\grqq{}\footcite[vgl.][19]{RenVetRexKet2005a}. Allerdings spricht
sie sodann gerade nicht die Forderungen von OS Lizenzen an, die diese an ihre
Lizenznehmer herantragen, sondern sie verweist auf das Urheberrecht, dass
das Vervielfältigungs- und Verbreitungsrecht \glqq{}[\ldots]ausschließlich dem
Urheber (gestatte)\grqq{}\footcite[vgl.][19]{RenVetRexKet2005a}:
\glqq{}Im Falle von Open Source Software muss also auf dieses Recht
explizit verzichtet werden, da jede Weitergabe der Software ansonsten
illegal wäre\grqq{}\footcite[][19f]{RenVetRexKet2005a}.

Dann listet sie detaillierter jedoch nur die
GPL, die
LGPL\footcite[vgl.][21]{RenVetRexKet2005a} und die
BSD\footcite[vgl.][21]{RenVetRexKet2005a} Lizenz auf. Damit konstituiert sich
mittlerbar ein dreistufiges Klassififikationsmodell, demzufolge 
\begin{itemize}
  \item (a) manche Lizenzen beim Einbau in ein größeres Konglomerat dessen
  Lizenzierung unter derselben Lizenz erzwängen
  (GPL)\footcite[vgl.][20]{RenVetRexKet2005a}
  \item während (b) die nächst schwächere Kategorie zwar den lizenzmäßig
  folgenlosen Einbau als Komponente erlaube, bei Änderungen an sich selbst aber
  die Re-Veröffentlichung von sich selbst unter derselben Lizenz fordere
  (LGPL)\footcite[vgl.][21]{RenVetRexKet2005a},
  \item wovon sich (c) der Typus der nahezu ganz freien Lizenzen abgrenze
  (BSD)\footcite[vgl.][21]{RenVetRexKet2005a}
\end{itemize}

Obwohl diese mittelbar herausgearbeitete Typologie der Sachlage am nächsten
kommt, fällt es auch den Autoren dieser Studie schwer, der Wesen der GPL
sprachlich zu erfassen. Sie schreiben:

\begin{quote}\glqq{}Darüber hinaus ist die Kerneigenschaft der GPL, dass
Modifikationen an einer der GPL unterliegenden Software im Falle einer
Weiterverarbeitung wiederum der GPL unterliegen müssen. Es ist also nicht
möglich, Teile von GPL Software in kommerziellen (Closed Source) Produkten zu
verwenden.\grqq{}\footcite[][20]{RenVetRexKet2005a}
\end{quote}

Dies ist ungenau gesagt. Kerneigenschaft der GPL ist in Wirklichkeit, dass sie
(a) bei Änderungen an einer ihr unterliegenden Software die Wiederöffentlichung
der Software revidierten Software fordert und dass sie (b) bei Einbau einer ihr
unterliegenden Software oder Teile davon in ein größeres Ganzes auch die
Veröffentlichung des Ganzen unter der GPL erzwingt. Geschuldet ist das dem
Begriff des 'abgeleiteten Werkes', wie er in der GPL vorkommt: er erfasst beide
Dimensionen, den der Veränderung des Originals und den des unveränderten Einbaus
in ein größeres Ganzes.

\subsection{Inkonsistenz}

Besagt Inkonsistenz bezieht sich darauf, dass unter dem Stichwort
\glqq{}Charakteristika von Open Source Software\grqq{} und unter der Überschrift
\glqq{}Nachteile von Open Source Software\grqq{} zunächst konstatiert wird, es
gäbe \glqq{}(keinen) Support durch
Entwickler\grqq{}\footcite[vgl.][17]{RenVetRexKet2005a}. Dieser 'Befund' wird am
Ende der Studien unter dem Stichwort \glqq{}Strategische Analyse\grqq{} bruchlos
als Faktum wiederholt\footcite[vgl.][172]{RenVetRexKet2005a}. In beiden
Befragungen - bei der innerhalb der Fraunhofer Institute und bei der in
deutschen Unternehmen - wird das Thema \glqq{}Support durch Entwickler\grqq{}
jedoch unter dem Stichwort \glqq{}Vorteile von Open Source Software\grqq{} ab
gehandelt: Von den Fraunhofer intern Befragten geben 76\% (54\% + 22\%) an, es
träfe eher oder voll zu, dass der \glqq{}Support durch Entwickler\grqq{} ein
\glqq{}Vorteil von Open Source Software\grqq{}
sei\footcite[vgl.][56]{RenVetRexKet2005a}. Und von den extern Befragten geben
immerhin 55\% (42\% + 13\%) an, es träfe eher oder voll zu, dass der
\glqq{}Support durch Entwickler\grqq{} ein \glqq{}Vorteil von Open Source
Software\grqq{} sei\footcite[vgl.][69]{RenVetRexKet2005a}. Das Beibehalten der
Behauptung, es geben keinen Support von Seiten der Open Source Entwickler kann
so nicht gerechtfertigt werden.

\subsection{Rechtsbestimmung}

Bei der rechtlichen Einschätzung konstatiert die Studie zunächst, dass
\glqq{}das Urheberrecht [\ldots] dem Autor das ausschließliche Recht zur
Weiterverbreitung, Veränderung und zur Anforderung von Kopien
(zugestehe)\grqq{}, was dann bruchlos auch für Open Source Sofwtare
gelte\footcite[vgl.][23]{RenVetRexKet2005a}.

Sodann vermerkt sie in Anlehnung an Jaeger\&Metzeger, dass \glqq{}nach
deutschem Recht [\ldots] Open Source Lizenzen als Allgemeine
Geschäftsbedingungen aufzufassen (seien)\grqq{}. Der AGB Charakter
unterbinde dann jenes Auschließen jeglicher Gewährungsleistung, wie es
gemeinhein in OS Lizenzen formuliert werde. Dies Lösung dafür bestehe in der
Anwendung des \glqq{}Schenkungsrechtes\grqq{}, denn immerhin werde dem Gegenüber
die Software lizenztechnisch \glqq{}kostenfrei\grqq{} zur
\glqq{}Nutzung\grqq{} übergeben. Insofern seien
\glqq{}[\ldots] Gewährleistungsansprüche auf Arglist beschränkt\grqq{},
\glqq{}[\ldots] Haftungsansprüche (hingegen) auf Vorsatz und grobe
Fahrlässigkeit\grqq{}\footcite[vgl.][23]{RenVetRexKet2005a}.

Eine weitere Komplikation - so die Studie - bestehe bei der Deutung der OS
Lizenz als AGB darin, dass AGBs angemessen präsentiert werden müssen, bevor sie
zu gelten beginnen können. Der Ausweg aus diesem Dilemma bestehe darin, dass man
berichtigertweise \glqq{}unterstellen\grqq{} dürfe, der Nutzer habe die AGBs
resp. die Liezenz zur Kenntnis genommen, wenn der die Software an dritte
weiterreiche oder verändere\glqq{}\footcite[vgl.][23]{RenVetRexKet2005a}. Beides
sind Aktionen, die eine Informationspflicht über die Rechtmäßigkeit beinhalten.

\subsection{Vor- und Nachteile}
Die Studie listet die \glqq{}Anpassbarkeit\grqq{} und die
\glqq{}Wiederverwendbarkeit\grqq{} von Open Source Software Code ebenso als
Vorteil auf, wie die \glqq{}höhere Produktqualität\grqq{}, die
\glqq{}Anbieterunabhängigkeit\grqq{}, die \glqq{}höhere Sicherheit\grqq{}, die
Nutzung \glqq{}offener Standards\grqq{} und der \glqq{}Wegfall von
Lizenzkosten\grqq{}\footcite[vgl.][16f]{RenVetRexKet2005a}.

Zumindest zwei der behaupteten Nachteile können unmittelbar entkräftet
werden:

\begin{itemize}
  \item Dass \glqq{}keine
  Gewährleistungsrechte\grqq{}\footcite[vgl.][17]{RenVetRexKet2005a} existieren,
  wird mittlerweile durch Firmen abgefedert, die das Pendant als Service an
  zubieten, als ihr Geschäftsmodell sehen.
  \item Dass \glqq{}kein
  Support durch Entwickler\grqq{}\footcite[vgl.][17]{RenVetRexKet2005a} existiere, kann
  aus den eigenen Befragungen gerade nicht abgeleitet werden.
  \item 
\end{itemize}

Die anderen Nachteile - der \glqq{}höhere Schulungsaufwand\grqq{},die 
\glqq{}ungewisse Weiterentwicklung\grqq{}, das Fehlen von Appliaktionen und die
\glqq{}magelhafte Interoperationalität mit kommerzieller
Software\grqq{}\footcite[vgl.][18]{RenVetRexKet2005a} - müßten noch gesondert diskutiert
werden.
\small
\bibliography{../bibfiles/oscResourcesDe}

\end{document}
