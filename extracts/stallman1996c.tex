% Telekom osCompendium extract template
%
% (c) Karsten Reincke, Deutsche Telekom AG, Darmstadt 2011
%
% This LaTeX-File is licensed under the Creative Commons Attribution-ShareAlike
% 3.0 Germany License (http://creativecommons.org/licenses/by-sa/3.0/de/): Feel
% free 'to share (to copy, distribute and transmit)' or 'to remix (to adapt)'
% it, if you '... distribute the resulting work under the same or similar
% license to this one' and if you respect how 'you must attribute the work in
% the manner specified by the author ...':
%
% In an internet based reuse please link the reused parts to www.telekom.com and
% mention the original authors and Deutsche Telekom AG in a suitable manner. In
% a paper-like reuse please insert a short hint to www.telekom.com and to the
% original authors and Deutsche Telekom AG into your preface. For normal
% quotations please use the scientific standard to cite.
%
% [ File structure derived from 'mind your Scholar Research Framework' 
%   mycsrf (c) K. Reincke CC BY 3.0  http://mycsrf.fodina.de/ ]

%
% select the document class
% S.26: [ 10pt|11pt|12pt onecolumn|twocolumn oneside|twoside notitlepage|titlepage final|draft
%         leqno fleqn openbib a4paper|a5paper|b5paper|letterpaper|legalpaper|executivepaper openrigth ]
% S.25: { article|report|book|letter ... }
%
% oder koma-skript S.10 + 16
\documentclass[DIV=calc,BCOR=5mm,11pt,headings=small,oneside,abstract=true, toc=bib]{scrartcl}

%%% (1) general configurations %%%
\usepackage[utf8]{inputenc}

%%% (2) language specific configurations %%%
\usepackage[]{a4,ngerman}
\usepackage[ngerman, english]{babel}
\selectlanguage{english}

% jurabib configuration
\usepackage[see]{jurabib}
\bibliographystyle{jurabib}
\input{../btexmat/oscJbibCfgEnInc}

% language specific hyphenation
\input{../btexmat/oscHyphenationEnInc}

%%% (3) layout page configuration %%%

% select the visible parts of a page
% S.31: { plain|empty|headings|myheadings }
%\pagestyle{myheadings}
\pagestyle{headings}

% select the wished style of page-numbering
% S.32: { arabic,roman,Roman,alph,Alph }
\pagenumbering{arabic}
\setcounter{page}{1}

% select the wished distances using the general setlength order:
% S.34 { baselineskip| parskip | parindent }
% - general no indent for paragraphs
\setlength{\parindent}{0pt}
\setlength{\parskip}{1.2ex plus 0.2ex minus 0.2ex}


%%% (4) general package activation %%%
%\usepackage{utopia}
%\usepackage{courier}
%\usepackage{avant}
\usepackage[dvips]{epsfig}

% graphic
\usepackage{graphicx,color}
\usepackage{array}
\usepackage{shadow}
\usepackage{fancybox}

%- start(footnote-configuration)
%  flush the cite numbers out of the vertical line and let
%  the footnote text directly start in the left vertical line
\usepackage[marginal]{footmisc}
%- end(footnote-configuration)

\begin{document}

%% use all entries of the bliography

%%-- start(titlepage)
\titlehead{Literaturexzerpt}
\subject{Autor(en): Stallman / Stallman1996c }
\title{Titel: What is Copyleft?}
\subtitle{Jahr: 1996 / 2002 }
\author{K. Reincke\input{../btexmat/oscLicenseFootnoteInc}}

%\thanks{den Autoren von KOMA-Script und denen von Jurabib}
\maketitle
%%-- end(titlepage)
%\nocite{*}

\begin{abstract}
\noindent
\cite[(in:)][]{StaGay2002a} \\
\noindent
\cite[(ist:)][]{Stallman1996c} \\
Das Werk / The work\footcite[][]{Stallman1996c} \\
\noindent \itshape
\ldots  Beschreibt 'Copyleft' als Methode, mit der Programme und all deren
Modifikationen und Ableitungen zu freier Software gemacht werden: Zunächst
unterstelle man es dem normalen Copyright. Dann erlaube man als
Copyright-Inhaber die freie Nutzung, Veränderung und Weitergabe, sofern diese
Erlaubnis nicht entfernt worden werde. [In Deutschland braucht der Urheber die
Urheberschaft nicht über eine Copyright-Notiz zu beanspruchen und also das
Gemeinfrei-Werden nicht extra zu verhindern. Als Urheber er ist immer schon der
Inhaber der Urheberrechte.]
\\
\noindent
\ldots This article describes Copyleft as a method for making a programm free
software and requiring all modified and extended versions of the program to be
free software as well: firstly one states that the code is copyrighted, than -
as copyright holder - one allows to use, modify, and redistribute the sourcecode
under the condition, that this permission is not deleted.
\end{abstract}
\footnotesize
%\tableofcontents
\normalsize

\section{Line of Thought}

\subsection{Purpose}
RMS declares that \glqq{}Copyleft is a [\ldots] method for making a programm
free software and requiring all modified and extended versions of the program to
be free software as well\grqq{}\footcite[cf][89]{Stallman1996c}.

\subsection{Meaning}

Following RMS summary \glqq{}Copyleft says that anyone who redistribute the
software, with or without changes, must pass along the freedom to further copy
and change it\grqq{}: \glqq{}Copyleft garantuees that every user has
freedom\grqq{}\footcite[cf][89]{Stallman1996c}

\subsection{Method}

The American way to \glqq{}copyleft\grqq{} eg.: to establish a copyleft for a
work requires two steps: firstly one states, \glqq{}[\ldots] that it is
copyrighted\grqq{} and secondly one adds \glqq{}[\ldots] distribution
terms, which are a legal instrument that gives everyone the rights to use,
modify, and redistribute the program's code or any program derived from it
but only if the distribution terms are unchanged\grqq{}. This is method by which 
\glqq{}the code and the freedoms become legally
inseparable\grqq{}\footcite[cf][89]{Stallman1996c}.

\subsection{Context}
Stallman also mentions that \glqq{}the simplest way to make a program
free is to put it in the public domain, uncopyrigthed\grqq{}. But - and that's
the disadvantage Stallman fights against - this \glqq{}[\ldots] allows
uncooperative people to convert the program into proprietary software\grqq{}:
\glqq{}They can make changes, many or few, and distribute the result as a
proprietary product\grqq{}\footcite[cf][89]{Stallman1996c}..


\subsection{Wording}

The word 'copyleft' was selected with respect to the assumed meaning of
the normal word 'copyright' \glqq{}Proprietary software developers use
copyright to take away the users' freedom; we use copyright to guarantee
their freedom\grqq{}. And because of this changing of the meaning Stallman says:
\glqq{}That's why we reverse the name changing 'copyright' into
'copyleft'\grqq{}\footcite[cf][89]{Stallman1996c}.

[for the meaning see also extract of The GNU Project : Don Hopkins]

\small
\bibliography{../bibfiles/oscResourcesEn}

\end{document}
