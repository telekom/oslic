% Telekom osCompendium cloak file English text
%
% (c) Karsten Reincke, Deutsche Telekom AG, Darmstadt 2011
%
% This LaTeX-File is licensed under the Creative Commons Attribution-ShareAlike
% 3.0 Germany License (http://creativecommons.org/licenses/by-sa/3.0/de/): Feel
% free 'to share (to copy, distribute and transmit)' or 'to remix (to adapt)'
% it, if you '... distribute the resulting work under the same or similar
% license to this one' and if you respect how 'you must attribute the work in
% the manner specified by the author ...':
%
% In an internet based reuse please link the reused parts to www.telekom.com and
% mention the original authors and Deutsche Telekom AG in a suitable manner. In
% a paper-like reuse please insert a short hint to www.telekom.com and to the
% original authors and Deutsche Telekom AG into your preface. For normal
% quotations please use the scientific standard to cite.
%
% [ File structure derived from 'mind your Scholar Research Framework' 
%   mycsrf (c) K. Reincke CC BY 3.0  http://mycsrf.fodina.de/ ]

\documentclass[DIV=calc,BCOR=5mm,12pt,headings=small,oneside,toc=bib]{scrbook}

%%% (1) general configurations %%%
\usepackage[utf8]{inputenc}

%%% (2) language specific configurations %%%
\usepackage[]{a4,ngerman}
\usepackage[ngerman, english]{babel}
\selectlanguage{english}

%language specific quoting signs
%default for language emglish is american style of quotes
\usepackage[english=british]{csquotes}

% jurabib configuration
\usepackage[see]{jurabib}
\bibliographystyle{jurabib}
\input{btexmat/oscJbibCfgEnInc}

% language specific hyphenation
\input{btexmat/oscHyphenationEnInc}

%%% (3) layout page configuration %%%

% select the visible parts of a page
% S.31: { plain|empty|headings|myheadings }
%\pagestyle{myheadings}
\pagestyle{headings}

% select the wished style of page-numbering
% S.32: { arabic,roman,Roman,alph,Alph }
\pagenumbering{arabic}
\setcounter{page}{1}

% select the wished distances using the general setlength order:
% S.34 { baselineskip| parskip | parindent }
% - general no indent for paragraphs
\setlength{\parindent}{0pt}
\setlength{\parskip}{1.2ex plus 0.2ex minus 0.2ex}

%%% (4) general package activation %%%
%\usepackage{utopia}
%\usepackage{courier}
%\usepackage{avant}
\usepackage[dvips]{epsfig}

% graphic
\usepackage{graphicx,color}
\usepackage{array}
\usepackage{shadow}
\usepackage{fancybox}

%- start(footnote-configuration)
%  flush the cite numbers out of the vertical line and let
%  the footnote text directly start in the left vertical line
% \usepackage[marginal,hang]{footmisc}
% \renewcommand\footnotemargin{1.5em}

% formatting the footnote with koma script tools
% \deffootnote[1em]{1.5em}{1em}{\textsuperscript{\thefootnotemark}}
\deffootnote[1.5em]{1.5em}{1.5em}{\textsuperscript{\thefootnotemark)\ }}


%\deffootnote[0em]{1.5em}{1em}{\textsuperscript{\thefootnotemark}}
%- end(footnote-configuration)


% %- start(endnote-configuration) uncomment to activate
% % Let all notes being marked with \endnote instead of \footnote
% % become endnotes. This set of endnotes replaces the next 
% % arising command \theendnotes - even if it is not located
% % at the end of the text.
% 
% \usepackage{endnotes}
% 
% % Format endnotes as Block with indention - Solution 1
% %\renewcommand\enoteformat{%
% %   \noindent\theenmark.) \ \hangindent .7\parindent%
% %}
% 
% % Format endnotes as Block with indention - Solution 2
% \makeatletter
% \def\enoteformat{\rightskip\z@ \leftskip0em \parindent=0em \parskip=0em
% \leavevmode\llap{\hbox{\@theenmark.~}}}
% \makeatother
% 
% \renewcommand\notesname{Annotations}
% % additionally we shall active a special jurabib option
% % if we want to get all jurabib footnotes as endnotes
% \jurabibsetup{citetoend=true}
% %- end(footnote-configuration)

% - additional packages

\usepackage{tikz}
\usetikzlibrary{arrows}
\usetikzlibrary{shapes,snakes}
\usetikzlibrary{positioning}
\usetikzlibrary{decorations.text}

\usepackage{multirow}

\usepackage{blindtext}
\usepackage{caption}

\usetikzlibrary{matrix}

\usepackage{amsmath,amsfonts}
\usepackage{amssymb}
\usepackage{wasysym}
\usepackage{pstricks, pst-node, pst-tree}
\usepackage{chngcntr}
\usepackage{nameref}



\counterwithout{footnote}{chapter}

\usepackage[intoc]{nomencl}
\let\abbr\nomenclature
% Modify Section Title of nomenclature
\renewcommand{\nomname}{Periodicals, Shortcuts, and Abbreviations}
%\renewcommand{\nomname}{Periodika, ihre Kurzformen und generelle Abkürzungen}

% insert point between abbrewviation and explanation
\setlength{\nomlabelwidth}{.24\hsize}
\renewcommand{\nomlabel}[1]{#1 \dotfill}
% reduce the line distance
\setlength{\nomitemsep}{-\parsep}
\makenomenclature

% depth of contents
\setcounter{secnumdepth}{5}
\setcounter{tocdepth}{5}
%%%%%%%%%%%%%%
\begin{document}

%% use all entries of the bliography
%% \nocite{*}

%%-- start(titlepage)
\titlehead{Version 0.98.5.4
 % -- \input{rel-date}
}
\subject{\small \itshape A Practical Guide for Developers, Managers, OS Experts, 
and Companies} 

\title{Open Source License Compendium}

\subtitle{How to Achieve Open Source License Compliance\input{btexmat/oscLicenseFootnoteInc}}
\author{
Karsten Reincke\thanks{Deutsche Telekom AG, Products \& Innovation, 
T-Online-Allee 1, 64295 Darmstadt}
\and
Greg Sharpe\thanks{Deutsche Telekom AG, Telekom Deutschland GmbH, 
Landgrabenweg, Bonn}}

\maketitle
%%-- end(titlepage)

\footnotesize
\begin{flushright} 

\parbox{100mm}{\itshape
The Open Source Community is a swarm: it is more powerful than a set of
arbritarily selected experts. We are proud to have its support. Gladly we thank
(in alphabetical order):
}

\parbox{50mm}{
\tiny
\begin{flushright}
Eitan Adler,\\
Stefan Altmeyer,\\
John Dobson, \\
Steffen Härtlein, \\
Ta'Id Holmes, \\
Michael Kern,\\
Michael Machado,\\
Thorsten Müller,\\
Thomas Quiehl,\\
Peter Schichl,\\
Helene Tamer,\\
Bernhard Tsai,\\
and all the others\ldots
\end{flushright}
}
\end{flushright}
\normalsize
\newpage

\footnotesize
\tableofcontents
\newpage
% Telekom osCompendium 'for being included' snippet template
%
% (c) Karsten Reincke, Deutsche Telekom AG, Darmstadt 2011
%
% This LaTeX-File is licensed under the Creative Commons Attribution-ShareAlike
% 3.0 Germany License (http://creativecommons.org/licenses/by-sa/3.0/de/): Feel
% free 'to share (to copy, distribute and transmit)' or 'to remix (to adapt)'
% it, if you '... distribute the resulting work under the same or similar
% license to this one' and if you respect how 'you must attribute the work in
% the manner specified by the author ...':
%
% In an internet based reuse please link the reused parts to www.telekom.com and
% mention the original authors and Deutsche Telekom AG in a suitable manner. In
% a paper-like reuse please insert a short hint to www.telekom.com and to the
% original authors and Deutsche Telekom AG into your preface. For normal
% quotations please use the scientific standard to cite.
%
% [ File structure derived from 'mind your Scholar Research Framework' 
%   mycsrf (c) K. Reincke CC BY 3.0  http://mycsrf.fodina.de/ ]

%


%% use all entries of the bibliography

%\chapter*{History}

\begin{table}
\footnotesize
\caption{History of the Open Source License Compendium}
\begin{center}
\begin{tabular}{|r|c|p{10cm}|}
\hline
\hline
    \texttt{2013-05-20}
  & \texttt{0.96.1} 
  & Linux Days release\newline    
    $\vartriangleright$ open source use cases and licenses specific usecase renamed\newline
    $\vartriangleright$ version matches the content of OSCAd\\
\hline
    \texttt{2013-04-15}
  & \texttt{0.95.2} 
  & FSFE LLW post release\newline
    $\vartriangleright$ to-do lists for nearly all popular OSI licenses\newline
    $\vartriangleright$ improved finder for GPL and EUPL\newline
    $\vartriangleright$ simplified form and improved structure of the OSLiC finder\newline    
    $\vartriangleright$ branches merged and new master published\\
\hline
    \texttt{2013-04-05}
  & \texttt{0.95.1} 
  & FSFE LLW pre release\newline
    $\vartriangleright$ to-do lists for all permissive and all week copyleft licenses\newline
    $\vartriangleright$ branches merged and new master published\\
\hline
    \texttt{2013-03-15}
  & \texttt{0.94.1} 
  & Chemnitzer Linux Day release\newline
    $\vartriangleright$ to-do lists for all permissive and some week copyleft licenses\newline
    $\vartriangleright$ branches merged and new master published\\
\hline
    \texttt{2013-03-08}
  & \texttt{0.90.1} 
  & CeBIT release\newline
    $\vartriangleright$ to-do lists for the some important licenses added\newline
    $\vartriangleright$ branches merged and new master published\\
\hline
    \texttt{2013-02-16}
  & \texttt{0.8.90} 
  & CeBIT pre release\newline
    $\vartriangleright$ new arguing structure focused on the topic license fulfillment\newline
    $\vartriangleright$ new classifying license review\newline   
    $\vartriangleright$ new top down introduction\\
\hline
    \texttt{2012-12-28}
  & \texttt{0.8.0} 
  & inmternal EOY release\newline
    $\vartriangleright$ many distributed improvents unified in branch kreinck\\
\hline
    \texttt{2012-08-25}
  & \texttt{0.5.2} 
  & expanded break through release\newline
    $\vartriangleright$ MIT license fulfilling to-do lists\newline
    $\vartriangleright$ using integrated Eclipse spell checking methods\\
\hline
    \texttt{2012-07-06}
  & \texttt{0.4.0} 
  & break through release\newline
    $\vartriangleright$ open source use case definition and taxonomy\newline 
    $\vartriangleright$ open source use case based general finder\newline 
    $\vartriangleright$ corresponding BSD specific mini finder\newline 
    $\vartriangleright$ BSD license fulfilling to-do lists\\
\hline
    \texttt{2012-03-22}
  & \texttt{0.2.1} 
  & $\vartriangleright$ framework published as first community edition\\
\hline
    \texttt{2012-01-31}
  & \texttt{0.1.8} 
  & $\vartriangleright$ renamed existing introduction as prolegomena\newline
    $\vartriangleright$ inserted a shorter top-down written introduction\newline
    $\vartriangleright$ inserted an OSLiC disclaimer\newline
    $\vartriangleright$ oscCopiedButNotRead.bib expanded\newline 
    $\vartriangleright$ list of periodicals and shortcuts added\newline
    $\vartriangleright$ many bibliographic data added\\
\hline
    \texttt{2011-09-29}
  & \texttt{0.1.4} 
  & $\vartriangleright$ document history integrated\newline
    $\vartriangleright$ typos erased\newline
    $\vartriangleright$ review of english teacher integrated\newline
    $\vartriangleright$ improvements of John integrated\\
\hline
    \texttt{2011-09-12}
  & \texttt{0.1.0} 
  & $\vartriangleright$ introduction completed: purpose and methods \\
\hline
\hline 
\end{tabular}
\end{center}
\end{table}



%\bibliography{../bibfiles/oscResourcesEn}

\normalsize

\chapter*{Disclaimer}
% Telekom osCompendium 'for beeing included' snippet template
%
% (c) Karsten Reincke, Deutsche Telekom AG, Darmstadt 2011
%
% This LaTeX-File is licensed under the Creative Commons Attribution-ShareAlike
% 3.0 Germany License (http://creativecommons.org/licenses/by-sa/3.0/de/): Feel
% free 'to share (to copy, distribute and transmit)' or 'to remix (to adapt)'
% it, if you '... distribute the resulting work under the same or similar
% license to this one' and if you respect how 'you must attribute the work in
% the manner specified by the author ...':
%
% In an internet based reuse please link the reused parts to www.telekom.com and
% mention the original authors and Deutsche Telekom AG in a suitable manner. In
% a paper-like reuse please insert a short hint to www.telekom.com and to the
% original authors and Deutsche Telekom AG into your preface. For normal
% quotations please use the scientific standard to cite.
%
% [ File structure derived from 'mind your Scholar Research Framework' 
%   mycsrf (c) K. Reincke CC BY 3.0  http://mycsrf.fodina.de/ ]

%

%%% \chapter*{Disclaimer} %%%

This book shall be thoroughly developed - together with the Open Source
Community. Finally it shall deliver reliable information, following the rule,
that the swarm knows more than the single fish.

But nevertheless it can't offer more than the opinion(s) of its authors and
contributors. It's only one voice of chorus discussing the topic of Open Source
Licenses. For protecting the authors and contributors from charges and
claims of idemnification we adopt the lightly modified GPL3 disclaimer:

THERE IS NO WARRANTY FOR THE OSLiC, TO THE EXTENT PERMITTED BY APPLICABLE LAW.
THE COPYRIGHT HOLDERS AND/OR OTHER PARTIES PROVIDE THE TEXT “AS IS” WITHOUT
WARRANTY OF ANY KIND, EITHER EXPRESSED OR IMPLIED, INCLUDING, BUT NOT LIMITED
TO, THE IMPLIED WARRANTIES OF MERCHANTABILITY AND FITNESS FOR A PARTICULAR
PURPOSE. THE ENTIRE RISK AS TO THE QUALITY AND PERFORMANCE OF THE OSLiC IS
WITH YOU. SHOULD THE OSLiC PROVE DEFECTIVE, YOU ASSUME THE COST OF ALL
NECESSARY SERVICING, REPAIR OR CORRECTION.

IN NO EVENT UNLESS REQUIRED BY APPLICABLE LAW OR AGREED TO IN WRITING WILL ANY
COPYRIGHT HOLDER, OR ANY OTHER PARTY WHO MODIFIES AND/OR CONVEYS THE OSLiC AS
PERMITTED ABOVE, BE LIABLE TO YOU FOR DAMAGES, INCLUDING ANY GENERAL, SPECIAL,
INCIDENTAL OR CONSEQUENTIAL DAMAGES ARISING OUT OF THE USE OR INABILITY TO USE
THE PROGRAM (INCLUDING BUT NOT LIMITED TO LOSS OF DATA OR DATA BEING RENDERED
INACCURATE OR LOSSES SUSTAINED BY YOU OR THIRD PARTIES OR A FAILURE OF THE
PROGRAM TO OPERATE WITH ANY OTHER PROGRAMS), EVEN IF SUCH HOLDER OR OTHER PARTY
HAS BEEN ADVISED OF THE POSSIBILITY OF SUCH DAMAGES.

%%%%%%%%%%%%
% Telekom osCompendium 'for being included' snippet template
%
% (c) Karsten Reincke, Deutsche Telekom AG, Darmstadt 2011
%
% This LaTeX-File is licensed under the Creative Commons Attribution-ShareAlike
% 3.0 Germany License (http://creativecommons.org/licenses/by-sa/3.0/de/): Feel
% free 'to share (to copy, distribute and transmit)' or 'to remix (to adapt)'
% it, if you '... distribute the resulting work under the same or similar
% license to this one' and if you respect how 'you must attribute the work in
% the manner specified by the author ...':
%
% In an internet based reuse please link the reused parts to www.telekom.com and
% mention the original authors and Deutsche Telekom AG in a suitable manner. In
% a paper-like reuse please insert a short hint to www.telekom.com and to the
% original authors and Deutsche Telekom AG into your preface. For normal
% quotations please use the scientific standard to cite.
%
% [ Framework derived from 'mind your Scholar Research Framework' 
%   mycsrf (c) K. Reincke 2012 CC BY 3.0  http://mycsrf.fodina.de/ ]
%


%% use all entries of the bibliography
%\nocite{*}


\chapter{Introduction}

% Abstract
\footnotesize \begin{quote}\itshape
This chapter briefly describes the idea behind the \oslic, the way it should be
used and the way it can be read---which is indeed not quite the same.
\end{quote}
\normalsize{}

% Content
This book focuses on just one issue: \emph{What needs to be done in order to act
in accordance with the licenses of those \emph{open source software} we use?}
The \emph{Open Source License Compendium} aims at reliably answering this
question---in a simple and easy to understand manner. However, it is not just
another book on \emph{open source} in ge\-ne\-ral.\footnote{Meanwhile, there are
tons of literature dealing with open source. Trying to expand your knowledge by
means of books and articles might let you get lost in literature: our list of
secondary literature may adumbrate this `danger of being overwhelmed'. But
nevertheless, our bibliography at the end of the \oslic{} is not complete.
Moreover, it is not intended to be complete. It is only an extract representing
the background information we did not directly cite in the \oslic. If we were
forced to indicate two books for attaining a good overview on the topic of
\emph{open source (licenses)} we would name (a) the `Rebel Code' (\cite[for a
German version cf.][\nopage passim]{Moody2001a}---\cite[for an English version
cf.][passim]{Moody2002a}) and (b) the `legal basic conditions'
(\cite[cf.][\nopage passim]{JaeMet2011a}). But fortunately, we are not forced to
do so.} The intention is, rather, for it to be a tool for simplifying the
activities for achieving license conformity.


This compendium was created out of necessity at \emph{Deutsche Telekom AG} to
counter a challenge some of its software developers and project managers were
facing: Of course, the company itself wants to behave as license compliantly as
its employees, but, unfortunately, they could not find a reference text which
simply lists what precisely must be done in order to comply with the license of
that piece of open source being used.

As some of these co-workers in Telekom projects, even we---the initial authors
of the \oslic---did not want to become open source license experts only for being
able to use open source software in accordance with their respective licenses. We
did not want to become lawyers. We just wanted to do more efficiently, what
in those days claimed much time and many resources. We were searching for clear
guidance instead of having to determine a correct way through the jungle of open
source licenses---over and over again, project for project. We loved using the
high-quality open source software to improve our performance. We liked using it
legally. But we did not like to laboriously discuss the legal constraints of the
many and different open source licenses.

What we needed was an easy-to-use handout which would lead us without any
detours to executable lists of work items. We wished to obtain to-do lists,
tailored to our usecases and our licenses. We needed reliable lists of tasks we
only had to execute for being sure that we were acting in accordance with the
open source license. When we started out, such a compendium did not exist.

For solving this problem our company took three decisions:

The first decision our company arrived at was to support a small group of
employees to act as \emph{a board of open source license experts}: They should
offer a service for the whole company. Projects, managers, and developers should
be able to ask this board what they have to do for complying with a specific
open Source License under specific circumstances. And this board should answer
with authoritative to-do lists whose executions would assure that the requestors
are acting according to the corresponding open source licenses. The idea behind
this decision was simple. It would save cost and increase quality if one had one
central group of experts instead of being obliged to select (and to train)
developers---over and over again, for every new project. So, the \emph{OSRB} 
(the \emph{Telekom Open Source Review Board}) was founded as an internal expert
group---as a self-organizing, bottom-up driven community.

The second decision our company took was to allow this \emph{Telekom OSRB} to
collect their results systematically---in the form of a reusable compendium.
The idea behind this decision was also simple: The more the internal service
became known, the more the workload would increase: the more work, the more
resources, the more costs. So, such a compendium should save costs and enable
the requestors to find answers by themselves without becoming license experts:
For all default cases, they should find an answer in the compendium instead of
having to request that their work is analyzed by the OSRB. Thus, the planned
\emph{Telekom Open Source License Compendium} will prevent the need
to increase the size of the OSRB in the future.

The third decision our company reached was to allow the \emph{Telekom OSRB} to
create the compendium in the same mode of cooperation that open source projects
usually use. Again, a simple reason evoked this ruling: If in the future---as a
rule---not a reviewing OSRB, but a simple manual should assure the open 
source license compliant behavior of projects, programmers, and managers, this
book had of course to be particularly reliable. There is a known feature of the
open source working model: the ongoing review by the cooperating community
increases the quality. Therefore, the decision not only to write an internal
`Telekom handout,' but to enable the whole community to use, modify, and 
redistribute a broader \emph{Open Source License Compendium} was a decision for
improving quality. Consequently, the \emph{OSRB} decided to publish the
\emph{\oslic} as a set of \LaTeX sources, publicly available via the open
repository GitHub.\footnote{Get the code by using the link
\texttt{https://github.com/dtag-dbu/oslic}; get project information by
\texttt{http://dtag-dbu.github.com/oslic/} or by
\texttt{http://www.oslic.org/}.}  And it licensed the \oslic{} under Creative
Commons Attribution-ShareAlike 3.0 Germany License.\footnote{This text is
licensed under the Creative Commons Attribution-ShareAlike 3.0 Germany License
(\texttt{http://creativecommons.org/licenses/by-sa/3.0/de/}): Feel free
\enquote{to share (to copy, distribute and transmit)} or \enquote{to remix (to
adapt)} it, if you \enquote{[\ldots] distribute the resulting work under the
same or similar license to this one} and if you respect how \enquote{you must
attribute the work in the manner specified by the author(s) [\ldots]}):
In an internet based reuse please mention the initial authors in a suitable
manner, name their sponsor \textit{Deutsche Telekom AG} and link it to
\texttt{http://www.telekom.com}. In a paper-like reuse please insert a short
hint to \texttt{http://www.telekom.com}, to the initial authors, and to their
sponsor \textit{Deutsche Telekom AG} into your preface. For normal citations
please use the scientific standard.}

But to publish the \emph{\oslic} as a free book has another important
connotation---at least for the \emph{Telekom OSRB}: It is also intended to be an
appreciative \emph{giving back} to the \emph{open source community} which has
enriched and simplified the life of so many employees and companies over so many
years. 

Altogether, the \oslic{} follows five principles:

\begin{description}
  \item[To-do lists as the core, discussions around them] Based on a simple
  form for gathering information concerning the use of a piece of open
  source software and its license, the \oslic{} shall offer an easy to use finder
  taking the requestor to the respective to-do list for ensuring license
  conformity. In addition, all these elements of the \oslic{} should comprehensibly
  be introduced and discussed without disturbing the usage itself.

  \item[Quotations with thoroughly specified sources]\label{QuotationPrinciple}
  The \oslic{} shall be revisable and reliable. It shall comprehensibly argue and
  explicitly specify why it adopts which information, from whom, in which
  version, and why.\footnote{For that purpose, we are using an `old-fashioned'
  bibliographic style with footnotes, instead of endnotes or inline-hints.
  We want to enable the users to review or to ignore our comments and hints just
  as they prefer---but on all accounts without being disturbed by large inline
  comments or frequent page turnings. We know that modern writer guides prefer
  less `noisy' styles (\cite[pars pro toto cf.][\nopage passim]{Mla2009a}). But
  for a reliable usage---challenged by the often modified internet
  sources---these methods are still a little imprecise (for details
  $\rightarrow$ \oslic, pp.\ \pageref{sec:QuotationAppendix}. For a short
  motivation of the style used in the \oslic{} \cite[cf.][\nopage
  passim]{Reincke2012a}. For a more elaborated legitimizing version
  \cite[cf.][\nopage passim]{Reincke2012b}).} 

  \item[Not clearing the forest, but cutting a swath] The \oslic{} has to deal with
  licenses and their legal aspects, no doubt. But it shall not discuss all
  details of every aspect. It shall focus on one possible way to act according
  to a license in a specific usecase---even if it is known that there might be
  alternatives.\footnote{The \oslic{} shall not counsel projects with respect to
  their specific needs. This must remain the task for lawyers and legal experts.
  The \oslic{} cannot and shall not replace a legal review or a legal advice by
  lawyers. It shall inspire developers, managers, open source experts, and
  companies to find good solutions, which they finally should have reviewed by
  legal counselors. For the German readers let us repeat again: The
  \oslic{} naturally respects the German \emph{Rechtsdienstleistungsgesetz}. It only
  contains legal reflections addressed to a general public. Its content may only
  be read as a \enquote{nur an die Allgemeinheit gerichtete Darstellung und
  Erörterung von Rechtsfragen}.}
  
  \item[Take the license text seriously!] The \oslic{} shall not give general
  lectures on legal discussions, much less shall it participate in them. It
  shall only find one dependable way for each license and each usecase to comply
  with the license. The main source for this analysis shall be the exact reading
  of the open source licenses themselves---based on and supported by the
  interpretation of benevolent lawyers and rationally arguing software
  developers. The \oslic{} shall respect that open source licenses are written for
  software developers (and sometimes by developers).
  
  \item[Trust the swarm!] The \oslic{} shall be open for improvements and
  adjustments encouraged and stimulated also by other people than employees of
  \emph{Deutsche Telekom AG}.
\end{description}

Based on these principles the \oslic{} offers two methods for being used:

First and foremost the readers expect to simply and quickly find those to-do
lists fitting their needs. Here is the respective process:%
  \footnote{For the well known `quick and dirty hackers'---as we tend to be, 
  too---we have integrated a shortcut: If you already know the license of the 
  open source package you want to use and if you are very familiar with the 
  meaning of the open source use cases we defined, then you might directly 
  jump to the corresponding license specific chapter, without `struggling' 
  with \textit{OSLiC 5 query form} ($\rightarrow$ \oslic{} p. 
  \pageref{OSLiCStandardFormForGatheringInformation}), the taxonomic
  \textit{Open Source Use Case Finder} ($\rightarrow$
  \pageref{OSLiCUseCaseFinder}) or the \textit{O}pen \textit{S}ource \textit{U}se
  \textit{C}ase page ($\rightarrow$ \pageref{OSUCList}ff.): Some of the chapters
  dedicated to specific open source licenses start with a license specific
  finder offering a set of license specific use cases---which, according to the
  complexity of the license, in some cases could be stripped down. But the
  disadvantage of this method is that you have to apply your knowledge about the
  use cases and their side effects by yourself without being systematically guided
  by the \oslic{} process.}

\tikzstyle{decision} = [diamond, draw, fill=gray!20, 
    text width=4.5em, text badly centered, node distance=4cm, inner sep=0pt]

\tikzstyle{preparation} = [rectangle, draw, fill=gray!30, 
    text width=11.5em, text centered, rounded corners, minimum height=3em]
 
\tikzstyle{lprocs} = [rectangle, draw, fill=gray!40, 
    text width=11.5em, text centered, rounded corners, minimum height=3em]
    
\tikzstyle{processing} = [rectangle, draw, fill=gray!40, node distance=2.4cm,
    text width=15em, text centered, rounded corners, minimum height=4em]
    
\tikzstyle{line} = [draw, -latex']

\tikzstyle{cloud} = [draw, ellipse, text centered, fill=gray!10]
 
    
\begin{tikzpicture}[node distance =1.5cm, auto]
\footnotesize
    % Place nodes
    
  \node [cloud, anchor=south, text width=7em] (start) at (1,10) 
    {$\forall$ open source \\ components};
  \node [preparation, below of=start] (select) 
    {select next open source component};     
  \node [preparation,  below of=select] (analyze) 
    {analyze its role as part of software system};
     
  \node [preparation,  below of=analyze] (determine) 
    {determine usage of final software product / service};     
    
  \node [preparation,  below of=determine] (detect) 
    {detect respective open source license};
    
  \node [lprocs,  below of=detect] (fillin)
    {\textbf{fill in the 5 query form} ($\rightarrow$ p.
    \pageref{OSLiCStandardFormForGatheringInformation})};
    
  \node [decision, right of=fillin] (success) {success?};
  
  \node [processing,  below of=success] (traverse)
    {\textbf{traverse} taxonomic \textbf{Open Source Use Case Finder}
    ($\rightarrow$ \pageref{OSLiCUseCaseFinder}) \& jump to indicated
    \textbf{O}pen \textbf{S}ource \textbf{U}se \textbf{C}ase page ($\rightarrow$
    \pageref{OSUCList}ff.)};
    
  \node [processing,  below of=traverse] (find)
    {\textbf{Determine} page of \textbf{license and use case specific to-do
    list} being presented in license specific chapter};
 
  \node [processing,  below of=find] (process)
    {Jump to indicated page \& \textbf{process license and use case specific
    to-do list} ($\rightarrow$ \pageref{OSUCToDoLists}ff.)};
    
  \node [decision, right of=process] (other) {more?};
  \node [cloud, right of=other, anchor=west] (stop) {stop};

  \path [line] (start) -- (select);  
     
  \path [line] (select) -- (analyze);      
  \path [line] (analyze) -- (determine);
  \path [line] (determine) -- (detect);       
  \path [line] (detect) -- (fillin);
  \path [line] (fillin) -- (success);
  
  \path [line] (success) |- node [near start] {no} (analyze);
  \path [line] (success) -- node [near start] {yes} (traverse);             
  
  \path [line] (traverse) -- (find);              
  \path [line] (find) -- (process);
  \path [line] (process) -- (other);

  \path [line] (other) |- node [near start] {yes} (select);
  \path [line] (other) -- node [near start] {no} (stop);                      

\end{tikzpicture}

Second, the readers might wish to comprehend the whole analysis. So, we briefly
discuss open source license taxonomies as the basis for a license compliant
behavior.\footnote{$\rightarrow$ OSLIC \enquote{\nameref{sec:LicenseTaxonomies}},
pp.\ \pageref{sec:LicenseTaxonomies}}  We consider some side effects of acting
according to the open source licenses.\footnote{$\rightarrow$ OSLiC
\enquote{\nameref{sec:SideEffects}}, pp.\ \pageref{sec:SideEffects}} Finally,
we study the structure of open source use cases.\footnote{$\rightarrow$ OSLiC
\enquote{\nameref{sec:OSUCdeduction}}, pp.\ \pageref{sec:OSUCdeduction}}

So, let us close our introduction by using, modifying, and (re)distributing a
well known wish of a well known man: Happy (Legally) Hacking.

%\bibliography{../../../bibfiles/oscResourcesEn}

% Local Variables:
% mode: latex
% fill-column: 80
% End:

%% Telekom osCompendium 'for beeing included' snippet template
%
% (c) Karsten Reincke, Deutsche Telekom AG, Darmstadt 2011
%
% This LaTeX-File is licensed under the Creative Commons Attribution-ShareAlike
% 3.0 Germany License (http://creativecommons.org/licenses/by-sa/3.0/de/): Feel
% free 'to share (to copy, distribute and transmit)' or 'to remix (to adapt)'
% it, if you '... distribute the resulting work under the same or similar
% license to this one' and if you respect how 'you must attribute the work in
% the manner specified by the author ...':
%
% In an internet based reuse please link the reused parts to www.telekom.com and
% mention the original authors and Deutsche Telekom AG in a suitable manner. In
% a paper-like reuse please insert a short hint to www.telekom.com and to the
% original authors and Deutsche Telekom AG into your preface. For normal
% quotations please use the scientific standard to cite.
%
% [ File structure derived from 'mind your Scholar Research Framework' 
%   mycsrf (c) K. Reincke CC BY 3.0  http://mycsrf.fodina.de/ ]

%
\newpage
\section{Form [only to demo our lib style. will be replaced]}
\begin{itemize}
  \item first initially quoted book\footnote{\cite[cf.][123ff]{Grassmuck2002a}
  (expected: complete bibl. data)} using LaTeX \texttt{$\backslash$footnote}
  \item second initially quoted book\footcite[cf.][120 (expected: complete bibl.
  data)]{Fogel2006a} using jurabib \texttt{$\backslash$footcite} (same
  appereance)
  \item initially mentioned collection /
  proceedings\footnote{\cite[cf.][123ff]{DjoGehGraKreSpi2008a} (expected: complete
  bibl. data)}
  \item first initially mentioned article in an initially mentioned collection /
  proceedings\footnote{\cite[cf.][123ff]{Spielkamp2008a} (expected: complete
  bibl. data of article, short title data of collection)} using LaTeX
  \texttt{$\backslash$footnote}
  \item second initially mentioned article in an already mentioned collection
  / proceedings\footcite[cf.][123ff (expected: complete
  bibl. data of article, short title data of collection)]{Kreutzer2008a} using
  jurabib \texttt{$\backslash$footcite} 
  \item rementioned book\footnote{\cite[cf.][120]{Fogel2006a} (expected: short
  title)}
  \item directly rementioned same book same
  page\footnote{\cite[cf.][120]{Fogel2006a} (expected: id., ibid, / ders.,
  ebda.,)}
  \item directly rementioned same book different
  page\footnote{\cite[cf.][121]{Fogel2006a} (expected: id., lc., / ders.,
  a.a.O. \& page)}
  \item rementioned collection article\footnote{\cite[cf.][120 ]{Kreutzer2008a} (expected: short
  title)}
  \item directly rementioned collection article same
  page\footnote{\cite[cf.][120]{Kreutzer2008a} (expected: id., ibid, / ders.,
  ebda.,)}
  \item directly rementioned collection article different
  page\footnote{\cite[cf.][121]{Kreutzer2008a} (expected: id., lc., / ders.,
  a.a.O. \& page)}
\end{itemize}

%\bibliography{../bibfiles/oscResourcesEn}

% Local Variables:
% mode: latex
% fill-column: 80
% End:


%%%%%%%%%%%%%%%
% Telekom osCompendium 'for being included' snippet template
%
% (c) Karsten Reincke, Deutsche Telekom AG, Darmstadt 2011
%
% This LaTeX-File is licensed under the Creative Commons Attribution-ShareAlike
% 3.0 Germany License (http://creativecommons.org/licenses/by-sa/3.0/de/): Feel
% free 'to share (to copy, distribute and transmit)' or 'to remix (to adapt)'
% it, if you '... distribute the resulting work under the same or similar
% license to this one' and if you respect how 'you must attribute the work in
% the manner specified by the author ...':
%
% In an internet based reuse please link the reused parts to www.telekom.com and
% mention the original authors and Deutsche Telekom AG in a suitable manner. In
% a paper-like reuse please insert a short hint to www.telekom.com and to the
% original authors and Deutsche Telekom AG into your preface. For normal
% quotations please use the scientific standard to cite.
%
% [ Framework derived from 'mind your Scholar Research Framework' 
%   mycsrf (c) K. Reincke 2012 CC BY 3.0  http://mycsrf.fodina.de/ ]
%

\chapter{Open Source: The Same Idea, Different Licenses}\label{sec:LicenseTaxonomies}

%% use all entries of the bibliography
%\nocite{*}
\footnotesize \begin{quote}\itshape This chapter describes different license
models which follow the common idea of free open source software. We want to
discuss existing ways of grouping licenses to underline the limits of building
such clusters: These groups are often used as `virtual prototypical licenses'
which are supposed to provide simplified conditions for acting according to
the respective real license instances. But one has to meet the requirements of a
specific license, not one's own generalized idea of a set of licenses.
Nonetheless, we, too,  offer a new way of structuring the world of the open
source licenses. We will use a novel set of grouping criteria by referring
to the common intended purpose of licenses: each license is designed to protect
something or someone against something or someone. Following this pattern, we
can indeed summarize all Open Source Licenses in a comparable way.
\end{quote}
\normalsize{}

Grouping open source licenses\footnote{Talking about licenses is sometimes a bit
tricky: Normally, they have a longer official name and a well known, often
abbreviating inofficial nickname. But that's not enough for talking about a
specific license adequately: one has additionally to refer to the version of the
license itself. The Linux Foundation offers a set of normalized names and
identifiers, to minimize the confusion how to denote a license correctly
(\cite[cf.][\nopage wp]{LinuxFounSpdxList2014a}). The OSLiC tries to use these
SPDX identifiers as far as possible. But sometimes the OSLiC wants to group
specific licenses by their authors without discriminating the release numbers.
Then, the OSLiC uses prefixes of the SPDX.} is commonly done.
Even the set of the \emph{open source li\-cen\-ses}\footcite[cf.][\nopage
wp]{OSI2012b} itself is already a cluster being established by a set of grouping
criteria: The \enquote{distribution terms} of each software license that intends
to become an open source license \enquote{[\ldots] must comply with the [\ldots]
criteria} of the \emph{Open Source De\-fi\-ni\-tion,}\footcite[cf.][\nopage
wp]{OSI2012a} maintained by the \emph{Open Source
Initiative}\footcite[cf.][\nopage wp]{OSI2012c} and often abbreviated as
\emph{OSD}. So, this \emph{OSD} demarcates `the group of [potential] open source
licenses' against `the group of not open sources licenses.'\footnote{More
precisely: meeting the OSD is only a necessary condition for becoming an
\emph{open source license}. The sufficient condition for becoming an \emph{open
source license} is the approval by the OSI, which offers a process for the
official approval of \emph{open source license} (\cite[cf.][\nopage
wp]{OSI2012d}).}

Another way to cluster the \emph{Free Software Licenses} is specified by the
\enquote{Free Software Definition.} This \emph{FSD} contains four conditions
which must be met by any free software license: any FSD compliant license must
grant \enquote{the freedom to run a program, for any purpose [\ldots]},
\enquote{the freedom to study how it works, and adapt it to (one's) needs
[\ldots]}, \enquote{the freedom to redistribute copies [\ldots]}, and finally
\enquote{the freedom to improve the program, and release your improvements
[\ldots]}\footcite[cf.][41]{Stallman1996a} Surprisingly this definition
implies that the requirement \emph{the sourcecode must be openly accessible}
is `only' a derived condition. If the \enquote{freedom to make changes and the
freedom to publish improved versions} shall be \enquote{meaningful}, then the
\enquote{access to the source code of the program} is a prerequisite.
\enquote{Therefore, accessibility of source code is a necessary condition for
free software.}\footcite[cf.][41]{Stallman1996a}

The difference between the OSD and the FSD has often been described as a
difference of emphasis:%
  \footnote{This is also the viewpoint of Richard M. Stallman: 
  On the one hand, he clearly states that the \enquote{Free Software
  movement} and the \enquote{open source movement} generally \enquote{[\ldots]
  disagree on the basic principles, but agree more or less on the practical
  recommendations} and that he \enquote{[\ldots] (does) not think of the open
  source movement as an enemy}.  On the other hand, he delineates the two
  movements by stating that \enquote{for the open source movement, the issue of
  whether software should be open source is a practical question, not an ethical
  one}, while \enquote{for the Free Software movement, non-free software is a
  social problem and free software is the solution}
  (\cite[cf.][55]{Stallman1998a}). \label{RmsFsPriority} Consequently, Richard
  M. Stallman summarizes the positions in a simple way: \enquote{[\ldots] `open
  source' was designed not to raise [\ldots] the point that users deserve
  freedom}. But he and his friends want \enquote{to spread the idea of freedom}
  and therefore \enquote{[\ldots] stick to the term `free software'}
  (\cite[][59]{Stallman1998a}). For a brush-up of this position, expressing
  again that \enquote{(o)pen source is a development methodology [and that] free
  software is a social movement} with an \enquote{ethical imparative}
  \cite[cf.][31]{Stallman2009a} }
Although both definitions \enquote{[\ldots]
(cover) almost exactly the same range of software}, the \emph{Free Software
Foundation}---as it is said---\enquote{prefers [\ldots] (to emphazise) the
idea of freedom [\ldots]} while the \emph{OSI} wants to underline the
philosophically indifferent \enquote{development methodology.}\footcite[pars pro
toto: cf.][232]{Fogel2006a}

A third method to group of free software and free software licenses is specified
by the \enquote{Debian Free Software Guideline}, which is embedded into the
\enquote{Debian Social Contract}. This \enquote{DFSG} contains nine defining
criteria, which---as Debian itself says---have been \enquote{[\ldots] adopted
by the free[sic!] software community as the basis of the Open Source
Definition.}\footcite[cf.][wp]{DFSG2013a}

A rough understanding of these methods might result in the conclusion that these
three definitions are extensionally equal and only differ intensionally.
But that is not true. To unveil the differences, let us compare the clusters
\emph{OSI approved licenses}, \emph{OSD compliant licenses}, \emph{DFSG
compliant licenses}, and \emph{FSD compliant licenses} extensionally, by asking
whether they \emph{could} establish different sets of licenses.\footnote{Indeed,
for analyzing the extensional power of the definition we have to regard all
potentially covered licenses, not only the already existing licenses, because
the subset of really existing licenses still could be expanded be developing new
licenses which fit the definition.}

First, the difference most easy to determine is that of an unidirectional
inclusion: By definition, the \emph{OSI approved licenses} and the \emph{OSD
compliant licenses} meet the requirements of the OSD.\footcite[cf.][\nopage
wp]{OSI2012a} But only the \emph{OSI approved licenses} have successfully
passed the OSI process\footcite[cf.][\nopage wp]{OSI2012a} and therefore are
officially listed as \emph{open source licenses.}\footcite[cf.][\nopage
wp]{OSI2012b}
% TODO: what does this mean? 
Hence, on the one hand, \emph{OSI approved licenses} are
\emph{open source licenses} and vice versa. On the other hand, both---the
\emph{OSI approved licenses} and the \emph{open source licenses}---are
\emph{OSD compliant licenses}, but not vice versa.

Second, a similar argumentation allows us to distinguish the \emph{DFSG compliant
licenses} from the \emph{OSI approved licenses}. As it is stated, the OSD
\enquote{[\ldots] is based on the Debian Free Software Guideline and any
license that meets one definition almost meets the
other.}\footcite[cf.][233]{Fogel2006a} But then again, meeting the definition is
not enough for being an official open source license: the license has to be
approved by the OSI.\footcite[cf.][\nopage wp]{OSI2012b} Thus, it follows that
all \emph{OSI approved licenses} are also \emph{DFSG compliant licenses}, but
not vice versa.

Third, by ignoring the \enquote{few exceptions} which have appeared
\enquote{over the years,}\footcite[cf.][233]{Fogel2006a} it can be said that,
because of their `kinsmanlike' relation, at least the \emph{OSD compliant
licenses} are also \emph{DFSG compliant licenses} and vice versa.

Last but not least, it must be stated that the (potential) set of free software
licenses must be greater than all the other three sets: On the one side, the FSD
requires that a license of free software must not only allow to read the
software, but must also permit to use, to modify, and to distribute
it.\footcite[cf.][41]{Stallman1996a} These conditions are covered by at least
the first three paragraphs of the OSD concerning the topics \enquote{Free
Redistribution,} \enquote{Source Code,} and \enquote{Derived
Works.}\footcite[cf.][\nopage wp]{OSI2012a} On the other side, the OSD contains
at least some requirements which are not mentioned by the FSD and which
nevertheless must be met by a license in order to be qualified as an OSD
compliant license.\footnote{For example, see the condition that \enquote{the
license must be technology-neutral} (\cite[cf.][\nopage wp]{OSI2012a}).} It
follows then that there may exist licenses which fulfill all conditions of the
FSD and nevertheless do not fulfill at least some conditions of the
OSD.\footnote{Again: we must consider the extensional potential of the
definitions, not the set of really existing licenses. In this context, it is
irrelevant that actually all existing Free Software Licenses like GPL, LGPL or
AGPL indeed are also classfied as open source licenses. We are referring to the
fact that there might be generated licenses which fulfill the FSD, but not the
OSD.} So, the set of all (potential) \emph{Free Software Licenses} must be
greater than the set of all (potential) \emph{open source licenses} and greater
than the set of \emph{OSD compliant licenses}.

All in all, we can visualize the situation as follows:

\begin{center}

\begin{tikzpicture}
\label{LICTAX}
\small

\node[ellipse,minimum height=5.8cm,minimum width=11.6cm,draw,fill=gray!10] (l0210) at (5,5)
{ };

\draw [-,dotted,line width=0pt,white,
    decoration={text along path,
              text align={center},
              text={|\itshape|All Software Licenses}},
              postaction={decorate}] (0,6.1) arc (120:60:10cm);

\node[ellipse,minimum height=4.4cm,minimum width=10cm,draw,fill=gray!20] (l0210) at (5,5)
{ };

\draw [-,dotted,line width=0pt,white,
    decoration={text along path,
              text align={center},
              text={|\itshape|FSD Compliant Licenses}},
              postaction={decorate}] (0,5.4) arc (120:60:10cm);
              

\node[ellipse,minimum height=3cm,minimum width=8.4cm,draw,fill=gray!30] (l0210) at (5,5)
{ };


             
\draw [-,dotted,line width=0pt,white,
    decoration={text along path,
              text align={center},
              text={|\itshape|OSD Compliant Licenses}},
              postaction={decorate}] (0,4.7) arc (120:60:10cm);
              
\draw [-,dotted,line width=0pt,white,
    decoration={text along path,
              text align={center},
              text={|\itshape|DFSG Compliant Licenses}},
              postaction={decorate}] (0,5) arc (240:300:10cm);
          

\node[ellipse,text width=4.4cm, text centered,minimum height=1.6cm,minimum width=6cm,draw,fill=gray!40] (l0210) at (5,5)
{ \textit{OSI approved licenses} = \\ \textit{\textbf{open source licenses}}
};

\end{tikzpicture}
\end{center}

It should be clear without longer explanations that these clusters don't allow
to extrapolate to the correct compliant behaviour according to the \emph{open source
licenses}: On the one hand, all larger clusters do not talk about the \emph{open
source licenses}. On the other hand, the \emph{open source license cluster}
itself only collects its elements on the basis of the OSD which does not stipulate
concrete license fulfilling actions for the licensee.

The next level of clustering \emph{open source licenses} concerns the inner
structure of these \emph{OSI approved licenses}. Even the OSI itself has recently
discussed whether a different way of grouping the listed licenses would better fit
the needs of the visitors of the OSI site.\footcite[cf.][\nopage wp]{OSI2013a}
And finally the OSI came up with the categories \enquote{popular and widely used
(licenses) or with strong communities,} \enquote{special purpose licenses,}
\enquote{other/miscellaneous licenses,} \enquote{licenses that are redundant
with more popular licenses,} \enquote{non-reusable licenses,} \enquote{superseded
licenses,} \enquote{licenses that have been voluntarily retired,} and \enquote{
uncategorized licenses.}\footcite[cf.][\nopage wp]{OSI2013b}

Another way to structure the field of open source licenses is to think in
\enquote{types of open source licenses} by grouping the \emph{academic
licenses}, \enquote{named as such because they were originally created by academic
institutions,}\footcite[cf.][69]{Rosen2005a} the \emph{reciprocal
licenses}, named as such because they \enquote{[\ldots] require the
distributors of derivative works to dis\-tri\-bu\-te those works under same
license including the requirement that the source code of those derivative works
be published,}\footcite[cf.][70]{Rosen2005a} the \emph{standard
licenses,} named as such because they refer to the reusability of
\enquote{industry standards,}\footcite[cf.][70]{Rosen2005a} and the
\emph{content licenses}, named as such because they refer to
\enquote{[\ldots] other than software, such as music art, film, literary works}
and so on.\footcite[cf.][71]{Rosen2005a}

Both kinds of taxonomies directly help to find the relevant licenses that should
be used for new (software) projects. But again: none of these categories 
allows us to infer license compliant behaviour, because the categories are
mostly defined based on license external criteria: whether a license is
published by a specific kind of organization or whether a license deals with
industry standards or other kind of works than software inherently does not
determine a license fulfilling behaviour.

Only the act of grouping into \emph{academic licenses} and 
\emph{reciprocal licenses} touches the idea of license fulfillment
tasks, if one---as it has been done---expands the definition of the  
\emph{academic licenses} by the specification that these licenses
\enquote{[\ldots] allow the software to be used for any purpose whatsoever with
no obligation on the part of the licensee to distribute the source code of
derivative works.}\footcite[cf.][71]{Rosen2005a} With respect to this additional
specification, the clusters \emph{academic licenses} and the
\emph{reciprocal licenses} indeed might be referred as the
\enquote{main categories} of (open source)
licenses:\footcite[cf.][179]{Rosen2005a} By definition, they are constituting
not only a contrary, but contradictory opposite. However, it must  be kept in
mind that they constitute an inherent antagonism, an antinomy inside of the set
of open source licenses.\footnote{Hence, it is at least a little confusing to say
that \enquote{the open source license (OSL) is a reciprocal license} and
\enquote{the Academic Free License (AFL) is the exact same license without the
reciprocity provisions} (\cite[cf.][180]{Rosen2005a}): If the BSD license is an
AFL and if an AFL is not an OSL and if the OSI approves only OSLs, then the BSD
license can not be an approved open source license. But in fact, it still is 
(\cite[cf.][\nopage wp]{OSI2012b}).}

Similiar in nature to the clustering into \emph{academic licenses} and 
\emph{reciprocal licenses} is the grouping into \emph{permissive licenses}, 
\emph{weak copyleft licenses}, and \emph{strong copyleft licenses}: 
Even Wikipedia uses the term \enquote{permissive free software licence} in the
meaning of \enquote{a class of free software licence[s] with minimal
requirements about how the software can be redistributed} and \enquote{contrasts}
them with the\enquote{copyleft licences} as those with \enquote{reciprocity%
\,/\,share-alike requirements.}\footcite[cf.][\nopage wp]{wpPermLic2013a}

Some other authors name the set of \emph{academic licenses} the
\enquote{permissive licenses} and specify the \emph{reciprocal licenses} as
\enquote{restrictive licenses}, because in this case---as a consequence of the
embedded \enquote{copyleft} effect---the source code must be published in case
of modifications. They also introduce the subset of \enquote{strong
restrictive licenses} which additionally require that an (overarching)
derivative work must be published under the same license.\footcite[pars pro toto
cf.][57]{Buchtala2007a} The next refinement of such clustering concepts
directly uses the categories \enquote{[open source] licenses with a strict
copyleft clause,}\footcite[Originally stated as \enquote{Lizenzen mit einer
strengen Copyleft-Klausel.} Cf.][24]{JaeMet2011a} \enquote{[open source]
licenses with a restricted copyleft clause,}\footcite[Originally stated as
\enquote{Lizenzen mit einer beschränkten Copyleft-Klausel.}
Cf.][71]{JaeMet2011a} and \enquote{[open source] licenses without any copyleft
clause.}\footcite[Originally stated as \enquote{Lizenzen ohne Copyleft-Klausel.}
Cf.][83]{JaeMet2011a} Finally, this viewpoint can directly be mapped to the
categories \emph{strong copyleft} and \emph{weak copyleft:} While on the one
hand, \enquote{only changes to the weak-copylefted software itself become
subject to the copyleft provisions of such a license, [and] not changes to the
software that links to it}, on the other hand, the \enquote{strong copyleft}
states \enquote{[\ldots] that the copyleft provisions can be efficiently imposed
on all kinds of derived works.}\footcite[cf.][\nopage wp]{wpCopyleft2013a}

Based on this approach to an adequate clustering and labeling,%
  \footnote{Finally, we should also mention that there exists still other
  classifications which might become important in other contexts. For example,
  the ifross license subsumes under the main category \enquote{Open Source
  Licenses} the subcategories \enquote{Licenses without Copyleft Effect,}
  \enquote{Licenses with Strong Copyleft,} \enquote{Licenses with Restricted
  Copyleft,} \enquote{Licenses with Restricted Choice,} or \enquote{Licenses
  with Privileges}---and lets finally denote these categories also licenses
  which are not listed by the OSI (\cite[cf.][\nopage wp]{ifross2011a}). This is
  reasonable if one refers to the meaning of the OSD (\cite[cf.][\nopage
  wp]{OSI2012a}). The \oslic{} wants to simplify its object of study by
  referring to the approved open source licenses (\cite[cf.][\nopage
    wp]{OSI2012d}) listed by the OSI (\cite[cf.][\nopage wp]{OSI2012b}).}
we can develop the following picture:

\begin{center}

\begin{tikzpicture}
\label{OSLICTAX}
\small

\node[ellipse,minimum height=8.5cm,minimum width=14cm,draw,fill=gray!10] (l0100) at (6.8,6.8)
{  };

\draw [-,dotted,line width=0pt,white,
    decoration={text along path,
              text align={center},
              text={|\itshape| OSI approved licenses}},
              postaction={decorate}] (-0.8,6.5) arc (142:38:9.5cm);

\draw [-,dotted,line width=0pt,white,
    decoration={text along path,
              text align={center},
              text={|\itshape|open source licenses}},
              postaction={decorate}] (-0.8,6.5) arc (218:322:9.5cm);
              
\node[ellipse,minimum height=6.2cm,minimum width=4cm,draw,fill=gray!20] (l0100) at (2.75,6.8)
{  };

\draw [-,dotted,line width=0pt,white,
    decoration={text along path,
              text align={center},
              text={|\itshape| permissive licenses}},
              postaction={decorate}] (0.9,7.4) arc (180:0:1.8cm);

\node[rectangle,draw,text width=1.3cm, text height=0.36cm, fill=gray!40, text
centered] (l0101) at (1.9,7.7) { \footnotesize  \textit{Apache-2.0}};
\node[rectangle,draw,text width=1.3cm, text height=0.36cm, fill=gray!40, text
centered] (l0102) at (3.6,7.7) { \footnotesize  \textit{BSD-X-Clause}};
\node[rectangle,draw,text width=1.3cm, text height=0.36cm, fill=gray!40, text
centered] (l0103) at (1.9,6.7) {  \footnotesize  \textit{MIT}};
\node[rectangle,draw,text width=1.3cm, text height=0.36cm, fill=gray!40, text
centered] (l0104) at (3.6,6.7) {  \footnotesize  \textit{MS-PL}};
\node[rectangle,draw,text width=1.3cm, text height=0.36cm, fill=gray!40, text
centered] (l0105) at (1.9,5.7) {  \footnotesize  \textit{Post-greSQL}};
\node[rectangle,draw,text width=1.3cm, text height=0.36cm, fill=gray!40, text
centered] (l0106) at (3.6,5.7) {  \footnotesize  \textit{PHP-3.X}};

\node[ellipse,minimum height=6cm,minimum width=8.5cm,draw,fill=gray!20] (l0200) at (9.2,6.5)
{  };

\draw [-,dotted,line width=0pt,white,
    decoration={text along path,
              text align={center},
              text={|\itshape| copyleft licenses}},
              postaction={decorate}] (7.5,8.5) arc (120:60:4cm);


\node[ellipse,minimum height=4.5cm,minimum width=4.2cm,draw,fill=gray!30] (l0210) at (7.45,6.5)
{  };

\draw [-,dotted,line width=0pt,white,
    decoration={text along path,
              text align={center},
              text={|\itshape| weak copyleft licenses}},
              postaction={decorate}] (5.4,6.2) arc (180:0:2cm);

\node[rectangle,draw,text width=1.2cm, text height=0.36cm, fill=gray!40, text
centered] (l0211) at (6.7,6.9) {  \footnotesize  \textit{EPL-1.X}};
\node[rectangle,draw,text width=1.2cm, text height=0.36cm, fill=gray!40, text
centered] (l0212) at (8.2,6.9) {  \footnotesize  \textit{EUPL-1.X}};
\node[rectangle,draw,text width=1.2cm, text height=0.36cm, fill=gray!40, text
centered] (l0213) at (6.7,5.7) {  \footnotesize  \textit{LGPL-Y.Y}};
\node[rectangle,draw,text width=1.2cm, text height=0.36cm, fill=gray!40, text
centered] (l0214) at (8.2,5.7) {  \footnotesize  \textit{MPL-X.Y}};

\node[ellipse,minimum height=4.5cm,minimum width=3cm,draw,fill=gray!30] (l0220) at (11.4,6.5)
{  };
 
% line width=0pt,white,
\draw [-,dotted,line width=0pt,white,
    decoration={text along path,
              text align={center},
              text={|\itshape| strong copyleft}},
              postaction={decorate}] (10.4,7) arc (180:0:1cm);

\draw [-,dotted,line width=0pt,white,
    decoration={text along path,
              text align={center},
              text={|\itshape| licenses}},
              postaction={decorate}] (10.4,5.4) arc (180:360:1cm);        

\node[rectangle,draw,text width=1.2cm, text height=0.36cm, fill=gray!40, text
centered] (l0221) at (11.4,6.8) {  \footnotesize  \textit{GPL-X.Y}};
\node[rectangle,draw,text width=1.2cm, text height=0.36cm, fill=gray!40, text
centered] (l0222) at (11.4,5.6) {  \footnotesize \textit{AGPL-3.X}};


\end{tikzpicture}
\end{center}

This extensionally based clarification of a possible open source license
taxonomy is probably well-known and often---more or less explicitly---%
referred to.\footnote{Even the FSF itself uses the term `permissive non-copyleft
free software license' (\cite[pars pro toto: cf.][\nopage wp/section `Original BSD
license']{FsfLicenseList2013a}) and contrasts it with the terms `weak copyleft'
and `strong copyleft' (\cite[pars pro toto: cf.][\nopage wp/section `European
Union Public License']{FsfLicenseList2013a})} Unfortunately, this taxonomy
still contains some misleading underlying messages:

\emph{Permissive} has a very positive connotation. So, the antinomy of
\emph{permissive licenses} versus \emph{copyleft licenses} implicitly signals,
that the \emph{permissive licenses} are in some sense better than the
\emph{copyleft licenses}. Naturally, this `conclusion' is evoked by
confusing the extensional definition and the intensional power of the labels.
But that is the way we---the human beings---like to think. 

Anyway, this underlying message is not necessarily `wrong.' It might be
convenient for those people or companies who only want to use open source
software without being restricted by the \emph{obligation to give something
back} as it has been introduced by the `copyleft.'\footnote{De facto,
\emph{copyleft} is not \emph{copyleft}. Apart from the definition, its effect
depends on the par\-ti\-cu\-ar licenses which determine the conditions for
applying the copyleft `method.' For example, in the GPL, the copyleft effect is
bound to the criteria of `being distributed.' Later on, we will collect these
conditions systematically (see chapter \emph{\nameref{sec:OSUCdeduction}}, pp.\
\pageref{sec:OSUCdeduction}). Therefore, here we still permit ourselves to use a
somewhat `generalizing' mode of speaking.} But there might be other people and
companies who emphasize the protecting effect of the copyleft licenses. And,
indeed, at least the open source license\footnote{Although RMS naturally prefers
to call it a \emph{Free Software License} (s. p.\ \pageref{RmsFsPriority})
} \emph{GPL}\footnote{As the original source \cite[cf.][\nopage
wp]{Gpl20FsfLicense1991a}. Inside of the \oslic, we constantly refer to the
license versions which are published by the OSI, because we are dealing with
officially approved open source licenses. For the `OSI-GPL' \cite[cf.][\nopage
wp]{Gpl20OsiLicense1991a}} has initially been developed to protect the freedom,
to enable the developers to help their \enquote{neighbours}, and to get the
modifications back:%
  \footnote{The history of the GNU project is multiply told. For
  the GNU project and its initiator \cite[cf.\ pars pro toto][\nopage
  passim]{Williams2002a}. For a broader survey \cite[cf.\ pars pro toto][\nopage
  passim]{Moody2001a}. A very short version is delivered by Richard M. Stallman
  himself where he states that---in the years when the early free community was
  destroyed---he saw the \enquote{nondisclosure agreement} which must be signed ,
  \enquote{[\ldots] even to get an executable copy} as a clear \enquote{[\ldots]
  promise not to help your neighbour}: \enquote{A cooperating community was
  forbidden.} (\cite[cf.][16]{Stallman1999a}).}
So, \enquote{Copyleft} is defined
as a \enquote{[\ldots] method for making a program free software and requiring
all modified and extended versions of the program to be free software as
well.}\footcite[cf.][89]{Stallman1996c} It is a method\footnote{Based on the
American legal copyright system, this method uses two steps: first one states,
\enquote{[\ldots] that it is copyrighted [\ldots]} and second one adds those
\enquote{[\ldots] distribution terms, which are a legal instrument that gives
everyone the rights to use, modify, and redistribute the program's code or any
program derived from it but only if the distribution terms are unchanged}
(\cite[cf.][89]{Stallman1996c}).} by which \enquote{[\ldots] the code and the
freedoms become legally inseparable}.\footcite[cf.][89]{Stallman1996c} Because
of these disparate interests of hoping not to be restricted and hoping to be
protected, it could be helpful to find a better label---an impartial name for
the cluster of \emph{permissive licenses}. But until that time, we should at
least know that this taxonomy still contains an underlying declassing message.

The other misleading interpretation is---counter-intuitively---prompted by using
the concept of `copyleft licenses.' By referring to a cluster of \emph{copyleft
licenses} as the opposite of the \emph{permissive licenses}, one implicitly also
sends two messages: First, that republishing one's own modifications is
sufficient to comply with the \emph{copyleft licenses}. And, second, that the
\emph{permissive licenses} do not require anything to be done for obtaining the
right to use the software. Even if one does not wish to evoke such an
interpretation, we---the human beings---tend to take the things as simple as
possible.\footnote{And indeed, in the experience of the authors sometimes
such simplifications gain their independent existence and determine decisions at
the management level. But that is not the fault of the managers. It is their
job to aggregate, generalize and simplify information. It is the job of the
experts to offer better viewpoints without overwhelming the others with
details.} But because of several aspects, this understanding of the antinomy of
\emph{copyleft licenses} and \emph{permissive licenses} is too misleading for
taking it as a serious generalization:

On the one hand, even the `strongly copylefted' GPL imposes other obligations
in addtion to republishing derivative works. For example, it also requires
giving \enquote{[\ldots] any other recipients of the [GPL licensed] Program a
copy of this License along with the Program.}\footcite[cf.][\nopage wp.\
§1]{Gpl20OsiLicense1991a} Furthermore, the `weakly copylefted' licenses require
also more and different criteria to be fulfilled for acting in accordance with
these licenses. For example, the EUPL requires that the licensor, who does not
directly deliver the binaries together with the sourcecode, must offer a
sourcecode version of his work free of charge,\footnote{The German version of the
EUPL uses the phrase \enquote{problemlos und unentgeltlich(sic!) auf den
Quellcode (zugreifen können)} (\cite[cf.][3, section 3]{EuplLicense2007de})
while the English version contains the specification \enquote{the Source Code is
easily and freely accessible} (\cite[cf.][2, section 3]{EuplLicense2007en})}
while the MPL requires that under the same circumstances a recipient
\enquote{[\ldots] can obtain a copy of such Source Code Form [\ldots] at a
charge no more than the cost of distribution to the recipient
[\ldots]}\footcite[cf.][\nopage section 3.2.a]{Mpl20OsiLicense2013a}
And last but not least, also the \emph{permissive licenses} require tasks 
to be fulfilled for a license compliant usage---moreover, they also require
different things. For example, the BSD license demands that \enquote{the
(re)distributions [\ldots] must (retain [and/or]) reproduce the above copyright
notice [\ldots]}. Because of the structure of the \enquote{copyright notice},
this compulsory notice implies that the authors / copyright holders of the
software must be publicly named.\footcite[cf.][\nopage wp]{BsdLicense2Clause} As
opposed to this, the Apache License requires that \enquote{if the Work includes
a \enquote{NOTICE} text file as part of its distribution, then any Derivative Works that
You distribute must include a readable copy of the attribution notices contained
within such NOTICE file} which often means that you have to present central
parts of such files publicly\footcite[cf.][\nopage wp.\ section
4.4]{Apl20OsiLicense2004a}---parts which can contain much more information than
only the names of the authors or copyright holders.

So, no doubt---and contrary to the intuitive interpretation of this taxonomy---%
each \emph{open source license} must be fulfilled by some actions, even the most
permissive one. And for ascertaining these tasks, one has to look into these
licenses themselves, not the generalized concepts of licenses taxonomies. Hence
again, we have to state that even this well known type of grouping of \emph{open
source licenses} does not allow to derive a specific license compliant behavior:
The taxonomy might be appropriate, if one wants to live with the implicit
messages and generalizations of some of its concepts. But the taxonomy is not an
adequate tool to determine, what one has to do for fulfilling an \emph{open
source license}. A license compliant behaviour for obtaining the right to use a
specific piece of \emph{open source software} must be based on the concrete
\emph{open source license} by which the licensor has licensed the software.
There is no shortcut.

Nevertheless, human beings need generalizing and structuring viewpoints for
enabling themselves to talk about a domain---even if they finally have to
regard the single objects of the domain for specific purposes. We think that
there is a subtler method to regard and to structure the domain of \emph{open
source licenses}. So, we want to offer this other possibility to cluster the
\emph{open source licenses}:\footnote{even if we also have to concede that,
ultimately, one has to always look into the license itself}

We think that, in general, licenses have a common purpose: they should protect
someone or something against something. The structure of this task is based on
the nature of the word `protect' which is a trivalent verb: it links someone or
something who protects, to someone or something who is protected and both
combined to something against which the protector protects and against the other
one is protected. Licenses in general do that. Moreover, to \enquote{protect}
the \enquote{rights} of the licensees is explicitly mentioned in the
GPL-2.0,\footcite[cf.][\nopage wp. Preamble]{Gpl20OsiLicense1991a} in the
LGPL-2.1,\footcite[cf.][\nopage wp. Preamble]{Lgpl21OsiLicense1999a} and the
GPL-3.0\footcite[cf.][\nopage wp. Preamble]{Gpl30OsiLicense2007a}---by which the
LGPL-3.0 inherits this purpose.\footcite[cf.][\nopage wp.
prefix]{Lgpl30OsiLicense2007a} Following this viewpoint, we want to generally
assume that open source licenses are designed to protect: They can protect
the user (recipient) of the software, its contributor resp.\ developer and/or
distributor, and the software itself. And they can protect them against
different threats:

\begin{itemize}
  \item First, we assume, that---in the context of open source software---the
  user can be protected against the loss of the right to use it, to modify it,
  and to redistribute it. Additionally, he can be protected against patent
  disputes.
  \item Second, we assume, that open source contributors and distributors can be
  protected against the loss of feedback in the form of code improvements and
  derivatives, against warranty claims, and against patent disputes.
  \item Third, we assume, that the open source programs and their specific forms%
  ---may they be distributed or not, may they be modified or not, may they be
  distributed as binaries or as sources---can be protected against the
  re-closing resp. against the re-privatization of their further development.
  \item Fourth, we want to assume that new on-top developments being based on
  open source components can be protected against the privatization for enlarging
  the world of freely usable software.\footnote{In a more rigid version, this
  capability of a license could also be identified as the power to protect the
  community against a stagnation of the set of open source software---but this
  description is at least a little to long to be used by the following pages}
\end{itemize}

With respect to these viewpoints, one gets a subtler picture of the license
specific protecting power. Thus, we are going to describe and deduce the
protecting power of each of the open source licenses on the following pages.
Table \ref{tab:powerOfLicenses} summarizes the results as a quick
reference.\footnote{$\rightarrow$ table \ref{tab:powerOfLicenses} on p.\
\pageref{tab:powerOfLicenses}. In February 2014, the Black Duck list of the
\enquote{Top 20 Open Source Licenses} additionally mentions the Artistic License
(AL), the Code Open Project License, the Common Public License, the zlib/png
License, the Academic Free License (AFL), the Microsoft Reciprocal License
(MS-RL) and the Open Software License (OSL) (\cite[cf.][\nopage
wp.]{wpBlackDuck2014a}). The Code Open Project License and Common Public License
are still not OSI approved open source licenses. (\cite[cf.][\nopage
wp.]{OSI2012b}). Thus, finally the OSLiC should additionally analyze not only
the AGPL and the CDDL, but also the AL, the AFL, the MS-RL, the OSL and the
zlib/png License for being able to justiufiably say, that the OSLiC covers the
most important open source licenses.
}

\begin{table}
\begin{minipage}{\textwidth}
\centering
\footnotesize
\caption{Open Source Licenses as Protectors}
\label{tab:powerOfLicenses}

\begin{tabular}{|c|c||c|c|c|c|c|c|c|c|c|c|c|c|c|c|c|}
\hline
  \multicolumn{2}{|c|}{\textit{Open}} &
  \multicolumn{13}{c|}{\textit{are protecting}}\\
\cline{3-15}
  \multicolumn{2}{|c|}{\textit{Source}} &
  \multicolumn{4}{c|}{ \textbf{Users}} &
  \multicolumn{3}{c|}{\textbf{Contributors}} &
  \multicolumn{5}{c|}{\textbf{Open Source Software}} &
  \multirow{4}{*}{\rotatebox{270}{\scriptsize{\textbf{On-Top Develop.\ }}}} 
  \\
\cline{10-14}
  \multicolumn{2}{|c|}{\textit{Licenses\footnote{'\checkmark' indicates that the
  license protects with respect to the meaning of the column, `$\neg$' indicates
  that the license does not protect with regard to the meaning of the column,
  and `--' indicates, that the corresponding statement must still be evaluated.
  \textit{Slanted names of licenses} indicate that these licenses are only
  listed in this table while the corresponding mindmap ($\rightarrow$ p.\
  \pageref{OSCLICMM}) does not cover them }}} &
  \multicolumn{4}{c|}{} &
  \multicolumn{3}{c|}{\tiny{(Distributors)}} &  
  not &
  \multicolumn{4}{c|}{distributed as} 
  & \\
\cline{3-9}\cline{11-14}
  \multicolumn{2}{|c|}{} &
  \multicolumn{4}{c|}{\scriptsize{\textit{who have already got}}} &
  \multicolumn{3}{c|}{\scriptsize{\textit{who spread open}}} & 
  dis- &
  \multicolumn{2}{c|}{unmodified} &
  \multicolumn{2}{c|}{modified} 
  & \\
  \cline{11-14}
  \multicolumn{2}{|c|}{} &
  \multicolumn{4}{c|}{\scriptsize{\textit{sources or binaries}}} &
  \multicolumn{3}{c|}{\scriptsize{\textit{source software}}} & 
  \parbox[t]{1cm}{tri\-bu\-ted} & 
 \rotatebox{270}{\footnotesize{sources\ }} &
 \rotatebox{270}{\footnotesize{binaries\ }} &
 \rotatebox{270}{\footnotesize{sources\ }} &
 \rotatebox{270}{\footnotesize{binaries\ }} 
 & \\
\cline{3-15}
  \multicolumn{2}{|c|}{} &
  \multicolumn{13}{c|}{\textit{against}}\\
\cline{3-15}
  \multicolumn{2}{|c|}{} &
  \multicolumn{3}{c|}{the loss of} & 
  \multirow{3}{*}{\rotatebox{270}{Patent Disputes}} &
  \multirow{3}{*}{\rotatebox{270}{Loss of Feedback}} & 
  \multirow{3}{*}{\rotatebox{270}{Warranty Claims}} & 
  \multirow{3}{*}{\rotatebox{270}{Patent Disputes}} & 
  \multicolumn{5}{c|}{}
  & \\
% no seperator line 
  \multicolumn{2}{|c|}{} &
  \multicolumn{3}{c|}{the right to} &
  & & & &
  \multicolumn{5}{c|}{\footnotesize{Re-Closings / Re-Privatization}} &
  \multirow{3}{*}{\rotatebox{270}{Privatization}}
   \\
\cline{3-5}
  \multicolumn{2}{|c|}{} & 
  \rotatebox{270}{use it} & 
  \rotatebox{270}{modify it} & 
  \rotatebox{270}{redistribute it\ } &
  &  &  &  &
  \multicolumn{5}{c|}{of already opened software}
  & \\
\hline
\hline
  Apache & 2.0 & \checkmark  & \checkmark  & \checkmark  &
  \checkmark & $\neg$ & \checkmark & \checkmark & $\neg$ &
   \checkmark  & $\neg$ & \checkmark & $\neg$ & $\neg$ \\
\hline
  \multirow{2}{*}{BSD} & 3-Cl & \checkmark & \checkmark  & \checkmark  & 
    $\neg$ & $\neg$ & \checkmark & $\neg$  &
    $\neg$ & \checkmark  & $\neg$ & \checkmark & $\neg$ & $\neg$ \\
\cline{2-15}
   & 2-Cl & \checkmark  & \checkmark  & \checkmark  & 
    $\neg$ & $\neg$ & \checkmark & $\neg$  &
    $\neg$ & \checkmark  & $\neg$ & \checkmark & $\neg$ & $\neg$ \\
\hline
  MIT & ~ & \checkmark  & \checkmark  & \checkmark  &
  $\neg$ & $\neg$ & \checkmark & $\neg$ & $\neg$ &
   \checkmark  & $\neg$ & \checkmark & $\neg$ & $\neg$ \\
\hline
  MS-PL & ~ & \checkmark  & \checkmark  & \checkmark  &
  \checkmark & $\neg$ & \checkmark & \checkmark & $\neg$ &
   \checkmark  & $\neg$ & \checkmark & $\neg$ & $\neg$ \\
\hline
  PostgreSQL & ~ & \checkmark  & \checkmark  & \checkmark  &
  $\neg$ & $\neg$ & \checkmark & $\neg$ & $\neg$ &
   \checkmark  & $\neg$ & \checkmark & $\neg$ & $\neg$ \\
\hline
  PHP & 3.0 & \checkmark  & \checkmark  & \checkmark  &
  $\neg$ & $\neg$ & \checkmark & $\neg$ & $\neg$ &
   \checkmark  & $\neg$ & \checkmark & $\neg$ & $\neg$ \\
\hline
\hline
  \textit{CDDL} & 1.0 & \checkmark & \checkmark & \checkmark &
  -- & -- & -- & -- & -- & -- & -- & -- & -- & -- \\
\hline
  EPL & 1.0 & \checkmark  & \checkmark  & \checkmark  &
  \checkmark  & \checkmark  & \checkmark & \checkmark & $\neg$ &
   \checkmark  & \checkmark & \checkmark & \checkmark & $\neg$ \\
\hline
  EUPL & 1.1 & \checkmark  & \checkmark  & \checkmark  &
  \checkmark  & \checkmark  & \checkmark & \checkmark & $\neg$ &
   \checkmark  & \checkmark & \checkmark & \checkmark & $\neg$ \\
\hline
  \multirow{2}{*}{LGPL} & 2.1 & \checkmark  & \checkmark  & \checkmark  &
   $\neg$ & \checkmark  & \checkmark & $\neg$ & $\neg$ &
   \checkmark  & \checkmark & \checkmark & \checkmark & $\neg$ \\
\cline{2-15}
   & 3.0 & \checkmark  & \checkmark  & \checkmark  &
   \checkmark & \checkmark  & \checkmark & \checkmark & $\neg$ &
   \checkmark  & \checkmark & \checkmark & \checkmark & $\neg$ \\
\hline
  \multirow{3}{*}{MPL} & 1.0 & --  & --  & --  &
   -- & --  & -- & -- & -- &
   --  & -- & -- & -- & -- \\
\cline{2-15}
   & 1.1 & --  & --  & --  &
   -- & --  & -- & -- & -- &
   --  & -- & -- & -- & -- \\
\cline{2-15}
   & 2.0 & \checkmark  & \checkmark  & \checkmark  &
  \checkmark  & \checkmark  & \checkmark & \checkmark & $\neg$ &
   \checkmark  & \checkmark & \checkmark & \checkmark & $\neg$ \\
\hline
  \textit{MS-RL} & ~ & \checkmark & \checkmark & \checkmark &
  -- & -- & -- & -- & -- & -- & -- & -- & -- & -- \\
\hline
\hline
  AGPL & 3.0 & \checkmark  & \checkmark  & \checkmark  &
   \checkmark & \checkmark  & \checkmark & \checkmark & \checkmark &
   \checkmark  & \checkmark & \checkmark & \checkmark & \checkmark \\
\hline
  \multirow{2}{*}{GPL} & 2.1 & \checkmark  & \checkmark  & \checkmark  &
   $\neg$ & \checkmark  & \checkmark & $\neg$ & $\neg$ &
   \checkmark  & \checkmark & \checkmark & \checkmark & \checkmark \\
\cline{2-15}
  & 3.0 & \checkmark  & \checkmark  & \checkmark  &
   \checkmark & \checkmark  & \checkmark & \checkmark & $\neg$ &
   \checkmark  & \checkmark & \checkmark & \checkmark & \checkmark \\
\hline
\hline

\end{tabular}

\end{minipage}
\end{table}

\section{\texorpdfstring{The protecting power of the}{The} GNU Affero General Public License (AGPL)}
\protectionlabel{AGPL}

[TODO...]

\section{\texorpdfstring{The protecting power of the}{The} Apache License
(Apache-2.0)}
\protectionlabel{APL}

As an approved \emph{open source license,}\footcite[cf.][\nopage wp]{OSI2012b}
the Apache License%
  \footnote{The Apache License, version 2.0 is maintained by the
  Apache Software Foundation (\cite[cf.][\nopage wp]{AsfApacheLicense20a}).  Of
  course, the OSI is hosting a duplicate of the Apache license
  (\cite[cf.][\nopage wp]{Apl20OsiLicense2004a}) and is listing it as an
  officially approved open source license (\cite[cf.][\nopage wp]{OSI2012b}). The
  Apache license 1.1 is classified by the OSI as \enquote{superseded
  license}(\cite[cf.][\nopage wp]{OSI2013b}). In the same spirit, the Apache
  Software Foundation itself classifies the releases 1.0 and 1.1 as
  \enquote{historic} (\cite[cf.][\nopage wp]{AsfLicenses2013a}). Thus, the \oslic{}
  only focuses on the most recent license Apache-2.0 version. For those who have
  to fulfill these earlier Apache licenses it could be helpful to read them as siblings of
  the BSD-2-Clause and BSD-3-Clause licenses.}
protects the user against the loss of the
right to use, to modify and/or to distribute the received copy of the source
code or the binaries.\footcite[cf.][\nopage wp. §2]{Apl20OsiLicense2004a}
Furthermore, based on its patent clause,\footnote{$\rightarrow$ \oslic{} pp.\
\patentpageref{APL}} the Apache-2.0 protects the users against patent
disputes.\footcite[cf.][\nopage wp. §3]{Apl20OsiLicense2004a} Because of this
patent clause and the \enquote{disclaimer of warranty} together with the
\enquote{limitation of liability,} the Apache license also protects the
contributors and distributors against patent disputes and warranty
claims.\footcite[cf.][\nopage wp. §3, §7, §8]{Apl20OsiLicense2004a} Finally, the
Apache-2.0 protects the distributed sources themselves \emph{against} a change of the
license which would \emph{convert} the work \emph{to closed software}, because,
first, one \enquote{[\ldots] must give any other recipients of the Work or
Derivative Works a copy of (the Apache) license,} second, \enquote{in the Source
form of any Derivative Works that (one) distributes}, one has \enquote{[\ldots]
to retain [\ldots] all copyright, patent, trademark, and attribution notices
[\ldots],} and third, one must \enquote{[\ldots] include a readable copy [\ldots
of the] NOTICE file} being supplied by the original package one has
received.\footcite[cf.][\nopage wp. §4]{Apl20OsiLicense2004a}

But the Apache License does not protect the contributors against the loss of
feedback because it does not `copyleft' the software: the Apache license does
not contain any sentence requiring that one has also to publish the source code.
In the same spirit, the Apache-2.0 does not protect the undistributed software or the
distributed binaries against re-closing (neither in unmodified nor in
modified form) because the Apache License allows to (re)distribute the
binaries without also supplying the sources---even if the binaries rest upon
sources modified by the distributor. Finally, the Apache-2.0 does not protect the
on-top developments against privatization.


\section{\texorpdfstring{The protecting power of the}{The} BSD licenses}
\protectionlabel{BSD2}
\protectionlabel{BSD3}

As approved \emph{open source licenses,}\footcite[cf.][\nopage wp]{OSI2012b} the
BSD Licenses%
  \footnote{BSD has to be resolved as \emph{Berkely Software Distribution}. 
  For details of the BSD license release and namings
  \cite[cf.][\nopage wp.\ editorial]{BsdLicense3Clause}} 
protect the user against
the loss of the right to use, to modify and/or to distribute the received copy
of the source code or the binaries.\footcite[cf.][\nopage wp. §1ff]{OSI2012a}
Additionally, they protect the contributors and/or distributors against warranty
claims of the software users, because these licenses contain a `No Warranty
Clause.'\footcite[one for all version cf.][\nopage wp]{BsdLicense2Clause} And
finally they protect the distributed sources against a change of the license
which closes the sources, because each modification and \enquote{redistributions
of [the] source code must retain the [\ldots] copyright notice, this list of
conditions and the [\ldots] disclaimer}:\footcite[cf.][\nopage
wp]{BsdLicense2Clause} Therefore it is incorrect to distribute BSD licensed
code under another license---regardless of whether it closes the sources or
not.%
  \footnote{In common sense based discussions you may have heard that BSD
  licenses allow to republish the work under another, an own license. Taking the
  words of the BSD License seriously that is not valid under all circumstances:
  Yes, it is true, you are not required to redistribute the sourcecode of a
  modified (derivative) work. You are allowed to modify a received version and to
  distribute the results only as binary code and to keep your improvements closed.
  But if you distribute the source code of your modifications, you have retain the
  licensing, because \enquote{Redistribution [\ldots] in source [\ldots], with or
  without modification, are permitted provided that [\ldots] (the) redistributions
  of source code [\ldots] retain the above copyright notice, this list of
  conditions and the following disclaimer} (\cite[cf.][\nopage
  wp]{BsdLicense2Clause})}

But the BSD Licenses protect neither the users nor the contributors
and/or distributors against patent disputes (because they do not contain any
patent clause). They do not protect the contributors against the loss of
feedback (because they do not `copyleft' the software). Moreover, they do not
protect the undistributed software or the distributed binaries against
re-closing---neither in unmodified nor in modified form---because they
allow to redistribute only the binaries without also supplying the source
code.\footnote{see both, the BSD-2-Clause License (\cite[cf.][\nopage
wp]{BsdLicense2Clause}), and the BSD-3Clause License (\cite[cf.][\nopage
wp]{BsdLicense3Clause})} Finally, the BSD licenses do not protect the on-top
developments against privatization.

\section{\texorpdfstring{The protecting power of the}{The} CDDL [tbd]}
\protectionlabel{CDDL}

As an approved \emph{open source license,}\footcite[cf.][\nopage wp]{OSI2012b}
the Common Develop and Distribution License protects the user
against the loss of the right to use, to modify and/or to distribute the
received copy of the source code or the binaries\footcite[cf.][\nopage wp. 
§?]{Cddl10OsiLicense2004a}

[\ldots]

\section{\texorpdfstring{The protecting power of the}{The} Eclipse Public License (EPL)}
\protectionlabel{EPL}

As an approved \emph{open source license,}\footcite[cf.][\nopage wp]{OSI2012b}
the Eclipse Public License%
  \footnote{The Eclipse Public License, version 1.0 is maintained by the Eclipse
  Software Foundation (\cite[cf.][\nopage wp]{Epl10EclipseFoundation2005a}).
  Of course, also the OSI is hosting a duplicate (\cite[cf.][\nopage
  wp]{Epl10OsiLicense2005a}).} 
protects the user against the loss of the right to use, to modify and/or to
distribute the received copy of the source code or the binaries\citeEPL{§2a}. 
Furthermore, based on its patent clause,\footnote{$\rightarrow$ \oslic{} 
pp.\ \patentpageref{EPL}} the EPL protects the users also against
patent disputes.\citeEPL{§2b \& §2c} Besides this patent clause, the EPL contains the
sections \enquote{no warranty} and \enquote{disclaimer of
liability.}\citeEPL{§5 \& §6} These three elements together protect the
contributors\,/\,distributors against patents disputes and warranty
claims. Finally, the EPL protects the distributed sources themselves
\emph{against} a change of the license which would \emph{reset} the work
\emph{as closed software}: First, the Eclipse Public Licenses requires that 
if a work---released under the EPL---\enquote{[\ldots] is made available in
source code form [\ldots] (then) it must be made available under this (EPL)
agreement, too} while this act of `making avalaible' \enquote{must} incorporate
a \enquote{copy} of the EPL into \enquote{each copy of the [distributed]
program} or program package.\citeEPL{§3} But in opposite to the permissive
licenses, the EPL does not only protect the distributed source code---regardless
whether it is modified or not. The EPL also protects the distributed modified or
unmodified binaries: The EPL allows each modifying \enquote{contributor} and
distributor \enquote{[\ldots] to distribute the Program in object code form
under (one's) own license agreement [\ldots]} provided this license clearly
states that the \enquote{source code for the Program is available} and where the
\enquote{licensees} can \enquote{[\ldots] obtain it in a reasonable manner on or
through a medium customarily used for software exchange.}\citeEPL{§3, esp. §3.b.iv}
Thus, one has to conclude that the EPL is a copyleft license.

But the Eclipse Public License is not a license with strong copyleft; the EPL
uses `only' a weak copyleft effect:%
  \footnote{Even if one can find contrary specifications in the
  internet. \cite[Pars pro toto cf.][\nopage wp]{ifross2011a}: This page is
  listing the EPL in the section \enquote{Other Licenses with strong Copyleft
  Effect}}
Indeed, the EPL says that for each EPL
licensed \enquote{program}---distributed in object form---a place must be made
known where one can get the corresponding source code.\citeEPL{§3, esp. §3.b.iv}
The term `Program' is defined as any \enquote{Contribution distributed in
accordance with [\ldots] (the EPL)} while the term `Contribution'
refers---besides other elements---to \enquote{changes to the Program, and
additions to the Program.}\citeEPL{§1} Unfortunately, this is a circular definition:
`Program' is defined by `Contribution'; and `Contribution' is defined by
`Program.' Nevertheless, one has to read the license benevolently.
Uncontroversial should be this: If one distributes any modified EPL licensed
program, library, module, or plugin, then one has to publish the modified source
code, too. If one \enquote{adds} some own plugins or additional libraries which
are used by an EPL licensed program (which on behalf of this use must have been
modified by adding [sic!] procedure calls) then one has to publish the code of
both parts: that of the program and that of the added elements. In this sense,
the EPL clearly protects the binaries against re-closings like other weak
copyleft using licenses. But if one distributes only an EPL licensed library
which is used as a component by another not EPL licensed on-top program, then
this library does not depend on the top development---provided that the library
itself does not call any (program) functions or procedures delivered by the
overarching on-top development. Hence, nothing is added to the library; and
hence, no other code than that of the library must be published. Therefore, the
EPL does not use the strong copyleft effect in the meaning of---for example --
the GPL.
 
\section{\texorpdfstring{The protecting power of the}{The} European Union Public License (EUPL)}
\protectionlabel{EUPL}

As an approved \emph{open source license,}\footcite[cf.][\nopage wp]{OSI2012b}
the European Union Public License%
  \footnote{The European Union Public License, version 1.1 is maintained by the
  European Union and hosted under the label \enquote{Joinup} 
  (\cite[cf.][\nopage wp]{EuplLicense2007en}).  This EUPL has officially been
  translated into many languages, among others into German 
  (\cite[cf.][\nopage wp]{EuplLicense2007de}). Because of this multi lingual
  instances, the OSI does not offer its own version, but just a landing page
  linked to the lading page of the European host \enquote{Joinup} 
  (\cite[cf.][\nopage wp]{Eupl11OsiLicense2007a}).} 
protects the user against the loss of the right to use, to modify and/or to
distribute the received copy of the source code or the binaries.\citeEUPL{§2}
Furthermore, based on its patent clause\footnote{$\rightarrow$ \oslic{}
pp.\ \patentpageref{EUPL}}, the EUPL protects the users against
patent disputes.\citeEUPL{§2, at its end} Besides this patent clause, the EUPL
additionally contains a \enquote{Disclaimer of Warranty} and a
\enquote{Disclaimer of Liability.}\citeEUPL{§7 \& §8} These three elements
together protect the contributors\,/\,distributors against patents disputes and
warranty claims. Finally, the EUPL also protects the distributed sources against
a re-closing\,/\,re-privatization and the contributors against the loss of
feedback. This protection is based on two steps: First, the European Public
License contains a particular paragraph titled \enquote{Copyleft clause} which
stipulates that \enquote{copies of the Original Work or Derivative Works based
upon the Original Work} must be distributed \enquote{under the terms of (the
European Union Public) License.}\citeEUPL{§5} Second, the EUPL requires that
each licensee---as long as he \enquote{[\ldots] continues to distribute and/or
communicate the Work}---has also to \enquote{[\ldots] provide [\ldots] the
Source Code}, either directly or by \enquote{[\ldots] (indicating) 
a repository where this Source will be easily and freely available
[\ldots]}\citeEUPL{§5} This condition seems to be so important for the EUPL that
the license repeats its message: in another paragraph the EUPL requires again
that \enquote{if the Work is provided as Executable Code, the Licensor provides
in addition a machine-readable copy of the Source Code of the Work along with
each copy of the Work [\ldots] or indicates, in a notice [\ldots], a repository
where the Source Code is easily and freely accessible for as long as the
Licensor continues to distribute [\ldots] the Work.}\citeEUPL{§3} Based on 
the meaning of \enquote{Work} which is defined by the EUPL as \enquote{the
Original Work and/or its Derivative Works}\citeEUPL{§1} it must be concluded
that the EUPL is a copyleft license. 

But nevertheless, the European Union Public License is not a license with strong
copyleft: On the one hand, if one takes the core of the EUPL then the license
seems to protect not only the modifications of the original work against
re-closings and (re-)privatization, but also the on-top developments because
normally you have to publish the source code in both cases. Understood in this
way, the EUPL would be a `strong copyleft license.' But on the other hand, the
EUPL additionally contains a \enquote{Compatibility clause} stating that
\enquote{if the Licensee Distributes [\ldots] Derivative Works or copies thereof
based upon both the Original Work and another work licensed under a Compatible
Licence, this Distribution [\ldots] can be done under the terms of this
Compatible Licence}\citeEUPL{§5}---while the term \enquote{Compatible Licence}
is explicitly defined by a list of compatible licenses, for example the Eclipse
Public License.\citeEUPL{Appendix}. Based on this compatibility clause the
obligation to publish the code of an on-top development can be subverted: As
% RPD: futile???
first step, you could release a little, more or less futile on-top application
licensed under the Eclipse Public License%
  \footnote{Taking the license text very seriously, it is not even necessary
  that this little futile application must depend on the EUPL library by calling
  functions of EUPL library. The license text only says that \enquote{another
  [any other] work licensed under a Compatible Licence} can be distributed
  together with \enquote{derivative works}. By this wording, the license itself
  is establishing a contrast between the derivative work and the other
  work---what indicates that the other work has not necessarily also to be a
  derivative work.} 
which uses a library licensed under the EUPL. As second step, you add this `EUPL
library' which you now may also distribute under the EPL instead of retaining
the EUPL licensing. So, finally you obtain the same work under the Eclipse
Public License which is a weak copyleft license\footnote{$\rightarrow$ \oslic,
p.\ \protectionpageref{EPL}}. Hence the protection of the EUPL-1.1 is
not as comprehensive as one might assume on the basis of the license text
itself,%
  \footnote{This kind of specifiying the protective power of the EUPL is
  initially presented by the FSF (\cite[cf.][wp.\ section `European Union
  Public License']{FsfLicenseList2013a}). The EU answers that publishing such a
  trick will comprise its user in the eyes of the open source community
  (\cite[cf.][wp]{FsfEuplRecomment2013}). That is undoubtely true. But
  unfortunately, this argument does not close the hole in the protecting shield
  put up by the EUPL.}
it can at most be a weak copyleft license---even if the reader might get the
impression that the authors of the EUPL wished to write a strong copyleft
license. Howsoever, the EUPL license does not protect the on-top developments
against a privatization. 

\section{\texorpdfstring{The protecting power of the}{The} GNU General Public License (GPL)}
\protectionlabel{GPL}

The GNU General Public License---also known as GPL---is maintained and offered
by the Free Software Foundation and hosted as part of the well known
\enquote{GNU operating system homepage.}\footcite[cf.][\nopage
wp]{FsfGnuOsLicenses2011a} Currently, there are two versions of the GPL which
are classified as OSI approved open source licenses\footcite[cf.][\nopage
wp]{OSI2012b}, the GPL-2.0%
  \footnote{For the original version, offered by the FSF 
  \cite[cf.][\nopage wp]{Gpl20FsfLicense1991a}. For the version, offered by the 
  OSI \cite[cf.][\nopage wp]{Gpl20OsiLicense1991a}.}
and the GPL-3.0.
  \footnote{For the original version, offered by the FSF 
  \cite[cf.][\nopage wp]{Gpl30FsfLicense2007a}. For the version, offered by the
  OSI \cite[cf.][\nopage wp]{Gpl30OsiLicense2007a}.}
Although both versions of the GPL aim for the same results and the same spirit,
they differ with respect to textual and arguing structure. Therefore, it is
helpful to treat these two licenses separately.  

\subsection{GPL-2.0}
\protectionlabel{GPL2}

The protecting power of the GPL-2.0 can easily be determined: First, the license
allows the users of a received software to \enquote{copy and distribute}
unmodified \enquote{copies of the [\ldots] source code}\citeGPLtwo{§1} as well
as to \enquote{[\ldots] modify [\ldots] copies [\ldots] or any portion of it,
[\ldots] and (to) distribute such modifications [\ldots]}\citeGPLtwo{§2}---%
not only in the form of source code, but also in the form of
binaries.\citeGPLtwo{§3} Thus---and in accordance of being an approved
\emph{open source license}\footcite[cf.][\nopage wp]{OSI2012b}---the GPL-2.0
protects the user against the loss of the right to use, to modify and/or to
distribute the received copy of the source code or the binaries. Second, it
protects the contributors against warranty claims\citeGPLtwo{§11, §12}
and---based on its copyleft effect\citeGPLtwo{§3}---also against the
loss of feedback. Third, the GPL-2.0 protects the source code itself in a nearly
complete mode against privatization: even if one initially distributes only the
binary version of a modification which one has generated (as a \enquote{work
based on the} original) by \enquote{copying} any {portion} of the original work
into this new derivative work,\citeGPLtwo{§2} then one has nevertheless to offer
a possibility to get the source code\citeGPLtwo{§4}---namely for \enquote{the
modified work as whole.}\citeGPLtwo{§3} This modified \enquote{work based on the
[original] Program} has to be read in a very broad sense; it \enquote{[\ldots]
means either the Program or any derivative work under copyright law: that is to
say, a work containing the Program or a portion of it, either verbatim or with
modifications and/or translated into another language.}\citeGPLtwo{§0} Hence, in
the context of software distribution, the GPL-2.0 does not only protect the
software against re-privatization, but also possible on-top developments against
privatization. 

But the GPL-2.0 does not protect against patent disputes\footnote{$\rightarrow$
\oslic, p. \patentpageref{GPL2}}---neither the users, nor the
contributors or distributors---and it does not protect the (modified) software
which is not distributed against (re-)privatization.%
  \footnote{This is a `lack' in the GPL which the AGPL wants to close: you are
  indeed allowed to modify and install a GPL-2.0 licensed server software on
  your own machine for offering a service based on this modified software
  without being obliged to give your improvements back to the
  community. But---at least in Germany---this viewpoint seems to have to respect
  rigorous limits. Sometimes, it is said that even distributing software over
  the parts of a holding is already a distribution which---in the case of 
  GPL-2.0 licensed software---would evoke the obligation to distribute the
  source code, too. [IMPORTANT: citation still needed!]}

\subsection{GPL-3.0}
\protectionlabel{GPL3}

An important modification of the GPL-3.0 is evoked by the use of the new wording
to \enquote{propagate} or to \enquote{convey} a \enquote{covered work}: On the
one hand a \enquote{covered work} denotes \enquote{either the unmodified Program
or a work based on the Program}. This \enquote{work based on the Program} is
defined as a \enquote{modified version} of an \enquote{earlier} instance of the
program which has been derived from this earlier instance by \enquote{(copying
it) from or (adapting) all or part of it} in way other than exactly copying the
earlier instance.\citeGPLthree{§0} On the other hand, \enquote{to propagate a
work} denotes \enquote{copying, distribution (with or without modification),
making available to the public} and any other kind of treating the work
\enquote{[\ldots] except executing it on a computer or modifying a private
copy.}%
  \footnote{\cite[cf.][\nopage wp. §0]{Gpl30OsiLicense2007a}. The GPL 3.0 wants
  to cover the copyright systems of all countries of the world without dealing
  with their particular constraints directly. Therefore it generally states,
  that the meaning of the phrase \enquote{to propagate a work}---in the spirit
  of the FSF---is whatever the specific copyright system wants to be covered by
  these words, \enquote{[\ldots] except executing it on a computer or modifying
  a private copy}.}
Third, the GPL 3.0 specifies that to \enquote{convey} a work \enquote{[\ldots]
means any kind of propagation that enables other parties to make or receive
copies.}\citeGPLthree{§0} This specification shall later on help to clarify that
it is an act of distribution if the recipient himself actively copies or fetches
a program. 

Referring to this new wording, the GPL-3.0 allows as a \enquote{basic
permission} to \enquote{[\ldots] make, run and propagate covered works
[\ldots] without conditions so long as your license otherwise remains in
force.}\citeGPLthree{§2} This might be read as \emph{anything is allowed without
any restrictions---provided there does not exist any rule which must be
respected}. Based on these specifications, the use and the modification of a
GPL-3.0 program only for yourself is not restricted.%
  \footnote{In general, you have to infer that you do not have to fulfill any
  tasks if you are using a piece of open source software only for
  yourself---namely based of the fact that the particular license rules focus 
  only on the distribution of the software, not on the private use. But in the
  GPL-3.0, this assertion concerning the private use becomes more explicit: It is
  one of your \enquote{basic permissions} to \enquote{[\ldots] make, run and
  propagate covered works that you do not convey, without conditions so long as
  your license otherwise remains in force}. And \enquote{to propagate a work}
  refers to anything \enquote{[\ldots] except executing it on a computer or
  modifying a private copy} 
  (\cite[cf.][\nopage wp. §2 and §0]{Gpl30OsiLicense2007a}). 
  Thus, the GPL-3.0 supports your total freedom on your own machine: Do whatever
  you want to do; anything goes---as long as you do not hand the result over to
  any third party in any sense.}  

So, in general---like all the other open source licenses and in accordance to
the OSD\footcite[cf.][\nopage wp]{OSI2012a}---also the GPL protects the user
against the loss of the right to use, to modify and/or to distribute the
received copy of the source code or the binaries.\citeGPLthree{§3, §4, §5, and~§6} 
Furthermore, based on its patent clauses, the GPL-3.0 protects the users and the
contributors of a software against patent disputes.%
  \footnote{$\rightarrow$ \oslic, p.\ \patentpageref{GPL3}}
Additionally, the GPL-3.0 tries to protect the contributors or distributors
against warranty claims by its well known \enquote{Disclaimer of
Warranty}\citeGPLthree{§15} and \enquote{Limitation of
Liability}\citeGPLthree{§16} which must explicitly made been known at least in  
each case of source code distribution.\citeGPLthree{§4} Finally, the most forceful
protection of the GPL-3.0 concerns the protection against the loss of feedback
and against the privatization: Whenever you distribute a GPL-3.0 licensed
program in the form of binaries, you have to make the source accessible,
too.\citeGPLthree{§6} Moreover, this obligation concerns every covered 
work, hence not only the unmodified original, but also any modification or
adaption derived by any other kind of copying parts of the original into the
\enquote{resulting work}:\citeGPLthree{§0} \enquote{You may convey a covered
work in object code form under the terms of sections 4 and 5, provided that you
also convey the machine-readable Corresponding Source under the terms of this
License.}\citeGPLthree{§6} So, no doubt: the GPL wants also the source code of
all on-top developments to be published, not only the modified programs and
libraries used as base of these on-top developments. The single mode of use, the
GPL does not protect against privatization, is the mode of using the software
only for yourself.%
  \footnote{Quite the contrary: The GPL-3.0 explicitly allows to
  delegate the modification to third parties and allows to distribute the source
  code as working base \enquote{[\ldots] to others for the sole purpose of having
  them make modifications exclusively for you [\ldots]} 
  (\cite[cf.][\nopage wp. §2]{Gpl30OsiLicense2007a}).}

\section{\texorpdfstring{The protecting power of the}{The} GNU Lesser General Public License (LGPL)}
\protectionlabel{LGPL}

The LGPL is maintained and offered by the Free Software Foundation and hosted as
part of the well known \enquote{GNU operating system
homepage.}\footcite[cf.][\nopage wp]{FsfGnuOsLicenses2011a} The meaning of the
name \emph{LGPL} was changed in the course of time. First, in 1991, it should be
resolved as \enquote{GNU Library General Public License} and should denote the
\enquote{first released version of the library GPL} which was \enquote{[\ldots]
numbered~2 because it goes with version~2 of the ordinary GPL.} Today, this
license is marked as \enquote{superseded by the GNU Lesser General Public
License}\footcite[cf.][\nopage wp]{Lgpl20FsfLicense1991a}. This newer
\emph{LGPL} version from 1999 was released as \enquote{the successor of the GNU
Library Public License, version 2, hence [as] the version
number~2.1.}\footcite[cf.][\nopage wp]{Lgpl21FsfLicense1999a} Finally, in June
2007, the---for now---last version of the \emph{LGPL} was released---namely with a
new structure: While GPL-2.0 and LGPL-2.1 are similar, but independent licenses,
the LGPL-3.0 has to be read as an addendum to GPL-3.0. At the beginning of the
LGPL-3.0 license, the content of the corresponding GPL-3.0 was included into
the LGPL by the sentence that \enquote{this version of the GNU Lesser General
Public License incorporates the terms and conditions of version~3 of the GNU
General Public License, supplemented by the additional permissions listed
below.}\footcite[cf.][\nopage wp]{Lgpl30FsfLicense2007a} Based on these
differences, it seems to be suitable to treat the different LGPLs separately.

\subsection{LGPL-2.1}
\protectionlabel{LGPL2}

Like the other versions of the GPL or LGPL, the LGPL-2.1 also explicitly
describes its purpose as the task to \enquote{protect} the \enquote{rights} of
the software users: it states that generally all \enquote{[\ldots] the GNU
General Public Licenses are intended to guarantee your freedom to share and
change free software [\ldots]}\citeLGPLtwo{Preamble} Of course, the LGPL-2.1 is
an approved \emph{open source license}\footcite[cf.][\nopage wp]{OSI2012b} which
protects the user against the loss of the right to use, to modify and/or to
distribute the received copy of the source code or the binaries.\citeLGPLtwo{§1, §2, §4} 
But the LGPL-2.1 does not offer any sentences to infer that it grants any patent
rights to the software user.\footnote{$\rightarrow$ \oslic,
p.\ \patentpageref{LGPL2}} So, it does not protect anyone against
patent disputes, neither the users, nor the
contributors\,/\,distributors. Instead of this, the LGPL-2.1 contains a special 
section \enquote{No Warranty} offering two paragraphs which together establish
the protection of the contributors and distributors against warranty
claims.\citeLGPLtwo{§15, §16} Finally, the LGPL-2.1 also protects the
distributed sources against a re-closing\,/\,re-privatization and the
contributors against the loss of feedback. For that purpose, the LGPL-2.1 on the
one hand states that the recipient \enquote{[\ldots] may modify (his) copy or
copies of the Library or any portion of it [\ldots] and copy and distribute such
modifications [\ldots]} provided that the results of these modifications are
\enquote{[\ldots] licensed at no charge to all third parties under the terms of
(the LGPL-2.1).}\citeLGPLtwo{§2} On the other hand, this LGPL version allows to
distribute such modifications \enquote{in object code or executable form}
provided that one accompanies these entities \enquote{[\ldots] with the complete
corresponding machine-readable source code} which itself must be distributed
under the terms of the LGPL-2.1.\citeLGPLtwo{§4}

But contrary to the GPL, the LGPL does not require to publish the code of an
overarching program or any on-top development: It distinguishes the
\enquote{work that \emph{uses} the Library} from the \enquote{work \emph{based
on} the Library}: First, it defines the \enquote{Library} as any
\enquote{software library or work} licensed under the LGPL-2.1 and adds that
\enquote{a `work \emph{based on} the Library' means either the Library or any
derivative work under copyright law.}\citeLGPLtwo{§0, emphasis ours} 
Second, it defines the \enquote{work that \emph{uses} the Library} as any
\enquote{[\ldots] program that contains no derivative of any portion of the
Library, but is designed to work with the Library by being compiled or linked
with it} whereas this \enquote{work that \emph{uses} the Library}---taken
\enquote{in isolation}---clearly \enquote{[\ldots] is not a derivative work of
the Library [\ldots]}% 
  \footnote{\cite[cf.][\nopage wp. §5, emphasis ours]{Lgpl21OsiLicense1999a}. To
  be exact: the LGPL states also, that this work can nevertheless become a
  derivative work under the particular circumstances of being linked to the
  library. But even then, the LGPL allows to treat this `derivative work' as a
  work which is not a derivative work, provided one fulfills some additional
  conditions. With respect to this viewpoint, the hint of the LGPL that the
  non-derivative work becomes a derivate work by linking it, seems not to be as
  crucial as one might expect.}
Third---and explictily \enquote{as an exception to the Sections above}---the
LGPL-2.1 allows to \enquote{[\ldots] combine or link a \enquote{work that uses
the Library} with the Library to produce a work containing portions of the
Library, and distribute that work under terms of (one's own) choice} provided
one \enquote{(accompanies) the work with the complete corresponding
machine-readable source code for the Library}. Together, these three
specifications clearly require that one must publish\,/\,distribute the source
code of the library itself---regardless, whether it is modified or not, and
regardless, whether one distributes the code directly or makes `only' written
offer for receiving the source code of the library separately.\citeLGPLtwo{§6}
But these  specifications do not require that one also must publish\,/\,distribute
the source code of the \emph{work that uses the library} or---as the \oslic{} is
using to say---the \emph{the on-top developments}.

Thus---no surprise---it has to be inferred that the LGPL does not protect the
on-top developments against a privatization. And of course, that is the reason why
it is called the \emph{GNU \emph{Lesser} General Public License}.


\subsection{LGPL-3.0}
\protectionlabel{LGPL3}

The LGPL-3.0 wants to be read as an extension of the GPL-3.0. For that purpose,
it explicitly \enquote{[\ldots] incorporates the terms and conditions of
version~3 of the GNU General Public License, supplemented by (some) additional 
permissions [\ldots]}\citeLGPLthree{just before §0} Thus, the LGPL-3.0 inherits
the most parts of the protecting power of the GPL-3.0---except those parts which
deal with the overarching on-top development: In opposite of the GPL-3.0, the
LGPL allows to embed LGPL-3.0 licensed libraries into libraries of higher
complexity\citeLGPLthree{§3}, into on-top applications\citeLGPLthree{§4}
and into sets of reorganized library systems.\citeLGPLthree{§5} Moreover, the
LGPL-3.0 allows to \enquote{convey} these overarching units \enquote{under terms
of (one's own) choice.}\citeLGPLthree{§4}  Therefore, one is not necessarily obliged to
publish the source code of these on-top developments, too%
  \footnote{To be exact:  The LGPL-3.0 wants to assure that \enquote{combined
  works} can be re-combined on the base of newer versions of the embedded
  library. For that purpose, one has either to use \enquote{a suitable shared
  libary mechanism} which allows to replace the embedded library without
  relinking the larger unit, or one has to publish at least \enquote{the minimal
  corresponding source [code]} and a set of binaries by which the user himself
  can relink the overarching unit on the base of a newer version ob the embedded
  library (\cite[cf.][\nopage wp. §4]{Lgpl30FsfLicense2007a})}%
---but, of course,  one is obliged to publish the source code of the (modified)
embedded libraries themselves. 

Based on the already described protecting power of the GPL-3.0%
\footnote{$\rightarrow$ \oslic, p.\ \protectionpageref{GPL3}}  
and on these additional specifications of the LGPL-3.0, one can summarize the
protecting power of the LGPL-3.0 this way:

First, the LGPL protects the users against the loss of the right to use, to
modify and/or to distribute the received software. Additionally, it protects
them against patent disputes. Second, it protects the contributors and
distributors against the loss of feedback, against warranty claims and against
patent disputes. Finally, it protects the distributed software itself against
re-privatization.

But the LGPL-3.0 does not protect the undistributed source code and does not
protect the on-top developments against privatization.

\section{\texorpdfstring{The protecting power of the}{The} MIT license}
\protectionlabel{MIT}

As an approved \emph{open source license,}\footcite[cf.][\nopage wp]{OSI2012b}
the MIT License%
  \footnote{`MIT' has to be resolved as \enquote{Massachusetts
  Institute of Technology} (\cite[cf.][\nopage wp]{wpMitLic2011a}).} 
protects the user against the loss of the right to use, to modify and/or to
distribute the received copy of the source code or the binaries.%
\footcite[cf.][\nopage wp 1ff]{OSI2012a} 
Additionally, it protects the contributors and/or distributors
against warranty claims of the software users, because it contains a `No
Warranty Clause.'\footcite[cf.][\nopage wp]{MitLicense2012a} And finally it
protects the distributed sources against a change of the license which would
close the sources, because the \enquote{permission [\ldots] to use, copy,
modify, [\ldots] distribute, [\ldots] (is granted) subject to the [\ldots]
conditions, [that] the [\ldots] copyright notice and this permission notice
shall be included in all copies or substantial portions of the
Software.}%
  \footnote{\cite[cf.][\nopage wp]{MitLicense2012a}. The argumentation
  why the source code is protected, but not the binary form follows that of the 
  BSD licenses: By these requirements, one is not obliged to redistribute the
  sourcecode of a modified (derivative) work. One is allowed to modify a received
  version and to distribute the results only in binary form and to keep one's
  improvements closed. But if one distribute the source code of the modifications,
  the licensing is retained, simply because the MIT \enquote{[\ldots] permission
  note shall be included in all copies or substantial portions of the software}.}

But the MIT License does not protect the users or the contributors and/or
distributors against patent disputes (because it does not contain any patent
clause). Additionally, it does not protect the contributors against the loss of
feedback (because it does not `copyleft' the software). Moreover, the MIT
license does not protect the undistributed software or the distributed binaries
against re-closings---neither in unmodified nor in modified form---because it
allows to redistribute only the binaries without also supplying the source
code.\footcite[cf.][\nopage wp]{MitLicense2012a} Finally, the MIT license does
not protect the on-top developments against a privatization.

\section{\texorpdfstring{The protecting power of the}{The} Mozilla Public License (MPL)}
\protectionlabel{MPL}
 
As an approved \emph{open source license,}\footcite[cf.][\nopage wp]{OSI2012b}
the Mozilla Public License%
  \footnote{In 2012, the Mozilla Public License 2.0 
  (\cite[cf.][\nopage wp]{Mpl20MozFoundation2012a}) has been released as a
  result of a longer \enquote{Revision Process}(\cite[cf.][\nopage
  wp]{Mpl11To20MozFoundation2013a}) by which the  Mozilla Public License 1.1 
  (\cite[cf.][\nopage wp]{Mpl11MozFoundation2013a}) has been ousted. The OSI is 
  also hosting its version of the MPL-2.0 (\cite[cf.][\nopage
  wp]{Mpl20OsiLicense2013a}) and is listing it as an OSI approved license 
  (\cite[cf.][\nopage wp]{OSI2012b}) while it classifies the MPL-1.1 as a
  \enquote{superseded license}(\cite[cf.][\nopage wp]{OSI2013b}). The Mozilla
  Foundation itself says concerning the difference between the two licenses that 
  \enquote{the most important part of the license---the file-level copyleft---is
  essentially the same in MPL~2.0 and MPL~1.1} (\cite[cf.][\nopage
  wp]{Mpl11To20MozFoundation2013a}). By reading the MPL-1.1, one could get the
  impression that fulfilling all conditions of the MPL-2.0 would imply also to act
  in accordance to the MPL-1.1. Thus the \oslic{} focuses on the MPL-2.0, at least
  for the moment. Nevertheless, in this section we want to use the general label
  `MPL' without any release number for indicating that with respect to its
  protecting power the MPL-2.0 and the MPL-1.1 can be taken as equipollent.}
protects the user against the loss of the right to use, to modify and/or to
distribute the received copy of the source code or the
binaries.\citeMPL{§2.1.a}  Furthermore, based on its split and distributed patent
clause,\footnote{$\rightarrow$ \oslic{} pp.\ \patentpageref{MPL}} the
MPL protects the users against patent disputes.\citeMPL{§2.1.b, §2.3, §5.2}
Besides this patent sections, the MPL additionally contains a
\enquote{Disclaimer of Warranty} and a \enquote{Limitation of
Liability.}\citeMPL{§6 \& §7} These three elements together protect the
contributors\,/\,distributors against patents disputes and warranty claims.
Finally, the MPL also protects the distributed sources against a
re-closing\,/\,re-privatization and the contributors against the loss of
feedback: The MPL clearly says that, on the one hand, \enquote{all distribution
of Covered Software in Source Code Form, including any Modifications [\ldots]
must be under the terms of this License}\citeMPL{§3.1} and that, on the other
hand, MPL licensed software \enquote{[\ldots] (distributed) in Executable
Form [\ldots] must also be made available in Source Code Form
[\ldots]}\citeMPL{§3.2} So, it must be inferred that the MPL is a copyleft
license. 

But nevertheless, the Mozilla Public License is not a license with strong
copyleft. It does not protect on-top developments against privatization: First,
the MPL does not use the term \emph{derivative work.}%
  \footnote{\cite[cf.][\nopage wp]{Mpl20OsiLicense2013a}. The MPL-1.1 uses the
  term \emph{derivative work} only in the context of writing new
  \enquote{versions of the license}, not in the context of licensing software
  (\cite[cf.][\nopage wp. §6.3]{Mpl11MozFoundation2013a}).}
Instead of this, the MPL denotes the
\enquote{[\ldots] (initial) Source Code Form [\ldots] and Modifications of such
Source Code Form} by the label \enquote{Covered Software}\citeMPL{§1.4}---while
the term \enquote{Modifications} refers to \enquote{any file in Source Code Form
that results from an addition to, deletion from, or modification of the contents
of Covered Software or any file in Source Code Form that results from an
addition to, deletion from, or modification of the contents of Covered
Software.}%
  \footnote{\cite[cf.][\nopage wp.\ §1.10]{Mpl20OsiLicense2013a}. The Mozilla
  Foundation denotes this reading by the term \enquote{file-level copyleft}
  (\cite[cf.][\nopage wp]{Mpl11To20MozFoundation2013a}).}
Second, the MPL contrasts the source code
form and its modifications with the \enquote{Larger Work} by specifying that the
larger work is \enquote{[\ldots] material, in a seperate file or files, that is
not covered software.}\citeMPL{§1.7}
Finally, the MPL states, that \enquote{you may create and distribute a Larger
Work under terms of Your choice, provided that You also comply with the
requirements of this License for the Covered Software.}\citeMPL{§3.3} Based on
these specifications, one has to reason that an on-top development which depends
on MPL licensed libraries by calling some of their functions, is undoubtably a
derivative work,%
  \footnote{This follows from the general meaning of a \emph{derivative work} as
  a benevolent software developer would read this term ($\rightarrow$ \oslic, pp.\
  \pageref{sec:BenevolentDerivativeWorkUnderstanding}). But again: The MPL does
  not focus on this general aspect; it uses its own concept of a \emph{larger
  work}.}
but also only a larger work in the meaning of the MPL so that code of
this on-top application needs not to be published---provided, that the library
and the on-top development are distributed as different files.%
  \footnote{It might be discussed whether integrating a declaration of a
  function, class, or method into the on-top development by including the
  corresponding header files indeed means that one is \enquote{including
  portions (of the Source Code Form)} into a file which therefore has to be
  taken as \enquote{Modification} (\cite[cf.][\nopage
  wp.\ §1.4]{Mpl11MozFoundation2013a}). From the viewpoint of a benevolent
  developer it should be difficult to argue that the including of declaring
  (header) files alone can evoke a derivative work. It is the call of the
  function in one's code which establishes the dependency. But that is not the 
  point, the MPL focuses. The MPL aims on the textual reuse of (defining) code
  snippets. Hence, one could ignore the textual integration of parts of the
  declaring header files: it should not trigger that one's own work becomes a
  modification in the eyes of the Mozilla Findation. But of course, one would
  circumvent the idea of the MPL if one hides defining code in header files and
  reuses that code by one's own compilation. This would undoubtably be an
  incorporation of portions and therefore would make the incorporating file
  becoming a modification of the MPL licensed initial work.} 
Hence, the MPL is license with a weak copyleft effect and does not protect the
on-top developments against privatization.

\section{\texorpdfstring{The protecting power of the}{The} Microsoft Public License (MS-PL)}
\protectionlabel{MSPL}

As an approved \emph{open source license,}\footcite[cf.][\nopage wp]{OSI2012b}
the Microsoft Public License protects the user against the loss of the right to
use, to modify and/or to distribute the received copy of the source code or the
binaries.\citeMSPL{§2} Furthermore, based on its patent
clause,\footnote{$\rightarrow$ \oslic{} pp.\ \patentpageref{MSPL}} the
MS-PL protects the users against patent disputes.\citeMSPL{§2.B and §3.B}
Because of this patent clause and of its 
concise \emph{disclaimer of warranty}, the MS-PL also protects the
contributors\,/\,distributors against patents disputes and warranty 
claims.\citeMSPL{§2B, §3B, §3D}
Finally, the Microsoft Public License protects the distributed sources
themselves---and even \enquote{portions of these sources}---\emph{against} a
change of the license which would \emph{reset} the work \emph{as closed
software}, because first, one \enquote{[\ldots] must retain all copyright,
patent, trademark, and attribution notices that are part of the
software,}\citeMSPL{§3C} and because, second, one must also incorporate
\enquote{a complete copy of this license} into one's own distribution premised
one distributes the source code.\citeMSPL{§3D}

But the Microsoft Public License does not protect the contributors against the
loss of feedback because it does not `copyleft' the software: The license does
not contain any sentence which requires that one has to publish the sources,
too.%
  \footnote{There seems to be some misunderstandings on the internet: The
  English wikipedia specifies the MS-PL as a permissive license and the MS-RL as
  a license with copyleft effect (\cite[cf.][\nopage wp]{wpMsSharedSources2013a}).
  The German wikipedia says that the MS-PL is a license with a \enquote{schwachen
  [weak] copyleft} (\cite[cf.][\nopage wp]{wpMspl2013a}). And it says also that
  the \enquote{Microsoft Reciprocal License} (MS-RL) is a license with weak
  copyleft, too (\cite[cf.][\nopage wp]{wpMsrl2013a}). But for the very
  thoroughly working \enquote{ifross license center}, the MS-RL is a license
  with restricted (weak) copyleft, while the MS-PL is a permissive license with
  some selectable options (\cite[cf.][\nopage wp]{ifross2011a}). Based on the
  license text itself and these other readings, we decided to take the MS-PL as
  a permissive license in accordance to the English wikipedia page and the
  ifross page.}
In the same spirit, the MS-PL does not protect the undistributed
software or the distributed binaries against re-closings---neither in
unmodified nor in modified form---because the MS-PL License allows to
(re)distribute the binaries without also supplying the sources---even if the
binaries rest upon sources modified by the distributor. Finally, also the MS-PL
does not protect the on-top developments against a privatization.


\section{\texorpdfstring{The protecting power of the}{The} Postgres License
(PostgreSQL)}
\protectionlabel{PGL}

As an approved \emph{open source license,}\footcite[cf.][\nopage wp]{OSI2012b}
the PostgreSQL License protects the user against the loss of the right to use,
to modify and/or to distribute the received copy of the source code or the
binaries.\citePGL{}
Because of its \emph{disclaimer of warranty}, the PostgreSQL also protects the
contributors\,/\,distributors against warranty claims.\citePGL{} Finally, the
PostgreSQL protects the distributed sources themselves \emph{against} a change of the
license which would \emph{reset} the work \emph{as closed software}, because the
\enquote{copyright notice} and the whole license must \enquote{[\ldots] appear
in all copies.}\citePGL{}

But the PostgreSQL License does not protect the contributors against the loss of
feedback because it does not `copyleft' the software: The license does not
contain any sentence which requires that one has to publish the sources, too. 
In the same spirit, the PostgreSQL does not protect the undistributed software or the
distributed binaries against re-closings---neither in unmodified nor in
modified form---because the PostgreSQL allows to (re)distribute the binaries without
also supplying the sources---even if the binaries rest upon sources modified by
the distributor. Finally, the PostgreSQL does not protect the on-top developments
against a privatization.


\section{\texorpdfstring{The protecting power of the}{The} PHP License}
\protectionlabel{PHP}

As an approved \emph{open source license,}\footcite[cf.][\nopage wp]{OSI2012b}
the PHP-3.0 License protects the user against the loss of the right to use, to
modify and/or to distribute the received copy of the source code or the
binaries.\citePHP{} Because of its \emph{disclaimer of warranty,} the PHP
license also protects the contributors\,/\,distributors against warranty
claims.\citePHP{}  Finally, the PHP license protects the distributed
sources themselves \emph{against} a change of the license which would
\emph{reset} the work \emph{as closed software,} because
\enquote{redistributions of source code must retain the [\ldots] copyright
notice, this list of conditions and the [\ldots] disclaimer.}\citePHP{}

But the PHP-3.0 License does not protect the contributors against the loss of
feedback because it does not `copyleft' the software: The license does not
contain any sentence which requires that one has to publish the sources, too. 
In the same spirit, the PHP license does not protect the undistributed software
or the distributed binaries against re-closings---neither in unmodified nor in
modified form---because the PHP license allows to (re)distribute the binaries
without also supplying the sources---even if the binaries rest upon sources
modified by the distributor.
  
\section{Summary}

All these specifications can not only be summarized by a
table,\footnote{$\rightarrow$ \oslic{}, p. \pageref{tab:powerOfLicenses}} but also
by a mindmap as it is shown at the end of this chapter. Moreover, based on these
specifications, one could generate new groups of open source licenses, new
classes, like `user protecting licenses,'\footnote{all of them because all of
them have to fulfill the OSD} `patent disputes fending licenses' up to more
sophisticated taxonomies.

However, one must keep in mind that all of these grouping viewpoints do not
legitimate the conclusion that all members of a group can be respected by
fulfilling the same requirements. This would only be possible if the grouping
criteria would directly refer to the fulfilling tasks. Indeed, nearly all open
source licenses do differ with respect to these criteria, and even if the
differences are very small, they can't be neglected.%
  \footnote{Pars pro toto:
  Both, the BSD license and the Apache license require that you provide an
  indication to the developers of the application. But in case of the BSD license
  you have to publish the copyright notice\,/\,line, while in case of the Apache
  license you have exactly to present the content of the notice file distributed
  together with the application.} 
So: reflecting on possible classes of open
source licenses is a good method to become familiar with the area of open source
licenses. But it is not a method to determine, what needs to be done to
obtain the right to use the software. For that purpose every license must be
considered individually.


\begin{tikzpicture}
\label{OSCLICMM}
\footnotesize

% (1.A) list of all licenses and their release numbers Level 5/6
\node[rectangle,draw,text width=1.4cm] (l0100) at (9,4)
{ \textit{BSD License} };
\node[text width=1.4cm] (l0101) at (8.25,3)
{ \scriptsize{3-Clauses} };
\node[text width=1.4cm] (l0102) at (10,3)
{ \scriptsize{2-Clauses} };
  
\node[rectangle,draw,text width=1.4cm] (l0200) at (10.2,5)
{ \textit{MIT License} }; 
  
\node[rectangle,draw,text width=1.4cm] (l0300) at (12,5.5)
{ \textit{\textbf{Apache} License}};
\node[text width=0.4cm] (l0301) at (12,4.5) {\scriptsize{2.0}};

\node[rectangle,draw,text width=1.4cm] (l0400) at (13,6.8)
{ \scriptsize{\textit{\textbf{M}icro\textbf{S}oft} \textbf{P}ublic
\textbf{L}icense} };
  
\node[rectangle,draw,text width=2.0cm] (l0500) at (13,8)
{\textit{\textbf{PostgreS[Q]} \textbf{L}icense}};
  
\node[rectangle,draw,text width=1.4cm] (l0600) at (13,9)
{\textit{\textbf{PHP} License}};
\node[text width=0.4cm] (l0601) at (14.5,9){\scriptsize{3.0}};
  

\node[rectangle,draw,text width=1.4cm] (l0800) at (13,10.7)
{ \textit{\textbf{M}ozilla \textbf{P}ublic \textbf{L}icense}};
\node[text width=0.4cm] (l0801) at (14.5,10.2){\scriptsize{1.1}};
\node[text width=0.4cm] (l0802) at (14.5,11.2){\scriptsize{2.0}};

\node[rectangle,draw,text width=1.4cm] (l0900) at (13,12.25)
{\textit{\textbf{E}clipse \textbf{P}ublic \textbf{L}icense}};
\node[text width=0.4cm] (l0901) at (14.5,12.25) {\scriptsize{1.0}};
 
\node[rectangle,draw,text width=1.5cm] (l1000) at (13,13.8)
{\textit{\textbf{E}uropean \textbf{P}ublic \textbf{L}icense}}; 
\node[text width=0.4cm] (l1001) at (14.5,13.3){\scriptsize{1.1}};
\node[text width=0.4cm,style=dotted] (l1002) at (14.5,14.3){\scriptsize{\textit{1.2}}};
  
\node[rectangle,draw,text width=1.4cm] (l1100) at (13,15.5)
{\textit{\textbf{L}esser \textbf{G}NU \textbf{P}ublic \textbf{L}icense}};

\node[text width=0.4cm] (l1101) at (14.5,15){\scriptsize{2.1}};
\node[text width=0.4cm] (l1102) at (14.5,16){\scriptsize{3.0} };

\node[rectangle,draw,text width=1.4cm] (l1200) at (13,17.5)
{\textit{\textbf{G}NU \textbf{P}ublic \textbf{L}icense}};

\node[text width=0.4cm] (l1201) at (14.5,17){\scriptsize{2.1}};
\node[text width=0.4cm] (l1202) at (14.5,18){\scriptsize{3.0} };

\node[rectangle,draw,text width=1.4cm] (l1300) at (13,19.5)
{ \textit{\textbf{A}ffero \textbf{G}NU \textbf{P}ublic \textbf{L}icense}};
\node[text width=0.4cm] (l1302) at (14.5,19.5){\scriptsize{3.0}};

% 2. the clustering concepts of licenses (level 4)
\node[rectangle,draw,text width=2.3cm] (n0100) at (10,8)
 { \textit{protecting the user, the con\-tri\-butor \& the initial code}\\
   \tiny{Permissive Licenses}      
 };

\node[rectangle,draw,text width=2.3cm] (n0200) at (10,12.5)
{ \textit{protecting the user, the con\-tri\-butor, the
  initial code, \& all di\-rect de\-ri\-va\-tions}\\
  \tiny{Weak Copyleft}        
};

\node[rectangle,draw,text width=2.3cm] (n0300) at (10,16.5)
{ \textit{protecting the user, the con\-tri\-bu\-tor, the 
  initial code, all di\-rect de\-ri\-va\-tions \& the 
  (in\-di\-rect\-ly de\-ri\-ved) on-top-deve\-lop\-ments}\\ 
  \tiny{Strong Copyleft}    
 };

% 3. the threats (level 3)
\node[ellipse,draw,text width=1.6cm] (c110000) at (4.5,0)
{ \textbf{\textit{Patent Disputes}}};

\node[ellipse,draw,text width=1.6cm] (c120000) at (4.5,2)
{ \textbf{\textit{Loss of Rights}} };

\node[ellipse,draw,text width=1.6cm] (c210000) at (4.5,4)
{ \textbf{\textit{Warranty Claims}} };
 
\node[ellipse,draw,text width=1.6cm] (c220000) at (4.5,6)
{ \textbf{\textit{Loss of Feeback}}};

\node[ellipse,draw,text width=0.6cm] (c311000) at (6.2,8)
{ \tiny{\textit{\textbf{reclos\-ings}}}};

\node[ellipse,,draw,text width=0.6cm] (c321000) at (6.2,10)
{ \tiny{\textit{\textbf{reclos\-ings}}} };

\node[ellipse,,draw,text width=0.6cm] (c331000) at (6.2,12)
{ \tiny{\textit{\textbf{reclos\-ings}}} };

\node[ellipse,,draw,text width=0.6cm] (c341000) at (6.2,14)
{ \tiny{\textit{\textbf{reclos\-ings}}} };

\node[ellipse,,draw,text width=0.6cm] (c351000) at (6.8,16.2)
{ \tiny{\textit{\textbf{reclos\-ings}}} };

\node[ellipse,,draw,text width=0.7cm] (c361000) at (7.5,17.5)
{ \tiny{\textit{\textbf{privati\-zings}}} };

\node[ellipse,,draw,text width=1.6cm] (c411000) at (6.5,19)
{ \textit{\textbf{clos\-ings}} };


% 4. the subtypes of protected entities (level 2)
\node[ellipse,draw,text width=1.5cm] (c310000) at (3,8)
 { \scriptsize{un\-modified} \textbf{Sources}};

\node[ellipse,draw,text width=1.5cm] (c320000) at (3.25,10)
 { \scriptsize{un\-modified} \textbf{Binaries}};

\node[ellipse,draw,text width=1.2cm] (c330000) at (3.5,12)
 { \scriptsize{modified} \textbf{Sources}};

\node[ellipse,draw,text width=1.4cm] (c340000) at (3.25,14)
 { \scriptsize{modified} \textbf{Binaries}};

\node[ellipse,draw,text width=2cm] (c350000) at (3.6,16)
 { \tiny{\textbf{part of} On-Top-Developments}};

\node[ellipse,draw,text width=2.9cm] (c360000) at (3.4,17.5)
 { \tiny{\textbf{On-Top-Developments}}};


% 5. the protected entities (level 1)
\node[ellipse,draw,text width=1cm] (c100000) at (1,1)
 { \textbf{Users} };

\node[ellipse,draw,text width=0.8cm] (c200000) at (1,5)
 { \textbf{Con\-tribu\-tors}};

\node[ellipse,draw,text width=0.8cm] (c300000) at (1,12)
 { distri\-buted \textbf{Soft\-ware}};
 
\node[ellipse,draw,text width=2.2cm] (c400000) at (1,19)
 { un\-distri\-buted \textbf{Soft\-ware}}; 

% 6. main node (leve 0)
\node[ellipse,draw,text width=1.3cm] (c000000) at (0,8)
{ \textbf{open source license}};

% a linking Licenses to their release numbers (Linking level 5 to 6)
\foreach \father/\daughter in {
  l0100/l0101/,
  l0100/l0102/,
  l0300/l0301/,
  l0600/l0601/,
  l0800/l0801/,
  l0800/l0802/,
  l0900/l0901/,
  l1000/l1001/,
  l1000/l1002/,
  l1100/l1101/,
  l1100/l1102/,
  l1200/l1201/,
  l1200/l1202/,
  l1300/l1302/
  }
  \draw[dashed] (\father) to  (\daughter) ;

% b) linking Licenses to license concepts (Linking level 5 to 4)
\foreach \father/\daughter/\outangle/\inangle in {
  n0100/l0100/270/150,       
  n0100/l0200/280/155,
  n0100/l0300/290/160,
  n0100/l0400/300/165,
  n0100/l0500/310/150,
  n0100/l0600/340/160,
  n0200/l0800/300/160,
  n0200/l0900/340/170,
  n0200/l1000/20/190,
  n0200/l1100/60/200,
  n0300/l1200/40/180,
  n0300/l1300/80/180 
  }
  %\draw[dashed] (\father) to [out=\outangle,in=\inangle] (\daughter) ;
  \draw[dashed] (\father) to  (\daughter) ;

% c) linking license concepts to the threats against they protect
% c.1) strong copyleft licenses
\foreach \father/\daughter/\outangle/\inangle in {
  c361000/n0300/0/180,
  c351000/n0300/0/180,
  c341000/n0300/45/190,
  c331000/n0300/50/200,
  c321000/n0300/55/210,
  c311000/n0300/60/220,
  c220000/n0300/25/225,
  c210000/n0300/25/230,
  c120000/n0300/25/235
  }
  \draw[<-,color=blue] (\father) to [out=\outangle,in=\inangle] (\daughter) ;
% c.2) weak copyleft licenses
\foreach \father/\daughter/\outangle/\inangle in {
  c341000/n0200/330/170,
  c331000/n0200/0/180,
  c321000/n0200/0/180,
  c311000/n0200/20/190,
  c220000/n0200/15/220,
  c210000/n0200/15/230,
  c120000/n0200/15/235
  }
  \draw[<-,color=cyan] (\father) to [out=\outangle,in=\inangle] (\daughter) ;
% c.3) permissive licenses
\foreach \father/\daughter/\outangle/\inangle in {
  c331000/n0100/355/150,
  c311000/n0100/0/180,
  c210000/n0100/5/210,
  c120000/n0100/10/230
  }
  \draw[<-,color=red] (\father) to [out=\outangle,in=\inangle] (\daughter) ;
%c.4 agpl license
\foreach \father/\daughter/\outangle/\inangle in {
  c411000/l1300/0/180    
}
  \draw[<-,color=green] (\father) to [out=\outangle,in=\inangle] (\daughter) ;


%d linking protected entities, their subtypes and the the relations
\foreach \father/\daughter/\edgetext/\outangle/\inangle in {
  c000000/c100000/protecting/260/120,
  c100000/c110000/against/360/180,
  c100000/c120000/against/360/180,
  c000000/c200000/protecting/270/180,
  c200000/c110000/against/340/150,
  c200000/c210000/against/0/180,
  c200000/c220000/against/0/180,
  c000000/c300000/protecting/90/230,
  c300000/c310000/as/300/180,
  c300000/c320000/as/330/180,
  c300000/c330000/as/0/180,
  c300000/c340000/as/30/180,
  c300000/c350000/as/60/180,
  c300000/c360000/as/70/180,
  c000000/c400000/protecting/100/240,
  c400000/c411000/against/0/180        
}
  \draw[->,dotted,
    decoration={text along path,
              text align={center},
              text={|\itshape|\edgetext}},
              postaction={decorate},] (\father) to [out=\outangle,in=\inangle] (\daughter) ;

\foreach \father/\daughter/\edgetext/\outangle/\inangle in {
  c310000/c311000/against/0/180,
  c320000/c321000/against/0/180,
  c330000/c331000/against/0/180,
  c340000/c341000/against/00/180,
  c350000/c351000/against/0/180,
  c360000/c361000/against/0/180      
}
  \draw[->,dotted,
    decoration={text along path,
              text align={center},
              text={|\itshape \tiny|\edgetext}},
              postaction={decorate},] (\father) to [out=\outangle,in=\inangle] (\daughter) ;

%f linking the patent clauses
\foreach \father/\daughter/\outangle/\inangle in {
  c110000/l1302/0/305,
  c110000/l1202/0/303,
  c110000/l1201/0/301,
  c110000/l1102/0/299,
  c110000/l1101/0/297,
  c110000/l1002/0/295,
  c110000/l0901/0/290,
  c110000/l0802/0/285,
  c110000/l0400/0/275,
  c110000/l0301/0/270   
}
  \draw[<-,color=gray] (\father) to [out=\outangle,in=\inangle] (\daughter) ;

\end{tikzpicture}




%\bibliography{../../../bibfiles/oscResourcesEn}

% Local Variables:
% mode: latex
% fill-column: 80
% End:



%%%%%%%%%%%%%%%

% Telekom osCompendium 'for being included' snippet template
%
% (c) Karsten Reincke, Deutsche Telekom AG, Darmstadt 2011
%
% This LaTeX-File is licensed under the Creative Commons Attribution-ShareAlike
% 3.0 Germany License (http://creativecommons.org/licenses/by-sa/3.0/de/): Feel
% free 'to share (to copy, distribute and transmit)' or 'to remix (to adapt)'
% it, if you '... distribute the resulting work under the same or similar
% license to this one' and if you respect how 'you must attribute the work in
% the manner specified by the author ...':
%
% In an internet based reuse please link the reused parts to www.telekom.com and
% mention the original authors and Deutsche Telekom AG in a suitable manner. In
% a paper-like reuse please insert a short hint to www.telekom.com and to the
% original authors and Deutsche Telekom AG into your preface. For normal
% quotations please use the scientific standard to cite.
%
% [ File structure derived from 'mind your Scholar Research Framework' 
%   mycsrf (c) K. Reincke CC BY 3.0  http://mycsrf.fodina.de/ ]
%

% Chapter Abstract
% ----------------
\chapter{Open Source: About Some Side Effects}\label{sec:SideEffects}

% Local Variables:
% mode: latex
% fill-column: 80
% End:

% Telekom osCompendium 'for being included' snippet template
%
% (c) Karsten Reincke, Deutsche Telekom AG, Darmstadt 2011
%
% This LaTeX-File is licensed under the Creative Commons Attribution-ShareAlike
% 3.0 Germany License (http://creativecommons.org/licenses/by-sa/3.0/de/): Feel
% free 'to share (to copy, distribute and transmit)' or 'to remix (to adapt)'
% it, if you '... distribute the resulting work under the same or similar
% license to this one' and if you respect how 'you must attribute the work in
% the manner specified by the author ...':
%
% In an internet based reuse please link the reused parts to www.telekom.com and
% mention the original authors and Deutsche Telekom AG in a suitable manner. In
% a paper-like reuse please insert a short hint to www.telekom.com and to the
% original authors and Deutsche Telekom AG into your preface. For normal
% quotations please use the scientific standard to cite.
%
% [ Framework derived from 'mind your Scholar Research Framework' 
%   mycsrf (c) K. Reincke 2012 CC BY 3.0  http://mycsrf.fodina.de/ ]
%

%% use all entries of the bibliography
%\nocite{*}

\section{The problem of implicitly releasing patents}
\footnotesize \begin{quote}\itshape In this chapter, we briefly analyze
the effects of patent clauses in open source licenses---not in general, but with
respect to the license fulfilling tasks they require, also known as the
`implicit acceptance of a patent use' by distributing open source software.
\end{quote}
\normalsize

At least the free software movement frowns on the existence of software
patents.%
  \footnote{For an early and elaborate description on the effects of
  software patents based on the viewpoint of the free software movement
  \cite[see][\nopage wp]{Stallman2001a}. This lecture seems to have been given
  more than once and printed later on (\cite[cf.][\nopage wp]{Stallman2002a}).
  Within the first decade of 2000, the focus switched to a more political fight
  against software patents (\cite[cf.][\nopage wp]{Stallman2004a}). But recently
  there seems to have appeared another turn in dealing with software patents:
  Not fighting against the patents, but mitigating their effects. The proposal is
  `[...] (to legislate) that developing, distributing, or running a program on
  generally used computing hardware does not constitute patent infringement'
  (\cite[cf.][\nopage wp]{Stallman2012a})}
One of the best known witnesses for that attitude is the GPL itself. Its
preamble purports that \enquote{[\ldots] any free program is threatened
constantly by software patents.}\citeGPLtwo{} One can read that the open
source community fears three risks: First, they are apprehensive of people who
hijack the idea of a piece of open source software they do not have developed,
register a corresponding patent, and finally try to earn money by preventing the
use of the software or by involving its users in patent
ligitations.\footcite[cf.][234]{JaeMet2011a} Second, they fear a bramble of
general software patents which practically prohibits them to develop open source
software legally.\footcite[cf.][234]{JaeMet2011a} Third, they anticipate the
possibility that (not quite benevolent) open source developers could try to
register patents with the intention of undermining the open source
principles.\footcite[cf.][235]{JaeMet2011a}

Howsoever, regardless whether one tries to fight against software patents or not,
software patents have become a reality. To abide by the law requires managing the
constraints of patents properly. Open source licenses know and respect this
necessity. Moreover, at least some of them try to manage the effect of software
patents by specific patent clauses\citeAPL[pars pro toto cf.]{§3} or by several
sentences distributed in the license text.\citeEPL[pars pro toto cf.]{wp} But why
does the \oslic{} have to deal with this topic, if the \oslic{} does not want to
participate in general discussions?

Opposite to the other conditions of the open source licenses, their patent
clauses or propositions in general do not directly refer to a specific set of
actions which have to be executed for acting in accordance with the licenses. Open
source patent clauses normally do not join in the game `paying by doing.' So,
actually, it does not seem to be necessary to mention the patent clauses here.

Unfortunately, although the patent clauses do not directly say \emph{`do this or
that in these or those circumstances,'} some of them nevertheless have side
effects which imply that the distributors of open source software already
have something done if they actually distribute a piece of open
source software. This implicit effect makes it necessary to deal with the patent
clauses even in an only pragmatic \oslic.

Patent clauses in open source licenses can have two different directions of
impact. They use two methods to protect the users of the open source software---%
and sometimes these methods are combined:

\begin{itemize}
  \item First, an open source license can assure that all contributors to and
  distributors of a piece of open source software grant to all users/%
  recipients not only the right to use the open source software itself, but
  automatically and implicitly also the right to use all those patents 
  belonging to the contributors/distributors which as patents are necessary
  to use the software legally.%
    \footnote{There might arise a legal discussion
    whether even a distributor who does not contribute to the software development
    has to grant the necessary rights of his patent
    portfolio. The \oslic{} does not want to participate in this discussion. We take a
    simple and pragmatic position: to be sure that you are acting according to
    an open source license with such a patent clause you should simply assume that
    you have to do so. If this default position is not reasonable for you it might
    be a good idea to consult legal experts which---perhaps---may find another
    way for you to use the software legally.} 
  So, let us---a little simplifying and therefore only on the following few
  pages---name such licenses the \emph{granting licenses}.
  \item Second, an open source license can try to automatically terminate the
  right to use, to modify, and to distribute the software if its user initiates
  litigations against any of the contributors/distributors with respect to an
  infringement of patent. That can be seen as a revocation of rights granted 
  earlier. So, let us name these license the \emph{revoking licenses.}
\end{itemize}

Later on, we will summarize the concrete patent clauses of all the licenses
discussed in the \oslic{} as a proof for the following classification:

\begin{small}

\begin{center}
\begin{tikzpicture}
\label{PATTAX}


\node[ellipse,minimum height=8.5cm,minimum width=14.2cm,draw,fill=gray!10] (l0100) at (6.7,6.8)
{  };

\draw [-,dotted,line width=0pt,white,
    decoration={text along path,
              text align={center},
              text={|\itshape|open source licenses}},
              postaction={decorate}] (-0.8,6.5) arc (218:322:9.5cm);
              
\node[ellipse,minimum height=6.2cm,minimum width=5cm,draw,fill=gray!20] (l0100)
at (2.5,6.8) {  };

\draw [-,dotted,line width=0pt,white,
    decoration={text along path,
              text align={center},
              text={|\itshape| without granting patent clauses}},
              postaction={decorate}] (0.75,7.5) arc (180:0:1.8cm);

\node[rectangle,draw,text width=1.2cm, text height=0.36cm, fill=gray!40, text
centered] (l0101) at (2.5,8.6) {\footnotesize \textit{MIT}};
\node[rectangle,draw,text width=1.2cm, text height=0.36cm, fill=gray!40, text
centered] (l0102) at (1.7,7.6) {\footnotesize \textit{BSD-X-Clause}};
\node[rectangle,draw,text width=1.2cm, text height=0.36cm, fill=gray!40, text
centered] (l0103) at (3.4,6.4) {\footnotesize \textit{LGPL-2.1}};
\node[rectangle,draw,text width=1.2cm, text height=0.36cm, fill=gray!40, text
centered] (l0104) at (3.4,7.6) {\footnotesize \textit{GPL-2.0}};
\node[rectangle,draw,text width=1.2cm, text height=0.36cm, fill=gray!40, text
centered] (l0105) at (1.7,6.4) {\footnotesize \textit{PHP-3.X}};
\node[rectangle,draw,text width=1.4cm, text height=0.36cm, fill=gray!40, text
centered] (l0106) at (2.5,5.2) {\footnotesize \textit{Post-greSQL}};

\node[ellipse,minimum height=6cm,minimum width=8.5cm,draw,fill=gray!20] (l0200)
at (9.4,6.5) {  };

\draw [-,dotted,line width=0pt,white,
    decoration={text along path,
              text align={center},
              text={|\itshape| with granting patent clauses}},
              postaction={decorate}] (2.2,2) arc (180:0:7cm);


\node[ellipse,minimum height=4.5cm,minimum width=5.6cm,draw,fill=gray!30]
(l0210) at (8.4,6.1) {  };

\draw [-,dotted,line width=0pt,white,
    decoration={text along path,
              text align={center},
              text={|\itshape| granting + revoking}},
              postaction={decorate}] (4.4,3.8) arc (180:0:4cm);

\node[rectangle,draw, text width=2cm, text height=0.34cm, fill=gray!40, text
centered] (l0212) at (7.2,6.9) {  \footnotesize \textit{Apache-2.0}};

\node[rectangle,draw, text width=2cm, text height=0.34cm, fill=gray!40, text
centered] (l0211) at (9.6,6.9) {  \footnotesize  \textit{EPL-1.X}};

\node[rectangle,draw, text width=2cm, text height=0.34cm, fill=gray!40, text
centered] (l0213) at (7.2,6.1) {  \footnotesize  \textit{MPL-X.Y}};

\node[rectangle,draw, text width=2cm, text height=0.34cm, fill=gray!40, text
centered] (l0214) at (9.6,6.1) {  \footnotesize  \textit{MS-PL}};

\node[rectangle,draw, text width=2cm, text height=0.34cm, fill=gray!40, text
centered] (l0213) at (7.2,5.3) {  \footnotesize  \textit{LGPL-3.X}};

\node[rectangle,draw, text width=2cm, text height=0.34cm, fill=gray!40,
text centered] (l0213) at (9.6,5.3) {  \footnotesize  \textit{GPL-3.0}};

\node[rectangle,draw,text width=1.6cm, text height=0.34cm, fill=gray!40, text
centered] (l0214) at (8.4,4.5) {  \footnotesize  \textit{AGPL-3.0}};
 
 
\node[rectangle,draw, text width=1.8cm, text height=0.34cm, fill=gray!40, text
centered] (l0221) at (11.6,8) {  \footnotesize  \textit{EUPL-1.X}};

\end{tikzpicture}
\end{center}

\end{small}


But regardless of the final textual form a license uses to express its
granting or revoking positions, in any case one has to consider some aspects: 

\begin{itemize}
  
  \item Overall, one has to keep in mind that of course no licensor, contributor
  and/or distributor can release the right to use any patents he does not own---%
  not even if he \emph{tries} to release them by an open source patent
  clause.%
    \footnote{The EPL is one of the licenses which insists on this aspect:
    It the second half of its patent clause, the EPL underlines that
    \enquote{[\ldots] no assurances are provided by any Contributor that the
    Program does not infringe the patent or other intellectual property rights of
    any other entity.} Moreover, it explicitly adds that \enquote{[\ldots] if a
    third party patent license is required to allow Recipient to distribute the
    Program, it is Recipient's responsibility to acquire that license before
    distributing the Program} (\cite[cf.][\nopage wp §2c]{Epl10OsiLicense2005a}).}
  Implictly touched patents of third parties not having contributed to the
  development and/or participated in the distribution can never be implicitly
  and automatically released on the base of such an (open source) patent clause:
  no rights, no right to release.% 
    \footnote{This is an important aspect which is sometimes not considered by
    programmers. Inside of DTAG we had a fruitful discussion evoked by Mr. Stephan
    Altmeyer who---as patent lawyer---patiently explained this constraint to us.} 
  Hence: even for those open source licenses which try to protect the users,
  finally the users themselves must nevertheless ensure that they do not violate
  the patents of third parties being unwillingly touched by the way the code
  works or the processes in which the software is used.%
    \footnote{Sometimes, this problem of willingly or
    unwillingly violated third party patents is seen as a weakness of open source
    software. But that is not true. It is a weakness of every software. Even a
    commercial licensor (developer) has only the right to license the use of those
    patents he really owns or he has `bought' for relicensing. Moreover, even
    commercial licensors can willingly or unwillingly violate patents of other
    persons.}
  
  \item In the context of a granting license, one has also to consider that
  contributing to and distributing a piece of software implicitly evokes that
  all patents of the contributor and/or distributor are `given free' which are
  necessary to use the software as whole---including the more or less deeply
  embedded libraries. So, if one wants to check whether some of the core patents
  of one's patent portfolio are afflicted by a patent clause (and whether one
  therefore better should not use/distribute the corresponding piece of open
  source software), one should not forget to check the embedded libraries, too.
  
  \item Finally, one has to consider in the context of a granting license that
  its patent clause only releases the use of the patents in the meaning of
  `allowed to be used for enabling the use of the distributed software.' The
  patent clause does not release the patents generally. Thus, the threat of
  (unwillingly) releasing patents by open source software is not as large as
  sometimes feared: the use of the patent is only granted in combination with
  the software. On the one hand, you may not use the open source software
  without having the right to use the patent because the use of the patent is
  inherently necessary for using the software---regardless, whether the open
  source software is embedded into a larger process or not. On the other hand,
  you are not allowed to use patents---released by the patent clause of an open
  source license---without exactly that open source software which has been
  licensed under this open source license, because the patent clause only refers
  to the use of just that open source software.
  % TODO: this is not completely accurat. OSS license grant downstream patent
  % licenses even for modified software.  The extent to which the software may
  % be modified varies between diffenrent licenses. (RPD)

  \item Summarized, one has to consider that the granting open source licenses
  automatically and implicitly force you to grant all the rights which are
  necessary to use the software legally. Open source contributors and
  distributors should know that.\footnote{Again: It might be debatable whether
  this is also valid for the distributors which do not contribute anything to
  the development. That's a legal discussion the \oslic{} does not wish to participate
  in. From the viewpoint of an open source user who only wants to have one
  reliable and secure way to use open source software compliantly, one should
  perhaps assume that there is no difference.}

  \item With respect to the revoking licenses, one has to consider that their
  patent clauses contain negative conditions which may be read as interdictions.
  The \oslic{} will integrate these conditions into specific `prohibits'-sections
  of its to-do lists.
  
  \item Finally one should mention that in some cases, the form of the
  revocation used by the revoking license refers to the use of the software, in
  other cases to the use of the patents. But nevertheless, one can reason that%
  ---from the pragmatic viewpoint of a benevolent open source software user---%
  this second case of patent revocation also implicitly terminates the right to
  use the software: If the use of a patent is necessary to use a piece of
  software legally, one is not allowed to use the software without having the
  right to use the patent, too; and if the use of the patent is not necessary
  for using the software, then the patent is not covered by the patent clause.
  So, in any case, this kind of patent clauses seems to terminate the right to
  use, distribute or modify the software. Hence, single users as well
  as companies or organizations should also respect such patent clauses if they
  want to be sure to use open source software compliantly.
\end{itemize}

The \oslic{} wants to support its readers not only to act according to the licenses
in general, but also according to its patent clause. Thus, we now briefly cite
and summarize the meaning of particular patent clauses:

\subsection{AGPL statements concerning patents}
\patentlabel{AGPL}

(prelimiary text)

The AGPL-3.0 is a license derived from the GPL-3.0: apart from the preamble and
the paragraphs §11 and §13, they contain nearly the same text.%
  \footnote{compare \cite[][\nopage]{Agpl30OsiLicense2007a} and
  \cite[][\nopage]{Gpl30OsiLicense2007a} in both §1 \ldots §11}
In §13, the AGPL explictly refers to the focus on a \enquote{remote network
interaction} which shall also be able to trigger the delivery of the
corresponding source code; and in §11, the AGPL establishes its specific patent
clause \cite[cf.][\nopage §11 and §13]{Agpl30OsiLicense2007a}.

Like the GPL-3.0, the AGPL-3.0 tries to protect all licensees against patent
claims. This kind of protection is then established by three steps:

First, the AGPL-3.0 assures that \enquote{each contributor grants a non
exclusive, worldwide, royalty free patent license under the contributor’s
essential patent claims, to make, use, sell offer for sale, import and
otherwise run, modify and propagate the contents of its contributor
version.}\citeAGPL{§11} Furthermore, the patent license defines that this patent
license granted by the contributor is automatically extended to all downstream
recipients who later on receive any version of the work even if they indirectly
receive them by third parties and even if they receive a covered work or work
based on the program.\citeAGPL{§11}

Second, the AGPL enforces not only the grant of patent licenses by the
\enquote{contributors,} the license even requires the same from licensees who
distributes the program unchanged: \enquote{If, pursuant to or in connection
with a single transaction or arrangement, you convey, or propagate by procuring
conveyance of, a covered work, and grant a patent license to some of the parties
receiving the covered work authorizing them to use, propagate, modify or convey
a specific copy of the covered work, then the patent license you grant is
automatically extended to all recipients of the covered work and works based on
it.}\citeAGPL{§11}

Finally, the AGPL-3.0 introduces an revoking clause by stating that a licensee
\enquote{[\ldots] may not initiate litigation (including a cross-claim or
counterclaim in a lawsuit) alleging that any patent claim is infringed by
making, using, selling, offering for sale, or importing the Program or any
portion of it}\citeAGPL{§10} and that this licensee \enquote{automatically}
loses the rights granted by the AGPL-3.0 \enquote{including any patent
licenses} if he tries to propagate or modify a covered work against the
regulations of the AGPL-3.0.\citeAGPL{§8} 

According to that, the AGPL-3.0 is like the GPL-3.0 a granting and a revoking
license: At first, one is granted the right to use all patents of all
contributors which are necessary to use the software legally. But if one
installs any litigation regarding an infringement of patents, then the rights
granted to him are revoked.


\subsection{Apache-2.0 statements concerning patents}
\patentlabel{APL}

Titled by the headline \enquote{Grant of Patent License}, the Apache License~2.0
contains a specific patent clause being comprised of two very long and condensed
sentences.\citeAPL{§3} Outside of this patent clause, the word \emph{patent} is
only used once again---for requiring that one \enquote{[\ldots] must retain, in
the (sources) [\ldots] all [\ldots] patent [\ldots] notices [\ldots]}\citeAPL{§4.3}

The one core message of the Apache-2.0 patent clause is that
\enquote{[\ldots] each Contributor hereby grants to You a perpetual, worldwide,
non-exclusive, no-charge, royalty-free, irrevocable [\ldots] patent license to
make, have made, use, offer to sell, sell, import, and otherwise transfer the
Work [\ldots]}%
  \footnote{\cite[cf.][\nopage wp §3]{Apl20OsiLicense2004a}. The
  \enquote{Contributor,} \enquote{Work,} and \enquote{You} are defined in §1:
  \emph{Contributor} refers to the original licensor and to all others whose
  contributions have been incorporated into the Work. The \emph{Work} denotes
  the result of the development process regardless of its form. \emph{You}
  denotes the licensees.}

The second core message of the Apache-2.0 patent clause is the statement that
\enquote{if You institute patent litigation against any entity [\ldots] alleging
that the Work [\ldots] constitutes [\ldots] patent infringement, then any patent
licenses granted to You [\ldots] shall terminate [\ldots]}\citeAPL{§3}

The third message of the Apache-2.0 patent clause is the statement, that the
\enquote{[\ldots] license applies only to those patent claims licensable by such
Contributor that are necessarily infringed by their Contribution(s) alone or by
combination of their Contribution(s) with the Work to which such Contribution(s)
was submitted}.\citeAPL{§3}

Thus, the Apache-2.0 is---as we use to say in this chapter---a granting and a
revoking license: At first you are granted to use all patents of all
contributors which are necessary to use the software legally. But if you---with
respect to the software---install any litigation concerning the infringement of
patents, then the rights granted to you are revoked.

\subsection{CDDL statements concerning patents}
\patentlabel{CDDL}

The patent clauses of the CDDL are similiar in spirit to the Apache License: 
The license grants rights to each contributors patents that are neccessarily
infringed by distributing or using the software. The license also revokes all
rights granted to someone who files a patent litigation with respect to the
software against any contributor.  The CDDL differs from other licenses in that
the litigant does not lose his rights automatically and immediately but gets a
grace period of 60 days. If he withdraws his claims during this period, the
license granted to him will not be terminated.

The actual wording used in the CDDL is complicated by the fact that the CDDL
distinguished between the \enquote{Initial Developer} and other
\enquote{Contributors.}  A \enquote{Contributor} receives a version of the
software to which he then adds some \enquote{Modifications} thus creating the
\enquote{Contributor Version.} For all practical purposes we can treat the
\enquote{Initial Developer} as another contributor who happens to not receive
any software and whose \enquote{Contributor Version} (officially called
\enquote{Original Software}) equals his \enquote{Modifications.}

The patent licenses are granted in the clause (b) of the sections titled
\enquote{The Initial Developer Grant}\citeCDDL{§2.1(b)} and \enquote{Contributor
  Grant.}\citeCDDL{§2.2(b)} Each contributor grants the licensee \enquote{a
  world-wide, royalty-free, non-exclusive license under Patent Claims infringed
  by the making, using, or selling of Modifications made by that Contributor
  either alone and/or in combination with its Contributor Version [\ldots], to
  make, use, sell, offer for sale, have made, and/or otherwise dispose of: (1)
  Modifications made by that Contributor [\ldots]; and (2) the combination of
  Modifications made by that Contributor with its Contributor Version [\ldots]} 
This limits the patent license to patents infringed by code present in the
contributor version. And clause (d) limits the grant even further to exclude
\enquote{infringements caused by[\ldots]third party modifications of Contributor
Version}\citeCDDL{§2.2(d)} or {Covered Software in the absence of Modifications
made by that Contributor.}\citeCDDL{§2.2(d)}
This ensures that no contributor is required to tolerate an infringement of his
patents caused by code modified after he made his contribution and, in
particular, it is not possible to remove the contributors modifications completely
without also removing all other causes of infringement of the patent claims
because the patent license does not carry over to such a use of the software.

The section titled \enquote{TERMINATION} contains the usual defense
against patent infringement claims by declaring that any such claim
against a \enquote{Participant%
  \footnote{The \enquote{Contributor} or \enquote{Initial Developer} against
  whom the claim is made}
[\ldots] alleging that the Participant Software [\ldots] directly or indirectly
infringes any patent, then any and all rights granted directly or indirectly to 
You\footnote{The party making the patent infringement claim}
[\ldots] under Sections 2.1 and/or 2.2 of this
License shall, upon 60 days notice from Participant terminate prospectively and
automatically at the expiration of such 60 day notice period, unless [\ldots] 
You withdraw Your claim [\ldots] against such Participant either unilaterally or
pursuant to a written agreement with Participant.}

Thus, not only has the Participant to actively initiate the termination of the
licenses, the licensee also has 60 days to either settle the case by an
agreement with the Participant or to withdraw his claims.


\subsection{EPL statements concerning patents}
\patentlabel{EPL}

The Eclipse Public License treats the patents necessary to use the program
in the same section and under the same headline \enquote{Grant of Rights} like
all the other rights: First, the EPL clearly states that \enquote{[\ldots] each
Contributor [\ldots] grants (the recipient) a non-exclusive, worldwide,
royalty-free patent license under Licensed Patents to make, use, sell, offer to
sell, import and otherwise transfer the Contribution of such Contributor, if
any, in source code and object code form.}\citeEPL{§2.b} Then the EPL delimits
the extend of this act of granting: Neither hardware patents of the contributors
are covered by this releasing patent clause, nor patents that concern aspects
out of the area of the initially intended software combination.\citeEPL{§2.b}
Finally, the EPL hints to the general fact that 3$^{rd}$ party patents not
belonging to the contributors can never be implicity be released by such a
patent clause. Moreover, it gives the example that \enquote{[\ldots] if a third
party patent license is required to allow Recipient to distribute the Program,
it is Recipient's responsibility to acquire that license before distributing the
Program.}\citeEPL{§2.c}

Like other open source licenses, the EPL announces at its end that
\enquote{if (a) Recipient institutes patent litigation against any entity
[\ldots] alleging that the Program [\ldots] infringes such Recipient's
patent(s), then such (granted) Recipient's rights [\ldots] shall terminate
[\ldots]}\citeEPL{§7}

Thus, the EPL, too, is a granting and a revoking license: 
At first you are granted the use of all patents of all
contributors which are necessary to use the software legally. But if you---with
respect to the software---install any litigation concerning an infringement of
patents, then the rights granted to you are revoked.

\subsection{EUPL statements concerning patents}
\patentlabel{EUPL}

The European Union Public License contains a very brief patent clause. It only
states, that \enquote{the Licensor grants to the Licensee royalty-free, non
exclusive usage rights to any patents held by the Licensor, to the extent
necessary to make use of the rights granted on the Work under this
Licence.}\citeEUPL{end of §2}
Furthermore the EUPL does not contain any patent specific revoking clause, but
only an abstract clause requiring that all \enquote{[\ldots] the rights granted
hereunder will terminate automatically upon any breach by the Licensee of the
terms of the Licence}\citeEUPL{§12}. Thus, the EUPL is---as we are using to say
in this chapter---a granting license but not a revoking license.

\subsection{GPL statements concerning patents}

Although the GPL versions 2.0 and 3.0 are aiming for the same results, they
differ heavily with respect to textual and arguing structure. Therefore, it
should be helpful to treat these two licenses separately.

\subsubsection{GPL-2.0}
\patentlabel{GPL2}

The GPL-2.0 does not contain any specific patent clause by which it would grant
(and revoke) the rights to use those patents belonging to the contributors and 
being necessary to use the software in accordance with the legal patent system.

Instead of this, the preamble of the GPL-2.0 alleges that \enquote{[\ldots] any
free program is threatened constantly by software patents} and that the authors
of the GPL---for tackling this threat---\enquote{[\ldots] had made it clear
that any patent must be licensed for everyone's free use or not licensed at
all}\citeGPLtwo{Preamble}. Unfortunately, this specification is only an indirect
claim which needs a lot of arguing for establishing a protective effect against
patent disputes. Howsoever, this paragraph of the GPL-2.0 does not directly
grant any rights to the software users to use necessary patents, too.

With respect to the patent problem, the GPL-2.0 also states that a licensee has
to fulfill the conditions of the GPL-2.0 completely, even if an existing patent
infringement---being \enquote{imposed} on the GPL licensee---\enquote{[\ldots]
contradicts the conditions of this license} so, that a waiver of the use of the
software is the only way to fulfill both constraints.\citeGPLtwo{§11} And
finally the GPL-2.0 allows the original copyright holder to \enquote{add an
explicit geographical distribution limitation excluding [\ldots] countries}
provided that these countries \enquote{[\ldots] (restict) the distribution
and/or use of the library [\ldots] by patents [\ldots]}\citeGPLtwo{§12}
Based on these statements, one cannot infer that the GPL-2.0 grants any patent
rights to the software user, neither directly, nor indirectly.

Thus, the GPL-2.0 is neither a granting nor a revoking license.

\subsubsection{GPL-3.0}
\patentlabel{GPL3}

Initially, the GPL-3.0 regrets that \enquote{[\ldots] every program is
threatened constantly by software patents} what should be seen as the
\enquote{[\ldots] danger that patents applied to a free program could make it
effectively proprietary}. And therefore---as the GPL-3.0 itself summarizes its
patent rules---\enquote{[\ldots] the GPL assures that patents cannot be used to
render the program non-free.}\citeGPLthree{Preamble}. This kind of protection is
then established by three steps. First, the GPL-3.0 stipulates that
\enquote{each contributor grants [\ldots the licensees] a non-exclusive,
worldwide, royalty-free patent license under the contributor's essential patent
claims, to make, use, sell, offer for sale, import and otherwise run, modify and
propagate the contents of its contributor version.}\citeGPLthree{§11}
Second, the GPL-3.0 defines that this patent license granted by the contributor
\enquote{[\ldots] is automatically extended to all recipients} who later on
receive any version of the work, even if they indirectly receive them by third
parties and even if they receive a \enquote{covered work} or \enquote{works
based on it.}\citeGPLthree{§11} Moreover, the GPL-3.0 also specifies that those
distributors of a \enquote{covered work} who have the right to use a patent
necessary for the use of the distributed software but who are not allowed to
relicense this patent to third parties must solve this problem by making the
source code available nevertheless, by \enquote{depriving} themselves or by
\enquote{extending the patent license to downstream recipients.}\citeGPLthree{§11} 
And finally, the GPL-3.0 also introduces a revoking clause by stating that a
licensee \enquote{[\ldots] may not initiate litigation [\ldots] alleging that
any patent claim is infringed by making, using, selling, offering for sale, or
importing the Program or any portion of it}\citeGPLthree{§10} and that this
licensee \enquote{automatically} loses the rights granted by the GPL-3.0
\enquote{including any patent licenses} if he tries to propagate or modify a
covered work against the rules of the GPL-3.0.\citeGPLthree{§8}

Thus, GPL-3.0 is a granting and a revoking license: At first, one is granted the
right to use all patents of all contributors which are necessary to use the
software legally. But if you---with respect to the software---install any
litigation concerning an infringement of patents, then the rights granted to you
are revoked. 


\subsection{LGPL statements concerning patents}

As already mentioned above, the LGPL versions 2.1 and~3.0 differ heavily with
respect to textual and arguing structure. Therefore, they should be treated
separately.

\subsubsection{LGPL-2.1}
\patentlabel{LGPL2}

Like the GPL-2.0, the LGPL-2.1 does not contain any specific patent clause by
which it would grant (and revoke) the rights to use those patents belonging to
the contributors and being necessary to use the software in accordance with the
legal patent system.

Instead of this, the preamble of the LGPL-2.1 says that \enquote{[\ldots]
software patents pose a constant threat to the existence of any free program}
and that the authors of the LGPL---for tackling this threat---%
\enquote{[\ldots] insist that any patent license obtained for a version of the
library must be consistent with the full freedom of use specified in this
license.}\citeLGPLtwo{Preamble}
Unfortunately, this specification is again only an indirect claim which needs a
lot of arguing to establish a protective effect against patent disputes.
Howsoever, this paragraph of the LGPL-2.1 does not directly grant any rights to
the software users to use necessary patents.

With respect to the patent problem, the LGPL-2.1 also states that a licensee has
to fulfill the conditions of the LGPL-2.1 completely, even if an existing patent
infringement---being \enquote{imposed} on the LGPL licensee---%
\enquote{[\ldots] contradicts the conditions of this license} so that a waiving
of the use of the software is the only way to fulfill both
constraints.\citeLGPLtwo{§11} And finally the LGPL-2.1 allows the original
copyright holder to \enquote{add an explicit geographical distribution limitation
excluding [\ldots] countries} provided that these countries \enquote{[\ldots]
(restict) the distribution and/or use of the library [\ldots] by patents
[\ldots]}\citeLGPLtwo{§12} Based on these statements, one cannot infer that 
the LGPL grants any patent rights to the software user, neither directly, nor
indirectly.

Thus, the LGPL-2.1 is neither a granting nor revoking license.

\subsubsection{LGPL-3.0}
\patentlabel{LGPL3}

The LGPL-3.0 is an extension of the GPL-3.0. Before starting with a section
\enquote{Additional Definitions}, the LGPL-3.0 states that it \enquote{[\ldots]
incorporates the terms and conditions of version~3 of the GNU General Public
License} and then \enquote{supplements} this GPL-3.0 content by some
\enquote{additional permissions.}\citeLGPLthree{wp} The LGPL-3.0 itself does not
contain the word `patent,' but the GPL-3.0 does.\citeGPLthree{§11}
So, the LGPL-3.0 inherits its patent clause from the GPL-3.0 which is---as we
already described\footnote{$\rightarrow$ \oslic{}, p.\
\patentpageref{GPL3}}---a granting and a revoking license.
 
\subsection{MPL statements concerning patents}
\patentlabel{MPL}

The MPL distributes its statements concerning the tolerated use of the patents
over three paragraphs: First, it clearly says that \enquote{each Contributor
[\ldots] grants [\ldots the licensee] a world-wide, royalty-free,
non-exclusive license [\ldots] under Patent Claims of such Contributor to
make, use, sell, offer for sale, have made, import, and otherwise transfer
either its Contributions or its Contributor Version}\citeMPL{§2.1,
esp. §2.1.b} Second, it hihlights some \enquote{limitations.}\citeMPL{§2.3}
And finally, the MPL introduces a revoking clause which signifies that the
rights, granted to the licensee \enquote{[\ldots] by any and all Contributors
[\ldots] shall terminate} if the licensee \enquote{initiates litigation
against any entity by asserting a patent infringement claim [\ldots] alleging
that a Contributor Version directly or indirectly infringes any patent
[\ldots]}\citeMPL{§5.2}

Thus, the MPL is a granting license and a revoking license.

\subsection{MS-PL statements concerning patents}
\patentlabel{MSPL}

First, the MS-PL contains a statement, by which \enquote{[\ldots] each 
contributor grants (the software users) a non-exclusive, worldwide, royalty-free 
license under its licensed patents to make, have made, use, sell, offer for 
sale, import, and/or otherwise dispose of its contribution in the software or 
derivative works of the contribution in the software.}\citeMSPL{§2.B} Second,
the MS-PL says that \enquote{if you bring a patent claim against any
contributor[\ldots] your patent license from such contributor to the software
ends automatically.}\citeMSPL{§3.B} 

Thus, the MS-PL is a granting and a revoking license: At first you are granted
to use all patents of all contributors which are necessary to use the software
legally. But if you install any litigation concerning an infringement of
patents with respect to the software, then the rights granted to you are revoked. 

% \bibliography{../../../bibfiles/oscResourcesEn}

% Local Variables:
% mode: latex
% fill-column: 80
% End:

% Telekom osCompendium 'for being included' snippet template
%
% (c) Karsten Reincke, Deutsche Telekom AG, Darmstadt 2011
%
% This LaTeX-File is licensed under the Creative Commons Attribution-ShareAlike
% 3.0 Germany License (http://creativecommons.org/licenses/by-sa/3.0/de/): Feel
% free 'to share (to copy, distribute and transmit)' or 'to remix (to adapt)'
% it, if you '... distribute the resulting work under the same or similar
% license to this one' and if you respect how 'you must attribute the work in
% the manner specified by the author ...':
%
% In an internet based reuse please link the reused parts to www.telekom.com and
% mention the original authors and Deutsche Telekom AG in a suitable manner. In
% a paper-like reuse please insert a short hint to www.telekom.com and to the
% original authors and Deutsche Telekom AG into your preface. For normal
% quotations please use the scientific standard to cite.
%
% [ Framework derived from 'mind your Scholar Research Framework' 
%   mycsrf (c) K. Reincke 2012 CC BY 3.0  http://mycsrf.fodina.de/ ]
%


%% use all entries of the bibliography
%\nocite{*}

\section{Excursion: What is a 'Derivative Work' - the basic idea of open source [tbd]}
\footnotesize
\begin{quote}\itshape
We will shortly discuss existing attempts to define the derivated works of
technical aspects, like dynamical or statical linking or not. We will
prove that linking can not deliver a definite criteria: 1) modules are only
unzipped libraries. 2) you can distribute software as modules added by a script,
which statically(sic!) links all modules before executing the program. 3) The
criteria of pipe-communication is good, but not sufficient. 4) All these
attempts do not match the constituting features of script languages. Therefore we
will follow Moglen(?) and will argue from the viewpoint of a developer: it is
only a question of a function, method or anything else which calls (jumps into)
a piece of code which has been licensed by a license protecting
on-top-developments and you have a derivated work.
\end{quote}
\normalsize
\ldots


%\bibliography{../../../bibfiles/oscResourcesEn}

\input{snippets/en/03C-osImportantMinorPoints/0303-licenseCompatibilityInc}
\input{snippets/en/03C-osImportantMinorPoints/0304-osAndMoneyInc}


%%%%%%%%%%%%%%%
% Telekom osCompendium 'for being included' snippet template
%
% (c) Karsten Reincke, Deutsche Telekom AG, Darmstadt 2011
%
% This LaTeX-File is licensed under the Creative Commons Attribution-ShareAlike
% 3.0 Germany License (http://creativecommons.org/licenses/by-sa/3.0/de/): Feel
% free 'to share (to copy, distribute and transmit)' or 'to remix (to adapt)'
% it, if you '... distribute the resulting work under the same or similar
% license to this one' and if you respect how 'you must attribute the work in
% the manner specified by the author ...':
%
% In an internet based reuse please link the reused parts to www.telekom.com and
% mention the original authors and Deutsche Telekom AG in a suitable manner. In
% a paper-like reuse please insert a short hint to www.telekom.com and to the
% original authors and Deutsche Telekom AG into your preface. For normal
% quotations please use the scientific standard to cite.
%
% [ Framework derived from 'mind your Scholar Research Framework' 
%   mycsrf (c) K. Reincke 2012 CC BY 3.0  http://mycsrf.fodina.de/ ]
%


%% use all entries of the bibliography
%\nocite{*}

\chapter{Open Source Use Cases: Concept and Taxonomy}\label{sec:OSUCdeduction}

\footnotesize \begin{quote}\itshape This chapter establishes our concept of
\emph{open source use cases} as a classification system for to-do lists. The
conditions of a specific license, in the context of a par\-ti\-cu\-lar
\emph{open source use case}, shall be satisfiable by following the corresponding
to-do list. Additionally this chapter introduces a taxonomy for these \emph{open
source use cases}. Later on, this taxonomy will organize the \emph{Open Source
Use Case Finder}.
\end{quote}
\normalsize{}

After all these introductory remarks, we can summarize our idea. We know that
the right to use open source software depends on the tasks required by the open
source licenses. As opposed to commercial licenses, you can not buy the right to
use a piece of open source software by paying money. It is embedded into the
\emph{Open Source Definition} that the right to use the software may not be
sold. The OSD states firstly that an open source license may \enquote{[\ldots]
not restrict any party from selling or giving away the software as a component
of (any) aggregate software distribution}, and adds secondly in the same context
that an open source license \enquote{[\ldots] shall not require a royalty or
other fee for such sale}\footcite[cf.][\nopage wp. §1]{OSI2012a}.

However, it would be wrong to conclude that you are automatically allowed to use
open source software without any service in return: generally you have to do
something to gain the right to use the software. In other words: open source
software is covered by the idea of ’paying by doing’. Accordingly, open source
li\-cen\-ses describe specific circumstances under which the user must execute
some tasks in order to be compliant with the licenses. So, if we want to offer
to-do lists for fulfilling license conditions, we must consider these tasks and
circumstances.

In practice, such circumstances are not linear and simple. They contain
combinations of (sometimes context sensitive) conditions which can be grouped
into classes of tokens. Such a class of tokens might denote a feature of the
software itself -- such as being an application or a library. Or it can refer to
the circumstances of using the software, such as 'using the software only for
yourself' or 'distributing the software also to third parties'.

At the end, we want to determine a set of specific OSUCs -- the \emph{open source
use cases}. And we want to deliver for each of these OSUCs and for each of the
considered open source licenses one list of actions which fulfills the license
in that context\footnote{Fortunately, sometimes one task list fulfills the
conditions of more than one use case -- a welcome reduction of complexity}.

Such an \emph{open source use case} shall be considered as a set of tokens
describing the circumstances of a specific usage. Hence, to begin, we must
specify the relevant classes of tokens, before we can determine the valid
combinations of these tokens -- our \emph{open source use cases}. Finally, based
on the tokens, we generate a taxonomy in the form of a tree. This tree will
become the base of the \emph{Open Source Use Case Finder} which will be offered
in the next chapter, and which leads you to your specific OSUC by evaluating
just a few questions and answers.

There are only a handful of tokens which are relevant to the circumstances of
open source software licenses:

\label{OsucTokens}
\begin{itemize}
  \item The \textbf{\underline{type} of the open source software}: On the one
  hand, we regard code snippets, modules, libraries and plugins, and on the
  other hand, autonomous applications, programs and servers. We will take the
  word ’snimolis’ for the first set, and ’proapses’ for the second. This is
  necessary, as we are not only talking about libraries and applications in the
  everyday sense, but rather in the broadest sense\footnote{Of course, our newly
  introduced concepts of 'snimoli' and 'proapse' are not absolutely one of the
  most elegant words. So, initially we tried to talk about 'applications' and
  'libraries', although in our context these words should denote more, than they
  traditionally do. But we couldn't minimize the irritations of our
  interlocutors. Too often we had to remind them that we were not talking about
  applications and libraries in the strict sense of the words. Finally we
  decided to find our own words -- and to stay open for better proposals ;-) }.
  More specifically, we will ask you, whether the open source software you want
  to use, is an includable code snippet, a linkable module or library, or a
  loadable plugin, or whether it is an autonomous application or server which
  can be executed or processed. In the first case, the answer should be 'it is a
  \underline{snimoli}', in the second 'it is a \underline{proapse}'.

  \item The \textbf{\underline{state} of the open source software}: It might be
  used exactly as one has received it. Or it can be modified, before being used.
  More specifically, we will ask you, whether you want to leave the open source
  software as you have received it, or whether you want to modify it before
  using and/or distributing it to 3rd parties. In the first case, the answer
  should be '\underline{unmodified}', in the second '\underline{modified}'.
  
  \item The \textbf{usage \underline{context} of the open source
  software}: On the one hand you might use the received open source software as a
  readily prepared application. On the other hand you might embed the received
  open source into a larger application as one of its components. More
  specifically, we will ask you, whether you are using the open source
  software as an autonomous piece of software, or whether you are using it as an
  embedded part of a larger, more complex piece of software. In the first case,
  the answer should be '\underline{independent}', in the second
  '\underline{embedded}'.
  
  \item The \textbf{\underline{recipient} of the open source software}:
  Sometimes you might wish to use the received open source software only for
  yourself. In other cases you might intend to hand over the software (also) to
  other people. More specifically, we will ask you, whether you are going to use
  the open source software only for yourself, or whether you plan to
  (re)distribute it (also) to third parties. In the first case, the answer
  should be '\underline{4yourself}', in the second '\underline{4others}'.
 
  \item The \textbf{\underline{mode} of combination}: In this case, we will ask
  you, whether you are going to combine or to embed the open source software
  with other software components by linking them statically or dynamically, or
  by textually including (parts of) the open source software into your larger
  product. In the first case, the answer should be '\underline{statically
  linked}', in the second '\underline{dynamically linked}', in the third
  '\underline{textually included}'
  
\end{itemize}

From a more programmatic point-of-view, we can summarize these tokens as
follows:

\begin{itemize}
  \item \texttt{type::snimoli} \emph{or} \texttt{type::proapse}
  \item \texttt{state::unmodified} \emph{or} \texttt{state::modified}
  \item \texttt{context::independent} \emph{or} \texttt{context::embedded}
  \item \texttt{recipient::4yourself} \emph{or} \texttt{recipient::4others}
  \item \texttt{mode::statically-linked} \emph{or} \texttt{mode::dynamically-linked}
   \emph{or} \\ \texttt{mode::textually-included}
\end{itemize}

We already defined an open source use case as a combination of these tokens. If
we simply combine all these tokens of all these classes with all the tokens of
the other classes\footnote{in the sense of the cross product TYPE $\times$ STATE
$\times$ CONTEXT $\times$ RECIPIENT $\times$ MODE}, we get 2*2*2*2*3 = 48 sets
of tokens -- or 48 \emph{open source use cases}. Fortunately, some of the
generated sets are invalid from an empirical or logical view, and some of these
sets are context sensitive:
\label{InvalidFinderTokenCombinations}

\begin{enumerate}
  
  \item It would be unreasonable to ask you whether you are going to combine the
  received software with other software components by linking them statically or
  dynamically, or by including it textually into a larger unit, if you already
  have answered that the received open source software is a \emph{proapse} or
  that it shall be used \emph{independently}: A readily prepared application or
  server can't be linked to another application or server which also contains a
  \texttt{main}-function. And using a \emph{proapse} or \emph{snimoli}
  \emph{independently} implies that it is not used \emph{in combination} with
  other units.
  
  \item If you already have specified that the used open source software is a
  \emph{proapse} -- an autonomous program, an application, or a server -- then
  your answer implies that the software is used independently and is not
  embedded with other components into a larger unit. But if you have specified
  that the used open source software is a \emph{snimoli} -- a snippet of
  code, a module, a plugin, or a library -- then it can indeed be used as an
  embedded component of a constructed larger application or server, or it can be
  used independently in case you 'only' re-distribute it to 3rd. parties.
  
  \item If you already have specified that the used open source software is a
  \emph{snimoli} -- a snippet of code, a module, a plugin, or a library -- and
  that this \emph{snimoli} shall be used only by yourself (not distributed to
  other 3rd.\ parties) then your answer must also imply that this \emph{snimoli}
  is used in combination, as an embedded part of a larger unit. A library can
  not be used autonomously, without using it as a component of another
  application. In this case, it would simply sit on the disk and would do
  nothing more than occupying space.

\end{enumerate}

Does this sound complex? We thought so, too. We spent much time explaining these
constraints to ourselves, and only when we had transposed all the combinations
and rules into a tree, the situation became clearer. The following diagrams
shall summarize this way of clarification:

\section{Overview of the OSUC classes and tokens}


\begin{footnotesize}
\begin{minipage}{\textwidth}
\pstree[treemode=R, levelsep=*0.2, treesep=0.6]{\Toval{tokens}}{ 
    \pstree[]{\Tr{\fbox{type?}}}{
      \Toval{proapse} 
      \Toval{snimoli}
    }
    \pstree[]{\Tr{\fbox{state?}}}{
      \Toval{unmodified} 
      \Toval{modified}
    }
    \pstree[]{\Tr{\fbox{context?}}}{
      \Toval{independent} 
      \pstree[]{\Toval{embedded}}{
        \pstree[]{\Tr{\fbox{mode?}}}{
          \Toval{statically linked}
          \Toval{dynamically linked}
          \Toval{textually included}
        }
      }
    }
    \pstree[]{\Tr{\fbox{recipient?}}}{
      \Toval{4yourself}
      \pstree[]{\Toval{4others}}{
      \pstree[]{\Tr{\fbox{\tiny \textit{form\footnote{For differentiating
      between distributing sources and binaries $\rightarrow$ OSLiC, p.\
      \pageref{sec:SourceBinaryDifference}}}?}}}{
          \Toval{\tiny \textit{sources}}
          \Toval{\tiny \textit{binaries}}
        }
      }
    }
  }
\end{minipage}
\end{footnotesize}  
  
\section{The OSUC taxonomy}

This is one of the possible trees 'collecting' the tokens and offering the
\emph{open source use cases} as their leafs\footnote{ Each of the invalid use
cases (= sets of tokens) [for details s. p.\
\pageref{InvalidFinderTokenCombinations}] is marked by an \lightning{} and leads
to an empty set (= $\varnothing$): A proapse can not be embedded with another
software unit, also containing a main-function. Using a software library only
for yourself and independent (not in combination with larger software unit), is
like having an unused heap of bytes on your disc.}:

\label{OsucDefinitionTree}
\begin{tiny}
\pstree[treemode=R,levelsep=*0.2, treesep=0.2]{\Toval{OSS}}{ 
    \pstree[]{\Tr{\fbox{type?}}}{
      \pstree[]{\Tr{\parbox{4em}{\texttt{\{\underline{proapse}\}}}}}{
        \pstree[]{\Tr{\fbox{state?}}}{
          \pstree[]{\Tr{\parbox{5.2em}{
                             \texttt{\{proapse,\\
                             \hspace*{0.5em}\underline{unmodified}\}}}}}{
            \pstree[]{\Tr{\fbox{context?}}}{
              \pstree[]{\Tr{\parbox{5.2em}{
                             \texttt{\{proapse,\\
                             \hspace*{0.5em}unmodified,\\
                             \hspace*{0.5em}\underline{independent}\}}}}}{
                \pstree[]{\Tr{\fbox{recipient?}}}{
                
                  \pstree[]{\Toval{\bfseries{OSUC-01}}}{
              
                    \Tr{\parbox{5.2em}{
                             \texttt{\{proapse,\\
                             \hspace*{0.5em}unmodified,\\
                             \hspace*{0.5em}independent,\\
                             \hspace*{0.5em}\underline{4yourself}\}}}}
                   }                            
                  \pstree[]{\Toval{\bfseries{OSUC-02}}}{
                             
                    \Tr{\parbox{5.2em}{
                             \texttt{\{proapse,\\
                             \hspace*{0.5em}unmodified,\\
                             \hspace*{0.5em}independent,\\
                             \hspace*{0.5em}\underline{4others}\}}}}
                    }         
                }
              }
              \pstree[]{\Tr{\parbox{5.4em}{
                             \texttt{\{proapse,\\
                             \hspace*{0.5em}unmodified,\\
                             \hspace*{0.5em}\underline{embedded}
                             \bfseries{\lightning}\}}}}}{
                \Tr{$\varnothing$}
              }              
            }
          }
          \pstree[]{\Tr{\parbox{5.2em}{
                             \texttt{\{proapse,\\
                             \hspace*{0.5em}\underline{modified}\}}}}}{
            \pstree[]{\Tr{\fbox{context?}}}{
              \pstree[]{\Tr{\parbox{5.2em}{
                             \texttt{\{proapse,\\
                             \hspace*{0.5em}modified,\\
                             \hspace*{0.5em}\underline{independent}\}}}}}{
                \pstree[]{\Tr{\fbox{recipient?}}}{
                  \pstree[]{\Toval{\bfseries{OSUC-03}}}{
                    \Tr{\parbox{5.2em}{
                             \texttt{\{proapse,\\
                             \hspace*{0.5em}modified,\\
                             \hspace*{0.5em}independent,\\
                             \hspace*{0.5em}\underline{4yourself}\}}}}
                   }
                   \pstree[]{\Toval{\bfseries{OSUC-04}}}{         
                    \Tr{\parbox{5.2em}{
                             \texttt{\{proapse,\\
                             \hspace*{0.5em}modified,\\
                             \hspace*{0.5em}independent,\\
                             \hspace*{0.5em}\underline{4others}\}}}}
                   }         
                }
              }
              \pstree[]{\Tr{\parbox{5.4em}{
                             \texttt{\{proapse,\\
                             \hspace*{0.5em}modified,\\
                             \hspace*{0.5em}\underline{embedded}
                             \bfseries{\lightning}\}}}}}{                          
                \Tr{$\varnothing$}
              }              
            }
          }
        }
      } 
      \pstree[]{\Tr{\parbox{4em}{\texttt{\{\underline{snimoli}\}}}}}{
        \pstree[]{\Tr{\fbox{state?}}}{
          \pstree[]{\Tr{\parbox{5.2em}{
                             \texttt{\{snimoli,\\
                             \hspace*{0.5em}\underline{unmodified}\}}}}}{
            \pstree[]{\Tr{\fbox{context?}}}{
              \pstree[]{\Tr{\parbox{5.2em}{
                             \texttt{\{snimoli,\\
                             \hspace*{0.5em}unmodified,\\
                             \hspace*{0.5em}\underline{independent}\}}}}}{
                \pstree[]{\Tr{\fbox{recipient?}}}{
                  \pstree[]{
                    \Tr{\parbox{5.8em}{
                             \texttt{\{snimoli,\\
                             \hspace*{0.5em}unmodified,\\
                             \hspace*{0.5em}independent,\\
                             \hspace*{0.5em}\underline{4yourself}
                             \bfseries{\lightning}\}}}}
                     }{\Tr{$\varnothing$}}        
                             
                             
                   \pstree[]{\Toval{\bfseries{OSUC-05}}}{        
                    \Tr{\parbox{5.2em}{
                             \texttt{\{snimoli,\\
                             \hspace*{0.5em}unmodified,\\
                             \hspace*{0.5em}independent,\\
                             \hspace*{0.5em}\underline{4others}\}}}}
                    }         
                }
              }
              
              
              \pstree[]{\Tr{\parbox{5.2em}{
                             \texttt{\{snimoli,\\
                             \hspace*{0.5em}unmodified,\\
                             \hspace*{0.5em}\underline{embedded}\}}}}}{
                \pstree[]{\Tr{\fbox{recipient?}}}{
                  \pstree[]{\Toval{\bfseries{OSUC-06}}}{
                    \pstree[]{\Tr{\parbox{5.2em}{
                             \texttt{\{snimoli,\\
                             \hspace*{0.5em}unmodified,\\
                             \hspace*{0.5em}embedded,\\
                             \hspace*{0.5em}\underline{4yourself}\}}}}}{
                               \pstree[]{\Tr{\fbox{$\cup$ mode?}}}{
                                 \Tr{\parbox{5.6em}{\underline{\textit{OSUC-06a}}\\
                                 \texttt{\{statically\\
                                  \hspace*{0.5em}linked\}}}}
                                \Tr{\parbox{5.6em}{\underline{\textit{OSUC-06b}}\\
                                \texttt{\{dynamically\\
                                  \hspace*{0.5em}linked\}}}}
                                \Tr{\parbox{5.6em}{\underline{\textit{OSUC-06c}}\\
                                \texttt{\{textually\\
                                  \hspace*{0.5em}included\}}}}
                               }
                             }
                    }
                    \pstree[]{\Toval{\bfseries{OSUC-07}}}{        
                      \pstree[]{\Tr{\parbox{5.2em}{
                             \texttt{\{snimoli,\\
                             \hspace*{0.5em}unmodified,\\
                             \hspace*{0.5em}embedded,\\
                             \hspace*{0.5em}\underline{4others}\}}}}}{
                               \pstree[]{\Tr{\fbox{$\cup$ mode?}}}{
                                \Tr{\parbox{5.6em}{\underline{\textit{OSUC-07a}}\\
                                \texttt{\{statically\\
                                  \hspace*{0.5em}linked\}}}}
                                \Tr{\parbox{5.26em}{\underline{\textit{OSUC-07b}}\\
                                \texttt{\{dynamically\\
                                  \hspace*{0.5em}linked\}}}}
                                \Tr{\parbox{5.6em}{\underline{\textit{OSUC-07c}}\\
                                \texttt{\{textually\\
                                  \hspace*{0.5em}included\}}}}
                               }
                             }
                       }     
                 }
              }
              
                            
            }
          }
          \pstree[]{\Tr{\parbox{5.2em}{
                             \texttt{\{snimoli,\\
                             \hspace*{0.5em}\underline{modified}\}}}}}{
            \pstree[]{\Tr{\fbox{context?}}}{
              
              \pstree[]{\Tr{\parbox{5.2em}{
                             \texttt{\{snimoli,\\
                             \hspace*{0.5em}modified,\\
                             \hspace*{0.5em}\underline{independent}\}}}}}{
                \pstree[]{\Tr{\fbox{recipient?}}}{
                  \pstree[]{
                    \Tr{\parbox{5.8em}{
                             \texttt{\{snimoli,\\
                             \hspace*{0.5em}modified,\\
                             \hspace*{0.5em}independent,\\
                             \hspace*{0.5em}\underline{4yourself}
                             \bfseries{\lightning}\}}}}
                     }{\Tr{$\varnothing$}}         
                             
                   \pstree[]{\Toval{\bfseries{OSUC-08}}}{              
                    \Tr{\parbox{5.2em}{
                             \texttt{\{snimoli,\\
                             \hspace*{0.5em}modified,\\
                             \hspace*{0.5em}independent,\\
                             \hspace*{0.5em}\underline{4others}\}}}}
                    }
                }
              }
              
              \pstree[]{\Tr{\parbox{5.2em}{
                             \texttt{\{snimoli,\\
                             \hspace*{0.5em}modified,\\
                             \hspace*{0.5em}\underline{embedded}\}}}}}{
                \pstree[]{\Tr{\fbox{recipient?}}}{
                  \pstree[]{\Toval{\bfseries{OSUC-09}}}{     
                    \pstree[]{\Tr{\parbox{5.2em}{
                             \texttt{\{snimoli,\\
                             \hspace*{0.5em}modified,\\
                             \hspace*{0.5em}embedded,\\
                             \hspace*{0.5em}\underline{4yourself}\}}}}}{
                               \pstree[]{\Tr{\fbox{$\cup$ mode?}}}{
                                 \Tr{\parbox{5.6em}{\underline{\textit{OSUC-09a}}\\
                                 \texttt{\{statically\\
                                  \hspace*{0.5em}linked\}}}}
                                \Tr{\parbox{5.6em}{\underline{\textit{OSUC-09b}}\\
                                \texttt{\{dynamically\\
                                \hspace*{0.5em}linked\}}}}
                                \Tr{\parbox{5.6em}{\underline{\textit{OSUC-09c}}\\
                                \texttt{\{textually\\
                                  \hspace*{0.5em}included\}}}}
                               }
                             }
                           }
                   \pstree[]{\Toval{\bfseries{OSUC-10}}}{           
                    \pstree[]{\Tr{\parbox{5.2em}{
                             \texttt{\{snimoli,\\
                             \hspace*{0.5em}modified,\\
                             \hspace*{0.5em}embedded,\\
                             \hspace*{0.5em}\underline{4others}\}}}}}{
                               \pstree[]{\Tr{\fbox{$\cup$ mode?}}}{
                                \Tr{\parbox{5.6em}{\underline{\textit{OSUC-10a}}\\
                                \texttt{\{statically\\
                                  \hspace*{0.5em}linked\}}}}
                                \Tr{\parbox{5.6em}{\underline{\textit{OSUC-10b}}\\
                                \texttt{\{dynamically\\
                                  \hspace*{0.5em}linked\}}}}
                                \Tr{\parbox{5.2em}{\underline{\textit{OSUC-10c}}\\
                                \texttt{\{textually\\
                                  \hspace*{0.5em}included\}}}}
                               }
                             }
                      }
                 }
              }             
            }
          }
        }
      }      
    }
  }
\end{tiny}

\section{About a suppressed degree of complexity}
\label{sec:SourceBinaryDifference}
Many licenses also draw a distinction between distributing the software as
sources and distributing them as binaries. This aspect is not covered by the
OSLiC taxonomy. Thus, for being truly complete, all the branches of the taxonomy
referring to the token \emph{4others} should have been split by tokens like
\emph{distributing sources} and \emph{distributing binaries}. But this would
have made the taxonomy unusable. So, we suppressed these criteria. But -- of
course -- we do not want to reduce the reality inadequately; we want to solve
the problem in a suitable manner. Therefore, we have implicitly split all
relevant use cases by writing two to-do lists for them, one concerning the
\emph{distribution of sources} and the other the \emph{distribution of
binaries}. Please select the relevant to-do list based on its title.

%\bibliography{../../../bibfiles/oscResourcesEn}


%%%%%%%%%%%%%%%
% Telekom osCompendium 'for being included' snippet template
%
% (c) Karsten Reincke, Deutsche Telekom AG, Darmstadt 2011
%
% This LaTeX-File is licensed under the Creative Commons Attribution-ShareAlike
% 3.0 Germany License (http://creativecommons.org/licenses/by-sa/3.0/de/): Feel
% free 'to share (to copy, distribute and transmit)' or 'to remix (to adapt)'
% it, if you '... distribute the resulting work under the same or similar
% license to this one' and if you respect how 'you must attribute the work in
% the manner specified by the author ...':
%
% In an internet based reuse please link the reused parts to www.telekom.com and
% mention the original authors and Deutsche Telekom AG in a suitable manner. In
% a paper-like reuse please insert a short hint to www.telekom.com and to the
% original authors and Deutsche Telekom AG into your preface. For normal
% quotations please use the scientific standard to cite.
%
% [ Framework derived from 'mind your Scholar Research Framework' 
%   mycsrf (c) K. Reincke 2012 CC BY 3.0  http://mycsrf.fodina.de/ ]
%


%% use all entries of the bibliography
%\nocite{*}

\chapter{Open Source Use Cases: Find the License Fulfilling To-do Lists}\label{sec:OSUCfinder}

\footnotesize
\begin{quote}\itshape
This chapter offers the \emph{Open Source Use Case Finder}: Based on the
information gathered by a form, it allows to traverse a tree whose leaves are
linked to the \emph{open source use cases} which finally refer to the respective
to-do lists.
\end{quote}
\normalsize{}

\section{A standard form for gathering the relevant information}
\label{OSLiCStandardFormForGatheringInformation}
 
{
% The 8.5cm below were found by trial and error - 8.6cm will generate an
% overfull hbox. 
\newcommand{\question[1]}{\parbox[c][#1][c]{8.5cm}}
\newcommand{\checkboxes}[2]{%
    \parbox{7.5em}{
        $\square$\hspace{1em}#1\\
        $\square$\hspace{1em}#2}}

\begin{small}
\begin{tabular}[h]{|l|l|l|l|}
\hline 
  \ & \textit{Which open source software do you want to use?} & \ \\
\hline 
  \ & \textit{Under which open source license is it released?} & \ \\
\hline
\hline 
\textbf{Focus} & \textbf{Questions} & \textbf{Answers}\\
\hline 
\hline 
  Type
  & \question[2.9cm]{
    \textit{Is the open source software you want to use a library in the
    broadest sense (an includable code \textbf{\underline{sni}}ppet, a linkable
    \textbf{\underline{mo}}dule or \textbf{\underline{li}}brary, or a loadable
    plugin), or is it an autonomous \textbf{\underline{pro}}gram,
    \textbf{\underline{ap}}plication, or \textbf{\underline{se}}rver which can be
    executed?}} 
  & \checkboxes{snimoli}{proapse}
  \\
\hline 
  State 
  & \question[1.7cm]{
    \textit{Do you want to leave the open source software
    \textbf{\underline{unmodified}} as you have received it, or are you going to
    create a \textbf{\underline{modified}} version of it?}} 
  & \checkboxes{unmodified}{modified}
  \\
\hline 
  Context 
  & \question[2.15cm]{ 
    \textit{Are you going to use / distribute the open source software as an
    \textbf{\underline{independent}} unit, or do you plan to integrate it as an
    \textbf{\underline{embedded}} component into a complexer piece of software?}}
  & \checkboxes{independent}{embedded}
  \\
\hline 
  Recipient 
  & \question[1.7cm]{ 
    \textit{Are you going to use the open source
    software only \textbf{\underline{for}} \textbf{\underline{yourself}}, or do
    you plan to (re)distribute it (also) \textbf{\underline{to}}
    \textbf{\underline{other}} third parties?}}
  & \checkboxes{4yourself}{2others}
  \\
\hline 
\hline
  Form 
  & \question[1.7cm]{
    \textit{Given you want to (re)distribute an open source based work [2others],
    do you focus on distributing the \textbf{\underline{binaries}} or the
    \textbf{\underline{sources}}?}}
  & \checkboxes{binaries}{sources}
  \\
\hline
  IoAccess 
  & \question[2.6cm]{
    \textit{Given you are using open source software [4yourself] by executing a
    modified os program [modified] or by creating \& executing a program using
    an os library [embedded], does this program distribute its IO data
    \textbf{\underline{only locally}} or \textbf{\underline{via internet}}?}} &
    \checkboxes{onlyLocally}{viaInternet}
  \\
\hline 
\hline
\end{tabular}
\end{small}
}

As discussed earlier, there are of course some invalid or irrelevant
combinations.\footnote{type::proapse excludes state::embedded;
recipient::4yourself excludes the combination with state::independent and
type::snimoli; any value of class 'mode' implies state::embedded; form is only
relevant if recipient::2others; ioAccess is only relevant if
recipient::4yourself[for details see page
\pageref{InvalidFinderTokenCombinations}]. If you have encountered one of these
invalid combinations, please check the corresponding explanations.} Here are
some extra explanations concerning the classes resp. the focuses:

\begin{description}
\item[Type:] A piece of (open source) software is a program, an application, or
a server, only if you can start its binary form with your normal program
launcher, or (in case of a text file which still must be interpreted by an
interpreter like php, perl, bash etc.) if you can start an interpreter which
takes the file as one of its arguments and executes the commands.
\item[State:] You are modifying a piece of (open source) software if you expand,
reduce or modify at least one of the received software files, and---in case of
dealing with binary object code---if you (re)compile and (re)link the modified
software to a new binary file. But if you only modify some of the configuration
files, you are not modifying the open source software itself.
\item[Context:] You are using a piece of open source software as an embedded
component of a larger unit \ldots
  \begin{itemize}
  \item  if one of your files of the larger unit contains a verbatim or a
  modified copy (i.e.\ a snippet) of the received open source software, or
  \item if your larger unit contains an include statement referring to a
  functionally defining file of the received open source software, or
  \item if your larger unit calls a function defined in the received open source
  software, or
  \item if your development environment contains a compiler or linker directive
  referring to the received open source software (binaries) and if your larger
  unit can't be executed without resolving this linker directive.
  \end{itemize}
\item[Recipient:] You are using the received open source software only for
yourself, if you as a person do not pass it to other entities like persons,
organizations, companies etc., or if you---as a member of a specific
development group---pass it only to the other members of your development
group. But if you store open source software on any device such as a mobile
phone, an USB stick, etc.\ or if you attach it to any transport medium like
email etc.\ and if you then sell, give away, or simply send this device or
transport medium to anyone (other than a direct member of your development
group) then you indeed hand the open source software over to third
parties.\footnote{Please remember that---at least in Germany---there are
opinions that even handing over software to another legal entity or department
of the same company is also a kind of distribution. It is always safest to take
the broadest possible meaning.}
\item[Form:] Open source software knows two ways to distribute the software: in
the form of binaries and in the form of sources. Mostly it is up to you to
decide whether you want to distribute only the binaries or whether you are
intentionally going to distribute the sources (too). At a first glance, the
concepts 'sources' and 'binaries' seems to be clearly distinguished.
On the one hand, compiled sources should be taken as binaries. On the other
hand, editable pieces of software are denoted by the concept 'sources'. But
sometimes the difference is not as clear as wished: For example, you can modify
even already compiled object files by using an hex-editor. Or it is very
difficult to modify the minimized versions of javascript files even if they are
indeed text files. Therefore, the OSLiC 'reuses' a famous \textbf{rule of
thumb}: \enquote{The source code for a work means the preferred form of the work
for making modifications to it}.\citeGPLtwo{§3} All other forms are denoted by
the concept of 'binaries'. Based on this specification, you can respect some
special conditions if you want to distribute the sources and/or the binaries.
\item[ioAccess:] If you execute an open source program or an own program using
an open source library, then (normally) you do not distribute that software.
Under these circumstances, the most open source licenses do not require anything
for executing the program compliantly - even if it is the base of a globally
used internet service. For closing this 'gap', the AGPL has been invented: Like
the GPL, the AGPL let the obligation to fulfill the well known set of GPL tasks
be triggered by distributing the software. But, it let these tasks also be
triggered by an established remote network interaction: whoever interacts with
the locally executed program remotely through a computer network gets all the
rights which normally the receiver of a distribution gets. Nevertheless, the
AGPL does not wish to cause an overhead of tasks: Only \emph{locally excuted
open source programs which have been modfied} or \emph{locally executed own
programs using an AGPL licensed library} shall indeed trigger the fulfillment of
the requirements. Thus, we introduced the features \emph{ioAccess:onlyLocally}
and \emph{ioAccess:viaInternet}: They are only relevant if you uses a program
only for yourself (4yourself) \textbf{and} [ (if that AGPL licensed program has
been modified \{proapse and modified\}) \textbf{or} (if that program uses an
embedded AGPL licensed library \{snimoli and embedded\}) ].

\end{description}

\section{The taxonomic Open Source Use Case Finder}

Now, after having gathered the necessary information, determine your 
open source use case by traversing the following tree and its corresponding
branches:

{
\newcommand{\choicetext}[2]{\tiny #1:\\ \textbf{\textit{#2}}}

\newcommand{\cunmodified}{\choicetext{state}{unmodified}}
\newcommand{\cmodified}{\choicetext{state}{modified}}
\newcommand{\cindependent}{\choicetext{context}{independent}}
\newcommand{\cembedded}{\choicetext{context}{embedded}}
\newcommand{\cyourself}{\choicetext{recipient}{4yourself}}
\newcommand{\cothers}{\choicetext{recipient}{2others}}
\newcommand{\csources}{\choicetext{form}{sources}}
\newcommand{\cbinaries}{\choicetext{form}{binaries}}
\newcommand{\conlylocal}{\choicetext{ioAccess}{onlyLocal}}
\newcommand{\cviainternet}{\choicetext{ioAccess}{viaInternet}}


\newcommand{\osuctxtshort}[1]{$\Rightarrow$ OSUC-#1: \textit{p.\ \pageref{OSUC-#1-DEF}}}
\newcommand{\osuctxtbreak}[1]{$\Rightarrow$ OSUC-#1\\ \textit{(see p.\ \pageref{OSUC-#1-DEF})}}
\newcommand{\osucchild}[1]{child { node[anchor=west] {#1} edge from parent[draw=none] }}

\tikzset{choice/.style={rectangle, draw, rounded corners}}

\begin{tikzpicture}[
    font=\scriptsize,
    align=left,
    grow'=right,
    level 1/.style={sibling distance=27em, level distance=18mm},
    level 2/.style={sibling distance=15em, level distance=18mm},
    level 3/.style={sibling distance=8em, level distance=24mm},
    level 4/.style={sibling distance=5em, level distance=24mm},
    level 5/.style={sibling distance=2.5em, level distance=24mm},
    level 6/.style={sibling distance=1em, level distance=18mm},
%     level 3/.style={sibling distance=7em,  level distance=18mm, anchor=west, minimum width=2cm},
%     level 4/.style={sibling distance=3em,  level distance=18mm, minimum width=1.65cm},
%     level 5/.style={sibling distance=3em,  level distance=18mm, minimum width=1.45cm},
%     level 6/.style={sibling distance=3em,  level distance=6mm},
]
\node [ellipse,draw] {OSS}
    child { node [choice] { \choicetext{type}{proapse} }
      child { node [choice] {\cunmodified}
        child { node [choice] {\cindependent}
          child { node[choice] {\cyourself} 
            \osucchild{\osuctxtshort{01}}
          }
          child { node[choice] {\cothers} 
            child { node[choice] {\csources} 
              \osucchild{\osuctxtbreak{02S}}
            }
            child { node[choice] {\cbinaries} 
             \osucchild{\osuctxtbreak{02B}}
            }
          }
        }
      }
      child { node [choice] {\cmodified}
        child { node [choice] {\cindependent}
          child { node[choice] {\cyourself} 
            child { node[choice] {\conlylocal} 
              \osucchild{\osuctxtbreak{03L}}
            }
            child { node[choice] {\cviainternet} 
              \osucchild{\osuctxtbreak{03N}}
            }
          }
          child { node[choice] {\cothers} 
            child { node[choice] {\csources} 
              \osucchild{\osuctxtbreak{04S}}
            }
            child { node[choice] {\cbinaries} 
              \osucchild{\osuctxtbreak{04B}}
            }
          }
        }
      }
    }
    child { node [choice] { \choicetext{type}{snimoli} }
      child { node [choice] {\cunmodified}
        child { node [choice] {\cindependent} 
          child { node[choice] {\cothers} 
            child { node[choice] {\csources} 
              \osucchild{\osuctxtbreak{05S}}
            }
            child { node[choice] {\cbinaries} 
              \osucchild{\osuctxtbreak{05B}}
            }
          }
        }
        child { node [choice] {\cembedded} 
          child { node[choice] {\cyourself} 
            child { node[choice] {\conlylocal} 
              \osucchild{\osuctxtbreak{06L}}
            }
            child { node[choice] {\cviainternet} 
              \osucchild{\osuctxtbreak{06N}}
            }
          }
          child { node[choice] {\cothers} 
            child { node[choice] {\csources} 
              \osucchild{\osuctxtbreak{07S}}
            }
            child { node[choice] {\cbinaries} 
              \osucchild{\osuctxtbreak{07B}}
            }
          }
        }
      }
      child { node [choice] {\cmodified}
        child { node [choice] {\cindependent} 
          child { node[choice] {\cothers} 
            child { node[choice] {\csources} 
              \osucchild{\osuctxtbreak{08S}}
            }
            child { node[choice] {\cbinaries} 
              \osucchild{\osuctxtbreak{08B}}
            }
          }
        }
        child { node [choice] {\cembedded} 
          child { node[choice] {\cyourself} 
            child { node[choice] {\conlylocal} 
              \osucchild{\osuctxtbreak{09L}}
            }
            child { node[choice] {\cviainternet} 
              \osucchild{\osuctxtbreak{09N}}
            }
          }
          child { node[choice] {\cothers} 
            child { node[choice] {\csources} 
              \osucchild{\osuctxtbreak{10S}}
            }
            child { node[choice] {\cbinaries} 
              \osucchild{\osuctxtbreak{10B}}
            }
          }
        }
      }
    };
\end{tikzpicture}
\label{OSLiCUseCaseFinder}
}


\section{The open source use cases and its to-do list references}

On the following pages, each \textbf{O}pen \textbf{S}ource \textbf{U}se
\textbf{C}ase is textually specified one more time and complemented by a list of
page numbers. Each of these pages covers the license-specific to-do list whose
items together offer a processable way for acting according to the license under
the circumstances of the described \textbf{O}pen \textbf{S}ource \textbf{U}se
\textbf{C}ase.


\begin{osucdefinitions}
\bgroup
\newcommand{\osuclinktable}[1]{%
  To see the \textit{specific, license fulfilling to-do lists}
  jump to the following pages:
  \begin{itemize}
    \item p.\ \pageref{OSUC-#1-AGPL} for the \textbf{AGPL-3.0}
      \textit{(= GNU Affero General Public License)} 
    \item p.\ \pageref{OSUC-#1-APL} for the \textbf{Apache-2.0}
      \textit{(= Apache License)}
    \item p.\ \pageref{OSUC-#1-BSD2} for the \textbf{BSD-2-Clause} License
      \textit{(= Berkeley Software Distribution)}
    \item p.\ \pageref{OSUC-#1-BSD3} for the \textbf{BSD-3-Clause} License
      \textit{(= Berkeley Software Distribution)}
    \item p.\ \pageref{OSUC-#1-CDDL} for the \textbf{CDDL-1.0}
      \textit{(= Common Develop and Distribution License)}  
    \item p.\ \pageref{OSUC-#1-EPL} for the \textbf{EPL-1.0}
      \textit{(= Eclipse Public License)}     
    \item p.\ \pageref{OSUC-#1-EUPL} for the \textbf{EUPL-1.1}
      \textit{(= European Union Public License)} 
    \item p.\ \pageref{OSUC-#1-GPL2} for the \textbf{GPL-2.0}
       \textit{(= GNU General Public License Version 2)} 
    \item p.\ \pageref{OSUC-#1-GPL3} for the \textbf{GPL-3.0}
       \textit{(= GNU General Public License Version 3)} 
    \item p.\ \pageref{OSUC-#1-LGPL2} for the \textbf{LGPL-2.1}
      \textit{(= GNU Lesser General Public License Version 2.1)}           
    \item p.\ \pageref{OSUC-#1-LGPL3} for the \textbf{LGPL-3.0}
      \textit{(= GNU Lesser General Public License Version 3)}           
    \item p.\ \pageref{OSUC-#1-MIT} for the \textbf{MIT} License
       \textit{(= Massachusetts Institute of Technology)} 
    \item p.\ \pageref{OSUC-#1-MPL} for the \textbf{MPL}
      \textit{(= Mozilla Public License)}     
    \item p.\ \pageref{OSUC-#1-MSPL} for the \textbf{MS-PL}
      \textit{(= Microsoft Public License)} 
    \item p.\ \pageref{OSUC-#1-PGL} for the \textbf{PostgreSQL}
      \textit{(= Postgres License)} 
    \item p.\ \pageref{OSUC-#1-PHP} for the \textbf{PHP-3.0} License 
  \end{itemize}}

\newcommand{\osucitem}[3]{%
  \osucdef{#1}{#2}{#3}
  \osuclinktable{#1}}

\begin{description}
\label{OSUCList}
\osucitem{01}{proapse, unmodified, independent, 4yourself}{%
Only for yourself, you are going to use an unmodified open source program,
application, or server just as you received it. But you do not combine it with
other components in the sense of software development} 

\osucitem{02S}{proapse, unmodified, independent, 2others, sources}{%
Just as you received it, you are going to distribute an unmodified open source
program, application, or server to third parties in the form of sources. In this
act of distribution, you do not combine this program, application, or server
with other software components in the sense of software development} 

\osucitem{02B}{proapse, unmodified, independent, 2others, binaries}{%
Just as you received it, you are going to distribute an unmodified open source
program, application, or server to third parties in the form of binaries. In
this act of distribution, you do not combine this program, application, or
server with other software components in the sense of software development} 
  
% \osucitem{03}{proapse, modified, independent, 4yourself}{%
% Only for yourself, you are going to modify an open source program, application,
% or server after you received it and before you will use it. But you do not
% combine it with other components in the sense of software development} 

\osucitem{03L}{proapse, modified, independent, 4yourself, onlyLocal}{% 
You are executing an open source program, application, or server which you have
modified (but not combined with other components in the sense of software
development) and which distributes its input/output only locally to you}

\osucitem{03N}{proapse, modified, independent, 4yourself, viaInternet}{% 
You are executing an open source program, application, or server which you have
modified (but not combined with other components in the sense of software
development) and which distributes its input/output to you or other users via the
internet}

\osucitem{04S}{proapse, modified, independent, 2others, sources}{%
You are going to modify an open source program, application, or server after you
received it and  before you will distribute it to third parties in the form of
sources. But you do not combine this modified program, application, or server
with other software components in the sense of software development}
  
\osucitem{04B}{proapse, modified, independent, 2others, binaries}{%
You are going to modify an open source program, application, or server after you
received it and before you will distribute it to third parties in the form of
binaries. But you do not combine this modified program, application, or server
with other software components in the sense of software development}

\osucitem{05S}{snimoli, unmodified, independent, 2others, sources}{%
Just as you received it, you are going to distribute an unmodified open source
library, code snippet, module, or plugin to third parties in the form of
sources. In this act of distribution, you do not combine this library, code
snippet, module, or plugin with other software components in the sense of
software development} 

\osucitem{05B}{snimoli, unmodified,independent, 2others, binaries}{%
Just as you received it, you are going to distribute an unmodified open source
library, code snippet, module, or plugin to third parties in the form of
binaries. In this act of distribution, you do not combine this library, code
snippet, module, or plugin with other software components in the sense of
software development} 

% \osucitem{06}{snimoli, unmodified, embedded, 4yourself}{%
% Only for yourself and just as you received it, you are going to combine an
% unmodified open source library, code snippet, module, or plugin into a larger
% software unit as one of its parts}  

\osucitem{06L}{snimoli, umodified, embedded, 4yourself, onlyLocal}{% 
You are executing any application which distributes input/output only locally to
you and which uses an unmodified embedded open source library, code snippet,
module, or plugin}

\osucitem{06N}{snimoli, umodified, embedded, 4yourself, viaInternet}{%
You are executing any application which distributes its input/output to you or
other users via the internet and which uses an unmodified embedded open source
library, code snippet, module, or plugin}


\osucitem{07S}{snimoli, unmodified, embedded, 2others, sources}{%
Just as you received it and before you will distribute it to third parties in
the form of sources and together with a larger software unit, you are going to
combine and embed an unmodified open source library, code snippet, module, or
plugin into that larger software unit in the sense of software development}

\osucitem{07B}{snimoli, unmodified, embedded, 2others, binaries}{%
Just as you received it and before you will distribute it to third parties in
the form of binaries and together with a larger software unit, you are going to
combine and embed an unmodified open source library, code snippet, module, or
plugin into that larger software unit in the sense of software development}

\osucitem{08S}{snimoli, modified, independent, 2others, sources}{%
Before you will distribute it to third parties in the form of sources, you are
going to modify an open source library, code snippet, module, or plugin. But you
do not combine it with other software components in the sense of software
development}  

\osucitem{08B}{snimoli, modified, independent, 2others, binaries}{%
Before you will distribute it to third parties in the form of binaries, you are
going to modify an open source library, code snippet, module, or plugin. But you
do not combine it with other software components in the sense of software
development} 

% \osucitem{09}{snimoli, modified, embedded, 4yourself}{%
% Only for yourself, you are going to modify an open source library, code snippet,
% module, or plugin, and you will combine it in the sense of software development
% into a larger software unit as one of its parts} 

\osucitem{09L}{snimoli, modified, embedded, 4yourself, onlyLocal}{% 
You are executing any application which distributes input/output only locally to
you and which uses an embedded open source library, code snippet, module, or
plugin -- being modified by you}

\osucitem{09N}{snimoli, modified, embedded, 4yourself, viaInternet}{%
You are executing any application which distributes its input/output to you or
other users via the internet and which uses an embedded open source library,
code snippet, module, or plugin -- being modified by you}

\osucitem{10S}{snimoli, modified, embedded, 2others, sources}{%
Before you will distribute it to third parties in the form of sources, you are
going to modify an open source library, code snippet, module, or plugin, which
you combine with other software components in the sense of software development}

\osucitem{10B}{snimoli, modified, embedded, 2others, binaries}{%
Before you will distribute it to third parties in the form of binaries, you are
going to modify an open source library, code snippet, module, or plugin, which
you combine with other software components in the sense of software development}
\end{description}
\egroup
\end{osucdefinitions}

%\bibliography{../../../bibfiles/oscResourcesEn}

% Local Variables:
% mode: latex
% fill-column: 80
% End:



%%%%%%%%%%%%%%%
  
% Telekom osCompendium 'for being included' snippet template
%
% (c) Karsten Reincke, Deutsche Telekom AG, Darmstadt 2011
%
% This LaTeX-File is licensed under the Creative Commons Attribution-ShareAlike
% 3.0 Germany License (http://creativecommons.org/licenses/by-sa/3.0/de/): Feel
% free 'to share (to copy, distribute and transmit)' or 'to remix (to adapt)'
% it, if you '... distribute the resulting work under the same or similar
% license to this one' and if you respect how 'you must attribute the work in
% the manner specified by the author ...':
%
% In an internet based reuse please link the reused parts to www.telekom.com and
% mention the original authors and Deutsche Telekom AG in a suitable manner. In
% a paper-like reuse please insert a short hint to www.telekom.com and to the
% original authors and Deutsche Telekom AG into your preface. For normal
% quotations please use the scientific standard to cite.
%
% [ File structure derived from 'mind your Scholar Research Framework' 
%   mycsrf (c) K. Reincke CC BY 3.0  http://mycsrf.fodina.de/ ]
%

% Chapter Abstract
% ----------------

\chapter{Open Source License Compliance: To-Do Lists}

\footnotesize
\begin{quote}\itshape
With respect to the defined open source use cases, this chapter lists what one
has to do for acting in accordance with the specific open source licenses.
\end{quote}
\normalsize{}

\section{Some general remarks on 'giving' someone a file}

This chapter has to be started with some general points which are relevant for
many of the to-do lists. So that the same points are not repeated too often, we
will start with these general remarks and refer to them throughout the chapter.

\label{DistributingFilesHint}
\begin{itemize}
  \item
  Sometimes when delivering a binary package containing open source software,
  the medium doesn’t allow the recipient to view all files contained in that
  package. For example, a lot of mobile devices don’t give the user access to
  the file system. But open source licenses often require ‘to give’ someone
  copies of text files, such as the license text, copyright notes, or specific
  notice file. The safe interpretation of ‘giving someone a text’ is that the
  receiver must be able to read it\footnote{To give someone anything they can't
  touch, feel or see is like not giving him the object ;-)}. Thus, on
  systems which offer a file browser and a suitable reader, it is sufficient, to
  put these file onto the files system. On the other systems, you \emph{must}
  present the content of the files  through the UI of your application---for
  example in a specific copyright screen\footnote{Additionally, in the open
  source community, it is a good tradition, to present these reference data
  voluntarily.}. The \oslic{} does not want to refine the taxonomies down to the
  level of operating systems, so it is up to the user to keep this in mind when
  reading the to-do lists.
  
  \item Sometimes a product which uses and distributes open source software
  tries to fulfill the requirement 'to give the recipients the license etc.' by
  presenting links to general versions of these licensing files hosted somewhere
  on the internet. But be aware: Although it is a good tradition---especially
  if you link to the homepages of the projects for being totally transparent---
  it is not sufficient to offer only the links. If you are required by the open
  source licenses to handover something to your users, \emph{you} must do it. It
  is not safe to delegate the task to anyone hoping that they will offer the
  files all the time your product is being distributed\footnote{Moreover, the
  advantage of doing the job oneself is that one has not to struggle with
  uncommunicated implicit modifications of the link targets.}. Even if it would
  be safe to assume that the link will remain valid forever, the point is: you
  have to fulfill the license, no one else.
\end{itemize}

\label{OSUCToDoLists}

% ==============================================================================
% Some commands common to all to-do lists

% Common footnote, used in many task lists
\newcommand{\passingFilesCorrectly}{%
  \footnote{For implementing the handover of files correctly $\rightarrow$
    OSLiC, p. \pageref{DistributingFilesHint}}}

% A LSUC that covers many OSUCs
% #1 -> List of covered OSUCS, e. g., ``OSUC-01, OSUC-07S, and OSUC-10S''
% #2 -> number of first OSUC covered, e. g., '07B'
% #3 -> number of last OSUC covered
\newcommand{\coversOsucs}[3]{\lsuccovers{#1}%
  \footnote{For details $\rightarrow$ \oslic, pp.\ \osucpageref{#2} -- \osucpageref{#3}}}

% A LSUC that maps to exactly one OSUC
% # -> number of the OSUC, e. g., '07B'
\newcommand{\mapsToOsuc}[1]{\lsuccovers{OSUC-#1}%
  \footnote{For details $\rightarrow$ \oslic, pp.\ \osucpageref{#1}}}


% Local Variables:
% mode: latex
% fill-column: 80
% End:

\input{snippets/en/06C-osFulfillmentByToDoLists/0601-apacheFulfillToDoList}
\input{snippets/en/06C-osFulfillmentByToDoLists/0602-bsdFulfillToDoList}
\input{snippets/en/06C-osFulfillmentByToDoLists/0603-mitFulfillToDoList}
% Telekom osCompendium 'for being included' snippet template
%
% (c) Karsten Reincke, Deutsche Telekom AG, Darmstadt 2011
%
% This LaTeX-File is licensed under the Creative Commons Attribution-ShareAlike
% 3.0 Germany License (http://creativecommons.org/licenses/by-sa/3.0/de/): Feel
% free 'to share (to copy, distribute and transmit)' or 'to remix (to adapt)'
% it, if you '... distribute the resulting work under the same or similar
% license to this one' and if you respect how 'you must attribute the work in
% the manner specified by the author ...':
%
% In an internet based reuse please link the reused parts to www.telekom.com and
% mention the original authors and Deutsche Telekom AG in a suitable manner. In
% a paper-like reuse please insert a short hint to www.telekom.com and to the
% original authors and Deutsche Telekom AG into your preface. For normal
% quotations please use the scientific standard to cite.
%
% [ Framework derived from 'mind your Scholar Research Framework' 
%   mycsrf (c) K. Reincke 2012 CC BY 3.0  http://mycsrf.fodina.de/ ]
%


%% use all entries of the bibliography
%\nocite{*}

\section{Microsoft Public License \ldots [tbd]}
\label{OSUC-01-MS-PL} \label{OSUC-03-MS-PL} 
\label{OSUC-06-MS-PL} \label{OSUC-09-MS-PL}

\label{OSUC-02-MS-PL} \label{OSUC-04-MS-PL} \label{OSUC-05-MS-PL}
\label{OSUC-07-MS-PL} \label{OSUC-08-MS-PL} \label{OSUC-10-MS-PL}

% TODO insert MS-PL specifc mini finder

\subsection{MS-PL specific use case 1}
(covers OSUC-X - OSUC-Z)
% \label{OSUC-10-MS-PL}
\begin{description}
\item[means] \ldots

\item[covers] OSUC-?? \ldots

\item[requires] the following tasks in order to fulfill the license conditions
\begin{itemize}
  \item \textbf{[mandatorily:]} Ensure \ldots
  \item \textbf{[voluntarily:]} Let \ldots
\end{itemize}

% \item[prohibits] nothing explicitly.
\item[prohibits] the following doings in order to fulfill the license conditions
\begin{itemize}
  \item \textbf{[directly:]} 
  \item \textbf{[indirectly:]}
\end{itemize}
\end{description}

\subsection{MS-PL specific use case n}
(covers OSUC-x - OSUC-z)
% \label{OSUC-10-MS-PL}
\begin{description}
\item[means] \ldots

\item[covers] OSUC-?? \ldots

\item[requires] the following tasks in order to fulfill the license conditions
\begin{itemize}
  \item \textbf{[mandatorily:]} Ensure \ldots
  \item \textbf{[voluntarily:]} Let \ldots
\end{itemize}

% \item[prohibits] nothing explicitly.
\item[prohibits] the following doings in order to fulfill the license conditions
\begin{itemize}
  \item \textbf{[directly:]} 
  \item \textbf{[indirectly:]}
\end{itemize}
\end{description}


%\bibliography{../../../bibfiles/oscResourcesEn}

\input{snippets/en/06C-osFulfillmentByToDoLists/0605-pglFulfillToDoList}
% Telekom osCompendium 'for being included' snippet template
%
% (c) Karsten Reincke, Deutsche Telekom AG, Darmstadt 2011
%
% This LaTeX-File is licensed under the Creative Commons Attribution-ShareAlike
% 3.0 Germany License (http://creativecommons.org/licenses/by-sa/3.0/de/): Feel
% free 'to share (to copy, distribute and transmit)' or 'to remix (to adapt)'
% it, if you '... distribute the resulting work under the same or similar
% license to this one' and if you respect how 'you must attribute the work in
% the manner specified by the author ...':
%
% In an internet based reuse please link the reused parts to www.telekom.com and
% mention the original authors and Deutsche Telekom AG in a suitable manner. In
% a paper-like reuse please insert a short hint to www.telekom.com and to the
% original authors and Deutsche Telekom AG into your preface. For normal
% quotations please use the scientific standard to cite.
%
% [ Framework derived from 'mind your Scholar Research Framework' 
%   mycsrf (c) K. Reincke 2012 CC BY 3.0  http://mycsrf.fodina.de/ ]
%


%% use all entries of the bibliography
%\nocite{*}

\section{PHP Licensed Software in the usage context of \ldots}
\label{OSUC-01-PHP} \label{OSUC-03-PHP} 
\label{OSUC-06-PHP} \label{OSUC-09-PHP}

\label{OSUC-02-PHP} \label{OSUC-04-PHP} \label{OSUC-05-PHP}
\label{OSUC-07-PHP} \label{OSUC-08-PHP} \label{OSUC-10-PHP}

% TODO insert php license specifc mini finder

\subsection{PHP specific use case 1}
(covers OSUC-X - OSUC-Z)
% \label{OSUC-10-PHP}
\begin{description}
\item[means] \ldots

\item[covers] OSUC-?? \ldots

\item[requires] the following tasks in order to fulfill the license conditions
\begin{itemize}
  \item \textbf{[mandatorily:]} Ensure \ldots
  \item \textbf{[voluntarily:]} Let \ldots
\end{itemize}

% \item[prohibits] nothing explicitly.
\item[prohibits] the following doings in order to fulfill the license conditions
\begin{itemize}
  \item \textbf{[directly:]} 
  \item \textbf{[indirectly:]}
\end{itemize}
\end{description}

\subsection{PHP specific use case n}
(covers OSUC-x - OSUC-z)
% \label{OSUC-10-PHP}
\begin{description}
\item[means] \ldots

\item[covers] OSUC-?? \ldots

\item[requires] the following tasks in order to fulfill the license conditions
\begin{itemize}
  \item \textbf{[mandatorily:]} Ensure \ldots
  \item \textbf{[voluntarily:]} Let \ldots
\end{itemize}

% \item[prohibits] nothing explicitly.
\item[prohibits] the following doings in order to fulfill the license conditions
\begin{itemize}
  \item \textbf{[directly:]} 
  \item \textbf{[indirectly:]}
\end{itemize}
\end{description}

%\bibliography{../../../bibfiles/oscResourcesEn}

\input{snippets/en/06C-osFulfillmentByToDoLists/0607-eplFulfillToDoList}
\input{snippets/en/06C-osFulfillmentByToDoLists/0608-euplFulfillToDoList}
% Telekom osCompendium 'for being included' snippet template
%
% (c) Karsten Reincke, Deutsche Telekom AG, Darmstadt 2011
%
% This LaTeX-File is licensed under the Creative Commons Attribution-ShareAlike
% 3.0 Germany License (http://creativecommons.org/licenses/by-sa/3.0/de/): Feel
% free 'to share (to copy, distribute and transmit)' or 'to remix (to adapt)'
% it, if you '... distribute the resulting work under the same or similar
% license to this one' and if you respect how 'you must attribute the work in
% the manner specified by the author ...':
%
% In an internet based reuse please link the reused parts to www.telekom.com and
% mention the original authors and Deutsche Telekom AG in a suitable manner. In
% a paper-like reuse please insert a short hint to www.telekom.com and to the
% original authors and Deutsche Telekom AG into your preface. For normal
% quotations please use the scientific standard to cite.
%
% [ Framework derived from 'mind your Scholar Research Framework' 
%   mycsrf (c) K. Reincke 2012 CC BY 3.0  http://mycsrf.fodina.de/ ]
%


%% use all entries of the bibliography
%\nocite{*}

\section{MPL-2.0 licensed software}

\begin{license}{MPL} % ends at end of file
\licensename{MPL-2.0}
\licensespec{Mozilla Public License 2.0}
\licenseversion{2.0}
\licenseabbrev{MPL}

The Mozilla Public License clearly distinguishes the distribution of source code
from the distribution of binaries: First, it allows the \enquote{Distribution of
Source Form}.\citeMPL{§3.1} Then, it specifies the conditions for a
\enquote{Distribution of Executable Form}.\citeMPL{§3.2} Additionally, the
MPL-2.0 contrasts the \enquote{distribution of Covered Software} with the
\enquote{distribution of a Larger Work}.\citeMPL{§3.3} So, taken as whole, the
MPL-2.0 mainly focusses on the distribution of software. Thus, for finding the
relevant executable task lists, the following MPL-2.0 specific open source use
case structure%
  \footnote{For details of the general OSUC finder $\rightarrow$ \oslic, 
    pp.\ \pageref{OsucTokens} and \pageref{OsucDefinitionTree}} 
can be used:
 
\tikzstyle{nodv} = [font=\small, ellipse, draw, fill=gray!10, 
    text width=2cm, text centered, minimum height=2em]

\tikzstyle{nods} = [font=\footnotesize, rectangle, draw, fill=gray!20, 
    text width=1.2cm, text centered, rounded corners, minimum height=3em]

\tikzstyle{nodb} = [font=\footnotesize, rectangle, draw, fill=gray!20, 
    text width=2.2cm, text centered, rounded corners, minimum height=3em]
    
\tikzstyle{leaf} = [font=\tiny, rectangle, draw, fill=gray!30, 
    text width=1.2cm, text centered, minimum height=6em]

\tikzstyle{edge} = [draw, -latex']

\begin{tikzpicture}[]

\node[nodv] (l71) at (4,10) {MPL-2.0};

\node[nodb] (l61) at (0,8.6) {\textit{recipient:} \\ \textbf{4yourself}};
\node[nodb] (l62) at (6.5,8.6) {\textit{recipient:} \\ \textbf{2others}};

\node[nodb] (l51) at (2.5,7) {\textit{state:} \\ \textbf{unmodified}};
\node[nodb] (l52) at (9.3,7) {\textit{state:} \\ \textbf{modified}};

\node[nods] (l41) at (1.8,5.4) {\textit{form:} \textbf{source}};
\node[nods] (l42) at (3.6,5.4) {\textit{form:} \textbf{binary}};
\node[nodb] (l43) at (6.5,5.4) {\textit{type:} \\ \textbf{proapse}};
\node[nodb] (l44) at (12,5.4) {\textit{type:} \\ \textbf{snimoli}};


\node[nods] (l31) at (5.4,3.8) {\textit{form:} \textbf{source}};
\node[nods] (l32) at (7.2,3.8) {\textit{form:} \textbf{binary}};
\node[nodb] (l33) at (10,3.8) {\textit{context:} \\ \textbf{independent}};
\node[nodb] (l34) at (13.5,3.8) {\textit{context:} \\ \textbf{embedded}};

\node[nods] (l21) at (9,2.2) {\textit{form:} \textbf{source}};
\node[nods] (l22) at (10.8,2.2) {\textit{form:} \textbf{binary}};
\node[nods] (l23) at (12.6,2.2) {\textit{form:} \textbf{source}};
\node[nods] (l24) at (14.4,2.2) {\textit{form:} \textbf{binary}};

\node[leaf] (l11) at (0,0) {\textbf{MPL-2.0-C1} \textit{using software only
for yourself}};

\node[leaf] (l12) at (1.8,0) { \textbf{MPL-2.0-C2} \textit{ distributing unmodified
software as sources}};

\node[leaf] (l13) at (3.6,0) { \textbf{MPL-2.0-C3}  \textit{ distributing unmodified
software as binaries}};

\node[leaf] (l14) at (5.4,0) { \textbf{MPL-2.0-C4}  \textit{ distributing modified
program as sources}};

\node[leaf] (l15) at (7.2,0) { \textbf{MPL-2.0-C5}  \textit{ distributing modified
program as binaries}};

\node[leaf] (l16) at (9,0) { \textbf{MPL-2.0-C6}  \textit{ distributing modified
library as independent sources}};

\node[leaf] (l17) at (10.8,0) { \textbf{MPL-2.0-C7} \textit{distributing modified
library as independent binaries}};

\node[leaf] (l18) at (12.6,0) { \textbf{MPL-2.0-C8}  \textit{distributing
modified library as embedded sources}};

\node[leaf] (l19) at (14.4,0) { \textbf{MPL-2.0-C9}  \textit{ distributing modified
library as embedded binaries}};


\path [edge] (l71) -- (l61);
\path [edge] (l71) -- (l62);
\path [edge] (l61) -- (l11);
\path [edge] (l62) -- (l51);
\path [edge] (l62) -- (l52);
\path [edge] (l51) -- (l41);
\path [edge] (l51) -- (l42);
\path [edge] (l52) -- (l43);
\path [edge] (l52) -- (l44);
\path [edge] (l41) -- (l12);
\path [edge] (l42) -- (l13);
\path [edge] (l43) -- (l31);
\path [edge] (l43) -- (l32);
\path [edge] (l44) -- (l33);
\path [edge] (l44) -- (l34);
\path [edge] (l31) -- (l14);
\path [edge] (l32) -- (l15);
\path [edge] (l33) -- (l21);
\path [edge] (l33) -- (l22);
\path [edge] (l34) -- (l23);
\path [edge] (l34) -- (l24);
\path [edge] (l21) -- (l16);
\path [edge] (l22) -- (l17);
\path [edge] (l23) -- (l18);
\path [edge] (l24) -- (l19);

\end{tikzpicture}

%% =============================================================================
%% Common building blocks
%%

% ------------------------------------------------------------------------------
% Ensure license elements are present

\newcommand{\keepLicenseElements}{Ensure that the licensing elements (especially
  all copyright notices, patent notices, disclaimers of warranty, or limitations
  of liability) are retained in your package in exactly the form that you have
  received.}

\newcommand{\addWhenCompiling}{If you compile the binary from the sources,
  ensure that all these licensing elements are also incorporated into the
  package.}

% ------------------------------------------------------------------------------
% Give the recipient a copy of the license

\newcommand{\giveLicenseText}{Give the recipient a copy of the MPL-2.0 license.
  If it is not already part of the software package, add it. If the licensing
  statement in the licensing file of the package does still not clearly state
  that the package is licensed under the MPL-2.0, additionally insert your own
  correct MPL-2.0 licensing file containing the sentence: 
  \emph{This Source Code Form is subject to the terms of the Mozilla Public
    License, v. 2.0. If a copy of the MPL was not distributed with this file,
    You can obtain one at http://mozilla.org/MPL/2.0/.}}

% ------------------------------------------------------------------------------
% Add license, name, and link to homepage to documentation

\newcommand{\auxAddToDoc}[1]{Let the documentation of your distribution
  and/or your additional material also reproduce the content of the existing
  \emph{copyright notice text files,} the name of #1, a link to its homepage,
  and a link to the MPL-2.0 license.}  

\newcommand{\acknowledgeMPLSoftware}{
  \auxAddToDoc{the software}}

\newcommand{\acknowlegdeEmbeddedLibrary}{
  \auxAddToDoc{the embedded MPL-2.0 licensed component}}

% ------------------------------------------------------------------------------
% Make the source code available

\newcommand{\auxMakeSourceAvailable}[1]{Make the source code of #1 accessible
  via a repository under your own control: Push the source code package into the
  repository and make it downloadable via the Internet. Do no charge any fees
  from the user for downloading the source. Ensure, that this repository is
  online for a reasonable period of time after you ceased distributing the
  software.} 

\newcommand{\makeSourceAvailable}{\auxMakeSourceAvailable{%
    the distributed software}}

\newcommand{\makeEmbeddedSourceAvailable}{\auxMakeSourceAvailable{%
    the embedded library}}

\newcommand{\describeHowToGetSource}{Insert an easy to find description into the 
  distribution package that explains how and where the code can be retrieved.}

% ------------------------------------------------------------------------------
% Ensure modifications are covered by MPL

\newcommand{\auxPlaceModificationsUnderMPL}[1]{Organize your modifications #1
  in such a way that they are covered by the existing MPL-2.0 licensing
  statements.}

\newcommand{\auxPlaceNewFilesUnderMPL}[1]{If you add new source code files#1,
  insert a header containing your copyright line and an MPL-2.0 adequate
  licensing the statement.}

\newcommand{\placeBinaryModificationsUnderMPL}{%
  \auxPlaceModificationsUnderMPL{}}

\newcommand{\placeSourceModificationsUnderMPL}{%
  \auxPlaceModificationsUnderMPL{}
  \auxPlaceNewFilesUnderMPL{}}

\newcommand{\placeEmbeddedBinaryUnderMPL}{%
  \auxPlaceModificationsUnderMPL{of the embedded library}}

\newcommand{\placeEmbeddedSourceUnderMPL}{%
  \auxPlaceModificationsUnderMPL{of the embedded library}
  \auxPlaceNewFilesUnderMPL{ to the library itself}}

% ------------------------------------------------------------------------------
% Create modification text file

\newcommand{\createChangeLog}{Create a \emph{modification text file}, if such a
  notice file still does not exist. \emph{Add} a general description of your
  modifications to the \emph{modification text file}. Incorporate the file into
  your distribution package.}

% ------------------------------------------------------------------------------
% Mark all modifications

\newcommand{\markAllModifications}{Mark all modifications of the source code
  thoroughly, preferably in the modified source itself.}

% ------------------------------------------------------------------------------
% Separate embedded library from enclosing program

\newcommand{\auxKeepSeparate}[1]{Arrange your #1 distribution so that the
  licensing elements (especially the MPL-2.0 license text and the
  \emph{licensing files}) clearly refer only to the embedded library and do not
  affect the licensing of your own overarching work. It's a good tradition to
  keep embedded components like libraries, modules, snippets, or plugins in
  separate directories, which contain also all additional licensing elements.}

\newcommand{\keepSourceSeparate}{\auxKeepSeparate{source code}}
\newcommand{\keepBinarySeparate}{\auxKeepSeparate{binary}}

% ------------------------------------------------------------------------------
% Do not modify or remove license elements

\newcommand{\dontAlterLicenseElement}{to remove or to alter any license elements
  (including copyright notices, patent notices, disclaimers of warranty, or
  limitations of liability) contained within the software package you have
  received.}

% ------------------------------------------------------------------------------
% Do not use trademarks and logos to promote your own work

\newcommand{\dontUseTrademarks}{to promote any of your services based on the
    this software by trademarks, service marks, or logos linked to this MPL-2.0
    software, except as required for reasonable and customary use in describing
    the origin of the software and reproducing the copyright notice.}

%% =============================================================================
%% Use Cases
%%

\subsection{MPL-2.0-C1: Using the software only for yourself}
\begin{lsuc}{MPL-2.0-C1}
  \linkosuc{01}
  \linkosuc{03L} 
  \linkosuc{03N} 
  \linkosuc{06L}
  \linkosuc{06N}
  \linkosuc{09L}
  \linkosuc{09N}

  \lsucmeans{that you received MPL-2.0 licensed software, that you will use it
    only for yourself, and that you do not hand it over to any third party in
    any sense.}

  \coversOsucs{OSUC-01, OSUC-03L, OSUC-03N, OSUC-06L, OSUC-06N, OSUC-09L, and
  OSUC-09N}{01}{09N}

  \begin{lsucrequiresnothing}
    \lsucitem{You are allowed to use any kind of MPL-2.0 software in any sense
      and in any context without being obliged to do anything as long as you do
      not give the software to third parties.}
  \end{lsucrequiresnothing}

  \begin{lsucprohibits}
    \lsucitem{\dontAlterLicenseElement}
    \lsucitem{\dontUseTrademarks}
  \end{lsucprohibits}
\end{lsuc}

% ------------------------------------------------------------------------------
\subsection{MPL-2.0-C2: Passing the unmodified software as source code}
\begin{lsuc}{MPL-2.0-C2}
  \linkosuc{02S} 
  \linkosuc{05S} 
  \linkosuc{07S} 

  \lsucmeans{that you received MPL-2.0 licensed software which you are now going
    to distribute to third parties in the form of unmodified source code files
    or as unmodified source code package. In this case it makes no difference if
    you distribute a program, an application, a server, a snippet, a module, a
    library, or a plugin as an independent or as an embedded unit.}

  \coversOsucs{OSUC-02S, OSUC-05S, OSUC-07S}{02S}{07S}

  \begin{lsucrequires}
    \lsucmandatory{\keepLicenseElements}
    \lsucmandatory{\giveLicenseText}\passingFilesCorrectly
    \lsucoptional{\acknowledgeMPLSoftware}
  \end{lsucrequires}

  \begin{lsucprohibits}
    \lsucitem{\dontAlterLicenseElement}
    \lsucitem{\dontUseTrademarks}
  \end{lsucprohibits}
\end{lsuc}

% ------------------------------------------------------------------------------
\subsection{MPL-2.0-C3: Passing the unmodified software as binaries} 
\begin{lsuc}{MPL-2.0-C3}
  \linkosuc{02B} 
  \linkosuc{05B} 
  \linkosuc{07B}

  \lsucmeans{that you received MPL-2.0 licensed software which you are now going
    to distribute to third parties in the form of unmodified binary files or as
    unmodified binary package. In this case it does not matter if you distribute
    a program, an application, a server, a snippet, a module, a library, or a
    plugin as an independent or an embedded unit.}

  \coversOsucs{OSUC-02B, OSUC-05B, OSUC-07B}{02B}{07B}

  \begin{lsucrequires}
    \lsucmandatory{\keepLicenseElements\ \addWhenCompiling}
    \lsucmandatory{\makeSourceAvailable}
    \lsucmandatory{\describeHowToGetSource}
  
    \lsucsourcedist{MPL-2.0-C2}
    \lsucoptional{\giveLicenseText}\passingFilesCorrectly
    \lsucoptional{\acknowledgeMPLSoftware}
  \end{lsucrequires}

  \begin{lsucprohibits}
    \lsucitem{\dontAlterLicenseElement}
    \lsucitem{\dontUseTrademarks}
  \end{lsucprohibits}
\end{lsuc}

% ------------------------------------------------------------------------------
\subsection{MPL-2.0-C4: Passing a modified program as source code}
\begin{lsuc}{MPL-2.0-C4}
  \linkosuc{04S} 

  \lsucmeans{that you received an MPL-2.0 licensed program, application, or
    server (proapse), that you modified it, and that you are now going to
    distribute this modified version to third parties in the form of source code
    files or as a source code package.}

  \mapsToOsuc{04S}

  \begin{lsucrequires}
    \lsucmandatory{\keepLicenseElements}
    \lsucmandatory{\giveLicenseText}\passingFilesCorrectly
    \lsucmandatory{\placeSourceModificationsUnderMPL}
    \lsucoptional{\createChangeLog}
    \lsucoptional{\markAllModifications}

    \lsucoptional{\acknowledgeMPLSoftware}
  \end{lsucrequires}
 
  \begin{lsucprohibits}
    \lsucitem{\dontAlterLicenseElement}
    \lsucitem{\dontUseTrademarks}
  \end{lsucprohibits}
\end{lsuc}

% ------------------------------------------------------------------------------
\subsection{MPL-2.0-C5: Passing a modified program as binary}
\begin{lsuc}{MPL-2.0-C5}
  \linkosuc{04B} 

  \lsucmeans{that you received an MPL-2.0 licensed program, application, or
    server (proapse), that you modified it, and that you are now going to
    distribute this modified version to third parties in the form of binary
    files or as a binary package.}

  \mapsToOsuc{04B}

  \begin{lsucrequires}
    \lsucmandatory{\keepLicenseElements\ \addWhenCompiling}
    \lsucmandatory{\makeSourceAvailable}
    \lsucmandatory{\describeHowToGetSource}
    \lsucsourcedist{MPL-2.0-C4}
    \lsucmandatory{\placeBinaryModificationsUnderMPL}
    \lsucoptional{\createChangeLog}
    \lsucoptional{\giveLicenseText}\passingFilesCorrectly
    \lsucoptional{\acknowledgeMPLSoftware}
  \end{lsucrequires}


  \begin{lsucprohibits}
    \lsucitem{\dontAlterLicenseElement}
    \lsucitem{\dontUseTrademarks}
  \end{lsucprohibits}
\end{lsuc}

% ------------------------------------------------------------------------------
\subsection{MPL-2.0-C6: Passing a modified library as independent source code}
\begin{lsuc}{MPL-2.0-C6}
  \linkosuc{08S}

  \lsucmeans{that you received an MPL-2.0 licensed code snippet, module,
    library, or plugin (snimoli), that you modified it, and that you are now
    going to distribute this modified version to third parties in the form of
    source code files or as a source code package, but without embedding it into
    another larger software unit.}

  \mapsToOsuc{08S}

  \begin{lsucrequires}
    \lsucmandatory{\keepLicenseElements}
    \lsucmandatory{\giveLicenseText}\passingFilesCorrectly
    \lsucmandatory{\placeSourceModificationsUnderMPL}
    \lsucoptional{\createChangeLog}
    \lsucoptional{\markAllModifications}
    \lsucoptional{\acknowledgeMPLSoftware}
  \end{lsucrequires}

  \begin{lsucprohibits}
    \lsucitem{\dontAlterLicenseElement}
    \lsucitem{\dontUseTrademarks}
  \end{lsucprohibits}
\end{lsuc}

% ------------------------------------------------------------------------------
\subsection{MPL-2.0-C7: Passing a modified library as independent binary}
\begin{lsuc}{MPL-2.0-C7}
  \linkosuc{08B}

  \lsucmeans{that you received an MPL-2.0 licensed code snippet, module,
    library, or plugin (snimoli), that you modified it, and that you are now
    going to distribute this modified version to third parties in the form of
    binary files or as a binary package but without embedding it into another
    larger software unit.}

  \mapsToOsuc{08B}

  \begin{lsucrequires}
    \lsucmandatory{\keepLicenseElements\ \addWhenCompiling}
    \lsucmandatory{\makeSourceAvailable}
    \lsucmandatory{\describeHowToGetSource}
    \lsucsourcedist{MPL-2.0-C6}
    \lsucmandatory{\placeBinaryModificationsUnderMPL}
    \lsucoptional{\createChangeLog}
    \lsucoptional{\giveLicenseText}\passingFilesCorrectly
    \lsucoptional{\acknowledgeMPLSoftware}
  \end{lsucrequires}

  \begin{lsucprohibits}
    \lsucitem{\dontAlterLicenseElement}
    \lsucitem{\dontUseTrademarks}
  \end{lsucprohibits}
\end{lsuc}

% ------------------------------------------------------------------------------
\subsection{MPL-2.0-C8: Passing a modified library as embedded source code}
\begin{lsuc}{MPL-2.0-C8}
  \linkosuc{10S}

  \lsucmeans{that you received an MPL-2.0 licensed code snippet, module,
    library, or plugin (snimoli), that you modified it, and that you are now
    going to distribute this modified version to third parties in the form of
    source code files or as a source code package together with another larger
    software unit which contains this code snippet, module, library, or plugin
    as an embedded component.}

  \mapsToOsuc{10S}

  \begin{lsucrequires}

    \lsucmandatory{\keepLicenseElements}
    \lsucmandatory{\giveLicenseText}\passingFilesCorrectly
    \lsucmandatory{\placeEmbeddedSourceUnderMPL}
    \lsucoptional{\keepSourceSeparate}
    \lsucoptional{\createChangeLog}
    \lsucoptional{\markAllModifications}
    \lsucoptional{\acknowlegdeEmbeddedLibrary}
  \end{lsucrequires}

  \begin{lsucprohibits}
    \lsucitem{\dontAlterLicenseElement}
    \lsucitem{\dontUseTrademarks}
  \end{lsucprohibits}
\end{lsuc}

% ------------------------------------------------------------------------------
\subsection{MPL-2.0-C9: Passing a modified library as embedded binary}
\begin{lsuc}{MPL-2.0-C9}
  \linkosuc{10B}

  \lsucmeans{that you received an MPL-2.0 licensed code snippet, module,
    library, or plugin (snimoli), that you modified it, and that you are now
    going to distribute this modified version to third parties in the form of
    binary files or as a binary package together with another larger software
    unit which contains this code snippet, module, library, or plugin as an
    embedded component.}

  \mapsToOsuc{10B}

  \begin{lsucrequires}
    \lsucmandatory{\keepLicenseElements \addWhenCompiling}
    \lsucmandatory{\makeEmbeddedSourceAvailable}
    \lsucmandatory{\describeHowToGetSource}
    \lsucsourcedist{MPL-2.0-C8}
    \lsucmandatory{\placeEmbeddedBinaryUnderMPL}
    \lsucoptional{\createChangeLog}
    \lsucoptional{\giveLicenseText}\passingFilesCorrectly
    \lsucoptional{\keepBinarySeparate}
    \lsucoptional{\acknowlegdeEmbeddedLibrary}
  \end{lsucrequires}

  \begin{lsucprohibits}
    \lsucitem{\dontAlterLicenseElement}
    \lsucitem{\dontUseTrademarks}
  \end{lsucprohibits}
\end{lsuc}

% ------------------------------------------------------------------------------

\subsection{Discussions and Explanations}
\label{MPLDiscussion}
The MPL-2.0 offers a section \enquote{Responsibilities} which contains nearly all
requirements.\citeMPL{§3} Only for some subordinate aspects, one has also to
reflect other paragraphs.\citeMPL{pars pro to cf.}{§3 - concerning the trademarks}
With respect to this structure, we can detect the following tasks:

\begin{itemize}

\item In a more general attitude, the MPL-2.0 states that it \enquote{[\ldots]
  does not grant any rights in the trademarks, service marks, or logos of any
  Contributor}---except as it may be necessary \enquote{to comply with} other
  requirements of the license.\citeMPL{§2.3} The \oslic{} rewrites the message
  as the interdiction to promote own services and products by and with such
  elements. 
  
\item The MPL-2.0 also generally prescribes that \enquote{you may not remove or
  alter the substance of any license notice (including copyright notices, patent
  notices, disclaimer of warranties, or limitations of liabiliy) contained
  within the Source Code Form [\ldots]}\citeMPL{§3.4} This focussing to the
  \enquote{substance of any license notice} refers to the allowance to
  \enquote{[\ldots] alter any license notices to the extent required to remedy
  known factual innacuracies}.\citeMPL{§3.4}  Following its principle to offer
  one reliable way and to ignore variants of secondary importance, the \oslic{}
  simplifies this condition to the general proscription to modify any licensing
  material for all use cases [MPL-2.0-C1 -- MPL-2.0-C9]. But for emphasizing
  that this is a job which must be activily done, the \oslic{} additionally
  rewrites this interdiction into all \emph{2others} use cases [MPL-2.0-C2 --
  MPL-2.0-C9] as the task to retain the licensing elements in the form one has
  obtained them. 
  
\item Moreover, the MPL-2.0 requires for all \enquote{distributions of [the]
  source [code] form} that all modifications of the software \enquote{[\ldots] 
  must be under the terms of (the MPL-2.0)} and that the distributor
  \enquote{[\ldots] must inform} all \enquote{recipients} that the software
  \enquote{[\ldots] is governed by the terms of (the MPL-2.0), and how (the
  recipients) can obtain a copy of this license}.\citeMPL{§3.1}  For the
  respective use case (MPL-2.0-C2, MPL-2.0-C4, MPL-2.0-C6, MPL-2.0-C8), the
  \oslic{} rewrites these conditions so that each MPL-2.0 source code package
  must neccessarily contain the MPL-2.0 itself as textfile and an additional 
  licensing file or statement strictly following the text given by the addendum
  of the MPL-2.0.\citeMPL{Exhibit A} Because the MPL-2.0 is only a license with
  weak copyleft, the \oslic{} proposes to separate the MPL-2.0 licensed,
  embedded component from the enclosing program (MPL-2.0-C8). 
  
\item But the MPL-2.0 does not explicitly require marking all modifications.
  Nevertheless, this is state of the art in computer emgineering. Therefore,
  with respect to the cases of distributing modified source code (MPL-2.0-C4,
  MPL-2.0-C6 and MPL-2.0-C8), the \oslic{} proposes to mark all modifications
  inside of the source code and to update the description of the functional
  changes. In case of distributing the modified software in the form of
  binaries, it should be sufficient to describe the modifications only on the
  functional level. 
  
\item Furthermore, the MPL-2.0 requires that the \enquote{Covered Software}---in 
  all cases of distributing it in an \enquote{Executable Form} (MPL-2.0-C3,
  MPL-2.0-C5, MPL-2.0-C7, MPL-2.0-C9)---\enquote{[\ldots] must also be made
  available in Source Code Form [\ldots]} and that the distributor
  \enquote{[\ldots] must inform recipients of the Executable Form how they can
  obtain a copy of such Source Code Form by reasonable means in a timely manner,
  at a charge no more than the cost of distribution to the
  recipient}.\citeMPL{§3.2.a}  The \oslic{} rewrites these conditions as the
  obligation to offer a download service at no charge and to point towards this
  services inside of the distributed package.
  
\item In this context, the MPL-2.0 allows to distribute the binaries under terms
  of another license \enquote{[\ldots] provided that that the license for the
  Executable Form does not attempt to limit or alter the recipients’ rights in
  the Source Code Form under this License.}\citeMPL{§3.2.b} This possibility
  might become important for those cases where the license compatibility must
  explicitly be managed. Normally, it should be sufficient also to distribute
  the binaries under the MPL-2.0. Thus, in case of distributing binaries
  (MPL-2.0-C3, MPL-2.0-C5, MPL-2.0-C7, MPL-2.0-C9), the \oslic{} proposes to
  insert into the distribution packages the MPL-2.0 itself and an additional
  licensing file or statement strictly following the text given by the addendum
  of the MPL-2.0.\citeMPL{Exhibit A} But again, because the MPL-2.0 is only a
  license with weak copyleft, the \oslic{} proposes to separate the MPL-2.0
  licensed embedded component from the overarching program (MPL-2.0-C9).
  
\item Finally, one clearly has to state that the distribution of the source code
  required by the previous rule must, of course, follow the rules of distributing
  the software. Thus, the \oslic{} requires in all cases of a binary distribution
  to execute also the task-lists of the respective source code use cases.

\end{itemize}

% ------------------------------------------------------------------------------

\end{license}
%\bibliography{../../../bibfiles/oscResourcesEn}

% Local Variables:
% mode: latex
% fill-column: 80
% End:

\input{snippets/en/06C-osFulfillmentByToDoLists/0610-lgplFulfillToDoList}
\input{snippets/en/06C-osFulfillmentByToDoLists/0611-agplFulfillToDoList}
\input{snippets/en/06C-osFulfillmentByToDoLists/0612-gplFulfillToDoList}

%%%%%%%%%%%%%%%
% Telekom osCompendium 'for being included' snippet template
%
% (c) Karsten Reincke, Deutsche Telekom AG, Darmstadt 2011
%
% This LaTeX-File is licensed under the Creative Commons Attribution-ShareAlike
% 3.0 Germany License (http://creativecommons.org/licenses/by-sa/3.0/de/): Feel
% free 'to share (to copy, distribute and transmit)' or 'to remix (to adapt)'
% it, if you '... distribute the resulting work under the same or similar
% license to this one' and if you respect how 'you must attribute the work in
% the manner specified by the author ...':
%
% In an internet based reuse please link the reused parts to www.telekom.com and
% mention the original authors and Deutsche Telekom AG in a suitable manner. In
% a paper-like reuse please insert a short hint to www.telekom.com and to the
% original authors and Deutsche Telekom AG into your preface. For normal
% quotations please use the scientific standard to cite.
%
% [ File structure derived from 'mind your Scholar Research Framework' 
%   mycsrf (c) K. Reincke CC BY 3.0  http://mycsrf.fodina.de/ ]
%

% Chapter Abstract
% ----------------
\chapter{Open Source Licenses and Their Legal Environments [tbd]}

\footnotesize
\begin{quote}\itshape
In this chapter we analyze why to know a license alone is not enough. At the end
you will know that open source licenses are embedded into the legal environment
of a state. And you will know in which sense the German legal environment
predetermines your readings of open source licenses.
\end{quote}
\normalsize{}


% Local Variables:
% mode: latex
% fill-column: 80
% End:


%%%%%%%%%%%%%%%
% Telekom osCompendium 'for being included' snippet template
%
% (c) Karsten Reincke, Deutsche Telekom AG, Darmstadt 2011
%
% This LaTeX-File is licensed under the Creative Commons Attribution-ShareAlike
% 3.0 Germany License (http://creativecommons.org/licenses/by-sa/3.0/de/): Feel
% free 'to share (to copy, distribute and transmit)' or 'to remix (to adapt)'
% it, if you '... distribute the resulting work under the same or similar
% license to this one' and if you respect how 'you must attribute the work in
% the manner specified by the author ...':
%
% In an internet based reuse please link the reused parts to www.telekom.com and
% mention the original authors and Deutsche Telekom AG in a suitable manner. In
% a paper-like reuse please insert a short hint to www.telekom.com and to the
% original authors and Deutsche Telekom AG into your preface. For normal
% quotations please use the scientific standard to cite.
%
% [ File structure derived from 'mind your Scholar Research Framework' 
%   mycsrf (c) K. Reincke CC BY 3.0  http://mycsrf.fodina.de/ ]
%

% Chapter Abstract
% ----------------
\chapter{Conclusion}

During the last 4 years, we have developed this \textbf{O}pen \textbf{S}ource
\textbf{Li}cense \textbf{C}ompendium. We had the honor and the pleasure to
discuss our ideas with many open source experts, for example with those, who
visit the European Legal and Licensing Workshop, organized by the FSFE. We were
invited to present our work on different conferences, in Germany, in Europe, and
even in Asia. We got a very encouring feedback. Today we know what we only
supposed when we started: We could indeed close an important gap by offering a
simple and reliable way to ascertain what one has to do for using open source
software compliantly. We are proud of having gone this long way. And we pride
ourselves on the fact that -- today -- the OSLiC is officially listed by the OSI
as one of those tools by which one can manage the open source
compliance\footnote{$\rightarrow$
http://osi.xwiki.com/bin/Projects/Process+and+Compliance+Resources}.

But, we also got adjusting feedback: Namely our initial premise was justifiably
not really accepted by the community. We were told that the software developers
themselves would never use our OSLiC. They would never read a book of more than
300 pages full of lists and tables -- as long as this book was not a
specification of a computer language. The OSLiC would be too large and too
complex for simplifying the daily life of the open source users. It would be an
excellent foundation for becoming an open source license expert -- but not a
tool for the desk. And indeed, it was simply silly to assume that software
developers, project managers, or IT managers can directly understand and use the
OSLiC: reading the OSLiC the first time has a discouraging shock effect. Today,
also we know this.

Nevertheless, it was very important for us to fall for the charme of this
illusion. Without this error, we never would have started the development of the
OSLiC. And thus, we never would have find the idea to organize the issue in form
of finders and a 5 question form. Without this error, we today would never have
a work which justifies and proves each single assertion by quoting the licenses
and the experts. And without this frightening feedback we received, we never
would have got one of our best and encouraging experiences: 

When we had accepted the feedback, we directly decided to develop an online
version of the OSLiC, the Open Source Compliance Advisor, also know as
OSCAd\footnote{$\rightarrow$ http://opensource.telekom.net/oscad/}.
We distributed it under the terms of the AGPL. Then, the company Amadeus decided
to take over the development of this online tool. We, on our side, inserted an
export interface into the OSLiC. They, on their side, rewrote the OSCAd and
integrated an import interface. So -- finally -- we both were able to focus on
only one specific aspect:  they took the responsibility for computing and
maintaining the online tool\footnote{$\rightarrow$
https://github.com/AmadeusITGroup/oscad}, we took the responsibility maintaining
for the fundamental analysis of the open source licenses\footnote{$\rightarrow$
https://github.com/dtag-dbu/oslic/}.

Thus, we concretely experienced the advantages of sharing ideas and sources,
which were so often emphazied: Playing the open source game actively means
giving a bit and getting back a lot. Playing the open source game actively means
saving the own resources.
 
Therefore, you may also take the fact that we finally could indeed publish the
version 1.0 of the OSLiC as a thankful profound curtsey to the open source
community!


%%%%%%%%%%%%%%%
\chapter{Appendices}

% Telekom osCompendium 'for being included' snippet template
%
% (c) Karsten Reincke, Deutsche Telekom AG, Darmstadt 2011
%
% This LaTeX-File is licensed under the Creative Commons Attribution-ShareAlike
% 3.0 Germany License (http://creativecommons.org/licenses/by-sa/3.0/de/): Feel
% free 'to share (to copy, distribute and transmit)' or 'to remix (to adapt)'
% it, if you '... distribute the resulting work under the same or similar
% license to this one' and if you respect how 'you must attribute the work in
% the manner specified by the author ...':
%
% In an internet based reuse please link the reused parts to www.telekom.com and
% mention the original authors and Deutsche Telekom AG in a suitable manner. In
% a paper-like reuse please insert a short hint to www.telekom.com and to the
% original authors and Deutsche Telekom AG into your preface. For normal
% quotations please use the scientific standard to cite.
%
% [ Framework derived from 'mind your Scholar Research Framework' 
%   mycsrf (c) K. Reincke 2012 CC BY 3.0  http://mycsrf.fodina.de/ ]
%

\section{Some Additional Remarks on the OSLiC Quotation Style}\label{sec:QuotationAppendix}

We have already characterized the general tone of our
footnotes\footnote{$\rightarrow$ p.\ \pageref{QuotationPrinciple} }. Let us now
briefly explain a little peculiarity of our bibliography:

Modern times have also changed the humanities. Formerly a book or an article
must be printed for being ripe to be quoted. Our statements relied on static,
readily prepared works. Nowadays even university libraries sometimes offer those
books and articles as PDF files which are printed in the original. As a scholar,
now you must rely on the equality of the printed version and the PDF file -- at
least with respect to the page numbers and the appearance. You can not verify the
equivalence -- at least to a certain degree.

Moreover: in case of such 'e-books' and 'e-articles' the libraries often do not
offer the pdf files themselves but links to the download pages of the publisher.
Formerly as a scholar you could trust that your readers would be able to
retrieve the quoted work if they want to verify your citations. It's one task of
our libraries to hold available our scientific sources. But now they do not buy
any longer the books, but the right to download files over the university net.
In this case these PDF files are not stored on the serves of the university
library. By using the link provided by the publisher each student or each reader
downloads his own file -- case by case. Therefore -- as a scholar -- you now have
to trust that the publisher, who provides the link, will not change that pdf
file that you have cited.

But it gets even worse: While it might be that publishers modify their work
secretly (even it is not very likely that they do it), it's a definite feature
of the web that its pages are fre\-quen\-tly changed. Hence we must ask
ourselves: Can we seriously argue on the basis of statements and documents which
might disappear? Can we quote such possibly volatile sources? The problem is: we
must do it, especially if we write about an internet topic -- and even if we want
to write a really reliable compendium.

So, what can we do? First, we must confide in our readers, that they either
will retrieve our sources or -- if they can not find them -- that they
believe that we really have found and read what we have written and
quoted. Second, we store all these e-wares\footnote{Take this little word as
(new) generalization of 'e-book', 'e-article', 'e-paper' and so on.} we
read\footnote{But because of the copyright we ourselves are naturally not
allowed to offer a download link for them or to send a copy of it to those who
want to verify our quotes.}. And thirdly we should lay open to our readers the
different levels of reliableness of our sources. Therefore we use
the following markers in our bibliographic data\footnote{And another hint: Nowadays sometimes
even scientific libraries don't offer exact 'e-copies' of the original. In
some cases one can only get html-versions of articles which formerly were
printed as part of journals. In these case the scholar has to use sources which
lost their original page-numbers. The same can happen to articles of proceedings
etc.\ which are now only offered as autonomous pdf files with an internal paging.
If we quote such kind of articles we try to specify the number of the quoted
article in the original row of articles, added -- if possible -- by an internal
page number. But naturally we also try to follow the bibliographic data
delivered by that organization which distributes these kind of copies.}:

\begin{itemize}
  \item Print / Copy:- The source is printed and we saw either the printed work
  really or we get an official copy by our library. Hence you should also be able
  to get the work in a library, at least in those we used (UB Frankfurt or ULB
  Darmstadt).
  \item BibWeb/[PDF/\ldots] :- The source might be printed, but we read only the
  electronic version (PDF or other type of format), offered by and over the
  net of our university libraries (UB Frankfurt or ULB Darmstadt).
  \item FreeWeb/[PDF/\ldots] :- We read the electronic version offered by the
  free web. In this case we add the url\footnote{Please note: Long urls often
  destroy the pleasing appearance of a text because it's difficult to wrap the
  lines acceptably. Hence we wished to make it easier for LaTeX to do this job.
  Therefor we sometimes split the urls and inserted blanks. So you have to erase
  all blanks if you want to verify our urls.} and the date when we downloaded /
  saw the text.
\end{itemize}


%\bibliography{../../../bibfiles/oscResourcesEn}

% Local Variables:
% mode: latex
% fill-column: 80
% End:


% Telekom osCompendium 'for being included' snippet template
%
% (c) Karsten Reincke, Deutsche Telekom AG, Darmstadt 2011
%
% This LaTeX-File is licensed under the Creative Commons Attribution-ShareAlike
% 3.0 Germany License (http://creativecommons.org/licenses/by-sa/3.0/de/): Feel
% free 'to share (to copy, distribute and transmit)' or 'to remix (to adapt)'
% it, if you '... distribute the resulting work under the same or similar
% license to this one' and if you respect how 'you must attribute the work in
% the manner specified by the author ...':
%
% In an internet based reuse please link the reused parts to www.telekom.com and
% mention the original authors and Deutsche Telekom AG in a suitable manner. In
% a paper-like reuse please insert a short hint to www.telekom.com and to the
% original authors and Deutsche Telekom AG into your preface. For normal
% quotations please use the scientific standard to cite.
%
% [ File structure derived from 'mind your Scholar Research Framework' 
%   mycsrf (c) K. Reincke CC BY 3.0  http://mycsrf.fodina.de/ ]

%


%% use all entries of the bibliography
%\nocite{*}


\section{Some Widespread Open Source Myths}

From the viewpoint of an internet student we have to consider that the web
offers a mass of rumors concerning the nature of open source software
(Licenses). Here are some of the myths\footcite[At least one time even a
scientific legally discussing book is talking about the \enquote{myth around open
source licenses} -- although only as part of  the title: cf][1ff,
especially 209ff]{GuiOvd2006a} we met:
 
\textbf{BE CAREFUL: THIS SECTION MUST THOROUGHLY BE REVIEWED AND REWRITTEN. 
IT'S ONLY AN OUTLINE!!! Do not quote part of it. It must be verified.}

\begin{description}
  \item[open source tries to improve the world ethically] :- No, there's a clear
  ban to exclude persons, groups, purposes. Thus, there is no chance to exclude
  anyone from using open source software because he is an ethical or moralic
  malefactor.
  \item[Changed open source software must be re-published] :- No, in a double
  sense! There are OS licenses which allow the proprietarization of the
  modified code. And even the LGPL and the GPL, which clearly try to prevent
  the proprietarization, do not require generally that a modified code must be
  (re-)published. Only if you give your modfied (L)GPL licensed application as
  binary to anybody, then you have to handover the modified code, too.
  \item[Modified open source software must be given back to the whole community]
  :- No. Again, there are OS licenses which allow the proprietarization of the
  modified code. And even the LGPL and the GPL -- which clearly require, that you
  also publish the modified code, if you give the modified binary to anybody --
  do not require that you distribute your modification around the world. LGPL and
  GPL clearly say that you have to hand over the code to those persons you
  give the binary to. And if you only give your improvement only one person or a
  group of persons, then you must handover your code only to that persons or
  only to all members of that group.
  \item[Published open source software is open for ever] :- No, if this myth
  says that also all future versions will have to be distributed under an open
  source license. The copyright holder ever holds the copyright. They can change
  the licence of next release of its software -- but only for the following
  release, not for the current or for former versions. Those releases, which
  already have been distributed under an open source license, indeed remain
  open.
  \item[Software can either be open source software or proprietary software] :-
  No. The copyright holders themselves can additionally distribute the code
  under other conditions when ever they want to do it. That's not a question of
  the licence, but of the copyright.  
  \item[The opposite of open source software is commercial Software] :- No.
  First, you are also allowed to use the open source software in any commercial
  purpose. There's only one point which is excluded in OSS: you are not allowed
  to ask for a licence fee if you distribute 'open source software'. Second,
  there are many other forms like freeware, public domain software or anything
  else which is neither open source software nor Commercial Software. It's
  pointless to take the question of money as a criterion for distinguish open
  source software and its opposite. Moreover: Proprietary Software as opposite
  of open source software should be defined ex negativo: all kind of software,
  which does not fit the OSD is proprietary.
  \item[open source software prohibits to earn money] :- No, you are allowed to
  invent each business model you want. There's only one exception: you are not
  allowed to ask for a licence fee if you distribute open source software. This
  limitation is based on the open source definition which clearly states that a
  license -- which wants to become an open source license -- \enquote{shall not
  restrict any party from selling or giving away the software as a component of
  an aggregate software distribution containing programs from several different
  sources} and that the license under this circumstances \enquote{[\ldots] shall
  not require a royalty or other fee for such sale}\footcite[cf.][§1]{OSI2012a}.
  If you combine this constraint with the requirements that an open source
  license \enquote{[\ldots] must not restrict anyone from making use of the
  program [\ldots]}\footcite[cf.][§6]{OSI2012a} and that it \enquote{[\ldots]
  must allow distribution in source code as well as compiled form
  [\ldots]}\footcite[cf.][§2]{OSI2012a}, you can generally conclude that none of
  the open source licenses may require a fee for using and/or distributing the
  program. But being paid for the service to install the program, to collect
  and compile a customer specific version, and/or to monitor the environment is
  of course not excluded by this condition.
  
  Historically this mistake might be evoked by Debian: The GNU project missed
  its kernel while the Linux kernel was already distributed as part of
  collections which also include GNU software. Then, in 1983? Ian Murdock was
  supported by RMS and its FSF to build a really free distribution (Debian)
  containg GNU software and the Linux kernel. But Ian Murdock states also, that
  Debian does not want to earn money.
% TODO find sources for indirect citations
% TODO: check, whether OSD requires license fee free distribution
  \item[Modifications of open source software must be marked] :- No. This is not
  a defining postulation of the OSD. The OSD allows licenses to require the mark
  of modifications. But it does not require from all licenses to require the mark
  modifications for being an open source license.
  \item[Modifications of open source software must be marked by your personal
  data] :- No, it is only required to mark modifications so that a reader could
  distinguish the modifications from the original code. It's required for saving
  the integrity of the original author. And therefore it is not required as a
  constitutive criterion by the OSD. It might be that a license additionally
  requires your name. But that is not feature of open source software in general.
  And at least the licenses discussed by us do not require to insert your name.
% TODO: check whether any of our licenses reuire that you mark modifications by
% your personal data / real name  
  \item[The open source Definition determines the conditions to use open source
  software] :- No. The \emph{Open Source Definition} determines which licenses
  are open source licenses, nothing more. The OSD is a set of necessary
  conditions to be an open source license. It determines the freedom and the
  responsibilities of a user as a set of more or less abstract rules. But it
  does not constitute a set of sufficient tasks which a user has to perform for
  fulfilling any open source license. Open source licenses may differ by
  instantiating the OSD criteria. So, if you want to know what you have to do to
  fulfill a license, you have to go back to the real license of that software
  you are using.
\end{description}

%\bibliography{../bibfiles/oscResourcesEn}

% Local Variables:
% mode: latex
% fill-column: 80
% End:


% Telekom osCompendium 'for being included' snippet template
%
% (c) Karsten Reincke, Deutsche Telekom AG, Darmstadt 2011
%
% This LaTeX-File is licensed under the Creative Commons Attribution-ShareAlike
% 3.0 Germany License (http://creativecommons.org/licenses/by-sa/3.0/de/): Feel
% free 'to share (to copy, distribute and transmit)' or 'to remix (to adapt)'
% it, if you '... distribute the resulting work under the same or similar
% license to this one' and if you respect how 'you must attribute the work in
% the manner specified by the author ...':
%
% In an internet based reuse please link the reused parts to www.telekom.com and
% mention the original authors and Deutsche Telekom AG in a suitable manner. In
% a paper-like reuse please insert a short hint to www.telekom.com and to the
% original authors and Deutsche Telekom AG into your preface. For normal
% quotations please use the scientific standard to cite.
%
% [ File structure derived from 'mind your Scholar Research Framework' 
%   mycsrf (c) K. Reincke CC BY 3.0  http://mycsrf.fodina.de/ ]
%

% Chapter Abstract
% ----------------

\footnotesize \begin{quote}\itshape This section outlines reflections by which
we initially focused ourselves on the question why we need an OSLiC and how its
content and form should be derivated from these needs.
\end{quote}
\normalsize{}

\subsection{Why}

Do we need another book about open source? Do \emph{you} need another book about
open source software? Let us address this question from the viewpoint of what we
already know, what we instinctively believe and what we may have heard. For
example you may presume one or more of the following statements are correct. Or
you may even have experienced similar perceptions from your peers or managers.
Or you have been told they describe 'open source':

\begin{itemize}
  \item The \emph{Open Source Definition} offers rules to use open source software.
  \item Modified open source software must be published.
  \item Modified open source software must be given back to the community.
  \item All generations of open source software will remain open for ever.
  \item Software can either be open source software or proprietary software.
  \item The opposite of open source software is commercial software.
  \item open source software prohibits to earn money.
  \item Modifications of open source software must be marked explicitly.
  \item Modifiers of open source software must identify themselves.
  \item When distributing an open source binary it’s enough point to a download
  page to obtain the source code.
  \item The aim of open source software is to improve the world ethically.
  \item open source software is viral and infectious.
\end{itemize}

Do these conceptions sound familiar to you? Unfortunately, whatever we might
believe or wish for, these concepts are incorrect. Naturally we will discuss
this issue later on. For the moment let us assume they are indeed
incorrect\footnote{For those who want directly verify our argumentation, we have
generated a condensed summary of the arguments and citations. You can find this
summary in our appendices.}.

So, again: Do \emph{we} need another book about open source software? \emph{We},
that is -- in this case and at least initially -- the large German company
\textit{Deutsche Telekom AG}. Arguing from the perspective of a large company
requires not only identifying the common misconceptions, but catering for the
unique needs of a large Enterprise. And indeed the very size of the company
brings its own problems.

Large companies use more open source software in more varied contexts than small
companies. There is an important question that every company should ask:
\emph{'Are we sure that we respect all those requirements of open source
software we have to respect?'}. But large companies cannot answer this question
as easily as small companies: the large number of diverse open source
deployments in different contexts mean that case by case governance, a model
that may work in small concerns, is far from appropriate for our needs. This
leads to wasting both time and money. Further, the chances of success are small:
training at least one employee in each software team as an open source software
License expert is unrealistic in terms of cost-efficiency and reliability.

Nevertheless even large companies want to and try to fulfill the rules of open
source software thoroughly -- especially \emph{Deutsche Telekom AG}. When this
company realized that the question \textit{Are we sure that we respect all those
rules of open source software correctly which we have to respect} could be
problematic, it directly asked some of its employees known as open source
enthusiasts to establish a service and a process for answering this question.

So, it is no surprise that we, the initial authors of this \textit{Open Source
License Compendium}, were asked by our employer \emph{Deutsche Telekom AG}.
Naturally we were proud to work on an open source topic officially. But while we
were doing our job we had to ask ourselves if \emph{we} perhaps needed another
book on open source. Our answer was \textit{Yes, we do!} Let us shortly explain,
why:

First, we already knew that there exists supporting software. These
meta-pro\-grams take the code of any other application and try to list those
open source components being 'covered' by that application\footnote{As general
examples let us mention Palamida (\texttt{http://www.palamida.com/}) and
BlackDuck (\texttt{http://www.blackducksoftware.com/}).}. But we had also
already realised that this supporting software did not always match the way we
thought the problem should be solved. Second, we recognized fairly quickly that
we need a reliable guide. We personally were asked to give the \emph{ok} for
projects of our company. We could not answer such requests on the base of
\textit{'Oh yes, I read this in the \emph{Heise-Ticker} a few days ago'} -- even
if the \emph{Heise-Ticker} had described the situation completely correctly. We
ourselves had to be more reliable than this\footnote{But of course, we have to
do ourselves the honor of conceding that we -- like many many other German open
source enthusiasts -- love using the \emph{Heise-Ticker} as main IT information
source. Unfortunately, its reputation is stil not high enough that its news can
directly be cited.}. Naturally we already knew a great deal about open source
software. Even so, our knowledge was not as systematic as necessary. We looked
for an open source compendium which adequately described what a project or
product development team had to do to fulfill the criteria of its open source
licenses. We wanted to use that compendium to the basis of our recommendations.

We were very thorough but we did not find what we were looking for. Our 'little'
bibliography attest our seriousness. What we found was a lot of information
releated to individual issues spread over many sources. We did not find answers
to our question even in the specific literature. Let us describe three little
steps to increase the understanding of the issue:

Without open source licenses there is no open source movement. Nevertheless in
dealing with open source licenses, this is sometimes neglected. Take the
\emph{Apache Web Server} as an example: No doubt, it is one of the most important
pieces of open source software\footnote{To prove that the \textit{Apache} is
really a piece of open source software one must execute a set of steps: First,
you have to note, that \emph{Apache} is something like a meta project, covered
by the \emph{Apache Software Foundation}, also known as \emph{ASF} (cf.
\texttt{http://www.apache.org/}, wp). Thus, you can not directly jump into
the \emph{Apache License}. First of all you have to visit the project site (cf.
\texttt{http://httpd.apache.org/}, wp) even if at the end its license link
leads you back to the general \emph{Apache License sub site} (cf.
\texttt{http://www.apache.org/licenses/}, wp) which announces, that \enquote{all
software produced by The Apache Software Foundation or any of its projects or
subjects is licensed according to the terms of the documents listed
below}. Only now you can use the offered link for switching to the
\emph{Apache License}, Version 2.0, if you want to check your rights and duties.
But that is difficult. There does not exist any simple list what you have to do
for fulfilling the license. Even the faq (cf.
\texttt{http://httpd.apache.org/docs/2.2/faq/}, wp) -- meanwhile being moved to
a wiki -- only says that the server \enquote{[\ldots] comes with an unrestrictive
license} and that you are allowed to put the code on a CD (cf.
\texttt{http://wiki.apache.org/httpd/FAQ}, wp). Hence, from the viewpoint of
the ASF the license itself shall answer all questions. [Reference download for
all urls: 2011-08-31] } with a specific license\footcite[cf.][\nopage
wp]{AsfApacheLicense20a}. Moreover: the success of the open source movement
in the commercial world depends directly on the decision of IBM to replace its
corresponding own component in the \textit{IBM WebSphere Application Server}
with the free \textit{Apache Web Server}\footcite[cf.][287ff]{Moody2001a}.
Meanwhile many companies use the \textit{Apache Web Server} to act as a web
provider. Currently the \emph{Apache http server} -- as it has to be named
correctly -- is used more than twice as much as all the other http server
software together\footcite[cf.][\nopage wp]{Netcraft2011a}. Hence many business
models depend on the Apache License. Another aspect is that even the famous
\emph{Apache Cookbook}, which explains the installation, the configuration, and
the maintainance of an Apache Web Server in details\footcite[cf.][\nopage et
passim]{CoaBow2004a}, does not mention anything about the license which allows
for installation, configuration and maintenance. Neither the index lists the
word 'license'\footcite[cf.][245ff, esp.\ p.\ 250]{CoaBow2004a}, nor the chapters
'Installation'\footcite[cf.][1ff]{CoaBow2004a} or the chapter
'Miscellaneous'\footcite[cf.][219ff]{CoaBow2004a} mentions the license question
in a serious way. There's only one short hint as to the advantage of open source
software, i.e.\ that everybody is allowed to install it\footcite[cf.][1: \enquote{%
\ldots einer der Vorzüge von open source software besteht darin, dass
je\-der\-mann die Erlaubnis zur Erzeugung eines eigenen Installationskits hat
}]{CoaBow2004a}. Can you be sure that you are allowed to do what you are
doing on the base of such a phrase?

Naturally, the \emph{Apache Cookbook} is not a book for lawyers, it is a book for
administrators and developers. They do not want to get bogged down by
legalities, they want to set up an Apache Web Server as fast as possible and get
down to work. Indeed, the Apache Cookbook offers a good support. But not only as
a company you have to ask yourself whether you are really allowed to do what you
are doing. Can you find the answer in the \emph{Apache Cookbook}? No. Can you
find it in the license itself? Yes, but it is difficult\footnote{And do we
really want our developers and maintainers to read the original licenses? Do we
really want them to discover that they also have to check the licenses of the
used modules?}. So again: Can you find your answer in another book, which is
\emph{Amazon's} current top recommendation for the search term \emph{'apache
server'}\footnote{Tested on \texttt{http://www.amazon.de/} at 2011-08-31.}? Not
really: Sascha Kersken's Apache 2.2 Handbook offers a license chapter, but it is
only two pages long\footcite[cf.][111f]{Kersken2009a}. Moreover, the rights and
duties are condensed into just 5 bullet points which taken together do not
explain when the software and the license have to be handed over to a customer
and when you are allowed to hide your
improvements\footcite[cf.][112]{Kersken2009a}.

This brings us to the question of what prevents us from using something like a
\emph{'general license cookbook'} which explains all the necessary details and
which offers  quick access to the relevant points:

Of course we also browsed the internet. At least for German speaking people
there is an excellent site concerning the topic \emph{open source licenses}.
offered by \textit{iffross}, which, loosely translated, means an
\textit{Institute for Legal Aspects of the Free and open source
software}\footnote{originally: \enquote{Institut für Rechtsfragen der Freien und
open source software}. Main entry point for its site is the URL
\texttt{http://www.ifross.org/}.}, founded in 2000 as a private institute to
track the phenomenon 'free software' from the viewpoint of (German)
lawyers\footcite[cf.][\nopage wp]{ifross2011b}. Besides many other
aspects this site offers a very well and thoroughly elaborated
FAQ\footcite[cf.][\nopage wp]{ifross2011c} and a large list of open
source licenses and other related licenses: moreover, evidently it is
classifying the open source licenses in those 'without copyleft-effect' (BSD),
in those with 'strict copyleft-effect' (GPL) and in those with 'restricted
copyleft-effect' (LGPL)\footcite[cf.][\nopage wp]{ifross2011a}.

However, even this excellent site does not fulfill our needs. It does not offer
those context specific to-do lists which companies, developers or project
managers can use to ensure their open source software is used in a regular
manner.

We therefore evaluated that standard book which is listed in the most legal
bibliographies\footnote{at least in that German judicial literature dealing with
open source}: the book of Jaeger and Metzger which concerns -- loosely translated
-- \textit{the judicial framework requirement for open source
software}\footcite[cf.][V -- It can not be any surprise that both authors,
Mr. Jaeger and Mr. Metzger are members of ifross (cf.
\texttt{http://www.ifross.org/personen/}, wp)]{JaeMet2002a}. Even the most
earliest edition of this book already had a clear structure in its chapter
'copyright': For each license mentioned (or at least for each license cluster)
it offered a subchapter for the rights and a subchapter for the
duties\footcite[cf.][30ff]{JaeMet2002a} of the software user\footcite[For
getting a good survey of the structure and the line of thought see the contents
cf.][VIIIf]{JaeMet2002a}. Many other important aspects of the topic
\textit{open source} are discussed, too\footcite[pars pro toto: have a
look at the chapter concerning the liability: cf.][137ff]{JaeMet2002a}.

But we needed more than this. Despite the quality of the book we were certain
that we could not hand over this book to our programmers with the recommendation
\textit{check your touched licenses and follow the instructions of the relevant
subchapters\ldots}. This book did not contain simply checkable to-do lists,
neither in the first edition\footcite[cf.][VIff]{JaeMet2002a} and in the second
edition\footcite[cf.][VIIff]{JaeMet2006a} nor in the recently published third
edition\footcite[cf.][VIIIff. Naturally we use this latest edition for adopting
or discussing systematical aspects]{JaeMet2011a}. So, how can a company or a
developer or a project manager be sure of fulfilling the requirements of the
open source licenses sufficiently if he/she does not have a verified list
telling him \textit{'do this, and in case of that, do that, and finally do also
this'}? Why should he himself implicitly become an open source licenses expert
who has to extract the necessary steps out of the literature?

While we were searching for an existing open source compendium, we found an
article with the title 'Compendium for the Publication of open source
software'\footnote{approximately translated}. It aims to be a 'pragmatic
guidebook' and an 'assistance' for 'publishing software under the conditions of
an open source license'\footcite[cf.][166f (originally: ein
\enquote{pragmatischer Ratgeber} zur \enquote{Veröffentlichung einer Software
unter den Rahmenbedingungen einer Open-Source-Lizenz}) ]{BreGlaGra2008a}.
Moreover, at the end of this article, its authors formulate ambitiously that
their 'guide' should be carried out, section by section -- for getting a legally
water tight process of publishing open source software\footcite[cf.][186
(originally: ein \enquote{Ratgeber}, der es erlaubt \enquote{ (\ldots) die zu
berücksichtigende Aspekte (strukturiert abzuarbeiten) (\ldots) } und einen
\enquote{rechtlich nicht angreifbaren Veröffentlichungsprozess} zu
ermöglichen) ]{BreGlaGra2008a}.

The authors of this article describe something close to what we were looking
for. Indeed, the article lists important aspects which have to be taken in
consideration if you want to deal open source software correctly: It announces
that no obligation exists to publish code either if you embed GPL code into your
proprietary code or if you modify the GPL code. It is only if you hand over your
binary to other persons that you have to distribute the code too, but only to
them and not to the general public\footcite[cf.][170 and 181]{BreGlaGra2008a}.
Additionally the articles explains exactly that software -- at least in Germany --
can only be acknowledged as open source software by transferring the rights to
use -- the \emph{'Nutzungsrechte'} -- to other people, while the copyright itself
-- the \emph{'Urheberpersönlichkeitsrecht'} -- is not transferable and belongs to
the author\footcite[cf.][173]{BreGlaGra2008a}. Moreover, besides other aspects
the articles briefly and deeply discusses the problem of the No-Warranty-Clauses
which are not valid in Germany and which will therefore automatically be
replaced by the liability rules for a
donation\footcite[cf.][177]{BreGlaGra2008a}. And last but not least this article
actually summarizes the idea of Copyleft and the differences between LGPL and
GPL\footcite[cf.][181]{BreGlaGra2008a}.

However some gaps remain. The article does not analyze in which cases a
University or a company perhaps \emph{must} publish its developments based
on open source software. It does not discern between different licenses
and conditions. It also does not discuss what Universities or companies,
which (re-)use and/or distribute open source software (internally), must do to
fulfill the touched open source licenses. And finally this article
does not offer the step by step list as promised.

We did, however, feel supported by this article, in two ways. First, it was a
well written summary of some main problems. Second, it stated the necessity to
have a compendium for being able to establish a legally 'water-tight' process of
publishing open source software\footcite[cf.][186]{BreGlaGra2008a}. We
seemed to be justified in our assumptions. But the open source compendium we
were looking for had to be more practical, more processable, more distinguishing
and more elaborated.

So again: Did we need a new book about open source software? We had looked for a
reliable integrated open source compendium. But we found separate pieces of
information and -- as we know today -- some rumors. Our answer was clear:
naturally we did not need a new general book about open source. But what was
lacking was a description of what responsible developers, project managers or
product developers require to fulfill open source licenses. We needed an
\textit{Open Source License Compendium}.

At the best such an \textit{Open Source License Compendium} would contain a set
of simply to process \textit{'For-Fulfilling-The-License-To-Do-Lists'}.
Additionally it should offer an intuitively user-friendly search option for
these lists. In any case, it should share developers and project managers the
effort of having to become open source license experts. For the other users, it
should also clearly explain why one has to do this and not that. Hence a
reliable \textit{Open Source License Compendium} should not only list what one
has to do, but should offer both, thoroughly verified reliable details and
clearly condensed guidance.

Although we did not find such an open source compendium we were familiar with
the spirit of the open source community. Hence we followed one of its most
simple rules: \emph{'what you miss you must develop on your own'}. Some
principles should help us to achieve our targets:

\begin{description}
  \item[To-do lists as the core, discussions around them]: Our work should be
  split into two parts. As its core we wanted to offer a
  set of to-do-Lists. Each of these lists should be relevant to one specific
  open source license and should be clustered by the open source specific use
  cases. Around this all those aspects of open source software which influence the
  interpretation of the licenses and the rules core should be precisely
  characterized. Nevertheless, the users should be able to skip
  details and go directly to the section they require.
  \item[Quotations with thoroughly specified sources]: Even if our users should
  not be obliged to read every part of the compendium they should not be
  required to believe us. We wanted to be revisable. Because our sources and our
  conclusions should be easily verifiable, we decided to use the academic
  citations and list bibliographic data extensively on the basis that our task
  should be to collect information, not to invent new 'facts'.
  \item[Not the internet alone, also books and articles]: We wanted to go back
  to the originals even if the internet was full of more or less modified
  copies. We wished to get reliable facts and descriptions. Therefore we decided
  to evaluate not only the internet but also scientific sources -- for example --
  offered by university libraries.
  \item[Not clearing out the forest land, but cutting out a swathe]: Even if we
  had to deal with licenses and their legal aspects we did not want to get lost
  in detailed discussions. It should not be our task to find out whether a
  specific kind of handling would still be legal or already forbidden.
  We did not want to fight against the licenses. We did not want to stretch
  their ambit or to test their boundary. We wished to accept open source
  licenses as they are: rules written from developers for developers. And even
  if some parts of these licenses would not be valid with respect to a legal
  system\footcite[And indeed for example for the GPL one can argue in this way:
  Even if you take the GPL as a contract of the type 'donation' respectively
  \enquote{Schenkung}, it is presented in the form of AGBs respectively
  \enquote{Allgemeine Geschäftsbedingungen} and must therefore follow the
  general AGB rules.'Regrettably' in Germany these general AGB rules do not
  allow to exclude each type of warranty. If we follow Oberhem, §11 and §12 of
  the GPL must be invalid in Germany because of these general AGB rules.
  Moreover, for Oberhem even §5 -- the important clause of the GPL by which you
  can only get the right to use and to distribute GPL software if you respect
  the rules of the GPL -- seems also to be invalid respectively
  \enquote{unwirksam}. But the good message is that the GPL as whole is not
  invalid even if it contains invalid clauses.][128, 133ff, 150ff, esp.\ 146,
  159]{Oberhem2008a}, we wanted to take them as our guideline -- at least while
  they do not violate more general laws\footnote{what they clearly do not do!}.
  We simply wanted to \emph{find one proven way} to cross the maybe slightly
  unsure forest of open source licenses. Even if indeed some clauses of the
  licenses finally were not enforceable against us we wanted to respect them
  'voluntarily'. We wanted to deliver a set of rules which support users and
  remove the possibility of becoming involved in license disputes with open
  source developers or the Free Software Foundation.
  \item[Take the text seriously]: On the other side we wanted to take our
  license texts as they were. If they lacked anything\footcite[The systematical
  underdetermination of licenses is a problem being also known in the open
  source respectively Free Software movement. Following the biography of RMS his
  main judicial counselor Moglen has stated, that \enquote{there is uncertainty
  in every legal process (\ldots) } and that it seemed to be silly to try
  \enquote{(\ldots) to take out all the bugs (\ldots)}. Nevertheless -- so
  Moglen resp.\ Williams -- the goal of Richard Stallman was \enquote{the complete
  opposite}: He tried \enquote{(\ldots) to remove uncertainty which is
  inherently impossible}. But -- and that's the nub of this analysis --
  Moglen had to follow Stallmann because of RMS character. And he had to
  summarize their work so, that \enquote{(\ldots) the resulting elegance (of the
  GPL; KR.), the resulting simplicity (of the GPL; KR.) in design almost
  achieves what it has to achieve}. Hence we are asked to take the license
  texts themselves seriously. cf.][177f]{Williams2002a}, we would interpret the
  open issues in the spirit of the open source idea. But where the text was
  clear and definite we wanted to take its propositions as a definite decision --
  even if that meaning stood against well known open source 'facts'.
  \item[Trust the swarm]: We did not want to use our own research alone as a
  basis. We knew that the swarm is ever stronger than a set of some randomly
  selected experts. Therefore we decided to publish our text as a still
  unfinished work, starting with an early release 0.2. And then we wanted to
  invite the community to complete the compendium together with us. We would
  elaborate our open source compendium as a set of LaTeX- and BibTeX files which
  could be developed and managed in GIT or any other version control system. And
  finally we would publish our text under a Creative Commons Attribution-Share
  Alike German 3.0 license, to allow other people to correct us, to help us or
  even to take our results for their own purposes.
\end{description}

And so we did. Here is the result. Feel free to use it -- according to our
licensing.

\subsection{What}

Now we can briefly explain how one should be able to use the compendium:

% TODO adopt real chapter structure into the prolegomena 
\begin{description}
  \item[The Same Idea, Different Licenses] :- Here you will find background
  information to help you interpret open source licenses in the sense of the
  \emph{Free Software movement}\footcite[At least at this place you are perhaps
  expecting that we use the logograms FLOSS, F/OSS, F/LOSS, or whatever. As you
  will read later on the word \textit{Free} is ambiguous and has strained the
  use of the concept \textit{Free Software}. Later on we will also talk about
  the invention of the concept \textit{open source} designed as a 'replacement'
  and acting as a 'splitter'. The mentioned logograms are introduced to
  re-establish or -- at least -- to underline the common history and the common
  center of 'both' movements, whereby the word \textit{Libre} shall resolve the
  ambiguity of the word \textit{Free}. For a first survey cf.] [\nopage
  wp]{wpFloss2011a}, the \emph{open source software movement}\footcite[For
  another brief and informative introduction cf.][231ff esp.\ p.\ 
  232f]{Fogel2006a}, or the GNU-Project\footnote{ We ourselves will stay with the
  concept \textit{open source} because the OSD specifies the scope of our
  analysis. But we do it with a deep obeisance to Stallmann and the FSF -- even
  if we know that this will not protect us from the thunderbolt of RMS.}. We discuss
  different ways to cluster open source licenses. Finally we present our own
  taxonomy based on the labels 'protecting the developer', 'protecting the
  licensed code' and 'protecting the on-top-developments'. If you are familiar
  with the methods of grouping different open source licenses and particular
  if you know that you can not authorize your doings on the base of descriptions
  of such license groups, then it is enough, in order to understand our line of
  thought, to briefly note our taxonomy and its wording.
  \item[The Problem of Derivated Works] :- This chapter is important. In the
  spirit of software developers we try to explain which kinds of programming
  evoke a derivated work and which not. Our to-do lists will refer to this
  analysis.
  \item[The Problem of Combining Different Licenses] :- You should
  not ignore this chapter. We will explain why and how combining software
  of different licenses is not as dangerous as it is often told. The results of
  this chapter influence the structure of our to-do lists.
  \item[open source software and Money] :- Here we will shortly
  discuss ways in which money is no problem. If you already know that it is only
  prohibited to require payment for the act of licensing a piece of open source
  software to second or third parties and if you already know that this is only
  forbidden by some licenses, and not by all, than you can postpone the reading
  of this chapter.
  \item[The Problem of Implicitly Freeing Patents] :- Here we
  will illuminate some aspects of software patents and how the are handled by
  some open source licenses. You should know what licenses implicitly do with
  your patents. But it is not our intention to write a software patent
  compendium.
  \item[Open Source Use Cases as Principle of Classification] :- This is an
  important chapter. We explain our categories 'Use as it is', 'Modify the
  Code', 'With Redistribution', 'Without Redistribution', 'Isolated Initial
  Development', 'On-Top-Development': we develop and discuss our taxonomy with
  respect to the side effects of 'combining different licenses' and 'generating
  derivated works'. This taxonomy will determine the following chapters.
  \item[open source licenses: Find Your Specific To-do Lists] :- This is a kind
  of summary which joins the relevant aspects and elaborates the 'finder
  for your to-do lists'. This is the chapter which you probably will reuse
  frequently, even if you do not want to read any of our explanations.
  \item[open source license Fulfillment: Classified To-do Lists] :- This chapter
  offers all classified to-do lists. The structure of its subchapters will
  match the structure of our finder and the structure of our taxonomy.
  \item[open source licenses and Their Legal Environments] :- Here we discuss
  why using open source software in a regular manner is not only a question of
  the licenses themselves but of the kind of the surrounding legal system.
  \item[Appendices: Some Widespread Open Source Myths] :- Here we make good on
  our promise to explain why all the propositions mentioned at the beginning of
  this chapter are wrong. You might read this chapter as a special introduction
  or a reminder epilogue whenever you want to do.
\end{description}


%\bibliography{../../../bibfiles/oscResourcesEn}

% Local Variables:
% mode: latex
% fill-column: 80
% End:



\small
%\theendnotes


\footnotesize
% Telekom osCompendium English Nomenclation Tokens Include Module 
%
% (c) Karsten Reincke, Deutsche Telekom AG, Darmstadt 2011
%
% This LaTeX-File is licensed under the Creative Commons Attribution-ShareAlike
% 3.0 Germany License (http://creativecommons.org/licenses/by-sa/3.0/de/): Feel
% free 'to share (to copy, distribute and transmit)' or 'to remix (to adapt)'
% it, if you '... distribute the resulting work under the same or similar
% license to this one' and if you respect how 'you must attribute the work in
% the manner specified by the author ...':
%
% In an internet based reuse please link the reused parts to www.telekom.com and
% mention the original authors and Deutsche Telekom AG in a suitable manner. In
% a paper-like reuse please insert a short hint to www.telekom.com and to the
% original authors and Deutsche Telekom AG into your preface. For normal
% quotations please use the scientific standard to cite.
%
% [ File structure derived from 'mind your Scholar Research Framework' 
%   mycsrf (c) K. Reincke CC BY 3.0  http://mycsrf.fodina.de/ ]


%\abbr[aaO]{a.a.O.}{am angegebenen Ort}
%\abbr[ds]{ds.}{kollektiv für ders., dies., \ldots}
\abbr[etseqq]{et seqq.}{and the following ones}
\abbr[id]{id.}{idem = latin for 'the same', be it a man, woman or a group\ldots}
\abbr[ibid]{ibid.}{ibidem = latin for 'at the same place'}
\abbr[ifross]{ifross}{Institut für Rechtsfragen der Freien und Open Source
Software}
\abbr[lc]{l.c.}{loco citato = latin for 'in the place cited'}
\abbr[np]{np.}{no page numbering}
\abbr[wp]{wp.}{webpage / webdocument without any internal (page)numbering}
\abbr[nst]{n.st.}{not stated}
\abbr[njear]{n.y.}{year not stated / no year}
\abbr[nlocation]{n.l.}{location not stated / no location}
\abbr[ub]{UB}{'Universitätsbibliothek' = library of university X}
\abbr[ulb]{ULB}{'Universitäts- \& Landesbibliothek' = library of university and state X}
\abbr[apl]{ApL}{Apache License}
\abbr[bsd]{BSD}{Berkeley Software Distrobution (License)}
\abbr[mit]{MIT}{Massachusetts Institute of Technology (License)}
\abbr[mspl]{Ms-PL}{Microsoft Public License}
\abbr[pgl]{PgL}{Postgres License}
\abbr[php]{PHP}{PHP (License)}
\abbr[epl]{EPL}{Eclipse Public License}
\abbr[eupl]{EUPL}{European Union Public License}
\abbr[lgpl]{LGPL}{GNU Lesser General Public License}
\abbr[mpl]{MPL}{Mozilla Public License}
\abbr[gpl]{GPL}{GNU General Public License}
\abbr[agpl]{AGPL}{GNU Affero General Public License}
\abbr[nabbr]{n.abbr.}{no abbreviation (known)}

% Local Variables:
% mode: latex
% fill-column: 80
% End:

% Telekom osCompendium English Nomenclation Tokens Include Module 
%
% (c) Karsten Reincke, Deutsche Telekom AG, Darmstadt 2011
%
% This LaTeX-File is licensed under the Creative Commons Attribution-ShareAlike
% 3.0 Germany License (http://creativecommons.org/licenses/by-sa/3.0/de/): Feel
% free 'to share (to copy, distribute and transmit)' or 'to remix (to adapt)'
% it, if you '... distribute the resulting work under the same or similar
% license to this one' and if you respect how 'you must attribute the work in
% the manner specified by the author ...':
%
% In an internet based reuse please link the reused parts to www.telekom.com and
% mention the original authors and Deutsche Telekom AG in a suitable manner. In
% a paper-like reuse please insert a short hint to www.telekom.com and to the
% original authors and Deutsche Telekom AG into your preface. For normal
% quotations please use the scientific standard to cite.
%
% [ Derived from 'mykeds Scholar Research Framework' 
%   mykeds-CSR-framework (c) K. Reincke CC BY 3.0  http://www.mykeds.net/ ]

%\abbr[]{[n.abbr.]}{ }
\abbr[zge]{ZGE / IPJ}{Zeitschrift für geistiges Eigentum [ISSN: 1867-237x]}
\abbr[itrb]{ITRB}{Der IT-Rechtsberater [ISSN: 1617-1527]}
\abbr[cri]{CRi}{Computer Law Review international [ISSN: 1610-7608]}
\abbr[btlj]{[n.abbr.]}{Berkeley Technology Law Journal}
\abbr[eclr]{E.C.L.R.}{European Competition Law Review}
\abbr[iesw]{[n.abbr.]}{IEEE Software [ISSN: 0740-7459]}
\abbr[cuitj]{[n.abbr.]}{Cutter IT Journal [ISSN: 1048-5600]}
\abbr[uoclr]{[n.abbr.]}{University of Chicago Law Review}
\abbr[uoilr]{[n.abbr.]}{University of Illinois Law Review}
\abbr[uoplr]{[n.abbr.]}{University of Pittsburgh Law Review}
\abbr[ddt]{DDT}{Drug Discovery Today [ISSN: 1359-6446]}
\abbr[rdm]{[n.abbr.]}{R\&D Management [ISSN: 1467-9310]}
\abbr[jleo]{JLEO}{Journal of Law, Economics, \& Organization [ISSN: 1465-7341]}
\abbr[ijomi]{[n.abbr.]}{International Journal of Medical Informatics [ISSN: 1386-5056]}
\abbr[slr]{[n.abbr.]}{Stanford Law Review [ISSN: 00389765]}
\abbr[bise]{BISE}{Business \& Information Systems Engineering [ISSN: 1867-0202]}
\abbr[joals]{[n.abbr.]}{Journal of Academic Librarianship [ISSN: 0099-1333]}
\abbr[eait]{[n.abbr.]}{Ethics and Information Technology [ISSN: 1388-1957]}
\abbr[jais]{JAIS}{Journal of the Association for Information Systems [ISSN:
1536-9323]}
\abbr[josas]{[n.abbr.]}{Journal of Systems and Software [ISSN: 0164-1212]}
\abbr[iialr]{[n.abbr.]}{International Information and Library Review [ISSN: 1057-2317]}
\abbr[sthv]{STHV}{Science, Technology \& Human Values [ISSN: 0162-2439]}
\abbr[cue]{[n.abbr.]}{Computers \& Education [ISSN: 0360-1315]}
\abbr[eer]{EER}{European Economic Review [ISSN: 0014-2921]}
\abbr[icc]{ICC}{Industrial and Corporate Change [ISSN: 0960-6491]}
\abbr[ca]{[n.abbr.]}{Cultural Anthropology [ISSN: 1548-1360]}
\abbr[sqj]{[n.abbr.]}{Software Qualilty Journal [ISSN: 0963-9314]}
\abbr[jmir]{JMIR}{Journal of Medical Information Research [ISSN: 1438-8871]}
\abbr[joce]{[n.abbr.]}{Journal of Comparative Economics [ISSN: 0147-5967]}
\abbr[orgsci]{[n.abbr.]}{Organization Science [ISSN: 1047-7039]}
\abbr[iam]{[n.abbr.]}{Information \& Management [ISSN: 0378-7206]}
\abbr[rp]{RP}{Research Policy [ISSN: 0048-7333]}
\abbr[jsis]{JSIS}{Journal of Strategic Information Systems [ISSN: 0963-8687]}
\abbr[isj]{ISJ}{Information Systems Journal [ISSN: 1365-2575]}
\abbr[jise]{JISE}{Journal of Information Science and Engineering [ISSN:
1016-2364]}
\abbr[dss]{DSS}{Decision Support Systems [ISSN: 0167-9236]}
\abbr[cihp]{CiHB}{Computers in Human Behavior [ISSN: 0747-5632]}
\abbr[iep]{IEaP}{Information Economics and Policy [ISSN: 0167-6245]}
\abbr[tosem]{ToSEM}{Transactions on Software Engineering Methodology [ISSN:
1049-331X]}
\abbr[commacm]{CotACM}{Communications of the ACM [ISSN: 0001-0782]}
\abbr[interactions]{[n.abbr.]}{interactions[ISSN: 1072-5520]}
\abbr[jcsc]{JCSC}{Journal of Computing Sciences in [Small] Colleges [ISSN:
1937-4771]}
\abbr[linuxjournal]{LJ}{Linux Journal [ISSN: 1075-3583]}
\abbr[networker]{[n.abbr.]}{netWorker [ISSN: 1091-3556]}
\abbr[queue]{[n.abbr.]}{Queue [ISSN: 1542-7730]}
\abbr[sigmisdb]{SIGMIS Database}{ACM SIGMIS - The Data Base for Advances in
Information Systems [ISSN: 0095-0033]}
\abbr[sigcas]{SIGCAS}{ACM SIGCAS Computers and Society [ISSN: 0095-2737]}
\abbr[sigsoft]{SIGSOFT SEN}{SIGSOFT Software Engineering Notes [ISSN:
0163-5948]}
\abbr[toit]{ToIT}{Transaction on Internet Technology [ISSN: 1533-5399]}
\abbr[sigbul]{SIGCSE Bulletin}{SIGCSE Bulletin [ISSN: 0097-8418]}
\abbr[ubiquity]{Ubiquity}{Ubiquity - The ACM IT Magazine and Forum [ISSN:
1530-2180]}
\abbr[bwv]{BWV}{Berliner Wissenschafts-Verlag GmbH}
\abbr[cr]{CR}{Computer und Recht. Zeitschrift für die Praxis des Rechts der
Informationstechnologien}

% Local Variables:
% mode: latex
% fill-column: 80
% End:

\printnomenclature

\bibliography{bibfiles/oscResourcesEn}


\end{document}
