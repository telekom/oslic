% Telekom osCompendium 'for being included' snippet template
%
% (c) Karsten Reincke, Deutsche Telekom AG, Darmstadt 2011
%
% This LaTeX-File is licensed under the Creative Commons Attribution-ShareAlike
% 3.0 Germany License (http://creativecommons.org/licenses/by-sa/3.0/de/): Feel
% free 'to share (to copy, distribute and transmit)' or 'to remix (to adapt)'
% it, if you '... distribute the resulting work under the same or similar
% license to this one' and if you respect how 'you must attribute the work in
% the manner specified by the author ...':
%
% In an internet based reuse please link the reused parts to www.telekom.com and
% mention the original authors and Deutsche Telekom AG in a suitable manner. In
% a paper-like reuse please insert a short hint to www.telekom.com and to the
% original authors and Deutsche Telekom AG into your preface. For normal
% quotations please use the scientific standard to cite.
%
% [ File structure derived from 'mind your Scholar Research Framework' 
%   mycsrf (c) K. Reincke CC BY 3.0  http://mycsrf.fodina.de/ ]
%

% Chapter Abstract
% ----------------

\footnotesize \begin{quote}\itshape This section outlines reflections by which
we initially focused ourselves on the question why we need an OSLiC and how its
content and form should be derivated from these needs.
\end{quote}
\normalsize{}

% Telekom osCompendium 'for being included' snippet template
%
% (c) Karsten Reincke, Deutsche Telekom AG, Darmstadt 2011
%
% This LaTeX-File is licensed under the Creative Commons Attribution-ShareAlike
% 3.0 Germany License (http://creativecommons.org/licenses/by-sa/3.0/de/): Feel
% free 'to share (to copy, distribute and transmit)' or 'to remix (to adapt)'
% it, if you '... distribute the resulting work under the same or similar
% license to this one' and if you respect how 'you must attribute the work in
% the manner specified by the author ...':
%
% In an internet based reuse please link the reused parts to www.telekom.com and
% mention the original authors and Deutsche Telekom AG in a suitable manner. In
% a paper-like reuse please insert a short hint to www.telekom.com and to the
% original authors and Deutsche Telekom AG into your preface. For normal
% quotations please use the scientific standard to cite.
%
% [ Framework derived from 'mind your Scholar Research Framework' 
%   mycsrf (c) K. Reincke 2012 CC BY 3.0  http://mycsrf.fodina.de/ ]
%


%% use all entries of the bibliography
%\nocite{*}

\subsection{Why}

Do we need another book about Open Source? Do \emph{you} need another book about
Open Source Software? Let us address this question from the viewpoint of what we
already know, what we instinctively believe and what we may have heard. For
example you may presume one or more of the following statements is correct. Or
you may even have experienced similar perceptions from your peers or managers.
Or you have been told they describe 'Open Source':

\begin{itemize}
  \item The Open Source Definition offers rules to use Open Source Software.
  \item Modified Open Source Software must be published.
  \item Modified Open Source Software must be given back to the community.
  \item All generations of Open Source Software will remain open for ever.
  \item Software can either be Open Source Software or proprietary software.
  \item The opposite of Open Source Software is commercial software.
  \item Open Source Software prohibits to earn money.
  \item Modifications of Open Source Software must be marked explicitly.
  \item Modifiers of Open Source Software must identify themselves.
  \item When distributing an Open Source binary it’s enough point to a download
  page to obtain the source code.
  \item The aim of Open Source Software is to improve the world ethically.
  \item Open Source Software is viral and infectious.
\end{itemize}

Do these conceptions sound familiar to you? Unfortunately, whatever we might
believe or wish for, these concepts are incorrect. Naturally we will discuss
this issue later on. For the moment let us assume they are indeed
incorrect\footnote{For those who want directly verify our argumentation, we have
generated a condensed summary of the arguments and citations. You can find this
summary in our appendices.}.

So, again: Do \emph{we} need another book about Open Source Software? \emph{We},
that is - in this case and at least initially - the large German company
\textit{Deutsche Telekom AG}. Arguing from the perspective of a large company
requires not only identifying the common misconceptions, but catering for the
unique needs of a large Enterprise. And indeed the very size of the company
brings its own problems.

Large companies use more Open Source Software in more varied contexts than small
companies. There is an important question that every company should ask:
\emph{'Are we sure that we respect all those requirements of Open Source
Software we have to respect?'}. But large companies can not answer this question
as easily as small companies: the large number of diverse Open Source
deployments in different contexts mean that case by case governance, a model
that may work in small concerns, is far from appropriate for our needs. This
leads to wasting both time and money. Further, the chances of success are small:
training at least one employee in each software team as an Open Source Software
License expert is unrealistic in terms of cost-efficiency and reliability.

Nevertheless even large companies want to and try to fulfill the rules of Open
Source Software thoroughly - especially \emph{Deutsche Telekom AG}. When this
company realized that the question \textit{Are we sure that we respect all those
rules of Open Source Software correctly which we have to respect} could be
problematic, it directly asked some of its' employees known as Open Source
enthusiasts to establish a service and a process for answering this question.

So, it is no surprise that we, the initial authors of this \textit{Open Source
License Compendium}, were asked by our employer \emph{Deutsche Telekom AG}.
Naturally we were proud to work on an Open Source topic officially. But while we
were doing our job we had to ask ourselves if \emph{we} perhaps needed another
book on Open Source. Our answer was \textit{Yes, we do!} Let us shortly explain,
why:

First, we already knew that there exists supporting software. These
meta-pro\-grams take the code of any other application and try to list those
Open Source components being 'covered' by that application\footnote{As general
examples let us mention Palamida (\texttt{http://www.palamida.com/}) and
BlackDuck (\texttt{http://www.blackducksoftware.com/}).}. But we had also
already realised that this supporting software did not always match the way we
thought the problem should be solved. Second, we recognized fairly quickly that
we need a reliable guide. We personally were asked to give the \emph{ok} for
projects of our company. We could not answer such requests on the base of
\textit{'Oh yes, I read this in the \emph{Heise-Ticker} a few days ago'} - even
if the \emph{Heise-Ticker} had described the situation completely correctly. We
ourselves had to be more reliable than this. Naturally we already knew a great
deal about Open Source Software. Even so, our knowledge was not as systematic as
necessary. We looked for an Open Source Compendium which adequately described
what a project or product development team had to do to fulfill the criteria of
its Open Source Licenses. We wanted to use that compendium to the basis of our
recommendations.

We were very thorough but we did not find what we were looking for. Our 'little'
bibliography attest our seriousness. What we found was a lot of information
releating to individual issues spread over many sources. We did not find answers
for our question even in the specific literature. Let us describe three little
steps to increase the understanding of the issue:


Without Open Source Licenses there is no Open Source movement. Nevertheless in
dealing with Open Source Licenses, this is sometimes neglected. Take the
\emph{Apache Web Server} as an example: No doubt, it's one of the most important
pieces of Open Source Software\footnote{To prove that the \textit{Apache} is
really a piece of Open Source Software one must execute a set of steps: First,
you have to note, that \emph{Apache} is something like a meta project, covered
by the \emph{Apache Software Foundation}, also known as \emph{ASF} (cf.
\texttt{http://www.apache.org/}, wp.). Thus, you can not directly jump into
the \emph{Apache License}. First of all you have to visit the project site (cf.
\texttt{http://httpd.apache.org/}, wp.) even if at the end its' license link
leads you back to the general \emph{Apache License sub site} (cf.
\texttt{http://www.apache.org/licenses/}, wp.) which announces, that \enquote{all
software produced by The Apache Software Foundation or any of its projects or
subjects is licensed according to the terms of the documents listed
below}. Only now you can use the offered link for switching to the
\emph{Apache License}, Version 2.0, if you want to check your rights and duties.
But that is difficult. There does not exist any simple list what you have to do
for fulfilling the license. Even the faq (cf.
\texttt{http://httpd.apache.org/docs/2.2/faq/}, wp.) - meanwhile being moved to
a wiki - only says that the server \enquote{[\ldots] comes with an unrestrictive
license} and that you are allowed to put the code on a CD (cf.
\texttt{http://wiki.apache.org/httpd/FAQ}, wp.). Hence, from the viewpoint of
the ASF the license itself shall answer all questions. [Reference download for
all urls: 2011-08-31] } with a specific license\footcite[cf.][\nopage
wp.]{AsfApacheLicense20a}. Moreover: the success of the Open Source movement
in the commercial world depends directly on the decision of IBM to replace its
corresponding own component in the \textit{IBM WebSphere Application Server}
with the free \textit{Apache Web Server}\footcite[cf.][287ff]{Moody2001a}.
Meanwhile many companies use the \textit{Apache Web Server} to act as a web
provider. Currently the \emph{Apache http server} - as it has to be named
correctly - is used more than twice as much as all the other http server
software together\footcite[cf.][\nopage wp]{Netcraft2011a}. Hence many business
models depend on the Apache License. Another aspect is that even the famous
\emph{Apache Cookbook}, which explains the installation, the configuration, and
the maintainance of an Apache Web Server in details\footcite[cf.][\nopage et
passim]{CoaBow2004a}, does not mention anything about the license which allows
for installation, configuration and maintenance. Neither the index lists the
word 'license'\footcite[cf.][245ff, esp. p. 250]{CoaBow2004a}, nor the chapters
'Installation'\footcite[cf.][1ff]{CoaBow2004a} or the chapter
'Miscellaneous'\footcite[cf.][219ff]{CoaBow2004a} mentions the license question
in a serious way. There's only one short hint as to the advantage of Open Source
Software, i.e. that everybody is allowed to install it\footcite[cf.][1: \enquote{
\ldots einer der Vorzüge von Open Source Software besteht darin, dass
je\-der\-mann die Erlaubnis zur Erzeugung eines eigenen Installationskits hat
}]{CoaBow2004a}. Can you be sure that you are allowed to do what you are
doing on the base of such a phrase?

Naturally, the \emph{Apache Cookbook} is not a book for lawyers, it's a book for
administrators and developers. They do not want to get bogged down by
legalities, they want to set up an Apache Web Server as fast as possible and get
down to work. Indeed, the Apache Cookbook offers a good support. But not only as
a company you have to ask yourself whether you are really allowed to do what you
are doing. Can you find the answer in the \emph{Apache Cookbook}? No. Can you
find it in the license itself? Yes, but it is difficult\footnote{And do we
really want our developers and maintainers to read the original licenses? Do we
really want them to discover that they also have to check the licenses of the
used modules?}. So again: Can you find your answer in another book, which is
\emph{Amazon's} current top recommendation for the request \emph{'apache
server'}\footnote{Tested on \texttt{http://www.amazon.de/} at 2011-08-31.}? Not
really: Sascha Kersken's Apache 2.2 Handbook offers a license chapter, but it's
only two pages long\footcite[cf.][111f]{Kersken2009a}. Moreover, the rights and
duties are condensed into just 5 bullet points which taken together do not
explain when the software and the license have to be handed over to a customer
and when you are allowed to hide your
improvements\footcite[cf.][112]{Kersken2009a}.

This brings us to the question of what prevents us from using something like a
\emph{'general license cookbook'} which explains all the necessary details and
which offers  quick access to the relevant points:

Of course we also browsed the internet. At least for German speaking people
there is an excellent site concerning the topic \emph{Open Source Licenses}.
offered by \textit{iffross}, which, loosely translated, means an
\textit{Institute for Legal Aspects of the Free and Open Source
Software}\footnote{originally: \enquote{Institut für Rechtsfragen der Freien und
Open Source Software}. Main entry point for its' site is the URL
\texttt{http://www.ifross.org/}.}, founded in 2000 as a private institute to
track the phenomenon 'free software' from the viewpoint of (German)
lawyers\footcite[cf.][\nopage wp]{ifross2011b}. Besides many other
aspects this site offers a very well and thoroughly elaborated
FAQ\footcite[cf.][\nopage wp]{ifross2011c} and a large list of Open
Source Licenses and other related licenses: moreover, evidently it is
classifying the Open Source Licenses in those 'without copyleft-effect' (BSD),
in those with 'strict copyleft-effect' (GPL)) and in those with 'restricted
copyleft-effect' (LGPL)\footcite[cf.][\nopage wp]{ifross2011a}.

However, even this excellent site does not fulfill our needs. It does not offer
those context specific to-do lists which companies, developers or project
managers can use to ensure their Open Source Software is used in a regular
manner.

We therefore evaluated that standard book which is listed in the most legal
bibliographies\footnote{at least in that German judicial literature dealing with
Open Source}: the book of Jaeger and Metzger which concerns - loosely translated
- \textit{the judicial framework requirement for Open Source
Software}\footcite[cf.][V - It can not be any surprise that both authors,
Mr. Jaeger and Mr. Metzger are members of ifross (cf.
\texttt{http://www.ifross.org/personen/}, wp.)]{JaeMet2002a}. Even the most
earliest edition of this book already had a clear structure in its' chapter
'copyright': For each license mentioned (or at least for each license cluster)
it offered a subchapter for the rights and a subchapter for the
duties\footcite[cf.][30ff]{JaeMet2002a} of the software user\footcite[For
getting a good survey of the structure and the line of thought see the contents
cf.][VIIIf]{JaeMet2002a}. Many other important aspects of the topic
\textit{Open Source} are discussed, too\footcite[pars pro toto: have a
look at the chapter concerning the liability: cf.][137ff]{JaeMet2002a}.

But we needed more than this. Despite the quality of the book we were certain
that we could not hand over this book to our programmers with the recommendation
\textit{check your touched licenses and follow the instructions of the relevant
subchapters\ldots}. This book did not contain simply checkable to-do lists,
neither in the first edition\footcite[cf.][VIff]{JaeMet2002a} and in the second
edition\footcite[cf.][VIIff]{JaeMet2006a} nor in the recently published third
edition\footcite[cf.][VIIIff. Naturally we use this latest edition for adopting
or discussing systematical aspects]{JaeMet2011a}. So, how can a company or a
developer or a project manager be sure of fulfilling the requirements of the
Open Source Licenses sufficiently if he/she does not have a verified list
telling him \textit{'do this, and in case of that, do that, and finally do also
this'}? Why should he himself implicitly become an Open Source Licenses expert
who has to extract the necessary steps out of the literature?

While we were searching for an existing Open Source compendium, we found an
article with the title 'Compendium for the Publication of Open Source
Software'\footnote{approximately translated}. It aims to be a 'pragmatic
guidebook' and an 'assistance' for 'publishing software under the conditions of
an Open Source License'\footcite[cf.][166f (originally: ein
\enquote{pragmatischer Ratgeber} zur \enquote{Veröffentlichung einer Software
unter den Rahmenbedingungen einer Open-Source-Lizenz}) ]{BreGlaGra2008a}.
Moreover, at the end of this article, its' authors formulate ambitiously that
their 'guide' should be carried out, section by section - for getting a legally
water tight process of publishing Open Source software\footcite[cf.][186
(originally: ein \enquote{Ratgeber}, der es erlaubt \enquote{ (\ldots) die zu
berücksichtigende Aspekte (strukturiert abzuarbeiten) (\ldots) } und einen
\enquote{rechtlich nicht angreifbaren Veröffentlichungsprozess} zu
ermöglichen) ]{BreGlaGra2008a}.

The authors of this article describe something close to what we were looking
for. Indeed, the article lists important aspects which have to be taken in
consideration if you want to deal Open Source Software correctly: It announces
that no obligation exists to publish code either if you embed GPL code into your
proprietary code or if you modify the GPL code. It is only if you hand over your
binary to other persons that you have to distribute the code too, but only to
them and not to the general public\footcite[cf.][170 and 181]{BreGlaGra2008a}.
Additionally the articles explains exactly that software - at least in Germany -
can only be acknowledged as Open Source Software by transferring the rights to
use - the \emph{'Nutzungsrechte'} - to other people, while the copyright itself
- the \emph{'Urheberpersönlichkeitsrecht'} - is not transferable and belongs to
the author\footcite[cf.][173]{BreGlaGra2008a}. Moreover, besides other aspects
the articles briefly and deeply discusses the problem of the No-Warranty-Clauses
which are not valid in Germany and which will therefore automatically be
replaced by the liability rules for a
donation\footcite[cf.][177]{BreGlaGra2008a}. And last but not least this article
actually summarizes the idea of Copyleft and the differences between LGPL and
GPL\footcite[cf.][181]{BreGlaGra2008a}.

However some gaps remain. The article does not analyze in which cases a
University or a company perhaps \emph{must} publish its' developments based
on Open Source Software. It does not discern between different licenses
and conditions. It also does not discuss what Universities or companies,
which (re-)use and/or distribute Open Source Software (internally), must do to
fulfill the touched Open Source Licenses. And finally this article
does not offer the step by step list as promised.

We did, however, feel supported by this article, in two ways. First, it was a
well written summary of some main problems. Second, it stated the necessity to
have a compendium for being able to establish a legally 'water-tight' process of
publishing Open Source software\footcite[cf.][186]{BreGlaGra2008a}. We
seemed to be justified in our assumptions. But the Open Source Compendium we
were looking for had to be more practical, more processable, more distinguishing
and more elaborated.

So again: Did we need a new book about Open Source Software? We had looked for a
reliable integrated Open Source Compendium. But we found separate pieces of
information and - as we know today - some rumors. Our answer was clear:
naturally we did not need a new general book about Open Source. But what was
lacking was a description of what responsible developers, project managers or
product developers require to fulfill Open Source Licenses. We needed an
\textit{Open Source License Compendium}.

At the best such an \textit{Open Source License Compendium} would contain a set
of simply to process \textit{'For-Fulfilling-The-License-To-Do-Lists'}.
Additionally it should offer an intuitively user-friendly search option for
these lists. In any case, it should share developers and project managers the
effort of having to become Open Source License experts. For the other users, it
should also clearly explain why one has to do this and not that. Hence a
reliable \textit{Open Source License Compendium} should not only list what one
has to do, but should offer both, thoroughly verified reliable details and
clearly condensed guidance.

Although we did not find such an Open Source Compendium we were familiar with
the spirit of the Open Source Community. Hence we followed one of its' most
simple rules: \emph{'what you miss you must develop on your own'}. Some
principles should help us to achieve our targets:

\begin{description}
  \item[To-do lists as the core, discussions around them]: Our work should be
  split into two parts. As it core we wanted to offer a
  set of To-Do-Lists. Each of these lists should be relevant to one specific
  Open Source License and should be clustered by the Open Source specific use
  cases. Around this all those aspects of Open Source Software which influence the
  interpretation of the licenses and the rules core should be precisely
  characterized. Nevertheless, the users should be able to skip
  details and go directly to the section they require.
  \item[Quotations with thoroughly specified sources]: Even if our users should
  not be obliged to read every part of the compendium they should not be
  required to believe us. We wanted to be revisable. Because our sources and our
  conclusions should be easily verifiable, we decided to use the academic
  citations and list bibliographic data extensively on the basis that our task
  should be to collect information, not to invent new 'facts'.
  \item[Not the internet alone, also books and articles]: We wanted to go back
  to the originals even if the internet was full of more or less modified
  copies. We wished to get reliable facts and descriptions. Therefore we decided
  to evaluate not only the internet but also scientific sources - for example -
  offered by university libraries.
  \item[Not clearing out the forest land, but cutting out a swathe]: Even if we
  had to deal with licenses and their legal aspects we did not want to get lost
  in detailed discussions. It should not be our task to find out whether a
  specific kind of handling would still be legal or already forbidden.
  We did not want to fight against the licenses. We did not want to stretch
  their ambit or to test their boundary. We wished to accept Open Source
  Licenses as they are: rules written from developers for developers. And even
  if some parts of these licenses would not be valid with respect to a legal
  system\footcite[And indeed for example for the GPL one can argue in this way:
  Even if you take the GPL as a contract of the type 'donation' respectively
  \enquote{Schenkung}, it is presented in the form of AGBs respectively
  \enquote{Allgemeine Geschäftsbedingungen} and must therefore follow the
  general AGB rules.'Regrettably' in Germany these general AGB rules do not
  allow to exclude each type of warranty. If we follow Oberhem, §11 and §12 of
  the GPL must be invalid in Germany because of these general AGB rules.
  Moreover, for Oberhem even §5 - the important clause of the GPL by which you
  can only get the right to use and to distribute GPL software if you respect
  the rules of the GPL - seems also to be invalid respectively
  \enquote{unwirksam}. But the good message is that the GPL as whole is not
  invalid even if it contains invalid clauses.][128, 133ff, 150ff, esp. 146,
  159]{Oberhem2008a}, we wanted to take them as our guideline - at least while
  they do not violate more general laws\footnote{what they clearly do not do!}.
  We simply wanted to \emph{find one proven way} to cross the maybe slightly
  unsure forest of Open Source Licenses. Even if indeed some clauses of the
  licenses finally were not enforceable against us we wanted to respect them
  'voluntarily'. We wanted to deliver a set of rules which support users and
  remove the possibility of becoming involved in license disputes with Open
  Source developers or the Free Software Foundation.
  \item[Take the text seriously]: On the other side we wanted to take our
  license texts as they were. If they lacked anything\footcite[The systematical
  underdetermination of licenses is a problem being also known in the Open
  Source respectively Free Software movement. Following the biography of RMS his
  main judicial counselor Moglen has stated, that \enquote{there is uncertainty
  in every legal process (\ldots) } and that it seemed to be silly to try
  \enquote{(\ldots) to take out all the bugs (\ldots)}. Nevertheless - so
  Moglen resp. Williams - the goal of Richard Stallman was \enquote{the complete
  opposite}: He tried \enquote{(\ldots) to remove uncertainty which is
  inherently impossible}. But - and that's the nub of this analysis -
  Moglen had to follow Stallmann because of RMS character. And he had to
  summarize their work so, that \enquote{(\ldots) the resulting elegance (of the
  GPL; KR.), the resulting simplicity (of the GPL; KR.) in design almost
  achieves what it has to achieve}. Hence we are asked to take the license
  texts themselves seriously. cf.][177f]{Williams2002a}, we would interpret the
  open issues in the spirit of the Open Source idea. But where the text was
  clear and definite we wanted to take its propositions as a definite decision -
  even if that meaning stood against well known Open Source 'facts'.
  \item[Trust the swarm]: We did not want to use our own research alone as a
  basis. We knew that the swarm is ever stronger than a set of some randomly
  selected experts. Therefore we decided to publish our text as a still
  unfinished work, starting with an early release 0.2. And then we wanted
  to invite the community to complete the compendium together with us. We would elaborate our Open
  Source Compendium as a set of LaTeX- and BibTeX files which could be developed
  and managed in GIT or any other version control system. And finally we would
  publish our text under a Creative Commons Attribution-Share Alike German 3.0
  license, to allow other people to correct
  us, to help us or even to take our results for their own purposes.
\end{description}

And so we did. Here is the result. Feel free to use it - according to our
licensing.

\subsection{What}

Now we can briefly explain how one should be able to use the compendium:

\begin{description}
  \item[The Same Idea, Different Licenses] :- Here you will find background
  information to help you interpret Open Source Licenses in the sense of the
  \emph{Free Software movement}\footcite[At least at this place you are perhaps
  expecting that we use the logograms FLOSS, F/OSS, F/LOSS, or whatever. As you
  will read later on the word \textit{Free} is ambiguous and has strained the
  use of the concept \textit{Free Software}. Later on we will also talk about
  the invention of the concept \textit{Open Source} designed as a 'replacement'
  and acting as a 'splitter'. The mentioned logograms are introduced to
  re-establish or - at least - to underline the common history and the common
  center of 'both' movements, whereby the word \textit{Libre} shall resolve the
  ambiguity of the word \textit{Free}. For a first survey cf.] [\nopage
  wp.]{wpFloss2011a}, the \emph{Open Source Software movement}\footcite[For
  another brief and informative introduction cf.][231ff esp. p.
  232f.]{Fogel2006a} or the GNU-Project\footnote{ We ourselves will stay with the
  concept \textit{Open Source} because the OSD specifies the scope of our
  analysis. But we do it with a deep obeisance to Stallmann and the FSF - even
  if we know that this will not protect us from the thunderbolt of RMS.}. We discuss
  different ways to cluster Open Source Licenses. Finally we present our own
  taxonomy based on the labels 'protecting the developer', 'protecting the
  licensed code' and 'protecting the on-top-developments'. If you are familiar
  with the methods of grouping different Open Source Licenses and particular
  if you know that you can not authorize your doings on the base of descriptions
  of such license groups, then it's enough, in order to understand our line of
  thought, to briefly note our taxonomy and its wording.
  \item[The Problem of Derivated Works] :- This chapter is important. In the
  spirit of software developer we try to explain which kinds of programming
  evoke a derivated work and which not. Our to-do lists will refer to this
  analysis.
  \item[The Problem of Combining Different Licenses] :- You should
  not ignore this chapter. We will explain why and how combining software
  of different licenses is not as dangerous as it's often told. The results of
  this chapter influence the structure of our to-do lists.
  \item[Open Source Software and Money] :- Here we will shortly
  discuss ways in which money is no problem. If you already know that it is only
  prohibited to require payment for the act of licensing a piece of Open Source
  Software to second or third parties and if you already know that this is only
  forbidden by some licenses, and not by all, than you can postpone the reading
  of this chapter.
  \item[The Problem of Implicitly Freeing Patents] :- Here we
  will illuminate some aspects of software patents and how the are handled by
  some Open Source Licenses. You should know what licenses implicitly do with
  your patents. But it's not our intention to write a software patent
  compendium.
  \item[Open Source: Use Cases as Principle of Classification] :- This is an
  important chapter. We explain our categories 'Use as it is', 'Modify the
  Code', 'With Redistribution', 'Without Redistribution', 'Isolated Initial
  Development', 'On-Top-Development': we develop and discuss our taxonomy with
  respect to the side effects of 'combining different licenses' and 'generating
  derivated works'. This taxonomy will determine the following chapters.
  \item[Open Source Licenses: Find Your Specific To-do Lists] :- This is a kind
  of summary which joins the relevant aspects and elaborates the 'finder
  for your to-do lists'. This is that chapter which you probably will reuse
  multiply, even if you do not want to read any of our explanations.
  \item[Open Source License Fulfillment: Classified To-do Lists] :- This chapter
  offers all classified to-do lists. The structure of its' subchapters will
  match the structure of our finder and the structure of our taxonomy.
  \item[Open Source Licenses and Their Legal Environments] :- Here we discuss
  why using Open Source Software in a regular manner is not only a question of
  the licenses themselves but of the kind of the surrounding legal system.
  \item[Appendices: Some Widespread Open Source Myths] :- Here we make good on
  our promise to explain why all the propositions mentioned at the beginning of
  this chapter are wrong. You might read this chapter as a special introduction
  or a reminder epilogue whenever you want to do.
\end{description}


%\bibliography{../../../bibfiles/oscResourcesEn}
