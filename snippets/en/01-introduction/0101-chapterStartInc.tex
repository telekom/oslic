% Telekom osCompendium 'for being included' snippet template
%
% (c) Karsten Reincke, Deutsche Telekom AG, Darmstadt 2011
%
% This LaTeX-File is licensed under the Creative Commons Attribution-ShareAlike
% 3.0 Germany License (http://creativecommons.org/licenses/by-sa/3.0/de/): Feel
% free 'to share (to copy, distribute and transmit)' or 'to remix (to adapt)'
% it, if you '... distribute the resulting work under the same or similar
% license to this one' and if you respect how 'you must attribute the work in
% the manner specified by the author ...':
%
% In an internet based reuse please link the reused parts to www.telekom.com and
% mention the original authors and Deutsche Telekom AG in a suitable manner. In
% a paper-like reuse please insert a short hint to www.telekom.com and to the
% original authors and Deutsche Telekom AG into your preface. For normal
% quotations please use the scientific standard to cite.
%
% [ Framework derived from 'mind your Scholar Research Framework' 
%   mycsrf (c) K. Reincke 2012 CC BY 3.0  http://mycsrf.fodina.de/ ]
%


%% use all entries of the bibliography
%\nocite{*}

There is only one leading question, which determines the purpose, the structure
and the style of this little \emph{Open Source License Compendium}: What do we
have to do for acting according to the licenses of the \emph{Open Source
Software} we use? 

Naturally, we want to behave license compliantly. But we had a specific
challenge. 'we', that was \emph{Deutsche Telekom AG} as whole and especially
some of its software developers and project managers.

Our problem was, that we could not find a compendium which would explain us
easily how we concretely should act in our specific circumstances. As workers in
projects, we did not want to become Open Source license experts for being able
to use Open Source software correctly. We did not want to become lawyer. We
wanted to do in a simplier way, what we had done before complicatedly expanded
by additional, disturbing license evaluating working steps - using good software
in a legal manner to improve our performance. What we needed, was a simply to
use handout which would lead us without any detours to executable lists of
working items. We wished to get To-Do-Lists, dedicated to our usecases and our
licenses. We needed reliable lists of tasks we only had to execute for being
sure that we were acting according to the Open Source License. As we started,
such a compendium did not exist.

The first decision our company came to was to allow to a small group of
employees to offer a service for the whole company: projects, managers, and
developers should be enabled to ask a board what they have to do for fulfilling
a specific Open Source License under specific circumstances. So, the Telekom
Open Source Review Board was founded as an internal bottom up community.

The second decision our company took was to allow this Telekom OSRB to collect
their results systematically and to publish them in the style of an Open Source
project. The idea behind that decision was simple: The more the internal service
would become known, there more work would be generated. And normally, more work
means more man power. So, it was a good cost saving idea, to enable the
requestors to find their answers in a compendium instead of making it necessary,
to let their work being analyzed by the OSRB. Finally, the OSRB should be \ldots






[TDB \ldots]

%\bibliography{../../../bibfiles/oscResourcesEn}
