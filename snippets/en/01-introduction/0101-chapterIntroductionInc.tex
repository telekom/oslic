% Telekom osCompendium 'for being included' snippet template
%
% (c) Karsten Reincke, Deutsche Telekom AG, Darmstadt 2011
%
% This LaTeX-File is licensed under the Creative Commons Attribution-ShareAlike
% 3.0 Germany License (http://creativecommons.org/licenses/by-sa/3.0/de/): Feel
% free 'to share (to copy, distribute and transmit)' or 'to remix (to adapt)'
% it, if you '... distribute the resulting work under the same or similar
% license to this one' and if you respect how 'you must attribute the work in
% the manner specified by the author ...':
%
% In an internet based reuse please link the reused parts to www.telekom.com and
% mention the original authors and Deutsche Telekom AG in a suitable manner. In
% a paper-like reuse please insert a short hint to www.telekom.com and to the
% original authors and Deutsche Telekom AG into your preface. For normal
% quotations please use the scientific standard to cite.
%
% [ Framework derived from 'mind your Scholar Research Framework' 
%   mycsrf (c) K. Reincke 2012 CC BY 3.0  http://mycsrf.fodina.de/ ]
%


%% use all entries of the bibliography
%\nocite{*}


\chapter{Introduction}

% Abstract
\footnotesize \begin{quote}\itshape This chapter shortly describes the idea of
the OSLiC, the way it should be used, and the way it can be read - what indeed
is not completely the same.
\end{quote}
\normalsize{}

% Content
This book focuses on only one issue: \emph{What do we have to do for acting
according to the licenses of those \emph{Open Source Software} we use?} The
\emph{Open Source License Compendium} wants to answer this question in a simply
to use and easily to understand manner. Thus, it is not another book on \emph{Open
Source} in ge\-ne\-ral\footnote{Meanwhile, there are tons of literature
dealing with Open Source. By improving your knowledge on the base of such books
and articles you might get lost in literature: our list of secondary literature
may adumbrate this 'danger of being overwhelmed'. But nevertheless, our
bibliography at the end of the OSLiC is not complete. Moreover, it's not
intended to be complete. It's only an extract which represents the background
knowledge we did not quoted in the OSLiC. If we were pressured to indicate two
books for getting a good survey on the topic \emph{Open Source (Licenses)} we
would name (a) the 'Rebel Code' (\cite[for a German version cf.][\nopage
passim]{Moody2001a} - \cite[for an English version cf.][passim]{Moody2002a}) and
(b) the 'legal basic conditions' (\cite[cf.][\nopage passim]{JaeMet2011a}). But
fortunately, we are not pressured to do so.}. It shall be no more than a
reliable tool to simplify the license compliant behavior.

This compendium was evoked by a challenge of \emph{Deutsche Telekom AG} and some
of its software developers and project managers: Naturally, they want to behave
license compliantly. But they could not find a reference text which simply lists
what they concretely had to do to fulfill the licenses of that Open Source, they
currently used. As some of these workers in such projects, initially even we did
not want to become Open Source license experts for being able to use Open Source
Software correctly. We did not want to become lawyers. We simply wanted to do in
a simplier way, what in those days claimed much time and many resources. We were
searching for a clear guidance instead of having to determine a correct way
through the djungel of Open Source Licenses - over and over again, project for
project. We liked to use the high-quality Open Source Software in a legal manner
to improve our performance. But we did not like to laboriously discuss the
relevant constraints of the many and different Open Source Licenses.

What we needed, was a simply to use handout which would lead us without any
detours to executable lists of working items. We wished to get To-Do-Lists,
dedicated to our usecases and our licenses. We needed reliable lists of tasks we
only had to execute for being sure that we were acting according to the Open
Source License. When we started, such a compendium did not exist. For solving
this problem our company took two decisions: 

The first decision our company came to, was to support a small group of
employees to act as \emph{a board of Open Source License experts}: They should
offer a service for the whole company. Projects, managers, and developers should
be enabled to ask this board what they have to do for fulfilling a specific Open
Source License under specific circumstances. And this board should answer with
reliable to-do lists, whose processing assures that the requestors are acting
according to their Open Source Licenses. The idea behind this decision was
simple. It save costs and quality if you have a central group of experts instead
of being obliged to select (and to train) developers on each project newly. So,
the \emph{OSRB} - the \emph{Telekom Open Source Review Board} - was founded as
an internal bottom up community.

The second decision our company took, was to allow this \emph{Telekom OSRB} to
collect their results systematically and to elaborate a compendium in that mode
of cooperation, Open Source projects used to do. The idea behind this decision
was also simple: The more the internal service would become known, the more the
workload would increase: the more work, the more recources, the more costs. So,
it was a cost saving idea, to enable the requestors to find their answer by
themselves - but simply without becoming licenses experts: In the default cases,
they should find their answers in a compendium instead of making it necessary,
to let their work being analyzed by the OSRB. Naturally, in these cases, where
experts would no longer directly be involved, the OSLiC had to be particularily
reliable. It is a known feature of the Open Source working model: the ongoing
review by the cooperating community increases the quality. Therefore, the
decision, not to write only an internal Telekom compendium, but to enable the
whole community to use, to modify and to redistribute this \emph{Open Source
License Compendium}, was a decision for improving the quality. Thus, the OSRB
published the OSLiC as a set von LaTex Sources. It decided to make them
accessible via the open repository github\footnote{Get the code by using the
link \texttt{https://github.com/dtag-dbu/oslic}; get project information by
\texttt{http://dtag-dbu.github.com/oslic/} or by
\texttt{http://www.oslic.org/}.}. And it licensed the OSLiC under Creative
Commons Attribution-ShareAlike 3.0 Germany License\footnote{ This text is
licensed under the Creative Commons Attribution-ShareAlike 3.0 Germany License
(\texttt{http://creativecommons.org/licenses/by-sa/3.0/de/}): Feel free
\enquote{to share (to copy, distribute and transmit)} or \enquote{to remix (to
adapt)} it, if you \enquote{[\ldots] distribute the resulting work under the
same or similar license to this one} and if you respect how \enquote{you must
attribute the work in the manner specified by the author(s) [\ldots]}):
In an internet based reuse please mention the initial authors in a suitable
manner, name their sponsor \textit{Deutsche Telekom AG} and link it to
\texttt{http://www.telekom.com}. In a paper-like reuse please insert a short
hint to \texttt{http://www.telekom.com}, to the initial authors, and to their
sponsor \textit{Deutsche Telekom AG} into your preface. For normal quotations
please use the scientific standard to cite.
\newline { \tiny \itshape [LaTeX form derived from myCsrf (= 'mind your Scholar
Research Framework') \copyright K. Reincke CC BY 3.0  http://mycsrf.fodina.de/)]
} }. 

But to publish the \emph{OSLiC} as a free book has another important meaning -
at least for the \emph{Telekom OSRB}: It is also intended to be a thankful
\emph{Giving Back} to the \emph{Open Source Community} which has enriched and
simplified the life of so many employees and companies over so many years.

Howsoever, overall, the OSLiC follows five principles:

\begin{description}
  \item[To-do lists as the core, discussions around them]: Based on simply to
  gather information concerning the concrete use of a piece of Open Source
  Software and its license, the OSLiC shall offer a simply processable finder
  which leads the requestor to the relevant, license compliance assuring to-do
  list. Additionally, all these elements of the OSLiC should comprehensibly
  be introduced and discussed without disturbing the usage itself.

  \item[Quotations with thoroughly specified sources]: The OSLiC shall be
  revisable and reliable. It shall comprehensibly argue and explictly specify
  why it overtakes which information from whom and why.

 \item[Not clearing out the forest land, but cutting out a swathe]: The OSLiC
  shall deal with licenses and their legal aspects. But it shall not discuss all
  details of all aspects. It shall focus on one possible way to act according a
  license in a specific usecase - even if the OSLiC knows, that there might be
  alternatives\footnote{The OSLiC shall not counsel projects with respect to
  their specific needs. This must remain the tasks lawyers and legal experts.
  They can answer the question, whether a project under its' specfic conditions
  could also / better use another way to fulfill the Open Source license.}.
  
  \item[Take the license text seriously]: The OSLiC shall not give general
  lectures on legal discussions. Much less, it shall participate in them. It
  shall only find one reliable way for each license and each usecase to fulfill
  the license. The main source for this analyze shall be the exact reading of
  the Open Source Licenses themselves - based and supported by the
  interpretation of benevolent lawyers and rational argueing software
  developers. The OSLiC shall respect, that Open Source Licenses are written for
  software developers (and sometimes by developers).
  
  \item[Trust the swarm]: The OSLiC shall be open for improvements and
  adjustments encouraged an stimulated also by other people than employees of
  \emph{Deutsche Telekom AG}.
\end{description}

And based on these principles the OSLiC offers two modes to be used:

On the one hand - and most frequently -  the readers want simply and quickly to
find those to-do list which fit their needs. Here is the corresponding process:

\tikzstyle{decision} = [diamond, draw, fill=gray!20, 
    text width=4.5em, text badly centered, node distance=4cm, inner sep=0pt]

\tikzstyle{preparation} = [rectangle, draw, fill=gray!30, 
    text width=10em, text centered, rounded corners, minimum height=4em]
 
\tikzstyle{lprocs} = [rectangle, draw, fill=gray!40, 
    text width=10em, text centered, rounded corners, minimum height=4em]
    
\tikzstyle{processing} = [rectangle, draw, fill=gray!40, node distance=2.5cm,
    text width=15em, text centered, rounded corners, minimum height=4em]
    
\tikzstyle{line} = [draw, -latex']

\tikzstyle{cloud} = [draw, ellipse, text width=6em, text centered, fill=gray!10]
 
    
\begin{tikzpicture}[node distance = 2cm, auto]
\footnotesize
    % Place nodes
    
  \node [cloud] (start) at (1,10) 
    {$\forall$ your \\ Open Source \\ components};
  \node [preparation] (select) at (5,10) 
    {select your next Open Source component};     
  \node [preparation,  below of=select] (analyze) 
    {analyze its' role in the software architecture and 
    determine the use of the app as whole};  
  \node [preparation,  below of=analyze] (determine) 
    {determine its Open Source License};
  \node [lprocs,  below of=determine] (fillin)
    {\textbf{fill in the 5 query form} ($\rightarrow$ p.
    \pageref{OSLiCStandardFormForGatheringInformation})};
    
  \node [decision, right of=fillin] (success) {success?};
  
  \node [processing,  below of=success] (traverse)
    {\textbf{traverse the} taxonomic \textbf{Open Source Use Case Finder}
    ($\rightarrow$ \pageref{OSLiCUseCaseFinder}) and jump to the indicated
    \textbf{O}pen \textbf{S}ource \textbf{U}se \textbf{C}ase page ($\rightarrow$
    \pageref{OSUCList}ff.)};
    
  \node [processing,  below of=traverse] (find)
    {\textbf{Determine} the page of \textbf{the license and use case specific to-do list}
    being presenetd in the license specific chapter};
 
  \node [processing,  below of=find] (process)
    {Jump to the indicated page and \textbf{process the license and use case specific
    to-do list} ($\rightarrow$ \pageref{OSUCToDoLists}ff.)};
    
  \node [decision, right of=process] (other) {more?};
  \node [cloud, below of=other] (stop) {stop};

  \path [line] (start) -- (select);  
     
  \path [line] (select) -- (analyze);      
  \path [line] (analyze) -- (determine);         
  \path [line] (determine) -- (fillin);
  \path [line] (fillin) -- (success);
  
  \path [line] (success) |- node [near start] {no} (analyze);
  \path [line] (success) -- node [near start] {yes} (traverse);             
  
  \path [line] (traverse) -- (find);              
  \path [line] (find) -- (process);
  \path [line] (process) -- (other);

  \path [line] (other) |- node [near start] {yes} (select);
  \path [line] (other) -- node [near start] {no} (stop);                      

\end{tikzpicture}

On the other hand, the readers might wish to comprehend the background of our
analyses. Perhaps they even want to refine our thinking. For that purpose, we
shortly discuss existing classifications of Open Source Licenses with respect to
their power to determine a license compliant behavior. Then we consider some
aspects, which evoke side effects for acting according to the Open Source
Licenses. Finally, we deduce our view on the fact that Open Source Licenses must be
concidered with respect to a sort of Use cases (\nameref{sec:OSUCdeduction}) and develop our Open Source use
Case finder on the base of these determinations.


[TDB \ldots]

%\bibliography{../../../bibfiles/oscResourcesEn}
