% Telekom osCompendium 'for beeing included' snippet template
%
% (c) Karsten Reincke, Deutsche Telekom AG, Darmstadt 2011
%
% This LaTeX-File is licensed under the Creative Commons Attribution-ShareAlike
% 3.0 Germany License (http://creativecommons.org/licenses/by-sa/3.0/de/): Feel
% free 'to share (to copy, distribute and transmit)' or 'to remix (to adapt)'
% it, if you '... distribute the resulting work under the same or similar
% license to this one' and if you respect how 'you must attribute the work in
% the manner specified by the author ...':
%
% In an internet based reuse please link the reused parts to www.telekom.com and
% mention the original authors and Deutsche Telekom AG in a suitable manner. In
% a paper-like reuse please insert a short hint to www.telekom.com and to the
% original authors and Deutsche Telekom AG into your preface. For normal
% quotations please use the scientific standard to cite.
%
% [ Framework derived from 'mind your Scholar Research Framework' 
%   mycsrf (c) K. Reincke 2012 CC BY 3.0  http://mycsrf.fodina.de/ ]
%


%% use all entries of the bibliography
%\nocite{*}

Based on this information we can no derive and define the use cases by which the
OSLiC classifies its license fulfilling to-do lists:

Firstly we have to discriminate the usage of Open Source software by the nature
of the software itself:

On the one side you can use an application intended to support the work of an
end-user. It takes input data and generates output data, mostly by using a more
or less elaborated end-user interface. On the other side you can use a computer
librabry intended to support the work of a software developer. It mostly offers
functions and/or objects and is embedded into an overarching work like an
application.

The use of an application is different from the use of a software library
because the use of a software library implies the act of developing a new
overarching piece of software.

Secondly we have to discriminate the usage of Open Source software by the
addressee:

On the one side you can intend to use the software directly for your own
purpose. On the other side you can intend to distribute the software to a third
party.

And thirdly we have to discriminate the usage of Open Source software by nature of
the usage itself:

One the one side you can intend to use the software as it is, respectively as
you got it. On the other side you can intend to modify the software before using
it.

Let's form the corresponding dichotomies

use-it-as-you-got-it <> modify-it
use-it-for-yourself <> distribute-it
application <> library

Now we have 8 possibilities to combine these attributes:

Use an application as you got it only for yourself
Use a library as you got it only for your self
Distribute an application as you got it to a third party
Distribute a library as you got it to a third party




%\bibliography{../../../bibfiles/oscResourcesEn}
