% Telekom osCompendium 'for beeing included' snippet template
%
% (c) Karsten Reincke, Deutsche Telekom AG, Darmstadt 2011
%
% This LaTeX-File is licensed under the Creative Commons Attribution-ShareAlike
% 3.0 Germany License (http://creativecommons.org/licenses/by-sa/3.0/de/): Feel
% free 'to share (to copy, distribute and transmit)' or 'to remix (to adapt)'
% it, if you '... distribute the resulting work under the same or similar
% license to this one' and if you respect how 'you must attribute the work in
% the manner specified by the author ...':
%
% In an internet based reuse please link the reused parts to www.telekom.com and
% mention the original authors and Deutsche Telekom AG in a suitable manner. In
% a paper-like reuse please insert a short hint to www.telekom.com and to the
% original authors and Deutsche Telekom AG into your preface. For normal
% quotations please use the scientific standard to cite.
%
% [ Framework derived from 'mind your Scholar Research Framework' 
%   mycsrf (c) K. Reincke 2012 CC BY 3.0  http://mycsrf.fodina.de/ ]
%


%% use all entries of the bibliography
%\nocite{*}

\section{Open Source and its history: some hints}
\footnotesize
\begin{quote}\itshape
Here we present main lines of the Open Source genesis: The start with the
bundling of hardware and software in the beginning on the one side and the
monopol of AT\&T and the free distribution of unix in the universities on the
other side - which together established the free hacker culture. We will shortly
describe the increase of the value of software evoked by the IBM unbundling
strategy and the antitrust suit against AT\&T which let become the software a
value itself worthful of protection and which destroyed the free exchange within
the early hacker community. Naturally we will illuminate the answer of RMS, the
GNU project, the founding of the FSF and the GPL. Then we will highlight the
introduction of the concept Open Source invented for dissolving the troubles to
talk about Free Software with managers of companies. We will hint to the Linux
kernel as an unwelcome completion of the GNU system. Finally we will outline the
convergency of business and Open Source, not only by Netscape/Mozilla, IBM
apache, Redhat, SUN/OpenOffice but also by IBM/eclipse, Sun/Java and so on. And
naturally we will highlight the meaning of 'the Cathedral and the Bazar', which
had not been written to contrast the working style of the Open Source Developemt
and the proprietary 'in company' development by for example microsoft, but for
dicern the working and leading style of RMS and Linus.
\end{quote}
\normalsize
\ldots

%\bibliography{../../../bibfiles/oscResourcesEn}
