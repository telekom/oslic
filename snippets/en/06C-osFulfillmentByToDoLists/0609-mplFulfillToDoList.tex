% Telekom osCompendium 'for being included' snippet template
%
% (c) Karsten Reincke, Deutsche Telekom AG, Darmstadt 2011
%
% This LaTeX-File is licensed under the Creative Commons Attribution-ShareAlike
% 3.0 Germany License (http://creativecommons.org/licenses/by-sa/3.0/de/): Feel
% free 'to share (to copy, distribute and transmit)' or 'to remix (to adapt)'
% it, if you '... distribute the resulting work under the same or similar
% license to this one' and if you respect how 'you must attribute the work in
% the manner specified by the author ...':
%
% In an internet based reuse please link the reused parts to www.telekom.com and
% mention the original authors and Deutsche Telekom AG in a suitable manner. In
% a paper-like reuse please insert a short hint to www.telekom.com and to the
% original authors and Deutsche Telekom AG into your preface. For normal
% quotations please use the scientific standard to cite.
%
% [ Framework derived from 'mind your Scholar Research Framework' 
%   mycsrf (c) K. Reincke 2012 CC BY 3.0  http://mycsrf.fodina.de/ ]
%


%% use all entries of the bibliography
%\nocite{*}

\section{MPL-2.0 licensed software}

\begin{license}{MPL} % ends at end of file
\licensename{MPL-2.0}
\licensespec{Mozilla Public License 2.0}
\licenseversion{2.0}
\licenseabbrev{MPL}

The Mozilla Public License clearly distinguishes the distribution of source code
from the distribution of binaries: First, it allows the \enquote{Distribution of
Source Form}.\citeMPL{§3.1} Then, it specifies the conditions for a
\enquote{Distribution of Executable Form}.\citeMPL{§3.2} Additionally, the
MPL-2.0 contrasts the \enquote{distribution of Covered Software} with the
\enquote{distribution of a Larger Work}.\citeMPL{§3.3} So, taken as whole, the
MPL-2.0 mainly focusses on the distribution of software. Thus, for finding the
relevant executable task lists, the following MPL-2.0 specific open source use
case structure%
  \footnote{For details of the general OSUC finder $\rightarrow$ \oslic, 
    pp.\ \pageref{OsucTokens} and \pageref{OsucDefinitionTree}} 
can be used:
 
\tikzstyle{nodv} = [font=\small, ellipse, draw, fill=gray!10, 
    text width=2cm, text centered, minimum height=2em]

\tikzstyle{nods} = [font=\footnotesize, rectangle, draw, fill=gray!20, 
    text width=1.2cm, text centered, rounded corners, minimum height=3em]

\tikzstyle{nodb} = [font=\footnotesize, rectangle, draw, fill=gray!20, 
    text width=2.2cm, text centered, rounded corners, minimum height=3em]
    
\tikzstyle{leaf} = [font=\tiny, rectangle, draw, fill=gray!30, 
    text width=1.2cm, text centered, minimum height=6em]

\tikzstyle{edge} = [draw, -latex']

\begin{tikzpicture}[]

\node[nodv] (l71) at (4,10) {MPL-2.0};

\node[nodb] (l61) at (0,8.6) {\textit{recipient:} \\ \textbf{4yourself}};
\node[nodb] (l62) at (6.5,8.6) {\textit{recipient:} \\ \textbf{2others}};

\node[nodb] (l51) at (2.5,7) {\textit{state:} \\ \textbf{unmodified}};
\node[nodb] (l52) at (9.3,7) {\textit{state:} \\ \textbf{modified}};

\node[nods] (l41) at (1.8,5.4) {\textit{form:} \textbf{source}};
\node[nods] (l42) at (3.6,5.4) {\textit{form:} \textbf{binary}};
\node[nodb] (l43) at (6.5,5.4) {\textit{type:} \\ \textbf{proapse}};
\node[nodb] (l44) at (12,5.4) {\textit{type:} \\ \textbf{snimoli}};


\node[nods] (l31) at (5.4,3.8) {\textit{form:} \textbf{source}};
\node[nods] (l32) at (7.2,3.8) {\textit{form:} \textbf{binary}};
\node[nodb] (l33) at (10,3.8) {\textit{context:} \\ \textbf{independent}};
\node[nodb] (l34) at (13.5,3.8) {\textit{context:} \\ \textbf{embedded}};

\node[nods] (l21) at (9,2.2) {\textit{form:} \textbf{source}};
\node[nods] (l22) at (10.8,2.2) {\textit{form:} \textbf{binary}};
\node[nods] (l23) at (12.6,2.2) {\textit{form:} \textbf{source}};
\node[nods] (l24) at (14.4,2.2) {\textit{form:} \textbf{binary}};

\node[leaf] (l11) at (0,0) {\textbf{MPL-2.0-C1} \textit{using software only
for yourself}};

\node[leaf] (l12) at (1.8,0) { \textbf{MPL-2.0-C2} \textit{ distributing unmodified
software as sources}};

\node[leaf] (l13) at (3.6,0) { \textbf{MPL-2.0-C3}  \textit{ distributing unmodified
software as binaries}};

\node[leaf] (l14) at (5.4,0) { \textbf{MPL-2.0-C4}  \textit{ distributing modified
program as sources}};

\node[leaf] (l15) at (7.2,0) { \textbf{MPL-2.0-C5}  \textit{ distributing modified
program as binaries}};

\node[leaf] (l16) at (9,0) { \textbf{MPL-2.0-C6}  \textit{ distributing modified
library as independent sources}};

\node[leaf] (l17) at (10.8,0) { \textbf{MPL-2.0-C7} \textit{distributing modified
library as independent binaries}};

\node[leaf] (l18) at (12.6,0) { \textbf{MPL-2.0-C8}  \textit{distributing
modified library as embedded sources}};

\node[leaf] (l19) at (14.4,0) { \textbf{MPL-2.0-C9}  \textit{ distributing modified
library as embedded binaries}};


\path [edge] (l71) -- (l61);
\path [edge] (l71) -- (l62);
\path [edge] (l61) -- (l11);
\path [edge] (l62) -- (l51);
\path [edge] (l62) -- (l52);
\path [edge] (l51) -- (l41);
\path [edge] (l51) -- (l42);
\path [edge] (l52) -- (l43);
\path [edge] (l52) -- (l44);
\path [edge] (l41) -- (l12);
\path [edge] (l42) -- (l13);
\path [edge] (l43) -- (l31);
\path [edge] (l43) -- (l32);
\path [edge] (l44) -- (l33);
\path [edge] (l44) -- (l34);
\path [edge] (l31) -- (l14);
\path [edge] (l32) -- (l15);
\path [edge] (l33) -- (l21);
\path [edge] (l33) -- (l22);
\path [edge] (l34) -- (l23);
\path [edge] (l34) -- (l24);
\path [edge] (l21) -- (l16);
\path [edge] (l22) -- (l17);
\path [edge] (l23) -- (l18);
\path [edge] (l24) -- (l19);

\end{tikzpicture}

%% =============================================================================
%% Common building blocks
%%

% ------------------------------------------------------------------------------
% Ensure license elements are present

\newcommand{\keepLicenseElements}{Ensure that the licensing elements (especially
  all copyright notices, patent notices, disclaimers of warranty, or limitations
  of liability) are retained in your package in exactly the form that you have
  received.}

\newcommand{\addWhenCompiling}{If you compile the binary from the sources,
  ensure that all these licensing elements are also incorporated into the
  package.}

% ------------------------------------------------------------------------------
% Give the recipient a copy of the license

\newcommand{\giveLicenseText}{Give the recipient a copy of the MPL-2.0 license.
  If it is not already part of the software package, add it. If the licensing
  statement in the licensing file of the package does still not clearly state
  that the package is licensed under the MPL-2.0, additionally insert your own
  correct MPL-2.0 licensing file containing the sentence: 
  \emph{This Source Code Form is subject to the terms of the Mozilla Public
    License, v. 2.0. If a copy of the MPL was not distributed with this file,
    You can obtain one at http://mozilla.org/MPL/2.0/.}}

% ------------------------------------------------------------------------------
% Add license, name, and link to homepage to documentation

\newcommand{\auxAddToDoc}[1]{Let the documentation of your distribution
  and/or your additional material also reproduce the content of the existing
  \emph{copyright notice text files,} the name of #1, a link to its homepage,
  and a link to the MPL-2.0 license.}  

\newcommand{\acknowledgeMPLSoftware}{
  \auxAddToDoc{the software}}

\newcommand{\acknowlegdeEmbeddedLibrary}{
  \auxAddToDoc{the embedded MPL-2.0 licensed component}}

% ------------------------------------------------------------------------------
% Make the source code available

\newcommand{\auxMakeSourceAvailable}[1]{Make the source code of #1 accessible
  via a repository under your own control: Push the source code package into the
  repository and make it downloadable via the Internet. Do no charge any fees
  from the user for downloading the source. Ensure, that this repository is
  online for a reasonable period of time after you ceased distributing the
  software.} 

\newcommand{\makeSourceAvailable}{\auxMakeSourceAvailable{%
    the distributed software}}

\newcommand{\makeEmbeddedSourceAvailable}{\auxMakeSourceAvailable{%
    the embedded library}}

\newcommand{\describeHowToGetSource}{Insert an easy to find description into the 
  distribution package that explains how and where the code can be retrieved.}

% ------------------------------------------------------------------------------
% Ensure modifications are covered by MPL

\newcommand{\auxPlaceModificationsUnderMPL}[1]{Organize your modifications #1
  in such a way that they are covered by the existing MPL-2.0 licensing
  statements.}

\newcommand{\auxPlaceNewFilesUnderMPL}[1]{If you add new source code files#1,
  insert a header containing your copyright line and an MPL-2.0 adequate
  licensing the statement.}

\newcommand{\placeBinaryModificationsUnderMPL}{%
  \auxPlaceModificationsUnderMPL{}}

\newcommand{\placeSourceModificationsUnderMPL}{%
  \auxPlaceModificationsUnderMPL{}
  \auxPlaceNewFilesUnderMPL{}}

\newcommand{\placeEmbeddedBinaryUnderMPL}{%
  \auxPlaceModificationsUnderMPL{of the embedded library}}

\newcommand{\placeEmbeddedSourceUnderMPL}{%
  \auxPlaceModificationsUnderMPL{of the embedded library}
  \auxPlaceNewFilesUnderMPL{ to the library itself}}

% ------------------------------------------------------------------------------
% Create modification text file

\newcommand{\createChangeLog}{Create a \emph{modification text file}, if such a
  notice file still does not exist. \emph{Add} a general description of your
  modifications to the \emph{modification text file}. Incorporate the file into
  your distribution package.}

% ------------------------------------------------------------------------------
% Mark all modifications

\newcommand{\markAllModifications}{Mark all modifications of the source code
  thoroughly, preferably in the modified source itself.}

% ------------------------------------------------------------------------------
% Separate embedded library from enclosing program

\newcommand{\auxKeepSeparate}[1]{Arrange your #1 distribution so that the
  licensing elements (especially the MPL-2.0 license text and the
  \emph{licensing files}) clearly refer only to the embedded library and do not
  affect the licensing of your own overarching work. It's a good tradition to
  keep embedded components like libraries, modules, snippets, or plugins in
  separate directories, which contain also all additional licensing elements.}

\newcommand{\keepSourceSeparate}{\auxKeepSeparate{source code}}
\newcommand{\keepBinarySeparate}{\auxKeepSeparate{binary}}

% ------------------------------------------------------------------------------
% Do not modify or remove license elements

\newcommand{\dontAlterLicenseElement}{to remove or to alter any license elements
  (including copyright notices, patent notices, disclaimers of warranty, or
  limitations of liability) contained within the software package you have
  received.}

% ------------------------------------------------------------------------------
% Do not use trademarks and logos to promote your own work

\newcommand{\dontUseTrademarks}{to promote any of your services based on the
    this software by trademarks, service marks, or logos linked to this MPL-2.0
    software, except as required for reasonable and customary use in describing
    the origin of the software and reproducing the copyright notice.}

%% =============================================================================
%% Use Cases
%%

\subsection{MPL-2.0-C1: Using the software only for yourself}
\begin{lsuc}{MPL-2.0-C1}
  \linkosuc{01}
  \linkosuc{03L} 
  \linkosuc{03N} 
  \linkosuc{06L}
  \linkosuc{06N}
  \linkosuc{09L}
  \linkosuc{09N}

  \lsucmeans{that you received MPL-2.0 licensed software, that you will use it
    only for yourself, and that you do not hand it over to any third party in
    any sense.}

  \coversOsucs{OSUC-01, OSUC-03L, OSUC-03N, OSUC-06L, OSUC-06N, OSUC-09L, and
  OSUC-09N}{01}{09N}

  \begin{lsucrequiresnothing}
    \lsucitem{You are allowed to use any kind of MPL-2.0 software in any sense
      and in any context without being obliged to do anything as long as you do
      not give the software to third parties.}
  \end{lsucrequiresnothing}

  \begin{lsucprohibits}
    \lsucitem{\dontAlterLicenseElement}
    \lsucitem{\dontUseTrademarks}
  \end{lsucprohibits}
\end{lsuc}

% ------------------------------------------------------------------------------
\subsection{MPL-2.0-C2: Passing the unmodified software as source code}
\begin{lsuc}{MPL-2.0-C2}
  \linkosuc{02S} 
  \linkosuc{05S} 
  \linkosuc{07S} 

  \lsucmeans{that you received MPL-2.0 licensed software which you are now going
    to distribute to third parties in the form of unmodified source code files
    or as unmodified source code package. In this case it makes no difference if
    you distribute a program, an application, a server, a snippet, a module, a
    library, or a plugin as an independent or as an embedded unit.}

  \coversOsucs{OSUC-02S, OSUC-05S, OSUC-07S}{02S}{07S}

  \begin{lsucrequires}
    \lsucmandatory{\keepLicenseElements}
    \lsucmandatory{\giveLicenseText}\passingFilesCorrectly
    \lsucoptional{\acknowledgeMPLSoftware}
  \end{lsucrequires}

  \begin{lsucprohibits}
    \lsucitem{\dontAlterLicenseElement}
    \lsucitem{\dontUseTrademarks}
  \end{lsucprohibits}
\end{lsuc}

% ------------------------------------------------------------------------------
\subsection{MPL-2.0-C3: Passing the unmodified software as binaries} 
\begin{lsuc}{MPL-2.0-C3}
  \linkosuc{02B} 
  \linkosuc{05B} 
  \linkosuc{07B}

  \lsucmeans{that you received MPL-2.0 licensed software which you are now going
    to distribute to third parties in the form of unmodified binary files or as
    unmodified binary package. In this case it does not matter if you distribute
    a program, an application, a server, a snippet, a module, a library, or a
    plugin as an independent or an embedded unit.}

  \coversOsucs{OSUC-02B, OSUC-05B, OSUC-07B}{02B}{07B}

  \begin{lsucrequires}
    \lsucmandatory{\keepLicenseElements\ \addWhenCompiling}
    \lsucmandatory{\makeSourceAvailable}
    \lsucmandatory{\describeHowToGetSource}
  
    \lsucsourcedist{MPL-2.0-C2}
    \lsucoptional{\giveLicenseText}\passingFilesCorrectly
    \lsucoptional{\acknowledgeMPLSoftware}
  \end{lsucrequires}

  \begin{lsucprohibits}
    \lsucitem{\dontAlterLicenseElement}
    \lsucitem{\dontUseTrademarks}
  \end{lsucprohibits}
\end{lsuc}

% ------------------------------------------------------------------------------
\subsection{MPL-2.0-C4: Passing a modified program as source code}
\begin{lsuc}{MPL-2.0-C4}
  \linkosuc{04S} 

  \lsucmeans{that you received an MPL-2.0 licensed program, application, or
    server (proapse), that you modified it, and that you are now going to
    distribute this modified version to third parties in the form of source code
    files or as a source code package.}

  \mapsToOsuc{04S}

  \begin{lsucrequires}
    \lsucmandatory{\keepLicenseElements}
    \lsucmandatory{\giveLicenseText}\passingFilesCorrectly
    \lsucmandatory{\placeSourceModificationsUnderMPL}
    \lsucoptional{\createChangeLog}
    \lsucoptional{\markAllModifications}

    \lsucoptional{\acknowledgeMPLSoftware}
  \end{lsucrequires}
 
  \begin{lsucprohibits}
    \lsucitem{\dontAlterLicenseElement}
    \lsucitem{\dontUseTrademarks}
  \end{lsucprohibits}
\end{lsuc}

% ------------------------------------------------------------------------------
\subsection{MPL-2.0-C5: Passing a modified program as binary}
\begin{lsuc}{MPL-2.0-C5}
  \linkosuc{04B} 

  \lsucmeans{that you received an MPL-2.0 licensed program, application, or
    server (proapse), that you modified it, and that you are now going to
    distribute this modified version to third parties in the form of binary
    files or as a binary package.}

  \mapsToOsuc{04B}

  \begin{lsucrequires}
    \lsucmandatory{\keepLicenseElements\ \addWhenCompiling}
    \lsucmandatory{\makeSourceAvailable}
    \lsucmandatory{\describeHowToGetSource}
    \lsucsourcedist{MPL-2.0-C4}
    \lsucmandatory{\placeBinaryModificationsUnderMPL}
    \lsucoptional{\createChangeLog}
    \lsucoptional{\giveLicenseText}\passingFilesCorrectly
    \lsucoptional{\acknowledgeMPLSoftware}
  \end{lsucrequires}


  \begin{lsucprohibits}
    \lsucitem{\dontAlterLicenseElement}
    \lsucitem{\dontUseTrademarks}
  \end{lsucprohibits}
\end{lsuc}

% ------------------------------------------------------------------------------
\subsection{MPL-2.0-C6: Passing a modified library as independent source code}
\begin{lsuc}{MPL-2.0-C6}
  \linkosuc{08S}

  \lsucmeans{that you received an MPL-2.0 licensed code snippet, module,
    library, or plugin (snimoli), that you modified it, and that you are now
    going to distribute this modified version to third parties in the form of
    source code files or as a source code package, but without embedding it into
    another larger software unit.}

  \mapsToOsuc{08S}

  \begin{lsucrequires}
    \lsucmandatory{\keepLicenseElements}
    \lsucmandatory{\giveLicenseText}\passingFilesCorrectly
    \lsucmandatory{\placeSourceModificationsUnderMPL}
    \lsucoptional{\createChangeLog}
    \lsucoptional{\markAllModifications}
    \lsucoptional{\acknowledgeMPLSoftware}
  \end{lsucrequires}

  \begin{lsucprohibits}
    \lsucitem{\dontAlterLicenseElement}
    \lsucitem{\dontUseTrademarks}
  \end{lsucprohibits}
\end{lsuc}

% ------------------------------------------------------------------------------
\subsection{MPL-2.0-C7: Passing a modified library as independent binary}
\begin{lsuc}{MPL-2.0-C7}
  \linkosuc{08B}

  \lsucmeans{that you received an MPL-2.0 licensed code snippet, module,
    library, or plugin (snimoli), that you modified it, and that you are now
    going to distribute this modified version to third parties in the form of
    binary files or as a binary package but without embedding it into another
    larger software unit.}

  \mapsToOsuc{08B}

  \begin{lsucrequires}
    \lsucmandatory{\keepLicenseElements\ \addWhenCompiling}
    \lsucmandatory{\makeSourceAvailable}
    \lsucmandatory{\describeHowToGetSource}
    \lsucsourcedist{MPL-2.0-C6}
    \lsucmandatory{\placeBinaryModificationsUnderMPL}
    \lsucoptional{\createChangeLog}
    \lsucoptional{\giveLicenseText}\passingFilesCorrectly
    \lsucoptional{\acknowledgeMPLSoftware}
  \end{lsucrequires}

  \begin{lsucprohibits}
    \lsucitem{\dontAlterLicenseElement}
    \lsucitem{\dontUseTrademarks}
  \end{lsucprohibits}
\end{lsuc}

% ------------------------------------------------------------------------------
\subsection{MPL-2.0-C8: Passing a modified library as embedded source code}
\begin{lsuc}{MPL-2.0-C8}
  \linkosuc{10S}

  \lsucmeans{that you received an MPL-2.0 licensed code snippet, module,
    library, or plugin (snimoli), that you modified it, and that you are now
    going to distribute this modified version to third parties in the form of
    source code files or as a source code package together with another larger
    software unit which contains this code snippet, module, library, or plugin
    as an embedded component.}

  \mapsToOsuc{10S}

  \begin{lsucrequires}

    \lsucmandatory{\keepLicenseElements}
    \lsucmandatory{\giveLicenseText}\passingFilesCorrectly
    \lsucmandatory{\placeEmbeddedSourceUnderMPL}
    \lsucoptional{\keepSourceSeparate}
    \lsucoptional{\createChangeLog}
    \lsucoptional{\markAllModifications}
    \lsucoptional{\acknowlegdeEmbeddedLibrary}
  \end{lsucrequires}

  \begin{lsucprohibits}
    \lsucitem{\dontAlterLicenseElement}
    \lsucitem{\dontUseTrademarks}
  \end{lsucprohibits}
\end{lsuc}

% ------------------------------------------------------------------------------
\subsection{MPL-2.0-C9: Passing a modified library as embedded binary}
\begin{lsuc}{MPL-2.0-C9}
  \linkosuc{10B}

  \lsucmeans{that you received an MPL-2.0 licensed code snippet, module,
    library, or plugin (snimoli), that you modified it, and that you are now
    going to distribute this modified version to third parties in the form of
    binary files or as a binary package together with another larger software
    unit which contains this code snippet, module, library, or plugin as an
    embedded component.}

  \mapsToOsuc{10B}

  \begin{lsucrequires}
    \lsucmandatory{\keepLicenseElements \addWhenCompiling}
    \lsucmandatory{\makeEmbeddedSourceAvailable}
    \lsucmandatory{\describeHowToGetSource}
    \lsucsourcedist{MPL-2.0-C8}
    \lsucmandatory{\placeEmbeddedBinaryUnderMPL}
    \lsucoptional{\createChangeLog}
    \lsucoptional{\giveLicenseText}\passingFilesCorrectly
    \lsucoptional{\keepBinarySeparate}
    \lsucoptional{\acknowlegdeEmbeddedLibrary}
  \end{lsucrequires}

  \begin{lsucprohibits}
    \lsucitem{\dontAlterLicenseElement}
    \lsucitem{\dontUseTrademarks}
  \end{lsucprohibits}
\end{lsuc}

% ------------------------------------------------------------------------------

\subsection{Discussions and Explanations}
\label{MPLDiscussion}
The MPL-2.0 offers a section \enquote{Responsibilities} which contains nearly all
requirements.\citeMPL{§3} Only for some subordinate aspects, one has also to
reflect other paragraphs.\citeMPL{pars pro to cf.}{§3 - concerning the trademarks}
With respect to this structure, we can detect the following tasks:

\begin{itemize}

\item In a more general attitude, the MPL-2.0 states that it \enquote{[\ldots]
  does not grant any rights in the trademarks, service marks, or logos of any
  Contributor}---except as it may be necessary \enquote{to comply with} other
  requirements of the license.\citeMPL{§2.3} The \oslic{} rewrites the message
  as the interdiction to promote own services and products by and with such
  elements. 
  
\item The MPL-2.0 also generally prescribes that \enquote{you may not remove or
  alter the substance of any license notice (including copyright notices, patent
  notices, disclaimer of warranties, or limitations of liabiliy) contained
  within the Source Code Form [\ldots]}\citeMPL{§3.4} This focussing to the
  \enquote{substance of any license notice} refers to the allowance to
  \enquote{[\ldots] alter any license notices to the extent required to remedy
  known factual innacuracies}.\citeMPL{§3.4}  Following its principle to offer
  one reliable way and to ignore variants of secondary importance, the \oslic{}
  simplifies this condition to the general proscription to modify any licensing
  material for all use cases [MPL-2.0-C1 -- MPL-2.0-C9]. But for emphasizing
  that this is a job which must be activily done, the \oslic{} additionally
  rewrites this interdiction into all \emph{2others} use cases [MPL-2.0-C2 --
  MPL-2.0-C9] as the task to retain the licensing elements in the form one has
  obtained them. 
  
\item Moreover, the MPL-2.0 requires for all \enquote{distributions of [the]
  source [code] form} that all modifications of the software \enquote{[\ldots] 
  must be under the terms of (the MPL-2.0)} and that the distributor
  \enquote{[\ldots] must inform} all \enquote{recipients} that the software
  \enquote{[\ldots] is governed by the terms of (the MPL-2.0), and how (the
  recipients) can obtain a copy of this license}.\citeMPL{§3.1}  For the
  respective use case (MPL-2.0-C2, MPL-2.0-C4, MPL-2.0-C6, MPL-2.0-C8), the
  \oslic{} rewrites these conditions so that each MPL-2.0 source code package
  must neccessarily contain the MPL-2.0 itself as textfile and an additional 
  licensing file or statement strictly following the text given by the addendum
  of the MPL-2.0.\citeMPL{Exhibit A} Because the MPL-2.0 is only a license with
  weak copyleft, the \oslic{} proposes to separate the MPL-2.0 licensed,
  embedded component from the enclosing program (MPL-2.0-C8). 
  
\item But the MPL-2.0 does not explicitly require marking all modifications.
  Nevertheless, this is state of the art in computer emgineering. Therefore,
  with respect to the cases of distributing modified source code (MPL-2.0-C4,
  MPL-2.0-C6 and MPL-2.0-C8), the \oslic{} proposes to mark all modifications
  inside of the source code and to update the description of the functional
  changes. In case of distributing the modified software in the form of
  binaries, it should be sufficient to describe the modifications only on the
  functional level. 
  
\item Furthermore, the MPL-2.0 requires that the \enquote{Covered Software}---in 
  all cases of distributing it in an \enquote{Executable Form} (MPL-2.0-C3,
  MPL-2.0-C5, MPL-2.0-C7, MPL-2.0-C9)---\enquote{[\ldots] must also be made
  available in Source Code Form [\ldots]} and that the distributor
  \enquote{[\ldots] must inform recipients of the Executable Form how they can
  obtain a copy of such Source Code Form by reasonable means in a timely manner,
  at a charge no more than the cost of distribution to the
  recipient}.\citeMPL{§3.2.a}  The \oslic{} rewrites these conditions as the
  obligation to offer a download service at no charge and to point towards this
  services inside of the distributed package.
  
\item In this context, the MPL-2.0 allows to distribute the binaries under terms
  of another license \enquote{[\ldots] provided that that the license for the
  Executable Form does not attempt to limit or alter the recipients’ rights in
  the Source Code Form under this License.}\citeMPL{§3.2.b} This possibility
  might become important for those cases where the license compatibility must
  explicitly be managed. Normally, it should be sufficient also to distribute
  the binaries under the MPL-2.0. Thus, in case of distributing binaries
  (MPL-2.0-C3, MPL-2.0-C5, MPL-2.0-C7, MPL-2.0-C9), the \oslic{} proposes to
  insert into the distribution packages the MPL-2.0 itself and an additional
  licensing file or statement strictly following the text given by the addendum
  of the MPL-2.0.\citeMPL{Exhibit A} But again, because the MPL-2.0 is only a
  license with weak copyleft, the \oslic{} proposes to separate the MPL-2.0
  licensed embedded component from the overarching program (MPL-2.0-C9).
  
\item Finally, one clearly has to state that the distribution of the source code
  required by the previous rule must, of course, follow the rules of distributing
  the software. Thus, the \oslic{} requires in all cases of a binary distribution
  to execute also the task-lists of the respective source code use cases.

\end{itemize}

% ------------------------------------------------------------------------------

\end{license}
%\bibliography{../../../bibfiles/oscResourcesEn}

% Local Variables:
% mode: latex
% fill-column: 80
% End:
