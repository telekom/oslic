% Telekom osCompendium 'for being included' snippet template
%
% (c) Karsten Reincke, Deutsche Telekom AG, and Ronald Dauster, GIDO GmbH
%     Darmstadt 2014
%
% This LaTeX-File is licensed under the Creative Commons Attribution-ShareAlike
% 3.0 Germany License (http://creativecommons.org/licenses/by-sa/3.0/de/): Feel
% free 'to share (to copy, distribute and transmit)' or 'to remix (to adapt)'
% it, if you '... distribute the resulting work under the same or similar
% license to this one' and if you respect how 'you must attribute the work in
% the manner specified by the author ...':
%
% In an Internet based reuse please link the reused parts to www.telekom.com and
% mention the original authors and Deutsche Telekom AG in a suitable manner. In
% a paper-like reuse please insert a short hint to www.telekom.com and to the
% original authors and Deutsche Telekom AG into your preface. For normal
% quotations please use the scientific standard to cite.
%
% [ Framework derived from 'mind your Scholar Research Framework' 
%   mycsrf (c) K. Reincke 2012 CC BY 3.0  http://mycsrf.fodina.de/ ]
%

% Common parametrized text blocks for GPL, LGPL, and AGPL
%
% All commands take (at least) one parameter: the name of the license, for
% example, 'GPL-2.0'.  This parameter is always present, even if it isn't used
% and it is always the first parameter.

% ------------------------------------------------------------------------------
% Keep license elements 
% #1 -> the license name

\newcommand{\gtbKeepLicenseElements}[1]{Ensure that the licensing elements
  (especially all notices that refer to the #1 and to the absence of any
  warranty) are retained in your package in the form in which you have received
  them.} 

% ------------------------------------------------------------------------------
% Give a copy of the license to the recipient of the software
% #1 -> the license name

\newcommand{\gtbGiveLicense}[1]{Give the recipient a copy of the #1 license.
  If it is not already part of the software package, add it.}

% ------------------------------------------------------------------------------
% Add license elements and acknowledgement to the documemtation
% #1 -> the license name

\newcommand{\gtbAddToDocumentation}[1]{Let the documentation of your
  distribution and/or your additional material also reproduce the content of the
  existing copyright notices, a hint to the software name, a link to its
  homepage, the respective disclaimer of warranty, and a link to the #1.}

% ------------------------------------------------------------------------------
% Keep all copyright notices intact
% #1 -> the license name

\newcommand{\gtbKeepCopyrightNotices}[1]{Retain all existing copyright notices.}

% ------------------------------------------------------------------------------
% Publish the source code
% #1 -> the license name

\newcommand{\gtbSourceRepository}[1]{Push the source code package into a
  repository under your control and make it downloadable via the Internet.
  Ensure, that this repository is online for at least 3 years after you ceased
  distributing the software package.}

% program or independent library, unmodified
\newcommand{\gtbMakeUnmodifiedSourceAvailable}[1]{Make the source code of the
  distributed software publicly available (even though you did not modify it):
  \gtbSourceRepository{#1}}

% program or independent library, modified
\newcommand{\gtbMakeModifiedSourceAvailable}[1]{Make the source code of the
  distributed software publicly available: \gtbSourceRepository{#1}} 

% embedded library, modified or unmodified, GPL and AGPL
\newcommand{\gtbMakeAllSourcesAvailable}[1]{Make the \emph{complete} source code
  of the program embedding the library publicly available (and, therefore, also
  the source code of the library itself): \gtbSourceRepository{#1}}

% embedded library, modified or unmodified, LGPL 
\newcommand{\gtbMakeEmbeddedSourcesAvailable}[1]{Make the source code of the 
  embedded library publicly available: \gtbSourceRepository{#1}}

% ------------------------------------------------------------------------------
% Explain where to find the sources
% #1 -> the license name

\newcommand{\gtbDescribeHowToGetSource}[1]{Insert an easy to find description
  into the distribution package that explains how and where the code can be
  retrieved.}

% ------------------------------------------------------------------------------
% Create and update the modification text file
% #1 -> the license name

\newcommand{\gtbCreateChangelog}[1]{Create a \emph{modification text file,} if
  such a file does not yet exist. \emph{Add} a description of your modifications
  on a functional level to the \emph{modification text file.}}

% ------------------------------------------------------------------------------
% Mark all modifications in the source files themselves
% #1 -> the license name

\newcommand{\gtbauxMarkChanges}[1]{Mark all modifications of the source code #1
  thoroughly within the source code and include the date of the modification.}

\newcommand{\gtbMarkEmbeddedModifications}[1]{%
  \gtbauxMarkChanges{of the embedded library (snimoli)}}

\newcommand{\gtbMarkLibraryModifications}[1]{%
  \gtbauxMarkChanges{of the library (snimoli)}}

\newcommand{\gtbMarkProgramModifications}[1]{%
  \gtbauxMarkChanges{the program (proapse)}}

% ------------------------------------------------------------------------------
% Ensure the copyright notice and the disclaimer (V2.x only) are present
% #1 -> the license name
% #2 -> type of distribution (binary or source code)

% GPL-3.0/LGPL-3.0/AGPL-3.0
\newcommand{\gtbVThreeCopyrightNotice}[2]{Ensure that the
  distributed #2 package contains a conspicuous, easy to find copyright notice.
  If this element is missing, add a new file containing the main copyright
  notice.} 

% GPL-2.0/LGPL-2.1
\newcommand{\gtbVTwoCopyrightNotice}[2]{Ensure that the distributed
  #2 package contains a conspicuous, easy to find copyright notice and
  disclaimer of warranty. If these elements are missing, add a new file
  containing the main copyright notice and the disclaimer of warranty in the
  form which is textually defined by the #1 license itself. (Yes, repeat
  the disclaimer although it is also part of the license itself and although you
  are required to hand the license itself over to the receiver.)}

% ------------------------------------------------------------------------------
% Make sure licensing statements apply to your modifications
% #1 -> the license name

\newcommand{\gtbauxArrangeChanges}[2]{Arrange your modifications of #2 in a way
  that they are covered by existing #1 licensing statements. If you add new
  source code files to the #2, insert a header containing your copyright line
  and a licensing statement in the form recommended by the #1.}

\newcommand{\gtbArrangeProgramChanges}[1]{%
  \gtbauxArrangeChanges{#1}{program}}

\newcommand{\gtbArrangeLibraryChanges}[1]{%
  \gtbauxArrangeChanges{#1}{library}}

\newcommand{\gtbArrangeEmbeddedChanges}[1]{%
  \gtbauxArrangeChanges{#1}{embedded library}}

\newcommand{\gtbHowToApplyTheseTerms}[1]{%
  \footnote{For details see section `How to Apply These Terms to Your New
    Programs' in the #1 license.}} 

% ------------------------------------------------------------------------------
% Forbid patent litigation
% #1 -> license name

\newcommand{\gtbNoPatentLitigation}[1]{%
  to institute a patent litigation against anyone alleging that the software
  constitutes patent infringement.} 

% Local Variables:
% mode: latex
% fill-column: 80
% End:
