% Telekom osCompendium 'for being included' snippet template
%
% (c) Karsten Reincke, Deutsche Telekom AG, Darmstadt 2011
%
% This LaTeX-File is licensed under the Creative Commons Attribution-ShareAlike
% 3.0 Germany License (http://creativecommons.org/licenses/by-sa/3.0/de/): Feel
% free 'to share (to copy, distribute and transmit)' or 'to remix (to adapt)'
% it, if you '... distribute the resulting work under the same or similar
% license to this one' and if you respect how 'you must attribute the work in
% the manner specified by the author ...':
%
% In an internet based reuse please link the reused parts to www.telekom.com and
% mention the original authors and Deutsche Telekom AG in a suitable manner. In
% a paper-like reuse please insert a short hint to www.telekom.com and to the
% original authors and Deutsche Telekom AG into your preface. For normal
% quotations please use the scientific standard to cite.
%
% [ Framework derived from 'mind your Scholar Research Framework' 
%   mycsrf (c) K. Reincke 2012 CC BY 3.0  http://mycsrf.fodina.de/ ]
%


%% use all entries of the bibliography
%\nocite{*}

\section{Excursion: Reverse Engineering and Open Source}

Beyond any doubt, the LGPL mentions \enquote{reverse engineering}
literally\footnote{For the LGPL-v2 \cite[cf.][\nopage wp.
§6]{Lgpl21OsiLicense1999a}; for the LGPL-v3 \cite[cf.][\nopage wp.
§4]{Lgpl30OsiLicense2007a} } for indicating that \enquote{reverse engineering}
in any sense must be allowed to use and distribute LGPL software compliantly:

\begin{quote}\noindent\emph{\enquote{[\ldots] you may [\ldots] distribute a
work (containing portions of the Library) under terms of your choice, provided
that the terms permit [\ldots] \emph{reverse engineering}
[\ldots]}\footcite[cf.][\nopage wp, §6]{Lgpl21OsiLicense1999a}}
\end{quote}

There are three strategies for dealing with such provisions. One can try to
fully honor its meaning, one can mitigate its meaning, or one can avoid to
discuss this requirement altogether:

A first group of well known open source experts take the sentence of the LGPL-v2
as a strict rule which requires that one has to allow reverse engineering of the
whole software product if one embeds any LGPL licensed component into that
product\footnote{For example, a very trustworthy German expert states that the
LGPL-2.1 generally requires that a distributor of a program which accesses a
LGPL-2.1 licensed library, must grant his customer also the right to modify the
accessing program and hence also the right to execute reverse engineering.
Literally the German text says:
\begin{quote}\enquote{Ziffer 6 LGPLv2.1 knüpft die Erlaubnis, das zugreifende
Programm unter beliebigen Lizenzbestimmungen verbreiten zu drüfen, an eine Reihe
von Verpflichtungen, die in der Praxis oft übersehen werden: Zunächst muss dem
Kunden, dem die Software geliefert wird, die Veränderung des zugreifenden
Programms gestattet werden und zu diesem Zweck auch ein Reverse Engineering zur
Fehlerbehebung. Dies dürfte alle Formen des Debugging und das Dekompilieren des
zugreifenden Programms umfassen.} (\cite[cf.][81]{JaeMet2011a}).\end{quote}
At first glance, also \enquote{copyleft.org} -- the \enquote{[...] collaborative
project to create and disseminate useful information, tutorial material, and new
policy ideas regarding all forms of copyleft licensing} (\cite[cf.][\nopage
wp.]{CopyLeftOrg2014a}) -- could be taken as another witness for such an
attitude of strict reading: Some of its contributors elucidate in a chapter
dealing with \enquote{special topics in compliance} that \enquote{the license of
the whole work must [sic!] permit \enquote{reverse engineering for debugging
such modifications} to the library} and that one therefore \enquote{ should take
care that the EULA used for the Application does not contradict this
permission}(\cite[cf.][86]{KuhSebGin2014a}}.

A second group of well known and knowledgeable open source experts signify that
the LGPL-v2 indeed literally contains a strict rule, but that this rule actually
is not meant as it sounds: For example, two of these explain that \enquote{these
requirements on the licensed combination require that the license chosen not
prohibit the customer’s modification and reverse engineering for debugging these
modifications in the work as a whole}. But then they directly add the
limitation, that \enquote{in practice, enforcement history suggests, it means
that the license terms chosen may not prohibit modification and reverse
engineering for debugging of modification in the LGPL’d code included in the
combination}\footnote{\cite[cf.][\nopage wp., chapter LGPLv2.1, section
6]{MogCho2014a}. Such a mitigation can also be found in the tutorial of
copyleft.org: After they have summarized the LGPL-v2 sentence as a strict rule,
they directly continue, that one \enquote{[\ldots] must refrain from license
terms on works based on the licensed work that prohibit replacement of the
licensed components of the larger non-LGPL'd work, or prohibit decompilation or
reverse engineering in order to enhance or fix bugs in the LGPL'd components}
(\cite[cf.][86]{KuhSebGin2014a}). This added specification indicates, that one
only has to facilitate the modification of the library and that reverse
engineering can be ignored as long as there are other ways to improve the
embedded library.}.

Finally, a third group of experts prefers not to discuss the problem of reverse
engineering, although this technique is literally mentioned in the license and
although they want explain how to use GPL/LGPL licensed software
complaintly\footnote{An article of Terry J. Ilardi might be taken as a first
witness of this third strategy: he profoundly explains the essence of the LGPL,
he especially discusses §6, and he delivers applicable rules like \enquote{DO
NOT statically link to LGPL [\ldots] code if you wish to keep your program
proprietary}. But he does not discuss \emph{reverse engineering}
(\cite[cf.][5f]{Ilardi2010a}). Similarily argues Rosen
(\cite[cf.][121ff]{Rosen2005a}). And -- despite their comments on reverse
engineering in the specific chapter \emph{special topics in compliance} -- the
copyleft.org document can also be taken as an instance of this attitude:
Although its' authors recommend to \enquote{study chapter 10 carefully} for
establishing an adequate \enquote{compliance with LGPLv2.1}
(\cite[cf.][86]{KuhSebGin2014a}), this chapter 10 -- dedicated to the meaning of
the \enquote{Lesser GPL} -- does not deal with reverse engineering, although it
discusses the §6 of the LGPLv2.1 in depth (\cite[cf.][56ff, esp.
60f]{KuhSebGin2014a}).}.

This situation must bother companies and people who want to use open source
software compliantly and who therefore are looking for guidance. Particularly it
disturbs those who want to protect their business relevant software. At the end,
they might consider that this sentence is not consistently understood by the
open source community itself. And -- as far as we know -- at least some of these
companies preventively prohibit their developers to embed LGPL licensed
components into programs which contain business relevant techniques.
Unfortunately, this consequence does not only obstruct access to a large set of
well written free software, but it is also difficult to obey such an
interdiction consequently: The glibc, which enables the programms to talk with
the kernel of the GNU/Linux system\footnote{cf.
http://www.gnu.org/software/libc/}, is licensed under the LGPL\footnote{cf.
http://en.wikipedia.org/wiki/GNU\_C\_Library}. And hence, this library is
indirectly linked to or combined with any program running on the GNU/Linux
system. So, if the LGPL-v2 indeed required, that reverse engineering of every
program must be allowed, which contains any LGPL license library, then every
GNU/Linux user would be allowed to examine every GNU/Linux progam by
\emph{reverse engineering}, simply, because finally every GNU/Linux program is
linked to the glibc\footnote{This conclusion might surprise the reader. But it
is inferred with exactly the same arguments as the conclusion, that without a
licence offering a weaker copyleft every program would have been licensed under
the GPL. The copyleft.org document explains this argumentation in great detail
(\cite[cf.][56f]{KuhSebGin2014a}).}. In other words: if the LGPL indeed required
the permission of reverse engineering, then every GNU/Linux program may be
reverse engineered.

But an exhaustive reading of the LGPL-v2 delivers a strong indicator for
another, more 'liquid' understanding of the LGPL: The preamble explains the
reason for offering another weaker license beside the GPL. It says that
\enquote{[\ldots] on rare occasions, there may be a special need to encourage
the widest possible use of a certain library, so that it becomes a de-facto
standard} and that therefore it could be strategicly necessary to \enquote{allow
[\ldots] non-free programs [\ldots] to use the library} without enforcing that
these programs become free software too\footcite[cf.][\nopage wp,
§preamble]{Lgpl21OsiLicense1999a}.

So, if the LGPL had indeed determined that every program linked or combined to
any LGPL libary may be reverse engineered, then the LGPL would have an effect
contrary to its own intention. It would have introduced something like security
by obscurity: First, the LGPL allows to protect the internals of your own work
against investigation because the code of the non-free programm using the
library does not necessarily have to be published as well\footnote{The weak
copyleft has been introduced for encouring the widest possible use of the
library}. But in the end the user would also be allowed to reverse engineer the
received binary -- and hence would nevertheless be able discover all
internals\footnote{It would only cost a little more effort - as security by
obscurity indicates.} By this means, the LGPL-v2 would have undermined its' own
raison d'$\grave{e}$tre introduced by its' inventors: under such circumstances
there probably would have been less hope that any LGPL library could have become
a defacto standard.

We know that the inventors of the GNU licenses and GNU software are very
sophisticated experts. They never would have published such an inconsistent
document. This dissent read in(to) the document is a strong indicator that there
must be a better way to understand the license. Thus, it is up to us, the
followers, to explicate a more adequate interpretation. Of course, such an
interpretation must be grounded on the written text. We, the scholars, must read
the license very strictly. We have to deduce 'understandings' only by matching
the interpretations explicitly, strictly, and reasonably back to the license text
itself.

Encouraged by the indication that a better understanding of the license may
exist and contrary to the other strategies, we are going to prove that, in
fact, none of the open source licenses\footnote{being discussed in the OSLiC} in
general require to allow reverse engineering of software containing a component
licensed under that open source license. In particular, we will prove, that even
the LGPL does not claim this permission generally: We want to explain, why the
LGPL only requires to permit reverse engineering if and only if the LGPL
licensed component is embedded into a statically linked and distributed piece of
software. Moreover, we want to show that in all other cases the LGPL allows 
to distribute packages without granting permission to reverse engineer the
software which uses the LGPL licensed library\footnote{By the way, our analysis
should also provide proof that the LGPL is not something like a 'poisoned'
license containing \enquote{an imprenetrable maze of technology babble} which
\enquote{[\ldots] should not be in a general-purpose software license}
(\cite[cf.][124]{Rosen2005a}). The challenge of the today's descendants is to
understand the former inventors of the GNU licenses and their way to think about
computing - including all the hassle the computing language C might provoke.}.
We hope that our analysis, grounded on the license text itself, will support
companies and people to compliantly use open source software more often and with
less scruples.

Hence, let us prove our position 'bottom up'. Let us firstly show that it is
true for the LGPL-v2 -- by explicating the license text lingually, then
logically, and finally empirically, before we infer the correct understanding.
Then let us show that it is also true for the LGPL-v3. And in the end let us
show that it is true for all other licenses, analysed by the OSLiC.

\subsection{Reverse Engineering in the LGPL-v2}
The LGPL-v2.1 contains one sentence which literally refers to the issues of 
\emph{reverse engineering}:

\begin{quote}\noindent\emph{\enquote{[\ldots] you may [\ldots] combine or link a
\enquote{work that uses the Library} with the Library to produce a work
containing portions of the Library, and distribute that work under terms of your
choice, provided that the terms permit modification of the work for the
customer's own use and \emph{reverse engineering} for debugging such
modifications.}\footcite[cf.][\nopage wp]{Lgpl21OsiLicense1999a}}
\end{quote}

\subsubsection{Lingistical Clarification}

For fulfilling our rule, to read the text strictly and deduce our
interpretations reasonably, let us firstly only highlight the syntactical
conjunctions for simplifying the understanding:

\begin{quote}\noindent\emph{\enquote{[\ldots] you may [\ldots] combine
\textbf{or} link a \enquote{work that uses the Library} with the Library to
produce a work containing portions of the Library \textbf{and} distribute that
work under terms of your choice, \textbf{provided that} the terms permit
modification of the work for the customer's own use \textbf{and} \emph{reverse
engineering} for debugging such modifications.}\footcite[cf.][\nopage wp. herv.
KR.]{Lgpl21OsiLicense1999a}}
\end{quote}

It is evident that the conjunction \emph{'provided that'} is splitting the
sentence into two parts: you are allowed to do something \emph{provided that} a
precondition is fulfilled. Additionally, both parts of the sentence --
the one before the conjunction \emph{'provided that'} and the part after it --
are syntactically condensed embedded phrases which also contain subordinated 
conjunctions and elliptical constructions\footnote{cf.
http://en.wikipedia.org/wiki/Ellipsis\_\%28linguistics\%29, wp.
}. These syntactical interconnections must be disbanded:

Let us firstly dissolve the syntactical compression before the conjunction
\emph{'provided that'}: It is established by using the two other conjunctions
\emph{and} and \emph{or} and introduced by the subordinating phrase \emph{you
may [\ldots]}. Unfortunately, from a formal point of view, one can read the
phrase \emph{you may (X or Y and Z)} as two different groupings: either as \emph{you
may ((X or Y) and Z)} or as \emph{you may (X or (Y and Z))}.

But, fortunately, we know from the semantic point of view that speaking about
\emph{\enquote{[\ldots] combining \textbf{or} linking [\ldots something] to
produce a work containing portions of the Library}} denotes two different
methods which both can \emph{join} the components \emph{\enquote{[\ldots] to
produce a work containing portions of the Library}}. So, let us for a moment
simply replace the string \emph{\enquote{combine or link}} by the string
\emph{\enquote{*join}}\footnote{When the LGPL and the GPL were initially
defined, the C programming language was the predominant model of software
development. Knowing this method eases the understanding of these licenses.
Thus, it is not totally wrong to take this token *join also as a curtsey to the
C programming language}. This reduces the syntactical structure of the sentence
back to the simple phrase \emph{you may (W and Z)} in which \emph{W} stands for
\emph{(X or Y)}.

Now, we see aso that the phrase \emph{you may (W and Z)} itself is a
condensed version of the explicit phrase \emph{ (you may W) and (you may Z)}.

Finally we have to note, that the phrase before the conjunction \emph{'provided
that'} contains also a linguistic ellipsis\footnote{cf.
http://en.wikipedia.org/wiki/Ellipsis\_\%28linguistics\%29, wp.
}: It says that you may *join the components \enquote{to produce \textbf{a work
containing portions of the Library} \textbf{and} distribute \textbf{that work}
under terms of your choice}. With respect to the English grammar, we may
conclude that the second term \emph{that work} refers back to the previously
introduced specification of \emph{a work containing portions of the Library}: if
a complete phrase has just been introduced explicitly, then the English language
allows to reduce its' next occurence syntactically while its' complete meaning
is retained. Hence, conversely, we are allowed to unfold the reduced form to
restore the complete phrase.

So -- overall -- we may understand the phrase before the conjunction
\emph{'provided that'} as a phrase with the structure \emph{(you may W) and (you
may Z')}:

\begin{quote}\noindent\emph{\textbf{((}you may [\ldots] \emph{*join} a
\enquote{work that uses the Library} with the Library to produce a work
containing portions of the Library\textbf{) and (}you may [\ldots] distribute
that work containing portions of the Library under terms of your
choice\textbf{))}} \textbf{provided that} [\ldots]\end{quote}

Theoretically, a reader of the OSLiC could reject our first dissolution of the
LGPL-v2-§6-sentence. But for reasonably denying our interpretation he has to
deliver other resolutions of the lingustic elliptical subphrases or other
dissolvations of the conjunctions. Fortunately, it seems to be evident that such
attempts must violate the English grammar.

Let us now dissolve the part after the conjunction \emph{'provided that'}: The
subphrase \emph{the terms permit} syntactically refer to both, the
\emph{modifcation} and the \emph{reverse engineering}. The embedded conjunction
\emph{'and'} establishes a more stylish compression. So, it should be clear,
that saying

\begin{quote}\noindent\textbf{provided that} \emph{the terms permit modification
of the work for the customer's own use \emph{\textbf{and}} reverse engineering
for debugging such modifications}\end{quote}

means

\begin{quote}\noindent\textbf{provided that} \emph{the terms permit
\textbf{(} modification of the work for the customer's own use \emph{\textbf{and}}
reverse engineering for debugging such modifications\textbf{)}}\end{quote}

and is totally equivalent to the sentence 

\begin{quote}\noindent[\ldots] \textbf{provided that} \emph{\textbf{((}the terms
permit modification of the work for the customer's own use\textbf{)}
\emph{\textbf{and}} \textbf{(}the terms permit reverse engineering for debugging
such modifications\textbf{))}}.
\end{quote}

We believe that there is no other possibility to understand this sentence with
respect to the rules of the English language. Nevertheless, formally this is the
next point where an OSLiC reader may formally disagree with us. But if he really
wants to object our dissolution, he must deliver another valid interpreation of
the scope of the conjunction \emph{and} or he must deliver another resolutions
of the liguistic ellipsis. But we reckon, that one can not reasonably argue for
such alternatives.

Finally, there are other deeply embedded ellipses, which need to be resolved
as well:

\begin{enumerate}
  \item  In the part before the splitting conjunction \emph{'provided that'} we
  had to expand the abridging \emph{'that work'} by its intended explicated
  version \emph{'that work containing portions of the Library'}. The first
  subphrase after the splitting conjunction \emph{'provided that'} also contains
  the term \emph{'the work'}. Formally, this term can either refer to \emph{'the
  work that uses the library'} as one of the components which are joined, or it
  can refer to \emph{'the work containing portions of the Library'} as the
  result of joining the components. We decide to constantly dissolve the
  elliptic abridgement by the phrase \emph{'the work containing portions of the
  Library'}.
  \item The first clause of the part after the splitting conjunction
  \emph{'provided that'} talks about the purpose of \enquote{permitting
  modification of the work} which we just had to unfold to the phrase
  \emph{'permitting modification of the work containing portions of the
  Library'}. The second clause talks about the purpose of \enquote{permitting
  reverse engineering}: it shall support the \enquote{debugging [of] such
  modifications}. The pronoun \emph{'such'} indicates that the word
  \emph{'modifications'} refers back to the just unfolded phrase
  \emph{modification of the work containing portions of the Library}. So, even
  the second sentence has to be expanded to that explicit phrase.
  \item Finally and only for being complete, we also have to unfold the clause
  \enquote{the terms} to the form which is predetermined by the first referred
  instance \enquote{the terms of your choice}
\end{enumerate}

So -- overall -- we are allowed to rewrite the first sentence of the LGPL-v2, §6
in the following form, namely without having changed its meaning:

\begin{verbatim}

( ( you may 
       *join a work that uses the Library with the Library
        to produce a work containing portions of the Library )
  AND 
  ( you may 
        distribute that work containing portions of the Library
        under terms of your choice 
) )
PROVIDED THAT
( ( the terms of your choice permit 
        modification of the work containing portions of 
        the Library for the customer's own use )
  AND
  ( the terms of your choice permit
        reverse engineering for debugging modifications 
        of the work containing portions of the Library   
) )
\end{verbatim}

At this point we must recommend all our readers to verify that this 'structured
explicated presentation' does exactly mean the same as the intially quoted
LGPL-v2-§6-sentence. We are now going to discuss some of its' logical aspects by
some formal transformations. For accepting our operations and linking the
results back to the original LGPL-v2-§6-sentence, it is very helpful to know
that one already had to accept the equivalence of this explicated form and the
more condensed original version. For reviewing the equivalence the reader should
ask himself which of our rewritings are wrong, why they are wrong and which
alternatives can reasonably be offered for solving the syntactical issues which
disposed us to chose our solutions. Again, we ourselves -- of course -- are
profoundly convinced that both versions are completely equivalent.

\subsubsection{Logical Clarification}

For simplifying our discussion let us now replace the meaningful terminal
phrases of our form by some logical variables:

\begin{description}
  \item[$\Gamma$] :- (you may *join a work that uses the Library with the
  Library to produce a work containing portions of the Library) 
  \item[$\Delta$] :- (you may distribute that work containing portions of the
  Library under terms of your choice)
  \item[$\Phi$] :- (the terms of your choice permit modification of the work 
  containing portions of the Library for the customer's own use)
  \item[$\Sigma$] :- (the terms of your choice permit reverse engineering for
  debugging modifications of the work containing portions of the Library)
  \item[$\Theta$] :- \emph{$\Gamma$ and $\Delta$}
  \item[$\Omega$] :- \emph{$\Phi$ and $\Sigma$}
\end{description}

Based on these definitions, we can syntactically reduce our LGPL-v2-§6-sentence
to the formula \emph{$(\Gamma$ and $\Delta)$ provided that $(\Phi$ and
$\Sigma)$} or -- even shorter -- to \emph{$(\Theta$ provided that
$\Omega)$}.

Now, we have to clarify the meaning of the conjunction \emph{'provided that'}:

Obviously, \emph{provided that} means something like \emph{under the
precondition that}. So, one might try to take this conjunction as another more
stylish version of the well known \emph{if(\ldots)then(\ldots)}-formula, also
known as logical implication\footnote{Actually the logical implication and the
computational if-then-construct are not equivalent. Fortunately, we can later on
show, that in the context of this discussion the difference can be ignored.} But
there might also exist people who tend to say that -- beyond this traditional
interpretation -- the conjunction \emph{provided that} has also a temporal
dimension: \emph{$\Phi$ and $\Sigma$} must be realized, \textbf{before}
\emph{$\Gamma$ and $\Delta$} can happen.

Nevertheless, at first we have to consider the process of sequencing the
linguistic form into a logical formula: if we take the conjunction
\emph{provided that} as another form of the logical implication, it is not
obviously clear, which part of the  linguistic sentence must become the premise,
and which the conclusion. Does \emph{$\Theta$ provided that $\Omega$} mean
\emph{if $\Theta$ then $\Omega$} or \emph{if $\Omega$ then $\Theta$}?

Obviously, \emph{provided that} wants to establish something like a
precondition. So, one might conclude that \emph{$(\Theta$ provided that
$\Omega)$} means \emph{(if $\Omega$ then $\Theta)$} or -- more logically notated
-- \emph{$((\Phi$ $\wedge$ $\Sigma)$ $\rightarrow$ $(\Gamma$ $\wedge$
$\Delta))$}. If this interpretation is adequate, it must fulfill the intended
purpose of the corresponding LGPL-v2-section: there, the LGPL wants to regulate
the distribution of works containing portions of LGPL libraries. 

For facilitating the understanding of our our argumentation. Let us first 
check whether our interpretation fits the purpose of the LGPL by unfolding
the slightly reduced version \emph{$(\Sigma$ $\rightarrow$ $\Delta))$} back to
the corresponding verbal form:

\begin{quote}\noindent\emph{\textbf{if (} [\ldots] the terms permit reverse
engineering for debugging modifications of the work containing portions of the
Library, \textbf{) then (} [\ldots] you may distribute that work containing
portions of the Library under the terms of your choice.\textbf{)}}\end{quote}

Now we can see the problem: An implication as a whole is false only if the
premise is true and the conclusion is false. In all other cases it is true.
Especially, it is true, if the premise is false: in this case the truth value of
the conclusion does not matter in any sense. If we take this implication as a
rule, which shall determine our behaviour, then this implication only supports
us, if we already have decided to permit reverse engineering. In this case we
are allowed to distribute the work containing portions of the Library.
But from the fact, that we have not permitted reverse engineering follows
nothing. Especially, it does not follow that we may not distribute the work
containing portions of the Library. So, from the viewpoint of the formal logic
this translation of the original conjunction \emph{'provided that'} says, that
if the terms of your own license do not permit reverse engineering for debugging
modifications of the work containing portions of the Library\footnote{The
premise is false.} then you may or may not distribute that work containing
portions of the Library under the terms of your choice\footnote{The truth value
of the conlusion is undetermined by the rule.}. Hence, we must state that this
interpretation does not fulfill the purpose of the LGPL-V2: if reverse
engineering is not allowed, the distribution of the work containing portions of
the Library is not regulated. We have to conclude, that this sequencing the
LGPL-v2-§6-sentence as a logical implication is simply wrong.

Does this conclusion also have to be deduced if we completely unfold the
complete formula \emph{$((\Phi$ $\wedge$ $\Sigma)$ $\rightarrow$ $(\Gamma$
$\wedge$ $\Delta))$}? The answer is yes: the premise \emph{$((\Phi$ $\wedge$
$\Sigma)$} contains a logical conjunction. So the truth value of the whole
formula depends on the truth value of on the terminal statement $\Sigma$. Hence
we can directly transfer our result, deduced for the slightly reduced formula
to the unfolded complete formula.

Let us now test the other possibility. Let us ask, whether \emph{$\Theta$
provided that $\Omega$} means \emph{if $\Theta$ then $\Omega$} or -- more
logically notated -- \emph{$((\Gamma$ $\wedge$ $\Delta)$ $\rightarrow$ $(\Phi$
$\wedge$ $\Sigma))$}. If we again for a moment focus on the reduced version
\emph{$(\Delta$ $\rightarrow$ $\Sigma)$} and dissolve our replacements, then we
get back the rule:

\begin{quote}\noindent\emph{\textbf{if (} [\ldots] you may distribute that work
containing portions of the Library under the terms of your choice, \textbf{)
then (} [\ldots] the terms permit reverse engineering for debugging
modifications of the work containing portions of the Library.
\textbf{)}}\end{quote}

Now we can see, that this version perfectly regulates the distribution of works
containing portions of LGPL libraries: If we are allowed to do so or -- in other
words: if we are compliantly distributing works containing portions of LGPL
libraries\footnote{The premise is true.}, then we have to permit reverse
engineering\footnote{The conclusion must be true, too}. This follows from
applying \emph{Modus Ponens} to the implication\footnote{A true premise evokes a
true conclusion based on the given truth of the implication / rule itself.}. And
if we do not permit reverse engineering\footnote{The conclusion is false.}, then
we are not allowed to distribute works containing portions of LGPL
libraries\footnote{The premise must be false, too}. This follows from applying
\emph{Modus Tollens} to the implication\footnote{A false conclusion evokes a
false premise based on the given truth of the implication / rule itself.}

Based on this clarification, we can reasonably replace the more stylish
conjunction \emph{'provided that'} by its more known equivalent
\emph{'implication'} which we do not notate as traditional
\emph{if-then}-construction\footnote{Here we can also see, that the difference
between the if-then-command as part of a procedural computer language and the
logical implication does not influence our results: In the conext of a
procedural if-then-command the truth of the premise triggers the execution of
the conclusion. In our discussion, this aspect is totally covered by the Modus
Ponens derivation of the logical interpretation. And the Modus Tollens
derivation of the logical interpretation on the other side does not play any
role in a procedural if-then-command. So, it was the right decision to
understand the LGPL-v2-§6-sentence logically and not as procedual command.}, but
as logical implication \emph{$\rightarrow$}. So, we now get

\begin{description}
  \item[\#] $\Theta$ $\rightarrow$ $\Omega$
  \item[$\equiv$] ($\Phi$ $\wedge$ $\Sigma$) $\rightarrow$ ($\Gamma$ $\wedge$
  $\Delta$)
  \item[$\equiv$]
\begin{alltt}   
  ( ( [\(\Phi\)] you may 
       *join a work that uses the Library with the Library
       to produce a work containing portions of the Library )
  \(\wedge\)
  ( [\(\Sigma\)] you may 
        distribute that work containing portions of the 
        Library under terms of your choice 
) )
\(\rightarrow\)
( ( [\(\Gamma\)] the terms of your choice permit 
        modification of the work containing portions of 
        the Library for the customer's own use )
  \(\wedge\)
  ( [\(\Delta\)] the terms of your choice permit
        reverse engineering for debugging modifications 
        of the work containing portions of the Library   
) )
\end{alltt}

\subsubsection{Empirical Clarification}

Now, we can simplify this formula once more by regarding some empirical facts:

If we are allowed to do $\Sigma$, in other words: to \emph{distribute that work
containing portions of the Library under terms of your choice} and if we do so,
then we must already have done $\Phi$, in other words: we must have
\emph{*joined a work that uses the Library with the Library to produce a work
containing portions of the Library} - whereby \emph{joining} means, that we have
\emph{combined or linked a work that uses the Library with the Library to
produce a work containing portions of the Library}.

So, obviously, the $\Sigma$ empirically implies $\Phi$ (but not vice versa). But
is this really correct? Let us check this statement by assuming the opposite:
If the contrary is true, then there must exist a \emph{work containing portions
of the Library} which has been gained from the library without that both
components have been linked or combined. To talk about a \emph{work containing
portions of the Library} and also to state, that this work is not combined with
the library in any sense, is indeed self-contradictory: based on the meaning of
\emph{being combined with} it seems empirically unpossible that a work
literally contains more or less parts of a library and at the same time is not
combined with the library.

But there is still one little empirical implication which also should become an
explicated equitation: $\Phi$ explains that the \emph{work that uses the
Library} and the used Library itself together are tansformed into a \emph{work
containing portions of the Library}. So, formally, one might ask, whether this
\emph{work containing portions of the Library} also still \emph{uses the
Library}. Unfortunately, it is empirically possible, that the process of
combining the two components (a) copies all original portions of the library
into a something like a 'dead end section' of the program which is never
excuted, and (b) replaces all original portions of the library by functionally
equivalent portions of another library.


The answer is of course 'Yes'. But let us nevertheless verify this
result by a proof by contradiction: Let us assume that the \emph{work containing
portions of the Library} does no longer use the Library although it has been
built on the base of \emph{work that uses the Library} and the Library itself
and although it indeed contains portions of the Library. This would be only
possible, if (a) every element of the Library inside of the work had been moved
to a 'dead code area' which will never be executed and if (b) at each place,
where a piece of the Library has been extracted functionally replaced by a
functional equivalent, but literally different foreign element

So, we can indeed
conclude that $\Sigma$ empirically implies $\Phi$; and we can reduce the
LGPL-v2-§6-sentence to its' real core:

\begin{quote}
\begin{alltt}   
(   [\(\Sigma\)] you may 
        distribute that work containing portions of the 
        Library under terms of your choice 
) 
\(\rightarrow\)
( ( [\(\Gamma\)] the terms of your choice permit 
        modification of the work containing portions of 
        the Library for the customer's own use )
  \(\wedge\)
  ( [\(\Delta\)] the terms of your choice permit
        reverse engineering for debugging modifications 
        of the work containing portions of the Library   
) )
\end{alltt}

This essence of the LGPL-v2-§6-sentence directly explain us under which
conditions we have to allow reverse engineering for using LGPL licensed
components compliantly. We have only to determine when a work 

 the outstanding rest can be analyzed straight
forward:

\end{quote}


\end{description}



\subsection{Reverse Engineering in the LGPL-v3}
Also the LGPL-v3.0 contains a paragraph explicitly referring the \emph{reverse
enigeering}:


\begin{quote}\emph{
\enquote{You may convey a Combined Work under terms of your choice that,
taken together, effectively \emph{do not restrict} modification of the portions
of the Library contained in the Combined Work and \emph{reverse engineering} for
debugging such modifications, if you also do each of the following
[\ldots]}\footcite[cf.][\nopage wp]{Lgpl30OsiLicense2007a}}
\end{quote}


\subsection{Reverse Engineering in the GPL}

\subsection{Reverse Engineering in the other Licenses}

\subsection{Reverse Engineeringing as Right}

Unfortunately, the meaning of \emph{reverse engineering} of software is not as
clear as one whishes it should be\footnote{quotings}. But fortunately we can
limit ourselves to the consensus, that \emph{reverse [software] engineering} is
the opposite of \emph{normal [software] engineering}: Normal software
engineering is done on the base of human readable and understandable software
sources\footnote{The GPL uses 'the normal' \ldots}. So we can conclude on the
one side, that there must exist another form of software which can not directly be
understood and easily be modified. One instance of this form is the machine
specific object code generated by compling the sources. Another instance might
be the JAVA compilation code. Let us -- as it is often done -- simply call this
other form of software the 'binary'. On the other side, we can conclude that
\emph{reverse engineering} is the process to understand and modify an application on the base of these other forms by using other methods than the normal software development
methods. One of these other methods is decompilation. Another might be using a
hex editor.

With respect to the open source concept, one has also to conclude that primarily
\emph{reverse engineering} is right which must be granted by the copyright
holders. There exist to ways to grant this right: Firstly, the license can grant
this right explcitly. In this case it must contain the word \emph{reverse
engineering} or -- for example -- decompilation. Secondly, the license can grant
the right to analyze, to understand, and to modify the application generally, so
that these rights also cover the work with

\subsection{\ldots}

Being sure that there does not exist any implicitly given admission to reverse 
engineer a program is important for those companies and developers who want to
keep their own work closed although it uses open source software as components.
To do so is no fault. The permissive open source licenses and the licenses with
weak copy left are designed to enable this kind of use. So it is worth to
investigate the facts of open source licenses and reverse enginerring.



Hence, open source licenses must either grant the right to use the software, to
study it and to modify it or they must explicitly grant the right of reverse
engineering or decompiling.

Generally, closed software is forwarded with the intention not to give the
recipients access to the source code. Therefore closed software is mostly
delivered in form of binaries. Hence, in case of closed software it is mostly
not allowed to decompile the distributed binaries.

As opposed to this, normally open source software is not affected by the problem
of reverse engineering - at least if the source code of the software is also
accessible: Licenses with strong or weak copyleft require that a distributed
binary must be accompanied either by the corresponding source code itself or by
an offer to get it. Permissive licenses allow also to distribute the software
only in form of binaries. If one wants to decompile a program licensed under
such a permissive license this license must explicitly or imlicitly also grant
the right to decompile the binary.




And in the LGPL-3.0, one can find the sentence


Based on these predications one can find the statement that those developers who
compliantly integrate an LGPL licensed library as component into an
'overarching' program, must also grant to all recipients of this program the
right to decompile the overarching program.  In other words: it is sometimes argued
that the LGPL allows reverse engineering of all works that use the
library\footnote{In general, the situation seems to be not as clear as possible.
For example, an important American author states, that the sections 5 and 6 of
the LGPL \enquote{[\ldots] are an impenetrable maze of technological babble} and
that they \enquote{[\ldots] should not be in a general-purpose software
license}\cite[cf.][124]{Rosen2005a}. And he does not discuss the granting of
the LGPL to execute a reverse engineering.}.

Despite of these substantial expressions and despite of the reputation of all
these witnesses, the OSLiC wants to show, that the LGPL actually requires to
allow a reverse engineering only in case of distributing a statically linked
program. For accepting this conclusion, one has to follow the sentences of the
LGPL-2.1 license very strictly:

First of all, the LGPL-2.1 distinguishes between a \enquote{work based the
Library} and a \enquote{work that uses the Library}. And a \enquote{Library} is
defined as \enquote{a collection of software functions and/or data prepared so
as to be conveniently linked with application programs (which use some of those
functions and data) to form executables}\footcite[cf.][\nopage
wp §0]{Lgpl21OsiLicense1999a}. This definition already contains an important
stipulation: neither the library itself (which delivers functions etc.) nor the
program (which wants to use the delivered elements) is an executable.
Executables are formed by linking the library and the program. Based on this
viewpoint, the LGPL-2.1 additionally determines that 

\begin{quote}\emph{\enquote{
A \enquote{work based on the Library} means either the Library or any derivative
work under copyright law: that is to say, a work containing the Library or a portion
of it, either verbatim or with modifications and/or translated straightforwardly
into another language.}\footcite[cf.][\nopage wp §0]{Lgpl21OsiLicense1999a} }
\end{quote}

Hence, in the wording of the LGPL-2.1, you have created a \enquote{derivative
work} of the library whenever you have expanded or reduced \enquote{the
collection of software functions and/or data}, whenever you have
\enquote{modified} some of its \enquote{functions and/or data}, and whenever you
have copied \enquote{portions} of the library into another work. And this
process of generating a derivative work does not depend on the format, neither
on that of the library nor on that of the other work.\footcite[cf.][\nopage wp
§0]{Lgpl21OsiLicense1999a}.

But the LGPL-2.1 knows that there nevertheless might exist software which - in
this sense - is not a derivative work of the Library. If such \enquote{a
program contains no derivative of any portion of the Library, but is
designed to work with the Library by being compiled or linked with it, [then
it] is called a \enquote{work that uses the Library}}\footcite[cf.][\nopage wp
§5]{Lgpl21OsiLicense1999a}.

Based on this distinction, the LGPL-2.1 can clearly assert, that \enquote{such
a work, in isolation, is not a derivative work of the Library, and therefore
falls outside the scope of this License}\footcite[cf.][\nopage wp
§5]{Lgpl21OsiLicense1999a}: if the isolated work is designed to work with the
library but still does not contain any derivative of any portion of the library,
it is not derived from the library.

From the view of a programmer -- especially from the view of a
C-programmer\footnote{When the GPL and the LGPL were designed, the programming
language C was the paradigm to generate executable software} -- this definition
causes a problem. Designing a program to use a library means including the
header files of the library. Such header files, which are developed and
delivered by the developer of the Library, may not contain declarations of
functions and data types, but also generally usable variants and inline
functions. Using these declarations is the common way to design a program to
work with a library. So lately such a designed program contains elements of the
library. So, a C-programmer might argue that designing a program to work with
the library let the program become a work based on the Library.

Thankfully, the LGPL-2.1 addresses this problem and generates a solution:

The LGPL-2.1 clearly states that the compiled version of a work that uses the
library -- in opposite to its source code version -- can indeed become a
derivative work of the library: 

\begin{quote}\emph{\enquote{ When a \enquote{work that uses the Library} uses
material from a header file that is part of the Library, the object code for the
work may be a derivative work of the Library even though the source code is
not.}\footcite[cf.][\nopage wp §5]{Lgpl21OsiLicense1999a} }
\end{quote}

From a viewpoint of a (C-) programmer this statement is adequate and clearly
comprehensible: the source code of the \enquote{work that uses the Library}
itself normally containes only pure include directives which refer to the names
of header files of the library. The act of compiling this source code can indeed 
enrich the work that uses the library by parts of the library: the preprocessor
will expand inline functions and data types (and therefore copy code of the
library into the \emph{work that therefore no longer only uses the Library}, the
compilation will add variables and references to variables into the object file,
and so on and so on. But if the include files only contain
fucntions declartions, then compiled onject code of the \enquote{work that uses
the Library} still does not contain elements of the library.

And finally, when both parts will be linked, the LGPL-2.1 regards the
result of the linking process indeed as a derivative work:
\begin{quote}\enquote{linking a \enquote{work that uses the Library} with the
Library creates an executable that is a derivative of the Library (because it
contains portions of the Library), rather than a \enquote{work that uses the
library}}\footcite[cf.][\nopage wp §5]{Lgpl21OsiLicense1999a}.
\end{quote}

So, the concrete status of the compiled version of the \enquote{work that uses
the Library} which still has not been linked to the library is vague. To give
clearness back to all users the LGPL defines that if the compiled but still
unlinked version of \enquote{work that uses the Library} indeed already contains
elements of the library, then is a derivative work of the libary. But if these
adopted elements are normal elements which are offered to design a work to work
with the library, then being a derivative work shall not have any effect. Or in
the words of the LGPL-2.1

\begin{quote}\enquote{ If such an object file [the compilation of the
\enquote{work that uses the Library}] uses only numerical parameters, data structure
layouts and accessors, and small macros and small inline functions (ten lines or
less in length), then the use of the object file is unrestricted, regardless of
whether it is legally a derivative work }\footcite[cf.][\nopage wp
§5]{Lgpl21OsiLicense1999a}.
\end{quote}

Hence, generally the LGPL-2.1 regards the \enquote{work that uses the Library}
as an independ unit which is not covered by the rules of the LGPL-2.1 license --
as long as both parts have not become an integrated entity as for example the
process of linking generates.

Based on these specifications one can clearly show that the LGPL-2.1 only
requires the permisson of reverse engineering in case of distributing a
statically linked integrated entity containing the \enquote{work that uses the
Library} and the library:

\begin{itemize}
\item First, anyone is allowed to distribute the LGPL-2.1 licensed library to
any third party. This follows from the freedom of free software.

\item Second, the copyright owners are allowed to distribute their \enquote{work
that uses the Library} to any third party, too. This follows from being the copyright
owner.

\item Third, if the copyright owner distribute their \enquote{work that
uses the Library} as object file which is not linked to the LGPL-Library, then
the LGPL does not oblige them to do anything. This follows from specicifation of
the LGPL under which conditions the use of the object file is unrestricted.

\item Hence, the recipient of the isolatedly distributed \enquote{work
that uses the Library} and the library can link these parts on his own machine
and under his own responsibility. [Follows from 1-3]
\end{itemize}

Now, we meet the question, whether the third party user which links the
seperatedly received parts on his own machine and on its behalf gets the right
to decompile the object file of the enquote{work that uses the Library} because it
is linked with the LGLP license library.

The answer is: No, he does not get the right to do so. The LGPL-2.1 clearly says
the right of reverse engineering is only given to third party, if 

%\bibliography{../../../bibfiles/oscResourcesEn}
