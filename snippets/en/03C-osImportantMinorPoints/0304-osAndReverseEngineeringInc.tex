% Telekom osCompendium 'for being included' snippet template
%
% (c) Karsten Reincke, Deutsche Telekom AG, Darmstadt 2011
%
% This LaTeX-File is licensed under the Creative Commons Attribution-ShareAlike
% 3.0 Germany License (http://creativecommons.org/licenses/by-sa/3.0/de/): Feel
% free 'to share (to copy, distribute and transmit)' or 'to remix (to adapt)'
% it, if you '... distribute the resulting work under the same or similar
% license to this one' and if you respect how 'you must attribute the work in
% the manner specified by the author ...':
%
% In an internet based reuse please link the reused parts to www.telekom.com and
% mention the original authors and Deutsche Telekom AG in a suitable manner. In
% a paper-like reuse please insert a short hint to www.telekom.com and to the
% original authors and Deutsche Telekom AG into your preface. For normal
% quotations please use the scientific standard to cite.
%
% [ Framework derived from 'mind your Scholar Research Framework' 
%   mycsrf (c) K. Reincke 2012 CC BY 3.0  http://mycsrf.fodina.de/ ]
%


%% use all entries of the bibliography
%\nocite{*}

\section{Excursion: Reverse Engineering and Open Source}

There exist some voices of important open source experts indicating that using
LGPL licensed libraries as part of a distributed 'on-top development' requires
to grant the right of reverse engineering to the receiver to that 'on-top
develoment'\footnote{So, one of these very trustworhty German experts says that
the LGPL-2.1 generally requires that a distributor of a program which accesses a
LGPL-2.1 licensed library, must grant his customer also the right to modify the
accessing program and hence also the right to execute a reverse engineering.
Originally the German text says: \enquote{Ziffer 6 LGPLv2.1 knüpft die
Erlaubnis, das zugreifende Programm unter beliebigen Lizenzbestimmungen
verbreiten zu drüfen, an eine Reihe von Verpflichtungen, die in der Praxis oft
übersehen werden: Zunächst muss dem Kunden, dem die Software geliefert wird, die
Veränderung des zugreifenden Programms gestattet werden und zu diesem Zweck auch
ein Reverse Engineering zur Fehlerbehebung. Dies dürfte alle Formen des
Debugging und das Dekompilieren des zugreifenden Programms
umfassen.}\cite[cf.][81]{JaeMet2011a}.}. Sometimes -- as part of the
simplification in a whisper down the lane -- this statement is reduced to
message, that we generally have the right of reverse engineering whenever we get
a programm containing any open source software component.

On the contrary, we are going to prove that none of the open source
licenses\footnote{being discussed in the OSLiC} generally requires to allow the
reverse engineering of a piece of software containing an open source component
which is licensed under that open source license. Particularly, we will to
prove, that even the LGPL does not claim this permission generally: we are going
to show, that the LGPL requires to permit the reverse engineering if and only if 
the LGPL licensed component is embedded into a statically linked and
distributed piece of software.

Let us prove our position 'bottom up'. Let us firstly show that it is true for
the LGPL-v2. Then let us show, that it is also true for the LGPL-v3. And finally
let us show that it is true for all other licenses, analysed by the OSLiC.

\subsection{Reverse Engineering in the LGPL-v2}
The LGPL-v2.1 contains only one paragraph which refers to the \emph{reverse
enigeering}:

\begin{quote}\noindent\emph{\enquote{[\ldots] you may [\ldots] combine or link a
\enquote{work that uses the Library} with the Library to produce a work
containing portions of the Library, and distribute that work under terms of your
choice, provided that the terms permit modification of the work for the
customer's own use and \emph{reverse engineering} for debugging such
modifications.}\footcite[cf.][\nopage wp]{Lgpl21OsiLicense1999a}}
\end{quote}

Following the general spirit of the OSLiC, we want to read the license text very
strictly. We want to derive 'understandings' by matching the interpretations
explictly and strictly back to the lincense text itself. For that purpose, let
us firstly only hightlight the syntactical conjunctions in the just quoted
sentence. This little graphical 'modification' simplifies the understanding:

\begin{quote}\noindent\emph{\enquote{[\ldots] you may [\ldots] combine
\textbf{or} link a \enquote{work that uses the Library} with the Library to
produce a work containing portions of the Library \textbf{and} distribute that
work under terms of your choice, \textbf{provided that} the terms permit
modification of the work for the customer's own use \textbf{and} \emph{reverse
engineering} for debugging such modifications.}\footcite[cf.][\nopage wp. herv.
KR.]{Lgpl21OsiLicense1999a}}
\end{quote}

It is evident that the conjunction \emph{'provided that'} is splitting the
sentence into two parts: you are allowed to do something \emph{provided that} a
precondition is fulfilled. Additionally, one has to state, that both parts --
the one before the conjunction \emph{'provided that'} and the part after it --
are syntactically condensed English sentences: they contain subordinated 
conjunctions and elliptical constructions\footnote{cf.
http://en.wikipedia.org/wiki/Ellipsis\_\%28linguistics\%29, wp.
}. Therefore, we must disband these syntactical interconnections:

Let us start with dissolving part after the conjunction \emph{'provided that'}:
It should be clear, that saying

\begin{quote}\noindent\textbf{provided that} \emph{the terms permit modification
of the work for the customer's own use \emph{\textbf{and}} reverse engineering
for debugging such modifications}\end{quote}

is totally equivalent to say that 

\begin{quote}\noindent[\ldots] \textbf{provided that} \emph{\textbf{((}the terms
permit modification of the work for the customer's own use\textbf{)}
\emph{\textbf{and}} \textbf{(}the terms permit reverse engineering for debugging
such modifications\textbf{))}}.
\end{quote}

We belief that there is no other possibility to understand this sentence with
respect to the English language rules. Nevertheless, formally this is the first
point where an OSLiC reader may disagree withn us. But if he wants to object our
dissolution, he must deliver another interpreation of scope of the
conjunction \emph{and} or he must deliver another denouement of the liguistic
ellipsis. We belief, that one can not reasonably argue for such alternatives.

Now, let us dissolve the syntactical compression positioned before the
conjunction \emph{'provided that'}. It is established by using the two other
conjunctions \emph{and} and \emph{or} and introduced by the subordinating phrase
\emph{you may [\ldots]}. Unfortunately from a formal point of view one can read
the phrase \emph{you may (X or Y and Z)} as two different groupings: either as
\emph{you may ((X or Y) and Z)} or as \emph{you may (X or (Y and Z))}. 

But fortunately we know from the semantic point of view that speaking about
\emph{\enquote{[\ldots] combining \textbf{or} linking [\ldots something] to
produce a work containing portions of the Library}} denotes two different
methods \emph{to join} the components \emph{\enquote{[\ldots] to produce a work
containing portions of the Library}}. So, let us for a moment simply replace the
string \emph{\enquote{combine or link}} by the sting
\emph{\enquote{*join}}\footnote{When the LGPL and the GPL were initially
defined, the C programming language was the predominant model of software
development. Knowing this method eases the understanding of these licenses. So,
it is not totally wrong to take this token *join also as a curtsey to the C
programming language}. This reduces the syntactical structure of the sentence
back to the simple phrase \emph{you may (W and Z)} in which \emph{W} stands
for \emph{(X or Y)}.

Additionally we know that the phrase \emph{you may (W and Z)} itself is a
condensed version of the explicit phrase \emph{ (you may W) and (you may Z)}.

Finally we have to note, that the phrase of the discussed LGPL sentence being
positioned before the conjunction \emph{'provided that'} contains
also an linguistic ellipsis\footnote{cf.
http://en.wikipedia.org/wiki/Ellipsis\_\%28linguistics\%29, wp.
}: It literally says that you may *join the components \enquote{to produce a
work containing portions of the Library \textbf{and} distribute that work under
terms of your choice}. But it means that the second term \emph{that work} refers
back to the previously introduced specification of \emph{a work containing
portions of the Library}.

Based on these explifications we are allowed understand the phrase before the
conjunction \emph{'provided that'} as \emph{(you may W) and (you may Z')}:

\begin{quote}\noindent\emph{\textbf{((}you may [\ldots] \emph{*join} a
\enquote{work that uses the Library} with the Library to produce a work
containing portions of the Library\textbf{) and (}you may [\ldots] distribute
that work containing portions of the Library under terms of your
choice\textbf{))}} \textbf{provided that} [\ldots]\end{quote}

Also this dissolution of the LGPL, §6 phrase could theoretically be rejected by
a reader of the OSLiC. But if he wants to deny our understanding he has to
deliver other resolutions of the lingustic elliptical subphrases in it or he has
to deliver another dissolvation of the logical conjunctions. It seems to be
evident that such attempts would violate the English language and its grammar.

Finally, there is another deeply embedded ellipis, which has to be resolved: In
the part before the splitting conjunction \emph{'provided that'} we had to
expand the abridging \emph{'that work'} by its intended explicated version
\emph{'that work containing portions of the Library}. In the phrase after the
splitting conjunction \emph{'provided that'} there is also mentioned the term
\emph{'the work'}. This term can either refer to \emph{'the work that uses the
library'} as one of the components which are joined, or it can refer to
\emph{'the work containing portions of the Library'} as the result of joining
the components. We decide to dissolve the elliptic abridgement by the phrase
\emph{'the work containing portions of the Library'} which is stronger than the
other.

So -- overall -- we are allowed to rewrite the first sentence of the LGPL-v2, §6
in the following form, namely without having changed its' meaning:

\begin{verbatim}

( ( you may *join a work that uses the Library with the Library
            to produce a work containing portions of the Library )
  AND 
  ( you may distribute that work containing portions of the Library
            under terms of your choice 
) )
PROVIDED THAT
( ( the terms permit modification of the work for containing 
                     portions of the Library the customer's own use )
  AND
  ( the terms permit reverse engineering for debugging such
                     modifications     
) )
\end{verbatim}

At this point we want to recommend the readers to verify that this 'structured
presentation' does exactly mean the same as the initially quoted first sentence
of the LGPL-v2, §6. But, if one of our reader wants to deny the equivalence, we
must also ask him to point out which of our rewritings is wrong, why it is wrong
and which alternative has to be selected for solving the syntactical challenge
which disposed us to select solution. We ourselves strongly believe that both
versions are completely equivalent.

Now, the heavy work is done and we can do some simple self evident steps: At
first, we are going to replace the meaningful terminal phrases our form by
logical variables:
\begin{description}
  \item[$\Gamma$] :- (you may *join a work that uses the Library with the
  Library to produce a work containing portions of the Library)
  \item[$\Delta$] :- (you may distribute that work containing portions of the
  Library under terms of your choice)
  \item[$\Phi$] :- (the terms permit modification of the work for containing
  portions of the Library the customer's own use)
  \item[$\Sigma$] :- (the terms permit reverse engineering for debugging such
  modifications)
\end{description}

Based on these definitions, we can syntatically reduce our LGPL-sentence to this
little formular

\begin{quote}\noindent $(\Gamma \; and \; \Delta) \; provided \;
that \; (\Phi \; and \; \Sigma)$
\end{quote}

Then, we want to reset the more stylish conjunction \emph{'$\Omega$ provided
that $\Theta$'} back to its more known equivalent, the implication \emph{'if
$\Theta$ then $\Omega$'}. So, we get another version of our LGPL sentence:

\begin{quote}\noindent $if \; (\Phi \; and \; \Sigma) \; then \;
(\Gamma \; and \; \Delta)$
\end{quote}

Now, we are free to take this formular either as token of a procedural
programming language. In this context, it means that if condition $\Phi$ and
condition $\Sigma$ are true then firstly the command $\Gamma$ is executed and
secondly the command $\Delta$. Or we can take this formular as a logical
expression indicating that whenever this formular as whole is true and $\Phi$
and $\Sigma$ as single terms are, then the conjuction of '$\Gamma$ and $\Delta$'
must also be true. The difference of these interpretations is very slight: it is
the difference of \emph{something has \textbf{to become} true} versus
\emph{something has \textbf{to be} true}. So, let us first start with the
logical view and discuss (not) existing differences at the end. For that
purpose, we can rewrite our LGPL-v2, §6 sentence like this:

\begin{quote}\noindent
$(\Phi \wedge \Sigma) \rightarrow (\Gamma \wedge \Delta)$
\end{quote}

Finally, we can formally derive that a knowledge base which contains only the
rule $(\Phi \wedge \Sigma) \rightarrow (\Gamma \wedge \Delta)$ is equivalent to
a knowledgebase containing the two rules $(\Phi \wedge \Sigma) \rightarrow
\Gamma)$ and $(\Phi \wedge \Sigma) \rightarrow \Delta)$. The equivalence can be
proved on the base of truth-tables\footnote{One has simply to show, that $(\Phi
\wedge \Sigma) \rightarrow (\Gamma \wedge \Delta)$ has the same truth values as
the conjunction of the isolated rules $(((\Phi \wedge \Sigma) \rightarrow
\Gamma) \wedge ((\Phi \wedge \Sigma) \rightarrow \Delta))$

 }:

\begin{enumerate}
\item $(\Phi \wedge \Sigma) \rightarrow \Gamma$
\item $(\Phi \wedge \Sigma) \rightarrow \Delta$
\end{enumerate}

So, now we know, that we can logically focus on one rule after antoher without
having to fear that we are violating the meaning of the LGPL-v2, §6 sentence.
Let us start\footnote{surprise, surprise} with the second rule $(\Phi
\wedge \Sigma) \rightarrow \Delta$ by replacing the placeholders by its
meanings. So, this rule says:

\begin{quote}\noindent\emph{ [ (the terms [of your license] permit
modification of the work for containing portions of the Library the customer's
own use) $\wedge$ (the terms permit reverse engineering for debugging such
  modifications) ] $\rightarrow$ [ you may distribute that work containing
  portions of the Library under terms of your choice ]
}
\end{quote}



\subsection{Reverse Engineering in the LGPL-v3}
Also the LGPL-v3.0 contains a paragraph explictly referring the \emph{reverse
enigeering}:


\begin{quote}\emph{
\enquote{You may convey a Combined Work under terms of your choice that,
taken together, effectively \emph{do not restrict} modification of the portions
of the Library contained in the Combined Work and \emph{reverse engineering} for
debugging such modifications, if you also do each of the following
[\ldots]}\footcite[cf.][\nopage wp]{Lgpl30OsiLicense2007a}}
\end{quote}


\subsection{Reverse Engineering in the GPL}

\subsection{Reverse Engineering in the other Licenses}





\subsection{Reverse Engineeringing as Right}

Unfortunately, the meaning of \emph{reverse engineering} of software is not as
clear as one whishes it should be\footnote{quotings}. But fortunately we can
limit ourselves to the consensus, that \emph{reverse [software] engineering} is
the opposite of \emph{normal [software] engineering}: Normal software
engineering is done on the base of human readable and understandable software
sources\footnote{The GPL uses 'the normal' \ldots}. So we can conclude on the
one side, that there must exist another form of software which can not directly be
understood and easily be modified. One instance of this form is the machine
specific object code generated by compling the sources. Another instance might
be the JAVA compilation code. Let us -- as it is often done -- simply call this
other form of software the 'binary'. On the other side, we can conclude that
\emph{reverse engineering} is the process to understand and modify an application on the base of these other forms by using other methods than the normal software development
methods. One of these other methods is decompilation. Another might be using a
hex editor.

With respect to the open source concept, one has also to conclude that primarily
\emph{reverse engineering} is right which must be granted by the copyright
holders. There exist to ways to grant this right: Firstly, the license can grant
this right explcitly. In this case it must contain the word \emph{reverse
engineering} or -- for example -- decompilation. Secondly, the license can grant
the right to analyze, to understand, and to modify the application generally, so
that these rights also cover the work with

\subsection{\ldots}

Being sure that there does not exist any implicitly given admission to reversely
engineer a program is important for those companies and developers who want to
keep their own work closed although it uses open source software as components.
To do so is no fault. The permissive open source licenses and the licenses with
weak copy left are designed to enable this kind of use. So it is worth to
investigate the facts of open source licenses and reverse enginerring.



Hence, open source licenses must either grant the right to use the software, to
study it and to modify it or they must explicitly grant the right of reverse
engineering or decompiling.

Generally, closed software is forwarded with the intention not to give the
recipients access to the source code. Therefore closed software is mostly
delivered in form of binaries. Hence, in case of closed software it is mostly
not allowed to decompile the distributed binaries.

As opposed to this, normally open source software is not affected by the problem
of reverse engineering - at least if the source code of the software is also
accessible: Licenses with strong or weak copyleft require that a distributed
binary must be accompanied either by the corresponding source code itself or by
an offer to get it. Permissive licenses allow also to distribute the software
only in form of binaries. If one wants to decompile a program licensed under
such a permissive license this license must explictly or imlicitly also grant
the right to decompile the binary.




And in the LGPL-3.0, one can find the sentence


Based on these predications one can find the statement that those developers who
compliantly integrate an LGPL licensed library as component into an
'overarching' program, must also grant to all recipients of this program the
right to decompile the overarching program.  In other words: it is sometimes argued
that the LGPL allows the reverse engineering of all works that use the
library\footnote{In general, the situation seems to be not as clear as possible.
For example, an important American author states, that the sections 5 and 6 of
the LGPL \enquote{[\ldots] are an impenetrable maze of technological babble} and
that they \enquote{[\ldots] should not be in a general-purpose software
license}\cite[cf.][124]{Rosen2005a}. And he does not discuss the granting of
the LGPL to execute a reverse engineering.}.

Despite of these substantial expressions and despite of the reputation of all
these witnesses, the OSLiC wants to show, that the LGPL actually requires to
allow a reverse engineering only in case of distributing a statically linked
program. For accepting this conclusion, one has to follow the sentences of the
LGPL-2.1 license very strictly:

First of all, the LGPL-2.1 distinguishes between a \enquote{work based the
Library} and a \enquote{work that uses the Library}. And a \enquote{Library} is
defined as \enquote{a collection of software functions and/or data prepared so
as to be conveniently linked with application programs (which use some of those
functions and data) to form executables}\footcite[cf.][\nopage
wp §0]{Lgpl21OsiLicense1999a}. This definition already contains an important
stipulation: neither the library itself (which delivers functions etc.) nor the
program (which wants to use the delivered elements) is an executable.
Executables are formed by linking the library and the program. Based on this
viewpoint, the LGPL-2.1 additionally determines that 

\begin{quote}\emph{\enquote{
A \enquote{work based on the Library} means either the Library or any derivative
work under copyright law: that is to say, a work containing the Library or a portion
of it, either verbatim or with modifications and/or translated straightforwardly
into another language.}\footcite[cf.][\nopage wp §0]{Lgpl21OsiLicense1999a} }
\end{quote}

Hence, in the wording of the LGPL-2.1, you have created a \enquote{derivative
work} of the library whenever you have expanded or reduced \enquote{the
collection of software functions and/or data}, whenever you have
\enquote{modified} some of its \enquote{functions and/or data}, and whenever you
have copied \enquote{portions} of the library into another work. And this
process of generating a derivative work does not depend on the format, neither
on that of the library nor on that of the other work.\footcite[cf.][\nopage wp
§0]{Lgpl21OsiLicense1999a}.

But the LGPL-2.1 knows that there nevertheless might exist software which - in
this sense - is not a derivative work of the Library. If such \enquote{a
program contains no derivative of any portion of the Library, but is
designed to work with the Library by being compiled or linked with it, [then
it] is called a \enquote{work that uses the Library}}\footcite[cf.][\nopage wp
§5]{Lgpl21OsiLicense1999a}.

Based on this distinction, the LGPL-2.1 can clearly assert, that \enquote{such
a work, in isolation, is not a derivative work of the Library, and therefore
falls outside the scope of this License}\footcite[cf.][\nopage wp
§5]{Lgpl21OsiLicense1999a}: if the isolated work is designed to work with the
library but still does not contain any derivative of any portion of the library,
it is not derived from the library.

From the view of a programmer -- especially from the view of a
C-programmer\footnote{When the GPL and the LGPL were designed, the programming
language C was the paradigm to generate executable software} -- this definition
causes a problem. Designing a program to use a library means including the
header files of the library. Such header files, which are developed and
delivered by the developer of the Library, may not contain declarations of
functions and data types, but also generally usable variants and inline
functions. Using these declarations is the common way to design a program to
work with a library. So lately such a designed program contains elements of the
library. So, a C-programmer might argue that designing a program to work with
the library let the program become a work based on the Library.

Thankfully, the LGPL-2.1 addresses this problem and generates a solution:

The LGPL-2.1 clearly states that the compiled version of a work that uses the
library -- in opposite to its source code version -- can indeed become a
derivative work of the library: 

\begin{quote}\emph{\enquote{ When a \enquote{work that uses the Library} uses
material from a header file that is part of the Library, the object code for the
work may be a derivative work of the Library even though the source code is
not.}\footcite[cf.][\nopage wp §5]{Lgpl21OsiLicense1999a} }
\end{quote}

From a viewpoint of a (C-) programmer this statement is adequate and clearly
comprehensible: the source code of the \enquote{work that uses the Library}
itself normally containes only pure include directives which refer to the names
of header files of the library. The act of compiling this source code can indeed 
enrich the work that uses the library by parts of the library: the preprocessor
will expand inline functions and data types (and therefore copy code of the
library into the \emph{work that therefore no longer only uses the Library}, the
compilation will add variables and references to variables into the object file,
and so on and so on. But if the include files only contain
fucntions declartions, then compiled onject code of the \enquote{work that uses
the Library} still does not contain elements of the library.

And finally, when both parts will be linked, the LGPL-2.1 regards the
result of the linking process indeed as a derivative work:
\begin{quote}\enquote{linking a \enquote{work that uses the Library} with the
Library creates an executable that is a derivative of the Library (because it
contains portions of the Library), rather than a \enquote{work that uses the
library}}\footcite[cf.][\nopage wp §5]{Lgpl21OsiLicense1999a}.
\end{quote}

So, the concrete status of the compiled version of the \enquote{work that uses
the Library} which still has not been linked to the library is vague. To give
clearness back to all users the LGPL defines that if the compiled but still
unlinked version of \enquote{work that uses the Library} indeed already contains
elements of the library, then is a derivative work of the libary. But if these
adopted elements are normal elements which are offered to design a work to work
with the library, then being a derivative work shall not have any effect. Or in
the words of the LGPL-2.1

\begin{quote}\enquote{ If such an object file [the compilation of the
\enquote{work that uses the Library}] uses only numerical parameters, data structure
layouts and accessors, and small macros and small inline functions (ten lines or
less in length), then the use of the object file is unrestricted, regardless of
whether it is legally a derivative work }\footcite[cf.][\nopage wp
§5]{Lgpl21OsiLicense1999a}.
\end{quote}

Hence, generally the LGPL-2.1 regards the \enquote{work that uses the Library}
as an independ unit which is not covered by the rules of the LGPL-2.1 license --
as long as both parts have not become an integrated entity as for example the
process of linking generates.

Based on these specifications one can clearly show that the LGPL-2.1 only
requires the permisson of reverse engineering in case of distributing a
statically linked integrated entity containing the \enquote{work that uses the
Library} and the library:

\begin{itemize}
\item First, anyone is allowed to distribute the LGPL-2.1 licensed library to
any third party. This follows from the freedom of free software.

\item Second, the copyright owners are allowed to distribute their \enquote{work
that uses the Library} to any third party, too. This follows from being the copyright
owner.

\item Third, if the copyright owner distribute their \enquote{work that
uses the Library} as object file which is not linked to the LGPL-Library, then
the LGPL does not oblige them to do anything. This follows from specicifation of
the LGPL under which conditions the use of the object file is unrestricted.

\item Hence, the recipient of the isolatedly distributed \enquote{work
that uses the Library} and the library can link these parts on his own machine
and under his own responsibility. [Follows from 1-3]
\end{itemize}

Now, we meet the question, whether the third party user which links the
seperatedly received parts on his own machine and on its behalf gets the right
to decompile the object file of the enquote{work that uses the Library} because it
is linked with the LGLP license library.

The answer is: No, he does not get the right to do so. The LGPL-2.1 clearly says
the right of reverse engineering is only given to third party, if 

%\bibliography{../../../bibfiles/oscResourcesEn}

% Local Variables:
% mode: latex
% fill-column: 80
% End:
