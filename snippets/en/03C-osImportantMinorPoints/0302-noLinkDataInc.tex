% Telekom osCompendium 'for being included' snippet template
%
% (c) Karsten Reincke, Deutsche Telekom AG, Darmstadt 2011
%
% This LaTeX-File is licensed under the Creative Commons Attribution-ShareAlike
% 3.0 Germany License (http://creativecommons.org/licenses/by-sa/3.0/de/): Feel
% free 'to share (to copy, distribute and transmit)' or 'to remix (to adapt)'
% it, if you '... distribute the resulting work under the same or similar
% license to this one' and if you respect how 'you must attribute the work in
% the manner specified by the author ...':
%
% In an internet based reuse please link the reused parts to www.telekom.com and
% mention the original authors and Deutsche Telekom AG in a suitable manner. In
% a paper-like reuse please insert a short hint to www.telekom.com and to the
% original authors and Deutsche Telekom AG into your preface. For normal
% quotations please use the scientific standard to cite.
%
% [ Framework derived from 'mind your Scholar Research Framework' 
%   mycsrf (c) K. Reincke 2012 CC BY 3.0  http://mycsrf.fodina.de/ ]
%


%% use all entries of the bibliography
%\nocite{*}

\section{Excursion: Why linking is a secondary criteria}
\footnotesize
\begin{quote}\itshape
Distributing statically or dynamically works is often discussed as a problem
(and sometimes as a solution) for acting compliantly. In this chapter, we
briefly discuss why this aspect can mostly be ignored and why it does not helps
to determine the existence of a derivative work.
\end{quote}
\normalsize

In an earlier version of the OSLiC, the its finder had also subclassified some
use cases with respect to the way an application was 'composed' as a larger
unit: In the form for gathering the necessary information, the OSLiC user had to
answer whether \emph{one was going to combine the received open source software
with other software components by linking all together statically, by linking
them dynamically, or by textually including (parts of) the open source software
into your larger unit}. From version 0.95.1 this question has totally be erased.
The authors discovered that it was not necessary to consider this aspect.

We know that being linked statically or dynamically is often and deeply
discussed by license experts. It seems to be an important aspect:

[TBD: Discussion of the literature]
%TODO Discus statically dynamicall discussion.

So, let us start with some undeinable facts: The OSLiC deals with the Apache-2.0
license\footcite [cf.][\nopage wp.]{Apl20OsiLicense2004a}, the BSD-2-Clause
license\footcite [cf.][\nopage wp.]{BsdLicense2Clause}, the BSD-3-Clause
license\footcite [cf.][\nopage wp.]{BsdLicense3Clause}, the MIT license\footcite
[cf.][\nopage wp.]{MitLicense2012a}, the MS-PL\footcite [cf.][\nopage
wp.]{MsplOsiLicense2013a}, the PgL\footcite [cf.][\nopage
wp.]{PglOsiLicense2013a} and the PHP license\footcite [cf.][\nopage
wp.]{Php30OsiLicense2013a} as instances of the permissive licenses.
Additionally, the OSLiC treats the EPL\footcite [cf.][\nopage
wp.]{Epl10OsiLicense2005a}, the EUPL\footcite [cf.][\nopage
wp.]{Eupl11OsiLicense2007a}, the LGPL\footnote{For LGPL-2.1 see \cite
[cf.][\nopage wp.]{Lgpl21OsiLicense1999a}. For LGPL-3.0 see \cite [cf.][\nopage
wp.]{Lgpl30OsiLicense2007a} }, and the MPL\footcite [cf.][\nopage
wp.]{Mpl20OsiLicense2013a} as licenses with weak copy left. Finally, the OSLiC
thoroughly discusses the GPL\footnote{For GPL-2.0 see \cite [cf.][\nopage
wp.]{Gpl20OsiLicense1991a}. For GPL-3.0 see \cite [cf.][\nopage
wp.]{Gpl30OsiLicense2007a} } and the AGPL\cite [cf.][\nopage
wp.]{Agpl30OsiLicense2007a} as licenses with strong copyleft\footnote{You can
find html based instances of these licenses in the OSLiC directory 'licenses'.
They have been downloaded from the OSI pages. All of the following statements
refer to these files.}

Only three of these licenses mention the word linking (or derivations of it):
Using the command \texttt{grep -i link * | grep -v
"<link\textbackslash{}|links\textbackslash{}|skip-link"} in a shell -- as an
operation on a set of html formatted license files -- directly shows that only
the AGPL-3.0, the ApL-2.0, the GPL-2.0, the GPL-3.0, the LGPL-2.1 and the
LGPL-3.0 are using derivations of the word linking. Additionally, the results of
the command \texttt{grep -i statical *} show that only the LGPL-2.1 uses this
'statical', while using the command \texttt{grep -i dynamical *} only hints to
the AGPL-3.0 and the GPL-3.0. This analysis already indicates that being
statically oder dynamically linked can not be as important for acting
compliantly as it often is suggested.
% 
If one reads the concrete statements, then one can see, that acting compliantly
depends only losely and only seldom on the kind of being 'combined':

\begin{description}
  \item[ApL] The Apache-2.0 license uses the word \emph{link} only once for stating
  that \enquote{[\ldots] Derivative Works shall not include works that [\ldots]
  link [\ldots] to the interfaces of, the Work and Derivative Works
  thereof}\footcite [cf.][\nopage wp.\ §0]{Apl20OsiLicense2004a}. Thus, the ApL
  does not use the criteria 'linkings' for determining a derivative work,
  neither 'being linked in general', nor 'being statically linked', nor being
  'dynamically linked'. Hence, for acting in accordance to the ApL these
  attributes can be ignored.
  \item[GPL-3.0] The GPL-3.0 uses the word \emph{link} three times: First, it defines the
  \enquote{\enquote{Corresponding Source} for a work in object code form [\ldots
  as] all the source code needed to generate, install, and [\ldots] run the
  object code and to modify the work [\ldots]} and explains that this definition
  includes \enquote{the source code for shared libraries and dynamically linked
  subprograms that the work is specifically designed to
  require}\footcite[cf.][\nopage wp.\ §0]{Gpl30OsiLicense2007a}. Second, the
  GPL-3.0 allows \enquote{[\ldots] to link or combine any covered work with a
  work licensed under version 3 of the GNU Affero General Public License into a
  single combined work, and to convey the resulting work}\footcite[cf.][\nopage
  wp.\ §13]{Gpl30OsiLicense2007a}. Finally, the GPL-3.0 explains that
  \enquote{the GNU General Public License [itself] does not permit incorporating
  your program into proprietary programs} and that the LGPL might be a better
  license for those licensors who have written a \enquote{subroutine library
  [\ldots] and may consider it more useful to permit linking proprietary
  applications with the library [\ldots]}\footcite[cf.][\nopage wp.\ last
  parapgraph]{Gpl30OsiLicense2007a}. Again, being statically or dynamically
  linked is not used to trigger any license fulfilling actions. The conditions
  for \enquote{Conveying Modified [] Versions}] refer to the \enquote{work based
  on the Program}\footcite[cf.][\nopage wp.\ §5]{Gpl30OsiLicense2007a} which
  itself is another expression for a \enquote{\enquote{modified version} of the
  earlier work}\footcite[cf.][\nopage wp.\ §0]{Gpl30OsiLicense2007a}. Moreover,
  one is required \enquote{[\ldots] to license the entire work, as a whole,
  under this License to anyone who comes into possession of a
  copy}\footcite[cf.][\nopage wp.\ §5]{Gpl30OsiLicense2007a} . Hence, from the
  viewpoint of the GPL-3.0 there is no difference based on the type of linking.
  One can ignore these features of the work if one wants to determine how to use
  the software compliantly.

%   \item[AGPL-3.0] Concerning the use and the meaning of the words
%   \emph{dynamically} and \emph{linking} the AGPL-3.0 exactly follows the
%   structure of the GPL-3.0: first they arise in the context of defining the
%   \enquote{Corresponding Source}\footcite[cf.][\nopage wp.\
%   §0]{Agpl30OsiLicense2007a}; then the word \emph{link} helps to say that AGPL
%   and GPL are compatible licenses\footcite[cf.][\nopage wp.\
%   §13]{Agpl30OsiLicense2007a}; and finally the word \emph{link}is used to hint
%   to the LGPL\footcite[cf.][\nopage wp.\ §5]{Agpl30OsiLicense2007a}. So, again,
%   one can ignore the feature of being statically or dynamically linked if one
%   wants to determine how to use the software compliantly.
%   
%   \item[GPL-2.0] In the GPL-2.0 the word \emph{link} only arises in the context
%   of hinting to the LGPL\footcite [cf.][\nopage wp.\ last
%   paragraph]{Gpl20OsiLicense1991a}. Moreover, the words \emph{statical} and
%   \emph{dynamical} are not used in this text -- not at all and in no sense: the
%   copy left feature of the GPL depends 'only' on a specification which refers to
%   a \enquote{work based on the Program [\ldots] that in whole or in part
%   contains or is derived from the Program or any part thereof [\ldots]}\footcite
%   [cf.][\nopage wp\. §2]{Gpl20OsiLicense1991a}. Thus, even in this old version
%   of the GPL, the criteria of being linked in which way ever does not trigger
%   any task for the software compliantly.
%   
%   \item[LGPL-3.0] In the LGPL-3.0 the word \emph{link} is used to define the
%   concept of a \enquote{Combined Work} which shall be the name for a
%   \enquote{[\ldots] work produced by combining or linking an Application with
%   the Library}\footcite [cf.][\nopage wp\. §0]{Lgpl30OsiLicense2007a}. Finally,
%   LGPL-3.0 allows to \enquote{[\ldots] convey a Combined Work under terms of
%   [his own] choice [\ldots]}, provided that one distributes also all material
%   (including the object files of the overarching on-top developments) being
%   necessary for enabling the receiver to relink the whole product with a later
%   incoming newer version of the library or that one presupposes the use of a
%   \eqnuote{suitable shared library mechanism} so that the receiver can update
%   the library by this method\footcite [cf.][\nopage wp\.
%   §4]{Lgpl30OsiLicense2007a}. For fulfilling these conditions it is sufficient
%   to require that a distributor shall \emph{either distribute the on-top
%   development and the library in the form of dynamically linkable parts or
%   distribute the statically linked application together with a written offer,
%   valid for at least three years, to give the user all object-files of the
%   on-top development and the library, so that he can relink the application on
%   its own behalf}.
% 
%   \item[LGPL-2.1] Even if the arguing structure of the LGPL-2.1 is more
%   complicate than all the others, in its preamble it clearly says it want to
%   evoke: \eqnuote{If you link other code with the library, you must provide
%   complete object files to the recipients, so that they can relink them with the
%   library after making changes to the library and recompiling
%   it[\ldots]}\footcite[cf.][\nopage wp\. preamble]{Lgpl21OsiLicense1999a}. For
%   that purpose, the LGPL-2.1 firstly states that if \enquote{a program is linked
%   with a library, whether statically or using a shared library, [thern] the
%   combination of the two is legally speaking a combined work, a derivative of
%   the original library}\footcite[cf.][\nopage wp\.
%   preamble]{Lgpl21OsiLicense1999a}: even if a \enquote{work that uses the
%   Libary} -- which is only \enquote{[\ldots] designed to work with the Library
%   by being compiled or linked with it [\ldots]} -- \enquote{[\ldots] in
%   isolation, is not a derivative work of the library [\ldots]}, it is no
%   question for the LGPL-2.1, that \enquote{linking a \enquote{work that uses the
%   Library} with the Library creates an executable that is a derivative of the
%   Library (because it contains portions of the Library)}\footcite[cf.][\nopage
%   wp\. §5]{Lgpl21OsiLicense1999a}. But then -- \enquote{as an
%   exeption} -- the LGPL-2.1 allows to \enquote{[\ldots] combine or link a "work
%   that uses the Library" with the Library to produce a work containing portions
%   of the Library, and distribute that work under terms of your choice} provided
%   that the distributor either presupposes the use of a \eqnuote{suitable shared
%   library mechanism} or that he distributes also all material (including the
%   object files of the overarching on-top developments) being necessary for
%   enabling the receiver to relink the whole product with a later incoming newer
%   version of the library\footcite[cf.][\nopage wp\. §6, §6b adn §6c together
%   with §6c]{Lgpl21OsiLicense1999a}. Again, For fulfilling these conditions it is
%   sufficient to require that a distributor shall \emph{either distribute the
%   on-top development and the library in the form of dynamically linkable parts
%   or distribute the statically linked application together with a written offer,
%   valid for at least three years, to give the user all object-files of the
%   on-top development and the library, so that he can relink the application on
%   its own behalf}.

\end{description}

So based, with respect this analysis, there is -- in general -- no need to
gather more or less complicate whether one wants to distributed statically or
dynamically linked applications if one wants to derive the necessary tasks to
distribute this software compliantly. One can directly incorporate the quoted
contion into the task list of the LGPL -- and only there. This we could simplify
our leading form


%\bibliography{../../../bibfiles/oscResourcesEn}
