% Telekom osCompendium 'for being included' snippet template
%
% (c) Karsten Reincke, Deutsche Telekom AG, Darmstadt 2011
%
% This LaTeX-File is licensed under the Creative Commons Attribution-ShareAlike
% 3.0 Germany License (http://creativecommons.org/licenses/by-sa/3.0/de/): Feel
% free 'to share (to copy, distribute and transmit)' or 'to remix (to adapt)'
% it, if you '... distribute the resulting work under the same or similar
% license to this one' and if you respect how 'you must attribute the work in
% the manner specified by the author ...':
%
% In an internet based reuse please link the reused parts to www.telekom.com and
% mention the original authors and Deutsche Telekom AG in a suitable manner. In
% a paper-like reuse please insert a short hint to www.telekom.com and to the
% original authors and Deutsche Telekom AG into your preface. For normal
% quotations please use the scientific standard to cite.
%
% [ Framework derived from 'mind your Scholar Research Framework' 
%   mycsrf (c) K. Reincke 2012 CC BY 3.0  http://mycsrf.fodina.de/ ]
%


%% use all entries of the bibliography
%\nocite{*}

Grouping Open Sources licenses is often applied. You may find taxonomies in your
prefered magazins\footnote{see For example the German Linux Magazine 04/05 2011
% TODO Linux Magazin 04/05:2011 wegen Taxonomie in bibtex aufnehmen und zitieren
} or in more or less specific legal literature\footnote{\ldots must be found
% TODO Taxonomien aus juristischer literature zitieren
}. Even in the mailing lists of the Open Source Initiative recently it has been
discussed whether the OSI should cluster the existing licenses into the group of
the known licenses and the group of the more or less unknown licenses\footnote{
\ldots belegen
% TODO OSI license grouping in mailingslist nahschlgane und zitieren
% TODO OSI site classifcation referieren.
}.

One of the most known license taxonomies is established by the grouping of the
'permissive' licenses, the 'weak copyleft licenses' and 'strong copyleft
licenses'\footnote{ \ldots belegen
% TODO permissive, weak & strong copyleft taxonomie belegen aus Buch
}:

% TODO Taxonomie permissive, weak, strong mit psztricks setzen

All these taxonomies are invented to simplify the talking about Open Source
licenses: One wants to talk about the groups instead of regarding each license.
And one wants to elaborate the internal structure of the amorph field of
licenses. Structuring by generalization is the first method of human beings to
adopt an area and to produce knowledge:

The main group is established by the 10 criteria of the Open Source Definition.
Each Open Source license must fit each of these 10 criteria for being an Open
Source license. These 10 criteria forms the external borders. So, for getting
internal groups of this cluster, one has to refer to other criteria. There exist
extrinsic criteria like 'being known' or 'being unknown' and intrinsic criteria
like 'being permissive', 'weak copylefted' or 'strong copyleft'.

The most known clustering 'permissive', 'weak copylefted' and 'strong
copylefted' refers to some consequences one has to respect for getting the right
to use the software. This is a problematic viewpoint. It seduces the users to
think that they have only to fulfill these conditions. But indeed, also for the
most permissive license one has to do specific things. So, with respect to
question what we have to get the right to use, to modify and/or to distribute
the software it is a bad solution to think in the categories of 'permissive',
'weak copylefted' and 'strong copylefted' Open Source licenses.

Nevertheless, the human beings needs structering elements for enabling
themselves to talk about an area. Hence, we should offer another possibility to
cluster the Open Source licenses:

We think, that in general, licenses have a common purpose: they should protect
someone or something against something\footnote{which lateron might be refined
by the type of methods they use}. This is based on the nature of the word
'protect' is a 3 valent verb: it links someone or something, who protects, to
someone or something, who is protected and both together to something against
the protector protects and against the otherone is protected. Licenses do so,
even Open Source Licenses: They can protect the user (receiver) of the software,
its' contributor resp. developer and/or distributor, and the software itself.
And they protect them against different threats. With respect to this viewpoint,
we can specify the Open Source licenses in a specific, purpose orientied way:

\begin{itemize}
  \item Firstly, the BSD Licenses protect the user against the loss of the right
  to use, to modify and/or to distribute the received copy of the source code or
  the binary - which directly depends on the Open Source Definition.
  Secondly they protect the contributor and/or distributor against warranty
  claims of the user, because they contain a 'No Warranty Clause'. Thirdly they
  protect distributed sources against a reclosing change of the
  license, because each modification and distribution must retain the existing
  copy right notes and the licensing remarks, so that distributing under another
  license is not allowed.
  
  But
  \item The MIT Licenses protect
\end{itemize}


All these specifications can also be covered by a table:

\begin{table}
\footnotesize
\caption{Open Source Licenses as Protectors}
\begin{center}

\begin{tabular}{|c|c|c|c|c|c|c|c|c|c|c|c|c|c|c|c|}
\hline
  \multicolumn{2}{|c|}{\textit{Open}} &
  \multicolumn{12}{c|}{\textit{are protecting}}\\
\cline{3-14}
  \multicolumn{2}{|c|}{\textit{Source}} &
  \multicolumn{4}{c|}{ \textbf{Users}} &
  \multicolumn{3}{c|}{\textbf{Contributors}} &
  \multicolumn{5}{c|}{\textbf{Software}} \\
\cline{10-14}
  \multicolumn{2}{|c|}{\textit{Licenses}} &
  \multicolumn{4}{c|}{} &
  \multicolumn{3}{c|}{\tiny{(Distributors)}} &  
  not &
  \multicolumn{4}{c|}{distributed as} \\
\cline{3-9}\cline{11-14}
  \multicolumn{2}{|c|}{} &
  \multicolumn{4}{c|}{\scriptsize{\textit{who have already got}}} &
  \multicolumn{3}{c|}{\scriptsize{\textit{who spread Open}}} & 
  distri- &
  \multicolumn{2}{c|}{modified} &
  \multicolumn{2}{c|}{unmodified} \\
  \cline{11-14}
  \multicolumn{2}{|c|}{} &
  \multicolumn{4}{c|}{\scriptsize{\textit{sources or binaries}}} &
  \multicolumn{3}{c|}{\scriptsize{\textit{Source software}}} & 
  buted & 
 \footnotesize{sources} &
 \footnotesize{binaries} &
 \footnotesize{sources} &
 \footnotesize{binaries} \\
\cline{3-14}
  \multicolumn{2}{|c|}{} &
  \multicolumn{12}{c|}{\textit{against}}\\
\cline{3-14}
  \multicolumn{2}{|c|}{} &
  \multicolumn{3}{c|}{the loss of} & 
  \multirow{3}{*}{\rotatebox{270}{Patent Disputes}} &
  \multirow{3}{*}{\rotatebox{270}{Loss of Feedback}} & 
  \multirow{3}{*}{\rotatebox{270}{Warranty Claims}} & 
  \multirow{3}{*}{\rotatebox{270}{Patent Disputes}} & 
  \multicolumn{5}{c|}{}\\
% no seperator line 
  \multicolumn{2}{|c|}{} &
  \multicolumn{3}{c|}{the right to} &
  & & & &
  \multicolumn{5}{c|}{\footnotesize{Re-Closings of}}\\
\cline{3-5}
  \multicolumn{2}{|c|}{} & 
  \rotatebox{270}{use it} & 
  \rotatebox{270}{modify it} & 
  \rotatebox{270}{redistribute it} &
  &  &  &  &
  \multicolumn{5}{c|}{already Opened Software}\\
\hline
\hline
  \multirow{2}{*}{BSD}\footnotemark & 3-Cl & \checkmark & \checkmark  & \checkmark  & 
    $\neg$ & $\neg$ & \checkmark & $\neg$  &
    $\neg$ & \checkmark  & $\neg$ & \checkmark & $\neg$ \\
\cline{2-14}
   & 2-Cl & \checkmark  & \checkmark  & \checkmark  & 
    $\neg$ & $\neg$ & \checkmark & $\neg$  &
    $\neg$ & \checkmark  & $\neg$ & \checkmark & $\neg$ \\
\hline
  MIT\footnotemark & 1.1 & \checkmark  & \checkmark  & \checkmark  &
  - & - & - & - & - & - & - & - & - \\
\hline
  APL & 2.0 & \checkmark  & \checkmark  & \checkmark & 
  - & - & - & - & - & - & - & - & - \\
\hline
  MsPL & 1.1 & \checkmark  & \checkmark  & \checkmark  & 
  - & - & - & - & - & - & - & - & - \\
\hline
  PHP & 2.0 & \checkmark  & \checkmark  & \checkmark  &
  - & - & - & - & - & - & - & - & - \\
\hline
  PGL & 2.0 & \checkmark & \checkmark  & \checkmark &
  - & - & - & - & - & - & - & - & - \\
\hline
\hline
  \multirow{2}{*}{MPL} & 1.0 & \checkmark & \checkmark & \checkmark &
  - & - & - & - & - & - & - & - & - \\
\cline{2-14}
  & 2.0 & \checkmark & \checkmark & \checkmark &
  - & - & - & - & - & - & - & - & - \\
\hline
  EPL & 1.0 & \checkmark & \checkmark & \checkmark &
  - & - & - & - & - & - & - & - & - \\
\hline
  EUPL & 2.0 & \checkmark & \checkmark & \checkmark &
  - & - & - & - & - & - & - & - & - \\
\hline
  \multirow{2}{*}{LGPL} & 2.1 & \checkmark & \checkmark & \checkmark &
  - & - & - & - & - & - & - & - & - \\
\cline{2-14}
   & 3.0 & \checkmark & \checkmark & \checkmark &
   - & - & - & - & - & - & - & - & - \\
\hline
\hline
  \multirow{2}{*}{GPL} & 2.1 & \checkmark & \checkmark & \checkmark &
   - & - & - & - & - & - & - & - & - \\
\cline{2-14}
  & 3.0 & \checkmark & \checkmark & \checkmark &
   - & - & - & - & - & - & - & - & - \\
\hline
  \multirow{2}{*}{AGPL} & 2.1 & \checkmark & \checkmark & \checkmark &
   - & - & - & - & - & - & - & - & - \\
\cline{2-14}
   & 3.0 & \checkmark & \checkmark & \checkmark &
    - & - & - & - & - & - & - & - & - \\
\hline
\hline

\end{tabular}
\end{center}
\end{table}
\addtocounter{footnote}{-1}
\footnotetext{Berkeley Software Distribution [License]}
\stepcounter{footnote}
\footnotetext{Massuchat Institute of Technique [License]}

And the content of this classifying table can also be transfered into a mindmap:

\begin{tikzpicture}
\label{LICTAX}
\footnotesize

% (1.A) list of all licenses and their release numbers Level 5/6
\node[rectangle,draw,text width=1.4cm] (l0100) at (10,0.5)
{ \textit{BSD License} };
\node[text width=1.4cm] (l0101) at (12,0)
{ \scriptsize{3-Clauses} };
\node[text width=1.4cm] (l0102) at (12,1)
{ \scriptsize{2-Clauses} };
  
\node[rectangle,draw,text width=1.4cm] (l0200) at (11,2)
{ \textit{MIT License} }; 
\node[text width=0.4cm] (l0201) at (12.5,2) {\scriptsize{1.1}};
  
\node[rectangle,draw,text width=1.4cm] (l0300) at (12.5,3)
{ \textit{\textbf{AP}ache \textbf{L}icense}};
   \node[text width=0.4cm] (l0301) at (14,3) {\scriptsize{2.1}};

\node[rectangle,draw,text width=1.4cm] (l0400) at (13,4.5)
{ \textit{\textbf{M}icro\textbf{s}oft \textbf{P}ublic \textbf{L}icense} };
\node[text width=0.4cm] (l0401) at (14.5,4){\scriptsize{1.0}};
\node[text width=0.4cm,style=dotted] (l0402) at (14.5,5){\scriptsize{\textit{1.1}}};
  
\node[rectangle,draw,text width=1.4cm] (l0500) at (13,6)
{\textit{\textbf{P}ost\textbf{g}res \textbf{L}icense}};
\node[text width=0.4cm] (l0501) at (14.5,6){ \scriptsize{1.1}};
  
\node[rectangle,draw,text width=1.4cm] (l0600) at (13,7)
{\textit{PHP License}};
\node[text width=0.4cm] (l0601) at (14.5,7){\scriptsize{1.1}};
  
\node[rectangle,draw,text width=1.4cm] (l0700) at (13,8)
{ \textit{XXX License}};
\node[text width=0.4cm] (l0701) at (14.5,8){\scriptsize{1.1}};

\node[rectangle,draw,text width=1.4cm] (l0800) at (13,9.5)
{ \textit{\textbf{M}ozilla \textbf{P}ublic \textbf{L}icense}};
\node[text width=0.4cm] (l0801) at (14.5,9){\scriptsize{1.1}};
\node[text width=0.4cm] (l0802) at (14.5,10){\scriptsize{2.0}};

\node[rectangle,draw,text width=1.4cm] (l0900) at (13,11.5)
{\textit{\textbf{E}clipse \textbf{P}ublic \textbf{L}icense}};
\node[text width=0.4cm] (l0901) at (14.5,11) {\scriptsize{1.0}};
\node[text width=0.4cm,style=dotted] (l0902) at (14.5,12){\scriptsize{\textit{1.1}}};
 
\node[rectangle,draw,text width=1.5cm] (l1000) at (13,13.5)
{\textit{\textbf{E}uropean \textbf{P}ublic \textbf{L}icense}}; 
\node[text width=0.4cm] (l1001) at (14.5,13){\scriptsize{1.0}};
\node[text width=0.4cm,style=dotted] (l1002) at (14.5,14){\scriptsize{\textit{1.1}}};
  
\node[rectangle,draw,text width=1.4cm] (l1100) at (13,15.5)
{\textit{\textbf{L}esser \textbf{G}NU \textbf{P}ublic \textbf{L}icense}};

\node[text width=0.4cm] (l1101) at (14.5,15){\scriptsize{2.1}};
\node[text width=0.4cm] (l1102) at (14.5,16){\scriptsize{3.0} };

\node[rectangle,draw,text width=1.4cm] (l1200) at (13,17.5)
{\textit{\textbf{G}NU \textbf{P}ublic \textbf{L}icense}};

\node[text width=0.4cm] (l1201) at (14.5,17){\scriptsize{2.1}};
\node[text width=0.4cm] (l1202) at (14.5,18){\scriptsize{3.0} };

\node[rectangle,draw,text width=1.4cm] (l1300) at (13,19.5)
{ \textit{\textbf{A}ffero \textbf{G}NU \textbf{P}ublic \textbf{L}icense}};
\node[text width=0.4cm, style=dotted] (l1301) at (14.5,19){\scriptsize{2.1}};
\node[text width=0.4cm] (l1302) at (14.5,20){\scriptsize{3.0}};

% 2. the clustering concepts of licenses (level 4)
\node[rectangle,draw,text width=2.3cm] (n0100) at (10,8)
 { \textit{protecting the user, the con\-tri\-butor \& the initial code}\\
   \tiny{Permissive Licenses}      
 };

\node[rectangle,draw,text width=2.3cm] (n0200) at (10,11.5)
{ \textit{protecting the user, the con\-tri\-butor, the
  initial code, \& all di\-rect de\-ri\-va\-tions}\\
  \tiny{Weak Copyleft}        
};

\node[rectangle,draw,text width=2.3cm] (n0300) at (10,16.5)
{ \textit{protecting the user, the con\-tri\-bu\-tor, the 
  initial code, all di\-rect de\-ri\-va\-tions \& the 
  (in\-di\-rect\-ly de\-ri\-ved) on-top-deve\-lop\-ments}\\ 
  \tiny{Strong Copyleft}    
 };

% 3. the threats (level 3)
\node[ellipse,draw,text width=1.6cm] (c110000) at (4.5,0)
{ \textbf{\textit{Patent Disputes}}};

\node[ellipse,draw,text width=1.6cm] (c120000) at (4.5,2)
{ \textbf{\textit{Loss of Rights}} };

\node[ellipse,draw,text width=1.6cm] (c210000) at (4.5,4)
{ \textbf{\textit{Warranty Claims}} };
 
\node[ellipse,draw,text width=1.6cm] (c220000) at (4.5,6)
{ \textbf{\textit{Loss of Feeback}}};

\node[ellipse,draw,text width=0.4cm] (c311000) at (6.2,8)
{ \tiny{\textit{\textbf{clos\-ings}}}};

\node[ellipse,,draw,text width=0.4cm] (c321000) at (6.2,10)
{ \tiny{\textit{\textbf{clos\-ings}}} };

\node[ellipse,,draw,text width=0.4cm] (c331000) at (6.2,12)
{ \tiny{\textit{\textbf{clos\-ings}}} };

\node[ellipse,,draw,text width=0.4cm] (c341000) at (6.2,14)
{ \tiny{\textit{\textbf{clos\-ings}}} };

\node[ellipse,,draw,text width=0.4cm] (c351000) at (6.2,16)
{ \tiny{\textit{\textbf{clos\-ings}}} };

\node[ellipse,,draw,text width=1.6cm] (c411000) at (6.2,19)
{ \textit{\textbf{clos\-ings}} };


% 4. the subtypes of protected entities (level 2)
\node[ellipse,draw,text width=1.5cm] (c310000) at (3,8)
 { \scriptsize{un\-modified} \textbf{Sources}};

\node[ellipse,draw,text width=1.5cm] (c320000) at (3.25,10)
 { \scriptsize{un\-modified} \textbf{Binaries}};

\node[ellipse,draw,text width=1.2cm] (c330000) at (3.5,12)
 { \scriptsize{modified} \textbf{Sources}};

\node[ellipse,draw,text width=1.4cm] (c340000) at (3.25,14)
 { \scriptsize{modified} \textbf{Binaries}};

\node[ellipse,draw,text width=1.4cm] (c350000) at (3,16)
 { \scriptsize{part of \textbf{On-Top-Develop\-ments}}};


% 5. the protected entities (level 1)
\node[ellipse,draw,text width=1cm] (c100000) at (1,1)
 { \textbf{Users} };

\node[ellipse,draw,text width=0.8cm] (c200000) at (1,5)
 { \textbf{Con\-tribu\-tors}};

\node[ellipse,draw,text width=0.8cm] (c300000) at (1,12)
 { distri\-buted \textbf{Soft\-ware}};
 
\node[ellipse,draw,text width=2.2cm] (c400000) at (1,19)
 { un\-distri\-buted \textbf{Soft\-ware}}; 

% 6. main node (leve 0)
\node[ellipse,draw,text width=1.3cm] (c000000) at (0,8)
{ \textbf{Open Source License}};

% a linking Licenses to their release numbers (Linking level 5 to 6)
\foreach \father/\daughter in {
  l0100/l0101/,
  l0100/l0102/,
  l0200/l0201/,   
  l0300/l0301/,
  l0400/l0401/,
  l0400/l0402/,
  l0500/l0501/,
  l0600/l0601/,
  l0700/l0701/,
  l0800/l0801/,
  l0800/l0802/,
  l0900/l0901/,
  l0900/l0902/,
  l1000/l1001/,
  l1000/l1002/,
  l1100/l1101/,
  l1100/l1102/,
  l1200/l1201/,
  l1200/l1202/,
  l1300/l1301/,
  l1300/l1302/
  }
  \draw[dashed] (\father) to  (\daughter) ;

% b) linking Licenses to license concepts (Linking level 5 to 4)
\foreach \father/\daughter/\outangle/\inangle in {
  n0100/l0100/270/150,       
  n0100/l0200/280/155,
  n0100/l0300/290/160,
  n0100/l0400/300/165,
  n0100/l0500/310/150,
  n0100/l0600/340/160,
  n0100/l0700/360/180,
  n0200/l0800/300/160,
  n0200/l0900/340/170,
  n0200/l1000/20/190,
  n0200/l1100/60/200,
  n0300/l1200/40/180,
  n0300/l1300/80/180 
  }
  %\draw[dashed] (\father) to [out=\outangle,in=\inangle] (\daughter) ;
  \draw[dashed] (\father) to  (\daughter) ;

% c) linking license concepts to the threats against they protect
% c.1) strong copyleft licenses
\foreach \father/\daughter/\outangle/\inangle in {
  c351000/n0300/0/180,
  c341000/n0300/45/190,
  c331000/n0300/50/200,
  c321000/n0300/55/210,
  c311000/n0300/60/220,
  c220000/n0300/25/225,
  c210000/n0300/25/230,
  c120000/n0300/25/235
  }
  \draw[<-,color=blue] (\father) to [out=\outangle,in=\inangle] (\daughter) ;
% c.2) weak copyleft licenses
\foreach \father/\daughter/\outangle/\inangle in {
  c341000/n0200/330/170,
  c331000/n0200/0/180,
  c321000/n0200/0/180,
  c311000/n0200/20/190,
  c220000/n0200/15/220,
  c210000/n0200/15/230,
  c120000/n0200/15/235
  }
  \draw[<-,color=cyan] (\father) to [out=\outangle,in=\inangle] (\daughter) ;
% c.3) permissive licenses
\foreach \father/\daughter/\outangle/\inangle in {
  c331000/n0100/355/150,
  c311000/n0100/0/180,
  c210000/n0100/5/210,
  c120000/n0100/10/230
  }
  \draw[<-,color=red] (\father) to [out=\outangle,in=\inangle] (\daughter) ;
%c.4 agpl license
\foreach \father/\daughter/\outangle/\inangle in {
  c411000/l1300/0/180    
}
  \draw[<-,color=green] (\father) to [out=\outangle,in=\inangle] (\daughter) ;


%d linking protected entities, their subtypes and the the relations
\foreach \father/\daughter/\edgetext/\outangle/\inangle in {
  c000000/c100000/protecting/260/120,
  c100000/c110000/against/360/180,
  c100000/c120000/against/360/180,
  c000000/c200000/protecting/270/180,
  c200000/c210000/against/0/180,
  c200000/c220000/against/0/180,
  c000000/c300000/protecting/90/230,
  c300000/c310000/as/300/180,
  c300000/c320000/as/330/180,
  c300000/c330000/as/0/180,
  c300000/c340000/as/30/180,
  c300000/c350000/as/60/180,
  c000000/c400000/protecting/100/240,
  c400000/c411000/against/0/180        
}
  \draw[->,dotted,
    decoration={text along path,
              text align={center},
              text={|\itshape|\edgetext}},
              postaction={decorate},] (\father) to [out=\outangle,in=\inangle] (\daughter) ;

\foreach \father/\daughter/\edgetext/\outangle/\inangle in {
  c310000/c311000/against/0/180,
  c320000/c321000/against/0/180,
  c330000/c331000/against/0/180,
  c340000/c341000/against/00/180,
  c350000/c351000/against/0/180        
}
  \draw[->,dotted,
    decoration={text along path,
              text align={center},
              text={|\itshape \tiny|\edgetext}},
              postaction={decorate},] (\father) to [out=\outangle,in=\inangle] (\daughter) ;

%f linking the patent clauses
\foreach \father/\daughter/\outangle/\inangle in {
  c110000/l1302/0/295,
  c110000/l0401/0/360,
  c110000/l0301/0/360   
}
  \draw[<-,color=gray] (\father) to [out=\outangle,in=\inangle] (\daughter) ;

\end{tikzpicture}


Finally, one could generate new groups of Open Source license, new classes, like
'user protecting licenses'\footnote{all of them because all of them have to
fulfill the OSD}, 'patent disputes abwehrende licenses', 'modified \ldots'.

But one has to know: all of these grouping viewpoints do not allow to conclude,
that all members of a group can be respected by fulfilling the same
requirements. This would only be possible, if the grouping criteria would
directly refer to the fulfilling tasks. And indeed, nearly all Open Source
licenses do differ with respect to these criteria, even if the differences are
very small, they can't be neglected\footnote{Pars pro toto: Both, the BSD
license and the Apache license require, that you give hint to the developers of
the application. But in case of the BSD license you xýou have to \ldots. In case
of the Apache license you have exactly to present the content of the notice file
distributed together with the application.}. So: reflecting on possible classes
of Open Source licenses is a got method to become familiar with the area of Open
Source licenses. But it is not a method to determine, what one has to do for
getting the right to use the software. For getting these information, one has to
consider each single license.


%\bibliography{../../../bibfiles/oscResourcesEn}
