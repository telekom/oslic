% Telekom osCompendium 'for beeing included' snippet
%
% (c) Karsten Reincke, Deutsche Telekom AG, Darmstadt 2011
%
% This LaTeX-File is licensed under the Creative Commons Attribution-ShareAlike
% 3.0 Germany License (http://creativecommons.org/licenses/by-sa/3.0/de/): Feel
% free 'to share (to copy, distribute and transmit)' or 'to remix (to adapt)'
% it, if you '... distribute the resulting work under the same or similar
% license to this one' and if you respect how 'you must attribute the work in
% the manner specified by the author ...':
%
% In an internet based reuse please link the reused parts to www.telekom.com and
% mention the original authors and Deutsche Telekom AG in a suitable manner. In
% a paper-like reuse please insert a short hint to www.telekom.com and to the
% original authors and Deutsche Telekom AG into your preface. For normal
% quotations please use the scientific standard to cite.
%
% [ File structure derived from 'mind your Scholar Research Framework' 
%   mycsrf (c) K. Reincke CC BY 3.0  http://mycsrf.fodina.de/ ]

%
Of course we also browsed the internet. At least for German speaking people
there is an excellent site concerning the topic \emph{Open Source Licenses}.
offered by \textit{iffross}, which, loosely translated, means an
\textit{Institute for Legal Aspects of the Free and Open Source
Software}\footnote{originally: \glqq{}Institut für Rechtsfragen der Freien und
Open Source Software\grqq{}. Main entry point for its' site is the URL
\texttt{http://www.ifross.org/}.}, founded in 2000 as a private institute to
tracke the phenomenon 'free software' from the viewpoint of (German)
lawyers\footnote{\cite[cf.][\nopage wp]{ifross2011b}}. Besides many other
aspects this site offers a very well and thoroughly elaborated
FAQ\footnote{\cite[cf.][\nopage wp]{ifross2011c}} and a large list of Open
Source Licenses and other related licenses: moreover, evidently it is
classifying the Open Source Licenses in those 'without copyleft-effect' (BSD),
in those with 'strict copyleft-effect' (GPL)) and in those with 'restricted
copyleft-effect' (LGPL)\footnote{\cite[cf.][\nopage wp]{ifross2011a}}.

However, even this excellent site does not fulfill our needs. It does not offer
those context specific to-do lists which companies, developers or project
managers can use to ensure their Open Source Software is used in a regular
manner.

We therefore evaluated that standard book which is listed in the most legal
bibliographies\footnote{at least in that German judicial literature dealing with
Open Source}: the book of Jaeger and Metzger which concerns - loosely translated
- \textit{the judicial framework requirement for Open Source
Software}\footnote{\cite[cf.][V - It can not be any surprise that both authors,
Mr. Jaeger and Mr. Metzger are members of ifross (cf.
\texttt{http://www.ifross.org/personen/}, wp.)]{JaeMet2002a}}. Even the most
earliest edition of this book already had a clear structure in its' chapter
'copyright': For each license mentioned (or at least for each license cluster)
it offered a subchapter for the rights and a subchapter for the
duties\footnote{\cite[cf.][30ff]{JaeMet2002a}} of the software user\footnote{For
getting a good survey of the structure and the line of thought see the contents
(\cite[cf.][VIIIf]{JaeMet2002a})}. Many other important aspects of the topic
\textit{Open Source} are discussed, too\footnote{\cite[pars pro toto: have a
look at the chapter concerning the liability: cf.][137ff]{JaeMet2002a}}.

But we needed more than this. Despite the quality of the book we were certain
that we could not hand over this book to our programmers with the recommendation
\textit{check your touched licenses and follow the instructions of the relevant
subchapters\ldots}. This book did not contain simply checkable to-do lists,
either in the first edition\footnote{\cite[cf.][VIff]{JaeMet2002a}} and in the
second edition\footnote{\cite[cf.][VIIff]{JaeMet2006a}} or in the recently
published third edition\footnote{\cite[cf.][VIIIff. Naturally we use this latest
edition for adopting or discussing systematical aspects]{JaeMet2011a}}. So, how
can a company or a developer or a project manager be sure of fulfilling the
requirements of the Open Source Licenses sufficiently if he/she does not have a
verified list telling him \textit{'do this' and 'in case of that, do that', and
'then do this'}? Why should he himself implicitly become an Open Source Licenses
expert which has to extract the necessary steps out of the literature?

%\bibliography{../bibfiles/oscResourcesEn}
